% modified on 2009 april 26 1am with three 1-word changes

%% Dan Willard's Final Draft (November 10) for
%% ``Some Specially Formulated Axiomizations for I$\Sigma_0$ Manage to
%%  Evade the Herbrandized Version of the Second Incompleteness Theorem''

\documentclass{elsart}

% Use the option doublespacing or reviewcopy to obtain double line spacing
% \documentclass[doublespacing]{elsart}

% if you use PostScript figures in your article
% use the graphics package for simple commands
% \usepackage{graphics}
% or use the graphicx package for more complicated commands
% \usepackage{graphicx}
% or use the epsfig package if you prefer to use the old commands
% \usepackage{epsfig}

% The amssymb package provides various useful mathematical symbols
\usepackage{amssymb}

\begin{document}

\begin{frontmatter}

% Title, authors and addresses

% use the thanksref command within \title, \author or \address for footnotes;
% use the corauthref command within \author for corresponding author footnotes;
% use the ead command for the email address,
% and the form \ead[url] for the home page:
% \title{Title\thanksref{label1}}
% \thanks[label1]{}
% \author{Name\corauthref{cor1}\thanksref{label2}}
% \ead{email address}
% \ead[url]{home page}
% \thanks[label2]{}
% \corauth[cor1]{}
% \address{Address\thanksref{label3}}
% \thanks[label3]{}

 \title{Some Specially Formulated Axiomizations for
I$\Sigma_0$ 
Manage to
Evade
the Herbrandized Version of the Second Incompleteness Theorem}


\def\f22{\baselineskip = 1.0 \normalbaselineskip}
\def\xpsi{\Psi}
\def\iiir{I$\Delta_0^R$}
\def\iiisr{I$\Delta_0^R~$}
\def\iiid{I$\Delta_0$}
\def\iiisd{I$\Delta_0~$}
\def\iisi{I$\Delta_0~$}
\def\jjsj{I$\Sigma_0~$}
\def\jjj{I$\Sigma_0$}
\def\iiia{I$\Delta_0$'s}
\def\thsap{Theorem 2's$~$}
\def\thscp{Theorem 2,$~$}
\def\thsp{Theorem 2$\,$ }
\def\rrr{\mbox{\bf R}}
\def\srrr{\mbox{\bf S}}
\def\mxm{\mbox{\bf m}}

\def\beq{\begin{equation}}
\def\enq{\end{equation}}

\def\bel{\begin{lemma}}
\def\enl{\end{lemma}}
\def\ent{\end{theorem}}


\def\bec{\begin{corollary}}
\def\enc{\end{corollary}}

\def\bed{\begin{description}}
\def\ennd{\end{description}}
\def\bee{\begin{enumerate}}
\def\ene{\end{enumerate}}


% use optional labels to link authors explicitly to addresses:
% \author[label1,label2]{}
% \address[label1]{}
% \address[label2]{}

\author{Dan E. Willard}

\address{University of Albany }

\thanks[thank]{Email=dew@cs.albany.edu.
This research was partially 
supported by NSF Grant CCR  99-02726}



\begin{abstract}
In 1981, Paris and Wilkie \cite{PW81}
indicated  it was an open question whether
I$\Sigma_0$ 
would satisfy
the Second Incompleteness Theorem for  Herbrand deduction.
We will show that
some 
specially formulated axiomizations for
I$\Sigma_0$  can evade the
Herbrandized version of the Second Incompleteness Theorem.
\end{abstract}



\begin{keyword}
 G\"{o}del's
Second Incompleteness Theorem, 
Herbrand Consistency
\newline
MSC:  03B52; 03F25; 03F45; 03H13 




% keywords here, in the form: keyword \sep keyword


% PACS codes here, in the form: \PACS code \sep code
\end{keyword}
%\begin{pacs}
%
%\end{pacs}

\end{frontmatter}

\setlength{\parindent}{1.0 em}


\addtolength{\topmargin}{.9in}

\newcommand{\xbar}[1]{\widetilde{\, {#1} \,}}
\newcommand{\ggd}[1]{ \, \lceil \, {#1} \, \rceil \,}
\newcommand{\underx}[1]{\underbrace{~ {#1} ~}}
\newcommand{\unders}[1]{\underbrace{\, {#1} \,}}
\newcommand{\eq}[1]{(\ref{#1})}
\newcommand{\ep}[1]{Equation (\ref{#1})}

\newcommand{\co}[1]{Corollary \ref{#1}}
\newcommand{\thx}[1]{Theorem \ref{#1}}
\newcommand{\lxem}[1]{Lemma \ref{#1}}

\newcommand{\tll}[1]{Tab$- {#1} -$List}
\newcommand{\txl}[1]{Tab$- {#1}$}
\newcommand{\tlxl}[1]{Tab$- {#1}$ }

\newcommand{\hll}[1]{Herb$- {#1} -$List}
\newcommand{\hxl}[1]{Herb$- {#1}$}
\newcommand{\hlxl}[1]{Herb$- {#1}$ }



\def\xpsi{\Psi}
\def\iiir{I$\Delta_0^R$}
\def\iiisr{I$\Delta_0^R~$}
\def\iiid{I$\Delta_0$}
\def\iiisd{I$\Delta_0~$}
\def\iisi{I$\Delta_0~$}
\def\jjsj{I$\Sigma_0~$}
\def\jjj{I$\Sigma_0$}
\def\iiia{I$\Delta_0$'s}
\def\thsap{Theorem 2's$~$}
\def\thscp{Theorem 2,$~$}
\def\thsp{Theorem 2$\,$ }
\def\rrr{\mbox{\bf R}}
\def\srrr{\mbox{\bf S}}
\def\mxm{\mbox{\bf m}}

\def\beq{\begin{equation}}
\def\enq{\end{equation}}

\def\bel{\begin{lemma}}
\def\enl{\end{lemma}}
\def\ent{\end{theorem}}


\def\bec{\begin{corollary}}
\def\enc{\end{corollary}}

\def\bed{\begin{description}}
\def\ennd{\end{description}}
\def\bee{\begin{enumerate}}
\def\ene{\end{enumerate}}


\newcommand{\overx}[1]{\, \overline{ {#1} } \,} 
\newcommand{\oveb}[1]{~ \overbrace{\, {#1} \, } } 
\newcommand{\ovec}[1]{ \overbrace{\, {#1} \, } } 


\newcommand{\newthmwithin}[3]{\newtheorem{#1q}{#2}[#3]
                        \newenvironment{#1}{\begin{#1q}\sf}{\end{#1q}}}

\newcommand{\newthm}[3]{\newtheorem{#1q}[#2q]{#3}
                        \newenvironment{#1}{\begin{#1q}\sf}{\end{#1q}}}
\newcommand{\newthmm}[3]{\newtheorem{#1q}[#2q]{#3}
                        \newenvironment{#1}{\begin{#1q}\rm}}

\newtheorem{theorem}{Theorem}

\newtheorem{definition}{Definition}
\newtheorem{lemma}{Lemma}
\newtheorem{corollary}{Corollary}
\newtheorem{fct}{Fact}


\newenvironment{proof}{{\it Proof:}}{$\Box$}



\setlength{\parindent}{1.0 em}
\setcounter{page}{1}



\normalsize


\section{Introduction}



 G\"{o}del's
Second Incompleteness
Theorem \cite{Go31} asserts
that neither Peano Arithmetic, nor any
consistent extension of it, can prove a theorem
affirming its own self-consistency under Hilbert
deduction. There have been numerous generalizations
and extensions
of G\"{o}del's seminal result
\cite{Ad2,AB1,AZ1,BS76,BI95,Da83,DPR61,Fe60,HB39,Ko7,Lo55,Ma93,PW81,Pu84,Pu85,Pu96,Sa1,Sm77,Sm85,So94,Sw3,TMR53,Ta0,Vi93,Vi5,WP87,ww0,ww2,ww5,ww6}.
For example, Solovay \cite{So94} has 
combined the work of
Nelson, Pudl\'{a}k  and Wilkie-Paris
\cite{Ne86,Pu85,WP87}
to establish
that essentially no axiom system that recognizes
Successor$(x)~=~x+1~$ as a total function
(and which treats addition and multiplication as 3-way
relations)
 can prove a theorem
affirming its own consistency under 
the conventional paradigm of
Hilbert deduction.



In 1981, Paris and Wilkie \cite{PW81} 
noticed that it was an open question whether
the axiom system \jjsj
did satisfy
the Second Incompleteness Theorem for 
semantic tableaux and Herbrand deduction.
Interestingly
at the same time,  Paris-Wilkie observed that
there was an available solution to this problem for Hilbert deduction.
Thus,  
\jjj +Exp is unable to prove the Hilbert consistency of even
an axiom system as simple as Q \cite{WP87}.
Subsequently.
 Adamowicz-Zbierski \cite{Ad2,AZ1} showed that \jjj $+ \Omega_1$ was unable
to verify its Herbrand  consistency,
and
Willard \cite{ww0,ww2}
developed an alternate variant of the
  Adamowicz-Zbierski formalism that showed at
least one type of natural encoding of the 
\jjsj axiom system would satisfy the semantic tableaux version
of the 
Second Incompleteness Theorem.

On 16 November 2005, we received a fascinating 
email communication
from 
L.A. Ko{\l}odziejczyk 
about this subject
(which was an outgrowth out of some conversations
he had with
 Zofia  Adamowicz and
Konrad  Zdanowksi). 
It
observed that there are  two
natural formalisms 
for axiomatizing \jjj , henceforth called
Ax-1 and Ax-2. Both  shall take  the Tarski-Mostowski-Robinson
axiom system $Q$ as their starting base.
In a context where $\phi(x,y)$ is
a $\Delta_0$ formula, these formalisms will use
respectively 
Equations \eq{hp} and \eq{wp} to denote
their induction schemes.
\begin{equation}
\label{hp}
 \forall x  ~~\{~~ \{~~
 \phi(x,0)\,\,\wedge\,\, \forall y ~[~
\phi(x,y) \, \Longrightarrow \, \phi(x,y')\, \, ] \,\,  \} \,~  \Longrightarrow \,~
 \forall y ~ \phi(x,y)~\, \}
\end{equation}
\begin{equation}
\label{wp}
\forall x  \,  \forall z  \,    \{  \,    
 \{ \,    \phi(x,0) \,  \, \wedge \,  \,  \forall y \leq z \, [ \, 
\phi(x,y) \, \Longrightarrow \, \phi(x,y') \,  \, ]   \, \}  \,  \,   \Longrightarrow  \,  \, 
 \forall y \leq z \,  \phi(x,y)  \, \}~~
\end{equation}
 Ko{\l}odziejczyk noticed 
that
logically equivalent axiom systems, such as Ax-1 and Ax-2,
do not necessarily have the same properties with regards to
the semantic tableaux and Herbrandized versions of the Second Incompleteness
Theorem. Thus,
Ko{\l}odziejczyk's email
asked 
whether \cite{ww2}'s semantic tableaux version of
the Second Incompleteness Theorem
will generalize for Ax-2's 
unconventional induction scheme?
It also inquired to what extent are
generalizations of the Second Incompleteness Theorem
germane to Herbrand deduction? One reason the second question
is especially intriguing is that 
Ko{\l}odziejczyk demonstrated in \cite {Ko6b,Ko7}
that there exists bounded arithmetics where Herbrand consistency
and tableaux consistency are provably not logically equivalent.


One part of our  answer to this question 
had appeared in the separate
paper 
\cite{ww7}. It  explained how our
prior results about Ax-1's  
semantic tableaux
incompleteness properties
have direct generalizations  for Ax-2.  

A second type of  answer to 
Ko{\l}odziejczyk's stimulating 
open
question will appear in this
paper. We will prove that there is a third type of axiomization of
\jjj, called Ax-3, which is logically equivalent to Ax-1 and Ax-2, but
which has the property that an extension of Ax-3 is capable of
recognizing its own Herbrand consistency.

This last  result is likely to raise almost as many questions
as it does answer. This is because there are many potential logically
equivalent axiomizations for \jjj.
$~$Thus,
one may ask for which 
potential 
particular
axiomizations $~\alpha~$ 
for \jjsj do
the Questions (1) and (2) below  have a positive answer?
\bee
\item Are all extensions of 
$~\alpha\,$'s axiomization for \jjsj unable to
prove a theorem verifying their own semantic tableaux consistency?
\item Likewise, are   all extensions of
$~\alpha\,$'s axiomization for \jjsj unable to
prove a theorem verifying their own Herbrand consistency?
\ene
Since the Second Incompleteness Theorem 
generalizes for most types of axiom systems, it is of course reasonable
to conjecture that 
most of
the answers to the Questions 1 and 2 (above)
will be
in a positive direction.
However, the point of this article is
that an automatic ``yes'' response to Questions 1 and 2 cannot be
always secured. Thus,
Ko{\l}odziejczyk \cite{Ko6b,Ko7} has shown that 
Herbrand consistency and semantic tableaux consistency are not always
equivalent to each other, and the current article will 
actually
construct a
particular 
formalization of \jjj, called
Ax-3, that manages to evade
at least
Question 2's
Herbrandized 
version of
the Second Incompleteness Theorem.


\section{  The Definition of a New Version of \jjj }
\label{s2}



This section will 
define  the  Ax-3  axiomatization for
\jjsj 
and  provide the
formal statement of our main theorem
(which will be subsequently proven in 
Sections \ref{s3} and \ref{main}). 
As the reader examines the 
formal definitions 
in the current section, 
it should be kept in mind that Ax-3 was
deliberately endowed with an unconventional method
for axiomatizing 
\jjsj so that its formalism will evade the Herbrandized version
of the Second Incompleteness Theorem.
(Moreover
because
of
 a
technical difference between the definitions of
Herbrand consistency and
semantic tableaux consistency,
it should be kept in mind that
 Ax-3's evasion of the
 Second Incompleteness Theorem
under a Herbrandized definition of consistency
does not generalize for semantic tableaux deduction.)



In our discussion,
a formula will be called 
 $~\Delta_0^R~$
iff it has a structure similar to a 
 $~\Delta_0~$ formula 
(appearing in for example the H\'{a}jek -Pudl\'{a}k textbook \cite{HP91})
except that its bounded quantifiers,
``$~\forall ~v \, \leq \, T~$'' and  ``$~\exists ~v \, \leq \, T,~$'',
are now
disallowed from using  the conventional arithmetic functions
of addition and multiplication  in their terms $~T~$.
Instead, the terms of a   $~\Delta_0^R~$ formula will
employ only  the maximum function as the only  permissible operator to define
a variable's bounded range.
(Arithmetic functions are allowed to appear elsewhere in the body of
a   $~\Delta_0^R~$ formula.)
Thus, \ep{ex1} is an example of a  
 $~\Delta_0^R~$ formula, and \eq{ex3}
is  an example of a  
 $~\Delta_0~$ formula that is not  $~\Delta_0^R~$.
\begin{equation}
\label{ex1}
\forall ~p \leq~ \mbox{Max}(x,y)   ~~~[~~(~ p + y \leq ~ x \, + \, 2*y)~
\vee ~(~ p * y~ \leq~ y*y*y~)~]~~
\end{equation}
\begin{equation}
\label{ex3}
\forall ~p \leq~x*y  ~~  ~~~[~~(~ p + y \leq ~ x \, + \, 2*y)~
\vee ~(~ p * y~ \leq~ y*y*x~)~]
\end{equation}


Let us say a  formula is
 $\Pi_1^R$ 
iff it can be written as
$~\forall \, v_1 \,  \forall \, v_2 \,  ...  \, \forall \, 
v_n~ \phi(v_1,v_2, \, ... \, ,\, v_n) $ where
$\phi(v_1,v_2,~ \ldots ,~ v_n)$ 
is a  $ \Delta_0^R $ formula. 
Each of  Ax-1, Ax-2 and Ax-3  will contain
a common set of nine
 $\Pi_1^R$ 
 axioms,  called $~Q_0~$ and listed below.
The main purpose of $~Q_0~$ will be to define
the constructs 
of addition, multiplication, integer-successor, maximum
and also $\, = \,$ and $\, \leq$.
\begin{equation}
\label{pw1}
 1 \, = \, 0'  ~ \, \wedge~ \, 
  2 \, = \, 1'  ~ \, \wedge~ \, 
 0 \, = \, 0  ~ \, \wedge~ \, 
0' \, \neq \, 0  ~ \, \wedge~ \, 
0  \,\leq \, 0 ~ \, \wedge~ \,  
\neg~[~ 0'  \,\leq \, 0~] 
\end{equation}
\begin{equation}
\label{pw2}
\forall ~x~~(~~ x+0 \, = \, x ~~ \wedge
~~ x \cdot 0 \, = \, 0 ~~ \wedge
~~ x \cdot 1 \, = \, x ~~ )
\end{equation}
\begin{equation}
\label{pw3}
\forall x ~ \forall y ~~ (~~ x' \, = \, y'~~ \Longleftrightarrow ~~ x \, = \, y ~~)
\enq
\begin{equation}
\label{pw4}
\forall x ~ \forall y ~~ (~~
x \, \leq \, y~~ \Longleftrightarrow ~~(~ x' \, \leq \, y ~ \vee ~ 
x \, = \, y ~~)
\end{equation}
\begin{equation}
\label{pw6a}
\forall x ~ \forall y ~~~
  ~~~~  x \cdot y'~=~ (x \cdot y)+x ~~~
\wedge ~~~ x+y'~=~ (x+y)'
\end{equation}
\begin{equation}
\label{pw6b}
\forall x ~ \forall y  ~ \forall z ~~~
[~x=y ~ \wedge ~ y=z ~] ~ \Rightarrow~ [~x=z ~\wedge ~z=x ~]
\end{equation}
\begin{equation}
\label{pw6c}
\forall x ~ \forall y  ~ \forall z ~~~
[~x=y ~ \wedge ~  y \leq z  ~] ~ \Rightarrow~ x \leq z
\end{equation}
\begin{equation}
\label{pw6.1}
\forall x ~ \forall y  ~ \forall z ~~~
[~x=y ~ \wedge ~  z \leq y  ~] ~ \Rightarrow~ z \leq x
\end{equation}
\begin{equation}
\label{pw6}
 \forall x  \,   \forall y  \,  
  \,  \, ( \,  \,  x \, \leq \, y \,  \, 
\Rightarrow   \,  \,  \mbox{Max}(x,y)=y \,  \, )
  \,  \,  \wedge \,  \, 
( \,  \,  y \, \leq \, x \,  \, 
\Rightarrow   \,  \,  \mbox{Max}(x,y)=x \,  \, )
\end{equation}
In the context of the above definition for $Q_0$,
 the Ax-1 and Ax-2 formalisms
will be respectively
defined
 as the union of $~Q_0~$ with all 
the instances of the 
respective
induction
schemes in
 Equations \eq{hp} and \eq{wp},
where $~\phi(x,y)~$ is a 
$\Delta_0$ formula.
Similarly,$~$ \iiisr will be defined as the union of
 $~Q_0~$ with all instances of 
\ep{wp}'s induction
schemas  where $~\phi(x,y)~$ is 
$\Delta_0^R$. 


This paragraph will define a set of
$\Pi_1^R$ sentences, called {\bf Trivial-R}, that has the property that
\iiir $\, + \,$Trivial-R proves the same set of theorems as the
more conventional Ax-1 and Ax-2
axiomatization for \jjj .
In our discussion, a
tuple $ ( a_0 ,  a_1 ,  a_2 , ~ \ldots ,~   a_N ) $
is called
a {\bf  Split Representation} of an non-negative 
integer $~ x~ $
when the following condition is satisfied:
\begin{equation}
\label{r111}
x \,  \, = \,  \,  \sum_{i=1}^{N} \, a_i  \,  * \,  (a_0+1)^{i-1}
    \,  \,  \,  \,  \mbox{AND}  \,  \,  \,  \, 
a_1  \,   \leq \,  a_0 \, \wedge \, a_2  \,   \leq \,  a_0 \, \wedge \, ... \, 
a_N  \,   \leq \,  a_0 \,  
\end{equation}
For a fixed integer $N$,
 let Split$^N( \, x \, , \, a_0 \, , \, a_1 \, , \, ~ \ldots ,~   \,  \, a_N \, ) \, $
denote a  $\Delta_0^R$ formula indicating  \eq {r111} is satisfied.

For each of the  arithmetic operators of
$\, + \,$, $\, * \,$, Max, $\, = \,$ and $\, \leq , \,$ the
axiom system Trivial-R will have available a family of
$\Delta_0^R$ predicates 
and $\Pi_1^R$ axioms
for simulating the 
operations of these functions when they
manipulate Split
Representations of integers.
Thus for a fixed triple
$~(I,J,K)$, let 
Mult$^{I,J,K}(a_0,a_1,  \, ..., \,  a_I,b_0,b_1, \, ..., \,  b_J,c_0,c_1,  \, ..., \,  c_K)$
designate a $\Delta_0^R$ predicate
simulating the action of integer multiplication when its input is the
two
split integers of 
$(a_0,a_1,~ \ldots ,~ a_I) ~$ and $(b_0,b_1,~ \ldots ,~ b_J)~$ and
its resultant is 
the multiplicative product of $~(c_0,c_1,~ \ldots ,~ c_K).~$
The accompanying 
$\Pi_1^R$ axiom 
of Trivial-R
that formalizes this predicate  will
be:

\bigskip

{\baselineskip = 1.0 \normalbaselineskip
\begin{center}
$\forall~x~~\forall~y~~\forall~z~~
\forall~a_0~~\forall~a_1~... ~\forall~a_I~~
\forall~b_0~~\forall~b_1~... ~\forall~b_J~~
\forall~c_0~~\forall~c_1~... ~\forall~c_K~~~$
\end{center}
\begin{center}
$\{~~~[~~ ~\mbox{Split}^I(x,a_0,~ ... \,  , ~a_I)
~\wedge ~\mbox{Split}^J(y,b_0, ~ ... \,  , ~b_J)
~\wedge ~\mbox{Split}^K(z,c_0, ~ ... \,  , ~c_K)~~~]
~~~\Longrightarrow ~~ $
\end{center}
\begin{equation}
\label{trivr-1}
[~~x*y=z~~~\Longleftrightarrow ~~ 
\mbox{Mult}^{I,J,K}(a_0, ~ ... \,  , ~a_I,b_0, ~ ... \,  , ~b_J,c_0, ~ ... \,  , ~c_K)   ~]~~\}~~~
\end{equation}}

\bigskip

\noindent
Likewise,  Trivial-R 
will have available  analogs
of \ep{trivr-1}
to
 simulate
 addition, maximum,  equality, and less-than-or-equals
among
split integers.
Thus, 
the predicates of
Add$^{I,J,K}(a_0,a_1, ~ \ldots , ~ a_I,b_0,b_1, ~ \ldots , ~ b_J,c_0,c_1, ~ \ldots , ~ c_K)$,
$~$Maxim$^{I,J,K}(a_0,a_1, ~ \ldots , ~ a_I,b_0,b_1, ~ \ldots , ~ b_,c_1, ~ \ldots , ~ c_K)$,
$~$Eq$^{I,J}(a_0,a_1, ~ \ldots , ~ a_I,b_0,b_1, ~ \ldots , ~ b_J)$
and
$~$LTE$^{I,J}(a_0,a_1, ~ \ldots , ~ a_I,b_0,b_1, ~ \ldots , ~ b_J)$
will be the
$\Delta_0^R$
 analogs of 
Mult$^{I,J,K}$
for these four structural relations.
 Their counterparts of 
 \ep{trivr-1}'s formal axiom  will then be:


\bigskip


{\baselineskip = 1.0 \normalbaselineskip


\begin{center}
$\forall~x~~\forall~y~~\forall~z~~
\forall~a_0~~\forall~a_1~... ~\forall~a_I~~
\forall~b_0~~\forall~b_1~... ~\forall~b_J~~
\forall~c_0~~\forall~c_1~... ~\forall~c_K~~~$
\end{center}
\begin{center}
$\{~~~[~~ ~\mbox{Split}^I(x,a_0...a_I)
~\wedge ~\mbox{Split}^J(y,b_0...b_J)
~\wedge ~\mbox{Split}^K(z,c_0...c_K)~~~]
~~~\Longrightarrow ~~ $
\end{center}
\begin{equation}
\label{trivr-2}
[~~x+y=z~~~\Longleftrightarrow ~~ 
\mbox{Add}^{I,J,K}(a_0...a_I,b_0...b_J,c_0...c_K) ~~]~~~\}
\end{equation}

\bigskip




\begin{center}
$\forall~x~~\forall~y~~\forall~z~~
\forall~a_0~~\forall~a_1~... ~\forall~a_I~~
\forall~b_0~~\forall~b_1~... ~\forall~b_J~~
\forall~c_0~~\forall~c_1~... ~\forall~c_K~~~$
\end{center}
\begin{center}
$\{~~~[~~ ~\mbox{Split}^I(x,a_0...a_I)
~\wedge ~\mbox{Split}^J(y,b_0...b_J)
~\wedge ~\mbox{Split}^K(z,c_0...c_K)~~~]
~~~\Longrightarrow ~~ $
\end{center}
\begin{equation}
\label{trivr-m}
[~~z~=~\mbox{Max}(x,y)~~~\Longleftrightarrow ~~ 
\mbox{Maxim}^{I,J,K}(a_0...a_I,b_0...b_J,c_0...c_K) ~~]~~~\}
\end{equation}

\bigskip


\begin{center}
$\forall~x~~\forall~y~~
\forall~a_0~~\forall~a_1~... ~\forall~a_I~~
\forall~b_0~~\forall~b_1~... ~\forall~b_J~~
  ~~~[~ \,\mbox{Split}^I(x,a_0...a_I)
~\wedge ~\mbox{Split}^J(y,b_0...b_J) ~]$
\end{center}
\begin{equation}
\label{trivr-3}
 ~~~~~~~~\Longrightarrow ~~ 
[~~x=y~~~\Longleftrightarrow ~~ 
\mbox{Eq}^{I,J}(a_0...a_I,b_0...b_J) ~~]
\end{equation}

\bigskip

\begin{center}
$\forall~x~~\forall~y~~
\forall~a_0~~\forall~a_1~... ~\forall~a_I~~
\forall~b_0~~\forall~b_1~... ~\forall~b_J~~
  ~~~[ \, ~\mbox{Split}^I(x,a_0...a_I)
~\wedge ~\mbox{Split}^J(y,b_0...b_J) ~]$
\end{center}
\begin{equation}
\label{trivr-5}
 ~~~~~~~~\Longrightarrow ~~ 
[~~x \leq y~~~\Longleftrightarrow ~~ 
\mbox{LTE}^{I,J}(a_0...a_I,b_0...b_J) ~~]
\end{equation}

}

\noindent
Henceforth,  {\bf Ax-3} will denote the
axiom system 
\iiir $\, + \,$Trivial-R.
Section \ref{s3} will prove that
Ax-3 proves the
same set of theorems as Ax-1 and Ax-2.



\bigskip

{\bf Definition 1: }
Let $ \, \alpha \, \supseteq \, \beta \, $ denote that $ \, \alpha$'s
set of formal axioms
includes all $ \, \beta \,$'s axioms.
(This definition of $ \, $``$ \, \supseteq \,$''$ \, $
is  stronger than the 
{\it more modest construct} that
$ \, \alpha \, $ proves all $ \, \beta \,$'s theorems.)
Also assuming 
$ \, \alpha \, $ denotes a {\bf consistent} axiom system
and $ \, D \, $ denotes a deductive method,
$(\alpha,D)$ will be called a
{\bf Threshold}
for the Second Incompleteness Effect iff
all consistent extensions
$ \, \alpha^* \, \supseteq \, \alpha \, $ have the property
that 
 $ \, \alpha^* \, $ is unable to prove the consistency
of its proofs using deduction method $D$.
Otherwise,  $(\alpha,D)$  will be called an 
{\bf Anti-Threshold}. (It  means that {\it some consistent}
$ \, \alpha^* \, \supseteq \, \alpha \, $ 
can prove a theorem affirming its own
consistency
under deduction method $D$.)


\bigskip

Using Definition 1's notation, the main theorem proven
in this article  will be
that Ax-3 is an anti-threshold for the Herbrandized version of the
Second Incompleteness Theorem.
This means that there must assuredly exist some {\it consistent}
system
$~\alpha^* ~ \supseteq ~$Ax-3, where $\alpha^*$ can prove
a theorem affirming its own Herbrand consistency.


\section{ Basic Framework and  Underlying Intuition}
\label{s3}


This section
will have two purposes. It
will formally
prove  Ax-3 proves the same set of
theorems as Ax-1 and Ax-2,
thus establishing that Ax-3 is a
permissible
 formal means for
axiomatizing \jjj . 
It will also provide an intuitive explanation of why
Ax-3 is able to evade the Herbrandized version of
the Second Incompleteness Theorem (by satisfying Definition 1's
``anti-threshold'' property
for Herbrand consistency).







\begin{theorem}
\label{tm4}
Each of 
Ax-1, Ax-2 and Ax-3 prove the
same set of theorems.
\end{theorem}


{\bf Proof Sketch:} It is well known  Ax-1 and Ax-2 prove the same
set of theorems. 
Thus to establish \thx{tm4}, we
need  only
show  Ax-2 and Ax-3.
also prove the same
set of theorems. 

Our proof will use the fact that 
 Paris and  Dimitracopoulos \cite{PD82} have observed that
in
a
 model-theoretic sense, there is a 1-to-1 correspondence
between $\Delta_0$ formulae and their equivalent representations
in a   $\Delta_0^R$ form.
Let $\psi(x,y)$ 
denote an arbitrary $\Delta_0^R$ formula. 
The  Paris and  Dimitracopoulos \cite{PD82} result easily implies
that
for any integer $k$, 
it is possible to construct 
a $\Delta_0^R$ formula 
 $\psi^*(x,y_0,y_1,~ \ldots , ~y_k)$ that is its
logical
 counterpart of
 $\psi(x,y)$ under split representations for an integer.
More precisely, it implies that one
can map
$\psi(x,y)$ onto
a $\Delta_0^R$ formula 
 $\psi^*(x,y_0,y_1,~ \ldots , ~y_k)$ 
such that the Ax-2 and Ax-3 representations of
\jjsj can both trivially prove 
the following property:
\begin{center}
\small
$\forall~x~
 ~\forall~y~~
\forall~y_0~~\forall~y_1~... ~\forall~y_k$
\end{center}
\begin{equation}
\small
\label{etm4}
 \{~\mbox{Split}^k(y,y_0,y_1,~ \ldots , ~y_k)~~
\Longrightarrow ~~
[~\psi(x,y)~~ \Longleftrightarrow~~
\psi^*(x,y_0,y_1,~ \ldots , ~y_k)~~]~ \}~~
\end{equation}

Let Size$_L(y_0,y_1,~ \ldots , ~y_k)$ denote a $\Delta_0^R$ formula indicating
that $(y_0,y_1,~ \ldots , ~y_k)$ represents an integer $\, \leq \, L.~$
Then it is not hard to show that
Ax-3 can use its Trivial-R axioms
to prove
 that
the two $\Delta_0$ formulae of
$~\exists y \leq x^k~ \, \psi(x,y)~~~$
and
$~~~\forall y \leq x^k~ \, \psi(x,y)~~~$
are equivalent to the respective  $\Delta_0^R$ formulae of:
\begin{equation}
\small
 \exists \,  y_0 \, \leq \, x \, \, 
\exists \,  y_1 \, \leq \, x \, \, ...~ \, 
\exists \,  y_k \, \leq \, x \, \, ~~
\mbox{Size}_{x^k}(y_0,y_1, \, ... \, , ~y_k)\wedge \psi^*(x,y_0,y_1, \, ... \, , ~y_k) ~~
\end{equation}
\begin{equation}
\footnotesize
 \forall \,  y_0 \, \leq \, x \, \, 
\forall \,  y_1 \, \leq \, x \, \, ...~ \, 
\forall \,  y_k \, \leq \, x \, \,~~ 
\mbox{Size}_{x^k}(y_0,y_1, \, ... \, , ~y_k)\Rightarrow \psi^*(x,y_0,y_1, \, ... \, , ~y_k) ~~
\end{equation}


 Thus by essentially applying $~n~$ iterations of this
technique
(and its obvious analogs)
 to any initial 
 $\Delta_0$ formula with $ \, n \,$ bounded quantifiers,
Ax-3 can transform an arbitrary
 $\Delta_0$ formula into a provably equivalent  $\Delta_0^R$
formula. It thus follows that although the Ax-3 system
contains technically only instances of
\ep{wp}'s axiom schema
 for   $\Delta_0^R$ 
formulae, it nevertheless
has an ability to formally prove as theorems all the
remaining instances of
Ax-2's axiom schema for  $\Delta_0$ formulae {\it as well.}
(In particular if 
$\phi(x,y)$ 
is 
a $\Delta_0$ formula which is not 
 $\Delta_0^R$ 
and if
$\phi^*(x,y)$ is a  
 $\Delta_0^R$ formula equivalent to 
$\phi(x,y)$,  
then Ax-3 can prove a theorem verifying
the validity of 
\ep{wp}'s axiom scheme for 
$\phi(x,y)$  by first observing that 
this axiom scheme is
 valid for 
$\phi^*(x,y)$
and then observing the
latter is equivalent to 
$\phi(x,y)$'s axiom scheme.)


Hence although Ax-3 contains technically only a
proper superset of Ax-2's induction axioms as formal axioms
within its own induction schema, it has the ability to
formally prove the validity of all of Ax-2's axioms.
The proof in the reverse direction (establishing that Ax-2 can prove all
Ax-3's axioms) is 
straightforward
 because Ax-2 can easily prove
all Ax-3's Trivial-R axioms.
Hence,
Ax-2 and Ax-3 will
prove  identical sets of theorems.
 $\Box$





\bigskip

{\bf Remark 1.} $~$
Our proof that Ax-3 is an anti-threshold for the 
Herbrandized version of the Incompleteness Theorem will
appear in Section \ref{main}. 
This result is quite surprising because 
there are of course many more generalizations of 
G\"{o}del's Second Incompleteness Theorem available in the
mathematical literature 
\cite{Ad2,AB1,AZ1,BS76,BI95,Da83,DPR61,Fe60,HB39,Ko7,Lo55,Ma93,PW81,Pu84,Pu85,Pu96,Sa1,Sm77,Sm85,So94,Sw3,TMR53,Ta0,Vi93,Vi5,WP87,ww0,ww2,ww5,ww6}
than there are
published 
examples of boundary-case style
exceptions to its formalism.

In order to explain  intuitively 
the core idea about
why our
paper is able to achieve this surprising evasion of
the 
Second Incompleteness Theorem,
it is helpful to
let $~\Upsilon_n~$ denote 
\ep{crazy}'s
$\Delta_0$ sentence.
Note 
that
this sentence
is comprised of
$~O(~n~)~$ logic symbols.
It asserts that
the variables $~v_0 \, , \,v_1 \, , \,v_2 \, , \, ~ \ldots , ~v_n \, , \,~$
satisfy $~v_i~=~2^{2^i}.~$
\begin{center}
$\exists~v_0 \, \leq \, 2~~
\exists~v_1 \, \leq \, v_0*v_0~~
\exists~v_2 \, \leq \, v_1*v_1~~...~~
\exists~v_n \, \leq \, v_{n-1}*v_{n-1}$
\end{center}
\beq
\label{crazy}
v_0 \, = \, 2~\wedge~
v_1 \, = \, v_0*v_0~\wedge~
v_2 \, = \, v_1*v_1~\wedge~ ... \wedge
v_n \, = \, v_{n-1}*v_{n-1}
\enq
It is easy to see  there exists
some $\Delta_0^R$ sentence, called say $~\Upsilon_n^R~,~$
that is the counterpart of \ep{crazy} written in a notation
using split integers. This sentence will
indicate the existence of a sequence of split integers
 $~S_0 \, , \,S_1 \, , \,S_2 \, , \, ~ \ldots , ~S_n \, , \,~$
where $~S_i~$ represents the quantity 
$~2^{2^i}.~$

However although they in some sense represent equivalent
concepts, there is a fundamental difference between
the $\Delta_0$ sentence  $~\Upsilon_n~$
and
its $\Delta_0^R$
counterpart $~\Upsilon_n^R.~$
This difference is easiest to explain if one uses
a logical language that has only 3 named constants, 0, 1 and 2,
and if split integers are encoded as base 2 numbers.
Then    $~\Upsilon_n^R~$ 
will be encoded as
a sequence of at least $~2^n~$ characters, but 
  $~\Upsilon_n$'s length has a
sharply smaller O$(n)$ magnitude.
As a consequence of this distinction
(and its generalizations), we 
realized that one could formulate an axiom system that was logically
equivalent to the more conventional encodings for \jjj ,
but which was
an anti-threshold to the Herbrandized version of
the Second Incompleteness Theorem (using Definition 1's terminology).

In particular,
it turns out that
the exponential difference between the lengths of the 
$\Delta_0$ sentence  $~\Upsilon_n~$
and
its $\Delta_0^R$
counterpart $~\Upsilon_n^R~$
plays a major role in 
understanding the 
central concept that 
motivated much of the research that stimulated the current article.
Thus, this paradigm (and its
formal generalization for arbitrarily more
complicated 
sequences of
sentences) are used by Section \ref{main}'s machinery
in a much more sophisticated context
to prove that it is feasible
to construct awkward encodings for \jjsj , similar to Ax-3,
that are 
actual
anti-thresholds to the Herbrandized version of the
Second Incompleteness Theorem.
(A reader
should not worry if he
 does not  follow
fully the intuitive idea,
sketched in this paragraph, 
about the significance of logically equivalent statements that
have exponentially different sizes in length.
This is
because a fully formalized
proof 
of our main theorem, that uses this effect,
will be explored in great detail in the next section.)



\section{ Main Analysis}
\label{main}


A sentence $ \psi $ in the propositional
calculus will be called
an {\bf Anti-Tautology} iff  $ \psi $
 is unsatisfiable (i.e. 
$ \neg \, \psi $ is a tautology). 
Our definition of Herbrand deduction
\cite{He30}
 will be identical to the definitions
used by Adamowicz,
H\'{a}jek-Pudl\'{a}k and 
Ko{\l}odziejczyk
\cite{Ad2,HP91,Ko7}, 
except that we will 
use a  dual version of this definition that follows from
De Morgan's Rule, where  disjunctions are  replaced with conjunctions
and where 
tautologies
are
 replaced with anti-tautologies. In other words,
our definition will use the well-known identity that
\begin{equation}
\label{prot.taut}
\bigvee_{i=1}^n~\neg \, \phi_i~~~~~=~~~~~
 \neg~\bigwedge_{i=1}^n~ \phi_i
\end{equation}
Our definition of Herbrand deduction
will differ from its more
conventional definitions by using the  
right (instead of left) side of \eq{prot.taut}'s identity. This change in
notation  
is 
unnecessary, but it does help
to considerably
 simplify our proofs.



 Let $~\Psi~$ denote an arbitrary prenex normal sentence such
as the prototype below, whose open subcomponent is denoted
as $~\xbar{\psi}~$.
\begin{equation}
\label{prot}
 ~~ \forall~ x_1 ~~ \exists~y_1  ~~ \forall~ x_2 ~~ \exists~y_2 
 ~~ .... ~~ ~~ \forall~ x_n ~~ \exists~y_n ~~~
\xbar{\psi} \,
(x_1,y_1...x_n,y_n)
\end{equation}
In a context where
$~f_1^{\psi}(x_1),~f_2^{\psi}(x_1,x_2)~...~f_n^{\psi}(x_1,x_2,~ \ldots , ~x_n)~$
are new function symbols,
\ep{prot2} is called the Skolemization of \ep{prot}.
\begin{equation}
\label{prot2} 
\forall  \, x_1  \,     \forall  \,  x_2
  \,    ....    \,  \,  \forall  \,  x_n  \,  \,  \,   
\xbar{\psi} \,[ \, \, x_1  \, , \, f_1^{\psi}(x_1)  \, , \, x_2  \,
 , \, f_2^{\psi}(x_1,x_2)
  \, ...  \,  x_n  \, , \, 
f_n^{\psi}(x_1,x_2,~ \ldots , ~x_n) \, ]~~
\end{equation}
In a context where
$ \, L \, $ is a logical language and
 $ \, \alpha \, $ is an axiom system, we will let
$ \, C_L \, $ and $ \, F_L \, $ denote the set of constant and function symbols
associated with $ \, L \, $. Similarly,
$ \, F_{\alpha} \, $ will denote the set of
``Skolemized'' function symbols
associated with $ \, {\alpha}  $'s axioms. 
Thus using   \eq{prot}
and \eq{prot2}'s notation,
let $\alpha$ denote a system of axioms
$ \, \Psi_1 \, , \, \Psi_2 \, , \,\Psi_3 \, ...  \, $,
and
for an arbitrary 
index $ \, i \, $ let its Skolemized function symbols 
carry names such as
$ \, f_i^{\psi_1}\, , \, f_i^{\psi_2}\, , \, f_i^{\psi_3}\, , \, ... \, $
The {\bf Herbrandized Terms} for this ordered pair
$ \, (\alpha,L) \, $ will then be defined to be the set of all terms
generated by the constants from the set $ \,  C_L \, $
combined with the functional operations from the set
 $ \, F_\alpha \cup F_L.$ 


A {\bf Herbrandized Instance} of a Skolemized axiom is a sentence identical
to this axiom except that all its universally quantified variables
are replaced by Herbrandized terms. For instance
in a context where $~T_1,T_2,T_3...~$ are
Herbrandized terms, \ep{prot3} is
such an instance of \eq{prot2}'s axiom:
\begin{equation}
\label{prot3}
\xbar{\psi} \,[ \,  \, T_1  \, , \, f_1^{\psi}(T_1)  \, , \, T_2  \,
 , \, f_2^{\psi}(T_1,T_2)
 ~ ~ \ldots , ~ ~ T_n  \, , \, 
f_n^{\psi}(T_1,T_2, ~ \ldots , ~T_n)~]
\end{equation}
Let $~\bot~$ denote the logical constant of FALSE.
A {\bf Herbrandized Proof} 
of  $~\bot~$ 
from  the axiom system $~\alpha~$ is defined as a finite collection
of Herbrandized instances 
of $~\alpha~$
together with a proof, in the pure propositional
calculus, that the conjunction of these instances is 
an anti-tautology.

\medskip

{\bf  Definition 2 : $~$} Using our revised notation convention,
 the theorem $~\Upsilon~$ will be said to have a {\bf Herbrandized
Proof} from the axiom system $~\beta~$ if and only if the union of
the axiom system $~\beta~$ with the
added sentence $~\neg \, \Upsilon~$
produces a Herbrandized proof of 
$~\bot~.~$

\medskip

{\bf More Notation:} $~$
Let us say that a function
$G(\, x_1 \, , \, x_2, \,  ~ \ldots , ~ \,x_n \,)$
is a {\bf Non-Growth Function} iff
$G(\, x_1 \, , \, x_2, \,  ~ \ldots , ~ \,x_n \,) ~
 \leq ~$Max$(\, x_1 \, , \, x_2, \,  ~ \ldots , ~ \,x_n \,).~$
Define a set $~S~$ of functions to be an 
{\bf Arithmetic Controlled Set} iff $~S~$
includes the arithmetic functions of addition, multiplication and successor
and all its other functions are non-growth functions. Also,
define a term $~t~$ to be an {\bf Arithmetically Controlled Term}
iff $~t~$ is a term that uses only the symbols of 0, 1 and 2
as its inputted constants and 
all its function symbols come from some 
Arithmetic Controlled Set $~S$.
Thus if
 $G_1$ and $G_2$ are non-growth
functions, \ep{silly} 
represents
an arithmetically controlled
term.
\begin{equation}
\label{silly}
G_1[~(2+1)*(1+1)~,~1+2~]~*~G_2(~2+2+0~,~2+2+1+0~)
\end{equation}
Also, in a context where
$~C_t~$ and $~F_t~$
denote the number of constant and function
symbols in $~t,~$  we will use the following notation:
\bee
\item
MinG$(t)$ will denote the quantity $~2^{C_t+F_t}$.
\item
Val$(t)$ will denote the quantity represented by the term $~t$.
\ene
For example if $~G_1(x,y)~=~|x-y|~$ and $~G_2(x,y)~=~$Min$(x,y)$
then \ep{silly}'s term $~t~$  will have
$~Val(t)~=~3*4~=~12~$  
and MinG$(t)~=~2^{25}~$ (because $~t~$ contains 12 function 
symbols and 13 constant symbols).


\begin{lemma}
\label{lem1}
Let  $~t~$ be an arithmetically controlled
 term which satisfies the inequality $~$Val$(t) \, \geq \, 4$.
Then
$~~~$Val$(t)~ < ~$MinG$(t)$.
\end{lemma}


{\bf Proof Sketch:} $~$
Suppose for some $~k \geq 2,~$ 
that Val$(t)~=~2^k.~$ Then it easy to see that $~t \,$'s maximally
compressed representation
as an {\it arithmetically controlled term} 
is ``$~2*2*....~2~$''. Thus
MinG$(t)=2^{2k-1}> \,$Val$(t) \, = \, 2^k$ is valid in this case
because the preceding product has
$ \, k \, $ appearances of the constant 2 connected by $ \, k-1 \, $ appearances
of the multiplication symbol.  
Moreover, it is easily proven that terms, which are not powers
of 2, are never represented in a more compressed form than the
greatest power of 2 that they exceed. Thus Lemma \ref{lem1}
is valid for all terms where 
 $ \, $Val$(t) \,  \geq \, 4$. $ \,  \, \Box$


\medskip


{\bf Definition 3.} $~$
For a fixed constant $~B>0,~$
a set $~S~$ of functions is defined to be a 
{\bf $B-$Bounded Arithmetic Set}  iff $~S~$
includes the arithmetic functions of addition, multiplication and successor
and all its other functions $~G~$
satisfy the constraint that
\begin{equation}
\label{bcs}
G( x_1  ,  x_2,   ~ ... \,  , ~ x_n ) ~
 \leq ~\mbox{Max}( x_1  ,  x_2,   ~ ... \,  , ~ x_n )~ \, 
\mbox{when}~ \,
\mbox{Max}( x_1  ,  x_2,   ~ ... \,  , ~ x_n )~<~B~~~
\end{equation}
Also, we will say a term $~t~$ is a {\bf
B-Bounded  Arithmetic Term}
iff $~t~$ is a term that uses only the symbols of 0, 1 and 2
as its inputted constants and 
all its function symbols come from some B-Bounded 
Arithmetic  Set $~S~$.
Lemma \ref{lem2} provides the  generalization of
 Lemma \ref{lem1} for B-bounded arithmetic terms.
Its proof is omitted because it
is an easy generalization 
(see footnote \footnote{The
intuitive  reason that
Lemmas \ref{lem1} and \ref{lem2} have
similar proofs
is that arithmetically controlled terms and
 $B-$bounded arithmetic terms
have precisely identical growth
rates until a construction process builds an
intermediate object $~t~$  with 
$~$MinG$(t)~\geq~B~$.} ) of  
Lemma \ref{lem1}'s
proof.


\begin{lemma}
\label{lem2}
Suppose that $~t~$ is a $B-$bounded arithmetic
 term with $~$MinG$(t)~<~B~$
and Val$(t)~ \geq ~4$. Then 
$~$Val$(t)~ < ~$MinG$(t)$.
\end{lemma}
 

{\bf Definition 4.} $~$ Let $~\Phi~$ denote the $\Pi_1^R$ sentence below
whose $\Delta_0^R$ subformula  is defined by $~\xbar{\phi}(~a_1,a_2~...~a_n~)~~$:
\begin{equation}
\label{bvp}
\forall    a_1    \forall    a_2    ...   
  \forall    a_n \,  \,     
\xbar{\phi}(  a_1,a_2  ...  a_n  ).
\end{equation}
For any $~B \,\geq \, 1,~~$  \ep{bvp} is
called
a {\bf $B-$Bounded Valid $\Pi_1^R$ sentence} iff 
\eq{bvpc} is valid under the standard model
of the natural numbers
\begin{equation}
\label{bvpc}
\forall ~ a_1 \, < \, B~~\forall ~ a_2  \, < \, B~~...
  ~~\forall ~ a_n  \, < \, B~~~~
\xbar{\phi}(~a_1,a_2, ~ \ldots , ~~a_n~).
\end{equation}

\bigskip

{\bf Definition 5.} $~$
An axiom system $~\alpha~$ 
will be said to satisfy
the {\bf Canonical Arithmetic Condition} 
when all $\alpha$'s axioms are $\Pi_1^R$ sentences
and they include $Q_0$'s nine axioms (i.e. Equations
\eq{pw1}--\eq{pw6} ). 

\bigskip

{\bf Definition 6.} $~$
Let $~\Theta~$ 
denote a methodology for assigning G\"{o}del numbers
to Herbrand proofs
(which are henceforth denoted as $~P ).~$ 
Let us recall that MinG$(t)$ was defined by Item (1) in this
section.
Define $ ~ \Theta ~$  to be a
{\bf Conventional Encoding Method} if  
$~\Theta(P)~>~$MinG$(t)~$ 
whenever the proof $~P~$ contains the
Herbrand term $~t~$. (Such encodings  are called
``conventional'' because all usual methods for encoding Herbrand
proofs satisfy $~\Theta(P)~>~$MinG$(t)~$.)

\medskip

\begin{theorem}
\label{tttccc}
			   Suppose $~\alpha~$ is
			  a canonical arithmetic
			axiom system  consisting of
		 $B-$Bounded Valid $\Pi_1^R$ sentences and
	  $~\Theta~$ again satisfies Definition 6's Conventional
			    Encoding property.
		     Then any Herbrand proof $~P~$ of
	$~\bot~$ from the axiom system $~\alpha~$ will satisfy the
		    inequality that $~\Theta(P)~>~B~$.
\end{theorem}

\medskip



{\bf General Comments about \thx{tttccc} and its Proof: $~$}
At an intuitive level, \thx{tttccc} can be viewed as a consequence of
the machineries of Lemma \ref{lem2} 
and Definitions 3-6. 
This is because the B-Bounded validity condition in \thx{tttccc}'s
hypothesis can be used to show that a Herbrand proof $~P~$ of
$~\bot~$ must contain some term $~t~$ where
Val$(t) \geq B \,$. In this context, the combination of  
Lemma  \ref{lem2}  and Definition 6 will
imply that such a term will force $~P \,$'s G\"{o}del number to
exceed the lower bound of $B$.

A  more detailed formal
proof of \thx{tttccc} 
appears in Appendix A. It explains the
precise role that Definition 3 and Lemma \ref{lem2}
play in establishing this theorem.
Our recommendation is that 
a reader postpone
examining Appendix A
until after
he
finishes
 the remainder of this section. It
will explain the significance of
\thx{tttccc} 
by showing how it
 enables us to prove the surprising result that the
Ax-3 axiomatization for \jjsj is an anti-threshold for the Herbrandized
version of the Second Incompleteness Theorem.


\medskip
 \bigskip

\begin{theorem}
\label{tm2}
For any arbitrary axiom system $\alpha$ and
deduction method $D$, let
Diagonal$(\alpha,D)$
and $~\alpha^D~$ 
denote the following two constructs:
\bed
\item[  A.   ]
{\rm Diagonal$(\alpha,D)~~$ will denote a logical
sentence that states: $~$ 
``There is no proof 
(using deduction method $~D~$)$~$
of the {\it falsity sentence} $~\bot~$
from the  union of
the axiom system $~\alpha~ $
with $~this~$
sentence  Diagonal$(\alpha,D) \,$ (looking at itself).''}
\item[  B.   ]
{\rm  $~\alpha^D~$  will denote the formal union of the
axiom system $~\alpha~$ with the sentence
Diagonal$(\alpha,D)$.}
\ennd
Let
Diag(Ax-3)
denote the special variant of 
Diagonal$(\alpha,D)  $ where $ \, \alpha \, = \, $Ax-3
and $~D~$ designates Herbrand deduction. 
Both these
two
 constructs 
and also  $~\alpha^D~$ 
are well defined. Also,
Diag(Ax-3) has a
$\Pi_1^R$ encoding.
\end{theorem}

\bigskip
\medskip



{\bf Abbreviated Sketch of \thx{tm2}'s proof.} $~$
As early as 1938, Kleene observed 
\cite{Kl38}
that
the Fixed Point Theorem implied that
 a type of cousin of the sentence 
 Diagonal$(\alpha,D)$ 
was well defined. More recently, Willard
\cite{ww1,ww5} showed how
 Diagonal$(\alpha,D)$ could
be formally endowed with a
 $\Pi_1$ encoding under
the conventional language of arithmetic. It is 
reasonably
straightforward to
generalize \cite{ww1,ww5}'s result to establish that 
Diag(Ax-3) also has
a well defined  $\Pi_1^R$
encoding (thus completing
\thx{tm2}'s  proof.)
The remainder of this proof sketch  
will summarize the ideas from 
\cite{ww1,ww5} for the benefit of those readers
who are unfamiliar with this topic.
Our discussion will employ the following notation:
\begin{description}
\item[  i]   $\mbox{Prf}_{\alpha}^D~( \, t \, , \, p \,)$ 
will denote a formula  designating
that $~p~$ is  a proof of the theorem $~t~$ from the axiom
system
  $~\alpha~$
 using the deduction method $~D.~$
\smallskip
\item[ ii]  
$\mbox{ExPrf}_{\alpha}^D~( \, h \, , \, t \, , \, p \,)$  
will be a formula stating that
$~p~$ is a proof
(using the deduction method $~D~)~$
 of
a theorem $~t~$ 
from the union 
of the axiom system
  $~\alpha~$
with the added axiom
sentence whose G\"{o}del number equals
$\, h \,$. 
\smallskip
\item[iii]  
 $\mbox{Subst} \, ( \, g \, , \, h \, )$ will denote
G\"{o}del's
classic substitution formula --- which yields TRUE when $\, g \,$
is an encoding of a formula
and $\, h \,$ is an encoding of a sentence
that replaces all occurrence of free variables in $\, g \,$ with
a mathematical term 
formalizing the
 G\"{o}del
number for representing  ``g''.
\smallskip
\item[ iv]   
$\mbox{SubstPrf}_{\alpha}^D~( \, g \, , \, t \, , \, p \,)$  
will denote the natural 
hybridizations of the constructs from Items (ii) and (iii)
which yields a Boolean value of TRUE exactly when there
exists some integer $~h~$ 
simultaneously 
satisfying
{\it both}
  $\mbox{Subst} \, ( \, g \, , \, h \, )$ and
$\mbox{ExPrf}_{~\alpha}^D~( \, h \, , \, t \, , \, p \,)$.
\end{description}
Each of (i)--(iv)
can be encoded as $\Delta_0^R$ formulae
when $\alpha$ is any recursively enumerable axiom system.
In particular,
Appendices C and D of  \cite{ww1}
essentially established
(see footnote \footnote{The results of Wrathall
\cite{Wr78} have been noted by
H\'{a}jek--Pudl\'{a}k 
\cite{HP91}
to imply that every LinH function 
\cite{HP91,Kr95,Wr78} 
has a $\Delta_0$ encoding. Using a slightly different
``$~\Delta_0^-~$''notation, the results from Appendices C and D
 of  \cite{ww1} 
explained how this result would imply that the 
each of 
$\mbox{Prf}_{\alpha}^D~(  t  ,  p )$,
$~\mbox{ExPrf}_{\alpha}^D~(  h  ,  t  ,  p )$  
and $\mbox{SubstPrf}_{\alpha}^D~(  g  ,  t  ,  p)$  
have ``$~\Delta_0^-~$''encodings. Since the
$~\Delta_0^R~$ class of formalae is broader than
$~\Delta_0^-~$, $~$
 these formulae must also have
$~\Delta_0^R~$ encodings.} ) 
that the first three of these predicates can receive
 $\Delta_0^R$ encodings when one applies 
the theory of LinH functions from
\cite{HP91,Kr95,Wr78} 
in a reasonably
routine manner.
In such a context, \ep{encode} illustrates
one possible  $\Delta_0^R$ encoding for 
$\mbox{SubstPrf}_{\alpha}^D \,(   g   ,   t   ,   p  )$'s 
graph. (It is
equivalent to 
$~$``$~\exists ~h~[~\mbox{Subst}    (    g    ,    h    )~\wedge~
\mbox{ExPrf}_{\alpha}^D(    h    ,    t    ,    p   )\, ] \, \,$''$,~$
  but \ep{encode} is 
 a $\Delta_0^R$ formula --- {\it unlike} the quoted
expression.)
\begin{equation}
\label{encode}
\mbox{Prf}_{~\alpha}^D~( \, t \, , \, p \,)~~~\vee~~~\exists ~h\leq p
~~[~ \mbox{Subst} \, ( \, g \, , \, h \, )~\wedge~
\mbox{ExPrf}_{~\alpha}^D~( \, h \, , \, t \, , \, p \,)~]  
\end{equation}




Utilizing \eq{encode}'s
  $\Delta_0^R$ encoding for 
$\mbox{SubstPrf}_{\alpha}^D(   g   ,   t   ,   p  )$, it is
easy to 
formulate a $\Pi_1^R$ encoding
for  the
axiom sentence Diagonal$(\alpha,D)$.
Thus,  let
$~\Gamma(g)~$ 
denote  \ep{encode2}'s formula, and let  $~N~$ denote $~\Gamma(g)$'s
G\"{o}del number. $~\,$Then  $~\Gamma(~ N~)~$
is a $\Pi_1^R$  encoding for  Diagonal$(\alpha,D)$.
\begin{equation}
\label{encode2}
 \forall   \, p \,   \,  \neg  \,  \, 
\mbox{SubstPrf}_{\alpha}^D  (  ~  g ~   , ~ \bot ~       , ~   p ~  )
\end{equation}

$\Box$

\bigskip





{\bf Clarifying Comment: }
One
should be somewhat cautious in interpreting the meaning of \thx{tm2}.
It does not indicate that  Diag(Ax-3)  is a logically valid
statement under the standard model of the
natural numbers. Rather, it merely indicates
 Diag(Ax-3)  is a well defined $\Pi_1^R$ sentence.
In order to establish prove that 
 Diag(Ax-3)  is also valid,
we will need the 
added
force of \thx{tm3} below.


\begin{theorem}
\label{tm3}
Let Ax-3* 
denote the union of
the axiom system
 Ax-3 with the sentence Diag(Ax-3).
Then Ax-3* is consistent. 
(Thus,
Ax-3 is an ``anti-threshold'' for the
Herbrandized version of the Second Incompleteness Theorem
under Definition 1's notation convention.) 
\end{theorem}

{\bf Proof of the Consistency Property of Ax-3* : $~$}
Suppose for the sake 
of establishing a
proof-by-contradiction that Ax-3* was inconsistent.
Then one could identify a
proof $~P~$ of  $~\bot~$ 
whose G\"{o}del number $~\Theta(P)~$ is
the smallest G\"{o}del number of a Herbrand proof of  
 $~\bot~$
from Ax-3*. We will now construct 
from $~P~$ an alternate Herbrand proof $~R~$ of
 $~\bot~$ where $~\Theta(R)~<~\Theta(P).~$
The formal
construction of such a $~R~$ will suffice for our proof by
contradiction to reach its desired end because
such a  $~R~$ cannot possibly exist
(on account of $P$'s minimality property).

Our strategy is to
 use \thx{tttccc}
to construct $~R~$ from $~P~$.
\thx{tttccc} is relevant to Ax-3* 
because all the  formal axiom
sentences of Ax-3* are
assuredly  $\Pi_1^R$
sentences (see 
footnote \footnote{ \thx{tm2}
implies that the Diag(Ax-3) axiom of 
Ax-3* has a $\Pi_1^R$ structure, and 
Section \ref{s2}'s definition of Ax-3 implies that
all the other axiom-sentences belonging to Ax-3 are certainly
also
$\Pi_1^R$. } for the justification of this claim).  
In such a context,
we may apply
\thx{tttccc} to conclude
that for some $~B~<~\Theta(P),~~$ at least one of
the  axiom sentences of Ax-3* 
must
fail to be a B-Bounded valid
$\Pi_1^R$ sentence. Moreover, it is obvious
that all the axioms of
Ax-3 possess an unbounded level of validity
(i.e they are $B-$Bounded valid {\it for all possible B}).
 Hence, these two
observations imply 
Diag(Ax-3) fails to be $B-$bounded valid
(simply because
some axiom from Ax-3* must fail to be $B-$bounded valid, and
 Diag(Ax-3) is the only 
available
axiom belonging to Ax-3* 
that is not  also a member of  Ax-3.)

The latter observation,
combined with Diag(Ax-3)'s definition implies 
(see footnote \footnote{ The 
strictly formalistic definition of
Diag(Ax-3) as the entity ``$~\Gamma(N)~$''
(using a self-reference principle)
can be found in the last sentence of 
\thx{tm2}'s
proof. The syntax of its \ep{encode2}
implies that if 
Diag(Ax-3) fails to be 
$B-$Bounded valid then another proof 
 $R$ must
assuredly exist for the sentence  $~\bot$
with  $ \, \Theta(R)< B \,$.} ) that
 some $~R~$ with
 $ \, \Theta(R)< B \,$ 
must 
be another proof of $~\bot$.
Hence
our last two inequalities
certainly
 imply that
  $ \, \Theta(R)< \, B \, < \, \Theta(P) \,$.
This finishes
our proof-by-contradiction  because
 $~P$'s previously 
presumed minimality has been contradicted by $R$.
 $~~\Box$


\bigskip

{\bf Remark 2.} $~$
Our discussion in this section had assumed that the terms $~T~$ in
a $\Delta_0^R$'s formula's bounded quantifiers included
{\it only the Maximum function symbol.}
The results of \thx{tm3} would actually also hold if these
quantifiers were also permitted to include the Addition function
symbol. (The only reason our discussion had omitted the
possibility that both 
the
addition and maximum
function symbols
 appear in
the $\Delta_0^R$ formula $\phi$'s bounded quantifiers in
\ep{wp} was for the sake of simplifying the presentation.)


\bigskip

{\bf Remark 3.} $~$
The attached Appendix B discusses 
a yet further reason  why 
\thx{tm3} is of interest.
The 
anonymous
referee had suggested we add
this appendix to the current paper.
Its methodologies  are related to
Ko{\l}odziejczyk's observation \cite {Ko6b,Ko7}
that semantic tableaux and Herbrand deduction can sometimes
have an exponential difference in their 
proof
lengths.
The  purpose of Appendix B is to
sketch how one 
can generalize \cite{ww7}'s
results for Ax-1 and Ax-2 to establish
that Ax-3 is also a threshold for the semantic tableaux
version of the Second Incompleteness Theorem.
In a context where \thx{tm3}
had established the polar opposite
result for Ax-3
under Herbrand deduction, 
this contrast is, of course, quite interesting.


\section{Discussion of Significance of Results}

A comparison between our research and the prior research
of  Kreisel-Takeuti and Pudl\'{a}k \cite{KT74,Pu85,Ta53} 
was  
postponed until the closing part  of this current article
because the results from
 Sections 3 and 4
were needed to precede this discussion.



In this section,
$~S(x)~$ will denote 
the ``successor'' operation that maps the
integer $x$ onto $x+1$. 
A formula
 $ \varphi(x) $ is called \cite{HP91} a
{\bf Definable Cut for}
an axiom system $ \alpha $   iff
$ \alpha $ can prove:
\begin{equation}
\label{initdefx}
\varphi(0)~\mbox{ AND }~
\forall~x~~ \{~\varphi(x)\Rightarrow\varphi[S(x)]~ \}
 ~\mbox{ AND }~
\forall~x ~\forall~y<x
~~ \{ ~\varphi(x)\Rightarrow\varphi(y) ~ \}~~
\end{equation}
A very extensive literature
\cite{Ad2,AZ1,BI95,GD82,Ha71,HP91,Ka91,Kr87,Kr95,Ne86,PD83,PW81,Pu83,Pu84,Pu85,Pu96,Sa1,Sm85,Sv83,Vi92,Vi93,Vi5,VH73,WP87,ww2,ww4}
has studied Definable cuts.
We have
published two 6-page summaries of this
literature
in the  review chapters of our articles  \cite{ww5,ww6}.
(Also, Pudlak's full-length survey article
\cite{Pu96}
 offers an excellent
review of the work done in this subject prior to 
approximately 1990.)


All axioms systems,
strictly weaker than Peano Arithmetic, contain some
definable cut that is not provably equivalent to the full set
of integers. 
In the proof-theory literature,
the definition of a 
``Definable Cut'' 
(see \cite{HP91})
is
formally
 unrelated to Gentzen's notion
of a sequent calculus
deductive ``cut rule'',
despite their very similar sounding names.

Let
$~\lceil ~ \Psi ~ \rceil ~$
denote $~\Psi\,$'s
G\"{o}del number, and
Prf$_\alpha^D \, \,(t,p)$
denote
that $~p~$ is a proof
of the theorem $~t~$ from the axiom system $~\alpha~$ using the
deduction method $~D.~$
Also, let
 $ \, \varphi(x) \, $
denote a definable cut.
Let us say that an axiom system
$~\alpha~$ can recognize its 
{\bf Cut-Localized  D-consistency 
under}
 $ \, \varphi(x) \, $
iff $ \, \alpha \, $ can 
{\it formally prove:}
\begin{equation}
\label{dcon}
\forall~p~~~~~~ \{ ~~~\varphi(p)~~\Rightarrow~~
\neg~\mbox{Prf}_\alpha^D \, ~(~\lceil 0=1 \rceil~,~p~)~~~ \}
\end{equation}
Pudl\'{a}k 
\cite{Pu85}
proved that no consistent extension of the Tarski-Mostowski-Robinson
\cite{TMR53}
system $~Q~$ 
can prove the validity of \ep{dcon} for any definable cut
when $~D~$ denotes either Hilbert deduction or a Gentzen sequent calculus
system with a deductive cut rule.

 
At the same time,
the literature about Definable Cuts has also defined some
circumstances where \eq{dcon} is provable 
from $\alpha$
when for example
$~D~$ denotes either  semantic tableaux or Herbrand
deduction.
The strongest result of this type was discovered by Pudl\'{a}k \cite{Pu85}.
For both Herbrand and semantic tableaux deduction,
\cite{Pu85} showed that
it is possible to construct a definable cut  
$~\varphi(p)~$
where $~\alpha~$ can prove the validity of
\ep{dcon}'s
 Cut-Localized  D-consistency property
 when
 $~\alpha~$
is an axiom system with finite cardinality that satisfies a
relatively minor
additional
 constraint called ``sequentiality''.


A second  class of
boundary-like
 exceptions to
the Second Incompleteness can be constructed via the
``Cut Free Analysis'' (CFA) systems of Kreisel and 
Takeuti \cite{KT74,Ta53}.
It has consisted of a Second Order Logic generalization
of the cut-free version of Gentzen's Sequent Calculus that
is able to prove some types of versions of statements
about its own consistency

The work of 
 Kreisel and 
Takeuti 
\cite{KT74,Ta53}
 had chronologically
preceded most of the literature about Definable Cuts.
There is a fascinating partial analogy between
the perspectives of these two approaches.
This is because the Kreisel-Takeuti papers \cite{KT74,Ta53}
 can be viewed as using an analog of
\ep{dcon} in an implicit 
manner. Thus \cite{KT74} employs
a set of
 objects, which we shall call $~I,~$
that includes all the standard integers 
plus plausibly {\it non-standard integers}
that can represent a proof of a contradiction.
It then uses $~I~$ to
construct a subset of it, called
 ``$~N~$'', which can be viewed as a representation
of  the natural numbers.
(More precisely, the pages 16-17 of  \cite{KT74} construct $~N~$ from
$~I~$ in this manner by employing Dedekind's
and Zermelo's inductive 
second-order logic
definitions of the natural numbers.
They thus formally treat the
 Dedekind
and Zermelo inductive 
definitions 
of $~N~$ 
as an analog of $~\varphi(p)~$
in \ep{dcon}'s Cut-Localized D-Consistency statement.)



A question which then naturally arises is whether or not one
can also develop some types of axiom systems which can prove
their own consistency without  relying upon
 \ep{dcon}'s
``Cut-Localized  D-consistency'' machinery
(or \cite{KT74}'s analog of it for a second-order generalization
of a Gentzen Cut-Free logic.)
Such an approach is desirable because one would 
ideally
like to 
characterize the properties of the natural numbers directly
--- instead of 
being required to 
view them as a subset of 
a potentially
larger set of objects,
called $~I~$.


It is indeed possible to make some progress in this area by
using a third approach, relying upon
the Diagonal$(\alpha,D)$ sentence,
which was formally defined by the
\thx{tm2}.
(Analogs of this Diagonal$(\alpha,D)$ sentence
have also been used in several of our previous
papers 
\cite{ww93,ww1,ww5,ww6}
in  some contexts
quite different from
the paradigm outlined by
 Theorem \ref{tm3}.)

It is  difficult to make
more
  detailed comparisons
between the partial exceptions to G\"{o}del's Second Theorem
that use our 
 Diagonal$(\alpha,D)$ sentence
 with the methods of Kreisel-Takeuti and Pudl\'{a}k
\cite{KT74,Pu85,Ta53}. Each technique has its own
distinct separate advantage. There also 
appears to be no
 natural way to hybridize
these techniques. (The point is that each
of these different methodologies is looking at
a different type of problem setting, where a different
form of solution method is available.)

In particular, our research
 has treated 
the sentence 
 Diagonal$(\alpha,D)$ as an axiom of $~\alpha~$ while
the 
 Kreisel-Takeuti and Pudl\'{a}k papers 
\cite{KT74,Pu85,Ta53} treat their analogs of
\ep{dcon} as
derived
 theorems.
In this context,
it is  very natural
 for a reader to inquire
whether is is preferable to treat an ``I am consistent'' statement
as a theorem rather than
as an axiom of the system $~\alpha~$ that
generates it?

There is surprisingly no easy answer to this question.
For instance,  if one's goal is to attempt
to return to the original objectives of Hilbert's consistency
program, then the 
 Kreisel-Takeuti and Pudl\'{a}k 
approaches
are quite significant  because they
indicate some respects in which an axiom system can 
indeed 
formally
prove at least
a reduced versions of its consistency statement.
However from an alternate perspective, there is a
substantial difficulty with
approaches that treat the statement ``I am consistent''
and its analogs
 as a theorem
rather than as an axiom.
The difficulty arises when the
relevant systems do not have an operating modus ponens or
Gentzen-like cut rule
(as is the case with
the axiom systems of \cite{KT74,Pu85}).
Such
systems cannot draw inferences when they
prove a theorem essentially
declaring  that
 ``I am consistent''. However,  a similar such
difficulty does not affect the self-justifying axiom systems
in our earlier papers 
\cite{ww93,ww1,ww5,ww5b,ww6}
or
in Theorem \ref{tm3} 
because they treat the statement
 Diagonal$(\alpha,D)$
 {\it as a formal axiom rather than
as a theorem.}
(The point here is that axiom sentences,
{\it quite unlike theorem sentences,} can be 
permissibly
used as intermediate
steps to generate other deductions under the inference rules
of semantic tableaux, the cut-free sequent calculus and/or Herbrand
deduction
$~$---$~$ thus causing their presence to have stronger
implications.)


Thus, there are 
quite
different insights
that arise
 from systems that
treat 
variants of
the ``I am consistent'' statement
as an axiom
 instead of as
a theorem
(because of the different connotations these two approaches
carry).
 Neither approach is uniformly
preferable
over the other.
 Another difference between our research and
the investigations of  
 Kreisel-Takeuti
and Pudl\'{a}k 
\cite{KT74,Pu85,Ta53} is that the sentence
 Diagonal$(\alpha,D)$ declares the consistency of
$~\alpha~$ in a global sense whereas 
\ep{dcon} refers only to the localized subset of integers
that lie within the domain of $~\varphi.~$ 
(In the case of 
 Kreisel-Takeuti second-order logic system, 
\ep{dcon} is used implicitly on the pages 16 and 17
of \cite{KT74}
to draw the distinction between 
what we have called
$I$
 and $N$ earlier in this section.)


When one compares our results in
  Theorem  \ref{tm3} or in our earlier papers
\cite{ww93,ww1,ww5,ww5b,ww6}
with the research of 
 Kreisel-Takeuti
and Pudl\'{a}k 
\cite{KT74,Pu85,Ta53},
it is best to remember that G\"{o}del's Second Incompleteness
Theorem precludes an exception to it from becoming too
powerful. Thus, it is possible to 
develop
different types of
partial evasions
 to the Second Incompleteness Theorem around its
periphery that shed different types of useful 
new perspectives. However, it is awkward to attempt to
hybridize these different types of partial evasions into
a stronger uniform evasion.
This is because if such a hybrid combined the strengths of
the different approaches X, Y and Z, then it would become
so strong that its properties could
potentially violate 
the statement of G\"{o}del's historic result.
Thus, there are trade-offs where different types of useful 
insights, stemming from different approaches
(that are difficult to hybridize into one unified methodology),
examine somewhat different problems and produce
different new perspectives.


Part of the reason that  
Ax-3's evasion of the Herbrandized version of the
Second Incompleteness Theorem is
 interesting is that it is
known that further improvements upon this result are very
difficult to obtain.
Thus,
 Adamowicz-Zbierski \cite{Ad2,AZ1} showed that \jjj $+ \Omega_1$ was unable
to verify its Herbrand  consistency,
Willard \cite{ww0,ww2,ww7} 
modified this result to
show the
Second Incompleteness Theorem
applied also to
Ax-1 and Ax-2's  versions
of
\jjj 's axiomatization
under semantic tableaux deduction,
and
 Salehi \cite{Sa1}
extended some of
the earlier results from \cite{Ad2,AZ1,ww0,ww2} to develop 
some further             incompleteness results for \jjsj under Herbrand
deduction.
Thus in a context where several earlier papers have explored
types of results where the Second Incompleteness Theorem
holds for 
 encodings for
\jjsj and its cousins ,  it is surprising that the
current article has documented 
a paradigm where at least the Ax-3 encoding for \jjsj evades 
the Herbrandized version of the Second Incompleteness
Theorem. 


\section{Broader Perspectives} 


Some added notation is useful so that we can examine 
\thx{tm3}'s
results from a broader perspective.
Let us say a  formula $~\Phi~$ is
 $\Sigma_1^R$ when it can be written in the form
$~\neg~\Psi~$ where $~\Psi~$ is 
 $\Pi_1^R$.
Some other fairly conventional notation  is
that a sentence will be called
$~\Pi_{k+1}^R~$
when it can be written in the form
$\forall \, v_1~ \forall \, v_2~ ... ~\forall \, v_n~
\phi(v_1,v_2, ~ \ldots , ~v_n)$ where $ \phi(v_1,v_2, ~ \ldots , ~v_n) $
is a  $ \Sigma_{k}^R $ formula. 
Likewise, a sentence will be called
$\Sigma_{k+1}^R~$
when  it can be written as
$\exists \, v_1~ \exists \, v_2~ ... ~\exists \, v_n~
\phi(v_1,v_2, ~ \ldots , ~v_n)$ where $ \phi(v_1,v_2, ~ \ldots , ~v_n) $
is a  $ \Pi_k^R $ formula. 
Also, a sentence will be said to be
Level$-k$ if it is either 
 $ \Pi_k^R $ or  $ \Sigma_{k}^R $.


\bigskip

{\bf Definition 7. }
Let
 $~H~$ denote a sequence of ordered pairs
$~(t_1,p_1),~(t_2,p_2),~...~(t_n,p_n),~$
where $~p_i~$ is a Herbrand 
 proof of the theorem $~t_i.~$
For an arbitrary integer $~k \geq 1~$,
this list $~H~$ will be defined
to be a 
\hxl{k} 
proof
of a theorem $~T~$ 
from the axiom system $~\alpha~$
  iff $~T=t_n~$
and also:
\begin{enumerate}
\parskip 3 pt
\item
Each axiom in  $ p_i$'s
proof is
either one of $ t_1,t_2, \, ... \, , t_{i-1} $ 
or comes from $\alpha$.
\item
Each of the ``intermediately derived theorems'' 
$~t_1,t_2, \, ... \, , t_{n-1}~$
must lie within the 
Level-k class of sentences.
\end{enumerate}
Intuitively, 
\hxl{k} deduction can be viewed as an extension of
Herbrand deduction that contains a type of Gentzen-like
deductive cut rule for Level-k sentences.

\bigskip
\medskip


The Definition 7's machinery  is  useful for
helping to describe both
the maximal generalizations as well as inherent limitations of
Section \ref{main}'s formalism.
Thus using 
 Definition 7's notation, one can establish the following two
tightly  complementary negative and
positive results:
\bed
\parskip 3 pt
\item[   I.  ]
There exists a 
logically valid
$\Pi_1^R$ sentence denoted as $~\Psi~$ such that no
consistent axiom system can contain $~\Psi~$ as an axiom and prove
a theorem affirming its own consistency under  
\hxl{2} deduction.
\item[  II.  ]
In contrast for each consistent axiom system $~A~$, there exists
a consistent axiom system $~I(A)~$ that can prove all $~A\,$'s
$\Pi_1^R$ theorems and recognize its own consistency under
\hxl{1} deduction.
\ennd
In other words, Items I and II indicate that there is a fundamental difference
between  \hxl{1} 
 and \hxl{2} deduction. Thus,
 \hxl{1} deduction allows for a type of robust evasion of the Second
Incompleteness Theorem under a formalism that contains a type of
limited Gentzen-style cut rule.
However, Result-II  precludes this evasion
from becoming too strong.


We will not prove results I and II here because each has a rather
long proof. Also,
partial
analogs of these two prior results have 
appeared in our prior work. Thus,
\cite{ww4} and \cite{ww5} have used the term
\txl{k} deduction to refer to a construct similar
to \hxl{k} deduction except for the following two changes:
\bee
\item
Under \txl{k} deduction, the proofs
$~p_1~,~p_2~,~ ~ \ldots , ~~p_n~$ associated with
the sequence
$~(t_1,p_1),~(t_2,p_2),~ ~ \ldots , ~~(t_n,p_n),~$
are semantic tableaux proofs instead of Herbrand proofs.
\item
Also under \txl{k} deduction, the 
intermediate results
$~t_1~,~t_2~,~ ~ \ldots , ~~t_{n-1}~$ 
will  technically represent what \cite{ww4,ww5}
had called 
 $ \Pi_k^* $ and  $ \Sigma_{k}^* $ sentences
(instead of being
 $ \Pi_k^R $ and  $ \Sigma_{k}^R $ sentences).
These 
 $ \Pi_k^* $ and  $ \Sigma_{k}^* $ sentences
can be intuitively viewed as being
roughly analogous to 
 $ \Pi_k^R $ and  $ \Sigma_{k}^R $ sentences 
---
except that they contain
no multiplication function symbol.
(They instead use a relation primitive
$M(x,y,z)$ to treat multiplication as a 3-way
relation.)
\ene
In essence Item-I's result can be viewed as being analogous
to \cite{ww4}'s main theorem.
Similarly, Item-II's result
can be viewed as following from a 
rather natural
hybridization of  
\cite{ww5}'s main result  with the machinery 
that
we had developed in Section \ref{main}.


For further details about this material,
the reader
should
examine \cite{ww4,ww5}'s generic formalisms.
The reason we had slightly 
changed the topic from
semantic tableaux to Herbrand deduction in the current paper
is because Herbrand deduction and its \hxl{1} generalization
possess one interesting  quality that semantic tableaux
and \txl{1} deduction 
simply
do not possess. This is that the
Ax-3 encoding of the axiom system \jjsj is an anti-threshold to
the Herbrandized (and also \hxl{1} ) versions  
of the Second Incompleteness Theorem.
However, the similar anti-threshold
effect does not  apply
also to semantic tableaux deduction
(as is explained in Appendix B).
Thus, the 
\hxl{1} deduction method will support certain types of
evasions of the Second Incompleteness Theorem that have
no analogs under
\txl{1} deduction. 
(The interested reader should also 
examine
Ko{\l}odziejczyk
work \cite {Ko6b,Ko7}, which observed how semantic tableaux and
Herbrand-styled proofs can sometimes have an exponential
difference in their lengths.)

In essence
the over-all goal of our research, 
in the current paper and in the previous papers
 \cite{ww93} -- \cite{ww7},
has
been to 
attempt to
sharpen the
academic community's understanding of the meaning of 
 the Second Incompleteness Theorem, by 
exploring both
 its maximal generalizations
and permitted allowed boundary-case exceptions.
We plan to prepare 
a new article about this subject in the near future,
describing the
underlying philosophical motivation for
much of this research.
One must
clearly  approach this 
subject matter carefully
because
the many
generalizations of the Second Incompleteness Theorem are
seemingly more significant
 than its occasional boundary-case exceptions.
The reason that the partial exceptions to the
Second Incompleteness Theorem should not be
ignored
is because
G\"{o}del's Incompleteness Theorem is often
regarded as 
the paramount discovery of 20th century mathematics.
It thus beckons  the academic
community to  explore both its
maximum
generalizations and
possible  boundary case exceptions,
so that an
understanding of the full meaning
of its historic result
 can be sharpened and made more
precise. Within such a
limited-but-precise framework, the anomalous behavior of Ax-3,
documented in this article, should be
of
scholarly interest.

\bigskip

{\bf Acknowledgments:} I warmly thank
Leszek Ko{\l}odziejczyk for the interesting questions that he had
emailed me on November 16, 2005. Those 
questions played a major role in
encouraging me to re-examine my 
earlier work in 
 \cite{ww2} for one more time.
I also thank Zofia  Adamowicz and
Konrad  Zdanowksi, who helped at least partially stimulate
the emailed questions from
L. Ko{\l}odziejczyk through their conversations with
him in Warsaw. 
Those questions 
from Leszek Ko{\l}odziejczyk
had greatly
stimulated my  research
in this article. I also thank the anonymous referee for his many
useful comments, including the  suggestion that I
add the Appendix B to this article.


\section*{Appendix A:  The Proof For  \thx{tttccc}}

This appendix will explain in further detail how
Definitions 3-6 and Lemma \ref{lem2}
may be used to prove
\thx{tttccc}.
Our discussion will begin
with one further definition and
two  further lemmas.


\bigskip

{\bf Definition 8. }
Consider the possibility that $~\Psi~$ is the
prenex normal sentence, whose open part is formalized
by $~\xbar{\psi} \,(x_1,y_1...x_n,y_n),~$
shown in \ep{2prot1} and whose Skolemized normalized
form is illustrated by \ep{2prot2}.
\begin{equation}
\label{2prot1}
\forall~ x_1 ~~ \exists~y_1  ~~ \forall~ x_2 ~~ \exists~y_2 
 ~~ .... ~~ ~~ \forall~ x_n ~~ \exists~y_n ~~~
\xbar{\psi} \,
(x_1,y_1...x_n,y_n)
\end{equation}
\begin{equation}
\label{2prot2}
\forall x_1  \,  \,   \forall  x_2
  \,  \,  ....  \,  \,   \,  \,  \forall  x_n  \,  \,  \,  \, 
\xbar{\psi} \,[ \,  x_1  \, , \, f_1^{\psi}(x_1)  \, , \, x_2  \,
 , \, f_2^{\psi}(x_1,x_2),
  \, ...  \,  x_n  \, , \, 
f_n^{\psi}(x_1,x_2  ...  ~x_n) \, ]~~
\end{equation}
For any $~B \, \geq \, 1,~$ 
Equations \eq{2prot1} and 
 \eq{2prot2} 
will be called
a {\bf $B-$Bounded Good Skolemization} iff
one can define 
 \eq{2prot2}'s Skolem functions
$ \, f_1^\Psi \, , \, f_2^\Psi, \,  ...  \, f_n^\Psi \, $
so they satisfy both 
Definition 3's  $B-$Bounded requirement
(see footnote \footnote{\label{83} The
function of Definition 3's  $B-$Bounded requirement
is that it assures that each of 
 \ep{2prot2}'s Skolem functions
of 
$f_1^\Psi \, , \, f_2^\Psi, \, ... ~$
satisfy the constraint
that
$f_i^\Psi( x_1  ,  x_2,   ~ ... \,  , ~ x_i ) ~
 \leq ~\mbox{Max}( x_1  ,  x_2,   ~ ... \,  , ~ x_i )~ $
whenever 
Max$( x_1  ,  x_2,   ~ ... \,  , ~ x_n )~<~B~.$ } ) 
 and
\ep{2prot3}  under the standard model of the natural
numbers.
\begin{center}
$\forall ~x_1 \, < \, B~~  \forall ~ x_2  \, < \, B
 ~~ .... ~~ ~~ \forall ~ x_n  \, < \, B$
\end{center}
\begin{equation}
\label{2prot3}
\xbar{\psi} \,[~ x_1  \, , \, f_1^{\psi}(x_1)  \, , \, x_2  \,
 , \, f_2^{\psi}(x_1,x_2)
 ~ ~ \ldots , ~ ~ x_n  \, , \, 
f_n^{\psi}(x_1,x_2 ~ \ldots , ~x_n)~]
\end{equation}
Likewise, we will say an axiom system $~\alpha~$ has a 
 {\bf $B-$Bounded Good Skolemization} iff all its axioms
are so Skolemized.



\bigskip



\begin{lemma}
\label{lem3}
Using the notation conventions from Definitions 4 and 8,
every  $B-$Bounded Valid $\Pi_1^R$ sentence can be 
rewritten into a logically equivalent form that
has a  $B-$Bounded Good Skolemization.
\end{lemma}


\medskip

{\bf  Proof. $~$} Follows immediately from the definitions of
Bounded Validity and Bounded Good Skolemizations 
(i.e. see Definitions 4 and 8).

\bigskip

{\bf Remark 4:} $~$
From Lemma \ref{lem3}, one can gain a further
intuitive
appreciation of the role that Definition 3 will play
in our proof
of
\thx{tttccc}.
This lemma indicated that
every  $B-$Bounded Valid $\Pi_1^R$ sentence 
had a  $B-$Bounded Good Skolemization, and Definition 8
indicated that 
such skolemizations
satisfied Definition 3's requirement that no
such 
invoked
Skolem function would grow faster than
the multiplication primitive. 
Such slow-growth Skolem functions characterize 
Ax-3 (but not also the
Ax-1 and Ax-2 systems).
The  intuitive reason
for this distinction is that the paradigm
in the footnote \ref{83} of Definition 8
applies only to Ax-3.


\begin{lemma}
\label{lllttt}
Using the notation 
conventions
from Definitions 5, 6 and 8, 
suppose that $~\alpha~$ is
a canonical arithmetic
system  consisting
of prenex sentences with
 $B-$Bounded Good Skolemizations
and that $~\Theta~$ satisfies the Conventional
Encoding property.
 Then any 
 Herbrand proof $~P~$ of
$~\bot~$ from the axiom system $~\alpha~$ will satisfy  $~\Theta(P)~>~B~$.
\end{lemma}

\bigskip

{\bf Proof-by-contradiction:}
Consider the contrary possibility that the 
inequality  $ \Theta(P) > B $ 
failed and that $ P $ is a 
Herbrand-proof of $ \bot $ from the system $ \alpha $ where 
 $ \Theta(P) \leq B $.
We shall denote this inequality as
 $~***~~~$.

\medskip
 
Definition 6 had  indicated
every term $~T~$ in the proof $~P~$ satisfies
the inequality of
$~\Theta(P)~>~$MinG$(T.)~$ Also,
Lemma \ref{lem2} implied Val$(T)<$MinG$(T)$.
These inequalities
and *** imply that every term $~T~$
in the proof $~P~$ satisfies
\begin{equation}
\label{nice}
\mbox{Val}(T)~<~B
\end{equation}
\ep{nice}
 implies all the terms
 $  T_1,T_2,T_3...  $ in the
 Herbrandized instances in the proof $  P  $
satisfy $  $Val$(T_i)    <    B.    $
The normalized form of an instance
of a  Skolemized axiom
is illustrated by  \ep{xprot3}. The combination of
our
 $  $Val$(T_i)    <    B  $ inequalities together with
\eq{2prot3}'s $B-$Bounded constraint on $\alpha$'s axioms 
implies that {\it each such instance of
\eq{xprot3} appearing in the proof $  P  $ must
{\bf be automatically valid} 
under the standard model
of the natural numbers.}
\begin{equation}
\label{xprot3}
\xbar{\psi} \,[ \,  \, T_1  \, , \, f_1^{\psi}(T_1)  \, , \, T_2  \,
 , \, f_2^{\psi}(T_1,T_2)
 ~, ~ \ldots , ~ ~ T_n  \, , \, 
f_n^{\psi}(T_1,T_2 ~ \ldots , ~T_n)~]
\end{equation}




The latter observation completes our proof-by-contradiction
because it  contradicts the  statement $***$ 
that had started our  proof.
More precisely $***$ had asserted that   
 $~P~$ was a 
Herbrand-proof of $~\bot~$ 
from the axiom system $~\alpha.~$
However, the Footnote
\footnote{ The point here is simply that a conjunction of
Skolemized instances
 can produce a proof of  $~\bot~$ 
only when there exists no model  $~M~$ where all these instances are
simultaneously valid. Hence when the preceding paragraph
shows that all these Skolemized instances are simultaneously valid
under the Standard Model of the Natural Numbers, it implies that
certainly
{\it no such  proof of  $~\bot~$ can feasibly  exist.}}
shows that  such is impossible when the last sentence of the preceding
paragraph indicated that 
each instance of \eq{xprot3}'s Skolemized axiom
is 
valid under the standard model of the natural numbers.
 $~~\Box$


\medskip

{\bf Finishing the Proof for \thx{tttccc}}.
 $~~$
It is easy to combine the machineries of
Lemmas  \ref{lem3} and 
 \ref{lllttt} to complete  the proof of \thx{tttccc}.
This is because Lemma \ref{lem3} had indicated
that every  $B-$Bounded Valid $\Pi_1^R$ sentence can be 
rewritten into a logically equivalent form that
has a  $B-$Bounded Good Skolemization.
Thus,  \thx{tttccc}
follows by simply taking such rewritten forms of
$~\alpha$'s axioms and then
applying Lemma \ref{lllttt}'s machinery. $~~\Box$


\section*{Appendix B: An Analysis of Ax-3's Semantic Tableaux Properties}

 
This appendix will illustrate
how the methods of \cite{ww7}
may be extended to prove that
Ax-3 satisfies the semantic
tableaux version of the Second Incompleteness
Theorem. 
Our discussion  will
be closely related to
Ko{\l}odziejczyk's observations \cite {Ko6b,Ko7} that
(in the context of Buss's Bounded Arithmetic \cite{Bu86})
semantic tableaux and Herbrand deduction can sometimes
have an exponential 
or greater 
difference in their 
proof
lengths.
In a context where
\thx{tm3} had showed that Ax-3 was an
anti-threshold  under Herbrand deduction,
the results in this appendix are noteworthy because
they imply  Ax-3 has  
 polar opposite qualities under semantic tableaux
and Herbrand deduction.


The discussion in this
abbreviated
 appendix will 
assume that the reader is familiar with
\cite{ww7}'s proof that Ax-1 and Ax-2 satisfy the semantic
tableaux version of the Second Incompleteness Theorem.
We will also often rely upon
the notation convention from
the second paragraph of 
 Section 2 of
\cite{ww7} (which defined semantic tableaux deduction's
eight elimination rules).


It is desirable to
examine 
a fourth
axiomization for \jjj , called Ax-4,
before considering 
Ax-3 because such
an approach will help make
the underlying intuitions
behind our methodologies
easier to appreciate.
In our discussion, the symbol $~\Psi~$ will denote
the $\Pi_1^R$ sentence defined by \ep{b.1}. Ax-4
will be defined to be an encoding of    
\jjsj that is identical to Ax-3 except that it includes
\ep{b.1}'s added sentence. (Since Ax-3 can trivially prove
the validity of \ep{b.1}, the Ax-4 system is clearly
logically equivalent to Ax-3. Thus while proof lengths
may differ in these two axiom systems, the final theorems
that they derive are the same.) 
\begin{equation}
\label{b.1} 
\forall ~z~~~ \forall ~q \, \leq \, z~~~~[~~ 
q*q~\leq~z~~\Rightarrow ~~ \exists ~r \, \leq~~z~~(~r=~q*q~)~~]
\end{equation}

\begin{lemma}
\label{lem.b1}
The axiom system Ax-4 satisfies the semantic tableaux version
of the Second Incompleteness Theorem. (In other words using
Definition 1's notation, Ax-4 is a 
``threshold
for the Second Incompleteness Effect'' under 
semantic tableaux deduction).
\end{lemma}

{\bf Proof Sketch:}
Our justification
of Lemma \ref{lem.b1}
 will employ the
definition of
semantic tableaux deduction
that had appeared in Section 2 of \cite{ww7}.
Its second paragraph listed eight elimination rules for
semantic tableaux deduction. 
For any term $~s,~$
its sixth rule applies
to bounded existential quantifiers appearing in expressions
similar to:
\begin{equation}
\label{b.2}
\exists~v~\leq~s~~~\Theta(v)
\end{equation}
For an arbitrary new constant symbol $~U~$ that does not
appear in the base axiom system $~\alpha~$ or in any higher
node in the semantic tableaux proof tree,
this rule 6 allows 
\ep{b.3} to be
 a descendant of \ep{b.2}'s
node.
\begin{equation}
\label{b.3}
~U~\leq~s~~\wedge ~~\Theta(U)
\end{equation}

Also 
consider 
\ep{b.4}'s
universally quantified
 sentence. 
\begin{equation}
\label{b.4}
\forall~v~~~\Phi(v)
\end{equation}
In this context for any term $~t~$
which is free in $\Phi$,
the seventh elimination rule in
Section 2 of \cite{ww7}
indicated that a descendant of \ep{b.4}'s
sentence in a semantic tableaux proof tree 
is allowed to be
any sentence of the form $~\Phi(t)~$.
In particular if we take $~t~$ to be a term of the
form ``$~U*U~$'' 
(where $~U~$ was defined in \ep{b.3} )
then this universal quantifier elimination
rule may produce the following reduction from \ep{b.4}.
\begin{equation}
\label{b.5}
\Phi(~U*U~)
\end{equation}


Lastly for any term $ \,\hat{s},~$
consider 
\ep{bc.4}'s sentence. 
\begin{equation}
\label{bc.4}
\forall~v~\leq~\hat{s}~~~\Phi(v)
\end{equation}
In this context for any term $~t~$
(again required to be free in $\Phi$),
the eighth elimination rule 
from Section 2 of \cite{ww7}
indicated that 
\ep{bc.5} is allowed to be
 descendant of \ep{bc.4}'s
sentence in a semantic tableaux proof. 
\begin{equation}
\label{bc.5}
t~\leq~\hat{s} ~~~\Rightarrow ~~~\Phi(t)
\end{equation}
Since 
\ep{bc.5}'s
 rule for eliminating universal quantifiers can apply to
any term $~t~$
(free in $\Phi$), it is applicable to the case where $~t~$ is
a term of the form 
``$~U~$'' where $~U~$ is a new constant 
created by \ep{b.3}'s elimination rule.
In this special case, \ep{bc.5}'s elimination rule can be
rewritten as:
\begin{equation}
\label{bc.6}
U~\leq~\hat{s} ~~~\Rightarrow ~~~\Phi(~U~)
\end{equation}


 
In essence, one may apply
to \ep{b.1} 
 $~n~$ iterations
of the elimination
rules from
Equations \eq{b.3}, \eq{b.5} 
and \eq{bc.6} 
to construct a sequence of
constants $ \, U_0 \, , \, U_1 \, , \, U_2 \, , \,  \ldots  \,  U_n \, $
such that $ \, U_0 \, = \, 2 \, $ and $ \, U_{i+1}  \, = \, U_i*U_i \, $.
(The 
footnote 
\footnote{The $ \, i-$th iteration of this process will
have $~z~,~$ $~q~$ and $~r~$ from 
\ep{b.1} be replaced by respectively $~U_i*U_i~,~$
$~U_i~$ and $~U_{i+1}~$ via the elimination
rules from respectively Equations
\eq{b.5}, \eq{bc.6} and \eq{b.3} to produce
an essentially thrice-revised form of
\ep{b.1} which implies that $~U_{i+1}~=~U_i*U_i~$.} summarizes 
the
structure of the $~i-$th round of these $~n~$ iterations.)
In a formal sense, these $~n~$ iterations may thus be simulated
by a fragment of a semantic tableaux proof tree, denoted as $~F~,~$
where all
the branches of $~F~$ are closed except for one
branch, called the pivotal branch, which contains the
parameter symbols of  $~U_0~,~U_1~,~U_2~,~ \ldots ~ U_n~$
together with a collection of sentences,
appearing in linear order, asserting
that  $~U_0~=~2~$ and $~U_{i+1} ~=~U_i*U_i~$.

Hence this ``pivotal branch'' 
of $~F~$
will imply that $~U_n~=~2^{2^n}~$. Moreover since the 
fragment $~F~$
will have only  $O(~n~)$ nodes, it will establish the
existence of a number  
 $~U_n~=~2^{2^n}~~,~$ whose binary encoding has a $~2^n~$ length
that is much larger than $~F \,$'s length.

The above invariant is essentially all that we need to generalize
\cite{ww7}'s semantic tableaux version of the Second Incompleteness
Theorem so that it also applies to Ax-4. (We obviously
have omitted many
details here.
However,  they are relatively straightforward
extensions of \cite{ww7}'s methodologies
 because the
super-exponential growth property, established by the prior paragraphs,
opens an avenue for introducing 
\cite{ww7}'s proof techniques, whose formal details are too
lengthy to 
be
fully duplicated during this abbreviated proof
sketch.) $~~\Box$



The remainder of this appendix will sketch how Lemma \ref{lem.b1}'s
variant of the semantic tableaux version of the Second Incompleteness
Theorem can be extended from the axiom system Ax-4 to
 Ax-3 (itself).
Before doing so, we wish to introduce one further preliminary lemma.



\begin{lemma}
\label{lembb2}
The axiom system Ax-4 
satisfies the same
 anti-threshold
property 
 for Herbrandized
deduction as did Ax-3 (in \thx{tm3} ).
\end{lemma}

{\bf Proof Sketch} 
Every aspect of the proof of \thx{tm3} 
(for Ax-3)
does generalize also for Ax-4's paradigm. This is because
\thx{tm3}'s proof generalizes for any extension of Ax-3 that consists
of
a finite number of additional 
logically valid
$\Pi_1^R$ sentences.
Thus, if $~\alpha~$ denotes any such an extension of
Ax-3 and if 
$~\alpha^*~$ denotes 
the extension of $~\alpha~$ which contains one additional
$\Pi_1^R$ sentence, asserting the consistency of 
$~\alpha^*~$ (analogous to \thx{tm2}'s Diagonal$(\alpha,D)$
sentence), then every aspect of
our prior analysis of Ax-3$^*$ applies also to
$~\alpha^*~$. 
Thus using the same reasoning as before, it follows that
$~\alpha~$ (and  hence 
also Ax-4) must
be anti-thresholds relative to Herbrand deduction. 
$~\Box$


The combination of Lemmas \ref{lem.b1}
and \ref{lembb2}
already shows that semantic tableaux and Herbrand
deduction have polar opposite threshold properties
with respect to Ax-4. (This is because the latter satisfies the semantic
tableaux 
version of the Second
Incompleteness Theorem,
but it does not also
satisfy
 its  Herbrandized 
variant.) The same
pair of
 polar opposite qualities
also applies to \jjj 's Ax-3 axiomization. However, the 
proof that Ax-3 satisfies the semantic tableaux version of
the Second Incompleteness Theorem is 
substantially
more complicated
than Lemma \ref{lem.b1}'s analogous 
result for Ax-4. 


The final goal in this appendix 
will 
thus be 
to explain how one may incrementally revise
Lemma \ref{lem.b1}'s 
proof-analysis
 for Ax-4 so that a similar incompleteness property also
applies to Ax-3.
In our discussion, $~\xbar{\psi}(v)~$ will denote
the following $~\Delta_0^R$ formula,
which is free only in $~v~~$:
\begin{equation}
\label{tb.1} 
\forall ~q \, \leq \, v~~~~[~~ 
q*q~\leq~v~~\Rightarrow ~~ \exists ~r \, \leq~~v~~(~r=~q*q~)~~]
\end{equation}
In this notation, the sentence $~\Psi~$ that constituted 
Ax-4's one additional $~\Pi_1^R~$ axiom-sentence (defined in
\ep{b.1} ) can be rewritten as:
\begin{equation}
\label{tb.2} 
\forall ~v~~~ \xbar{\psi}(v)
\end{equation}
Note that Ax-3, unlike Ax-4, does not contain 
\ep{tb.2}'s $\Pi_1^R$ sentence as an axiom of its formalism.
However using an analog of 
``passive induction'' from our paper \cite{ww7}, 
we will show Ax-3 contains a counterpart of \ep{tb.2} 
within its inductive schema that has comparable properties.



In particular, let $~\underx{\psi}(z)~$ denote the following
$\Delta_0^R$ formula:
\begin{equation}
\label{apb.wp1}
 \{ \,  \,  \xbar{\psi}( 0) \,  \, \wedge \,  \,  \forall y \leq z \, [ \, 
\xbar{\psi}( y) \, \Longrightarrow \, \xbar{\psi}( y') \,  \, ]  \, \, \}  \,  \,   \Longrightarrow  \,  \, 
 \forall y \leq z \,  \xbar{\psi}( y) 
\end{equation}
Then  \ep{tb.3} represents an encoding of one
of Ax-3's induction axioms.
\begin{equation}
\label{tb.3} 
\forall ~z~~~ \underx{\psi}(z)
\end{equation}
In order to explain the significance of this 
added 
notation, let $~\Psi~$ again denote
\ep{b.1}'s axiom sentence (which we saw was identical to
\ep{tb.2}'s sentence except that the latter uses a
slightly different
notation). Also, let $~\Psi^*~$ denote
\ep{tb.3}'s sentence. In this notation,
$~\Psi^*~$ 
can be viewed as a cousin of $~\Psi~$ which has the property
that although $~\Psi~$ is not an axiom of Ax-3, its cousin
$~\Psi^*~$ is a formal axiom of Ax-3.  

The latter property is important because we need a vehicle
for formally translating proofs 
from the Ax-4 
system
to proofs in
the Ax-3  system.
Since $~\Psi~$ is the only axiom belonging to Ax-4 which is not also
in Ax-3, 
\ep{apb.wp1}'s formal counterpart
of it,
called $~\Psi^*~$,
will provide 
a means for doing this translation. 
The implications of this translation mechanism
is described by our next lemma. 

\begin{lemma}
\label{lembb3}
Let $~\Omega~$ 
denote an arbitrary theorem that is proven
from the axiom system Ax-4 under a semantic tableaux
proof $~T~$ that consists of 
$~n~$ applications of
the $~\forall ~$ elimination to rule to
 \ep{tb.2}'s axiom sentence of $~\Psi~$.
For some term $~t_i~,~$
let us assume that
the $~i-$th such application of 
this rule
replaces 
\ep{tb.2} 
with the reduced sentence of
\begin{equation}
\label{tb.2r} 
\xbar{\psi}(~t_i~)
\end{equation}
Then for a constant $~k~$ whose value is
entirely
independent of $~n~$, one can construct a proof $~T^*~$
of the same theorem 
$~\Omega~$ from the axiom system Ax-3 where the difference
between the number of node-sentences appearing in the proofs
of $~T~$ and of $~T^*~$ is bounded by
the quantity of $~kn~$.
\end{lemma}

{\bf Proof:}
The justification of Lemma \ref{lembb3}
is fairly straightforward. On each occasion
where $~T \,$'s proof contains a node similar to
\ep{tb.2r}, the comparable structure in $~T^*~$
will replace this sentence with a tree-fragment
that consists of the following  
four components:
\bee
\item
An initial node-sentence of the form \eq{tb.cute}
(which is justified because it is an instance of
\ep{tb.3}'s
 axiom sentence of $~\Psi^*~$ ).
\begin{equation}
\label{tb.cute}
\underx{\psi}(~ t_i~)
\end{equation}
\item
A branch separation will appear
directly below \ep{tb.cute}'s
node that consists of the two
sibling
 sentences of
\eq{tb.cute1} and \eq{tb.cute2}. In light of
\ep{apb.wp1}'s definition of $~\underx{\psi}~,~$
this binary separation is justified by the semantic tableaux
rule for eliminating
the $~\Rightarrow~$ symbol (which
was formalized by Item 4 
from the second paragraph of \cite{ww7}'s 
 Section 2).
\begin{equation}
\label{tb.cute1} 
\neg ~~ \{ \,  \,  \xbar{\psi}( 0) \,  \, \wedge \,  \,  \forall y \leq t_i \, [ \, 
\xbar{\psi}( y) \, \Longrightarrow \, \xbar{\psi}( y') \,  \, ]  \, \, \}
\end{equation}
\begin{equation}
\label{tb.cute2}
 \forall y \leq t_i \,  \xbar{\psi}( y) 
\end{equation}
\item
Since Ax-3 
can prove 
\ep{b.1}'s sentence of $~\Psi~$ as a theorem,
it is trivial to establish that
for some fixed constant $~k_1~$ which does not depend on $~i~$,
one may insert a closed semantic tableaux proof
of length $~k_1~$
 under \ep{tb.cute1}'s
sentence, 
$~$which accomplishes the desired effect of showing that \ep{tb.cute1}
is inherently contradictory. 
\item
Using one more time
\cite{ww7}'s
 semantic tableaux rules for eliminating the bounded 
universal quantifiers and  the $~\Rightarrow~$ symbol,
one may easily insert a branch below \ep{tb.cute2} 
that ends with the pair of sibling sentences 
given in equations 
\eq{cute3} and \eq{cute4} below. 
Moreover, it is evident that
\ep{cute3} is inherently contradictory. Thus for some fixed constant
$~k_2~$ (which again does not depend on $~i~)~$,
the net consequence of this step will be to consist of a sequence of
$~k_2~$ node sentences that includes the sentences given in
\eq{cute3} and \eq{cute4} and closes the proof-tree that descends 
from \ep{cute3} with the desired forced contradiction.
\begin{equation}
\label{cute3}
\neg ~~(~~ t_i~ \leq ~ t_i~~)
\end{equation}
\begin{equation}
\label{cute4}
\xbar{\psi}( ~t_i~) 
\end{equation}
\ene

Let $~k~$ denote a constant that equals the quantity
of $~k_1 \, + \,~k_2 \, + \, 3~$. Then each iteration of
the above 4-step procedure will
replace 
\ep{tb.2r}'s sentence with a subtree fragment
consisting of $~k~$ 
new
sentences.
Note that the final sentence at the end of each such iteration
(given in \ep{cute4} )
contains the
identical logical statement as 
was given in 
\ep{tb.2r}'s initial sentence.
Since all the other branches of our new sub-structure
are closed and since Equations
\eq{tb.2r} and 
\ep{cute4} are identical
to each other, it follows that after we finish
performing $~n~$ iterations of the above process,
introducing $~kn~$ new sentences, our 
new revised semantic tableaux tree will 
prove the theorem $~\Omega~$ from Ax-3 instead of Ax-4. $~~~\Box$






\begin{theorem}
\label{tapb}
The axiom system Ax-3
(similar to Ax-4)
 satisfies the semantic tableaux version
of the Second Incompleteness Theorem. (Thus using
Definition 1's notation, Ax-3 is also a 
``threshold
for the Second Incompleteness Effect'' under 
semantic tableaux deduction).
\end{theorem}

{\bf Proof Sketch}: 
For the sake of brevity, we will only outline
the intuition behind \thx{tapb}'s proof.
It will be essentially a consequence of the combination of
Lemmas \ref{lem.b1} and \ref{lembb3}.


In particular, the core idea behind Lemma \ref{lem.b1}'s
analysis of Ax-4 was to employ a sequence of $\,n\,$ iterated
applications of \ep{b.1}'s axiom of $\,\Psi\,$
so as to construct a 
formal
sequence of
constants $\,U_0\,,\,U_1\,,\,U_2\,,\, \ldots \, U_n\,$
such that $\,U_0\,=\,2\,$ and $\,U_{i+1} \,=\,U_i*U_i\,$.
By Lemma \ref{lembb3}, the precisely identical 
sequence of
$\,U_0\,,\,U_1\,,\,U_2\,,\, \ldots \, U_n\,$
can be constructed from Ax-3 using only $\,kn\,$ additional
sentences, $\,$for some fixed constant $\,k\,$ whose value is independent
of $\,n\,$.

Since $~U_n~$ represents the quantity of $~2^{2^n}~$ whose binary
encoding has a length of $~2^n~,~$ this binary length is clearly
much larger than $~kn~$ as $~n~$ approaches of infinity. As a
result of this exponential difference in lengths, it is relatively routine
to revise 
Lemma \ref{lem.b1}'s
proof of the semantic tableaux version of the Second Incompleteness
Theorem for Ax-4 so that it may also apply to Ax-3. (The remaining details
are analogous to the constructions that we used in
\cite{ww2,ww7} and are omitted 
for the sake of brevity.) $~~\Box$


{\bf Summing Up What Has Been Done in this Appendix :} 
We have outlined abbreviated proofs
showing that the
Ax-3 and Ax-4 encodings for \jjsj satisfy the semantic tableaux
versions of the Second Incompleteness Theorem. We visited
this topic twice because the result is easier to establish
for Ax-4, 
$~$although
      it is stronger for Ax-3. (The latter
is 
stronger
because Ax-3 contains one less
axiom sentence than
Ax-4.) 
Both our 
results for 
 Ax-3 and Ax-4 are
interesting 
 because
these formalisms are 
anti-thresholds
for the Herbrandized version of the Second Incompleteness
while they are thresholds 
when the paradigm is changed to focus upon
semantic tableaux style deduction.


We close this appendix by once again reminding the reader that a
different 
type of axiomatic setting
where semantic tableaux and Herbrandized proofs have
a sharp difference in length was formalized by
Ko{\l}odziejczyk in \cite {Ko6b,Ko7}.
(In particular, Ko{\l}odziejczyk  \cite {Ko6b,Ko7}
focused his discussion on
Buss's bounded arithmetic
of \cite{Bu86}.) 


\begin{thebibliography}
\small
\footnotesize



\bibitem{Ad2}
Z. Adamowicz,
``Herbrand Consistency and Bounded
Arithmetic'', 
{\it Fundamenta Mathematica}
171 (2002) 279-292.



\bibitem{AB1}
Z. Adamowicz and T. Bigorajska,
``Existentially Closed Structures and G\"{o}del's Second Incompleteness
Theorem'', {\it Journal of Symbolic Logic}
66 (2001), 349-356.


\bibitem{AZ1}
Z. Adamowicz and P. Zbierski,
``On Herbrand consistency in weak theories'', 
{\it Archive for Mathematical Logic}
40 (2001) pp. 399-413.


\bibitem{BS76}
A. Bezboruah and J. 
 Shepherdson,
``G\"{o}del's Second Incompleteness Theorem for Q'',
{\it Journal of Symb Logic} 41 (1976)  503-512.


\bibitem{Bu86}
S. Buss, Bounded Arithmetic,  in
Proof Theory Lecture Notes, Vol. 3, published
by Bibliopolis, Naples, 1986.


\bibitem{BI95} S.  Buss and A. Ignjatovic, ``Unprovability of Consistency
Statements in Frag. of Bounded Arithmetic'', {\it Annals  Pure and
Applied Logic} 74 (1995)221-244.



\bibitem{Da83}
M. Davis, 
{\it Computability and Unsolvability}
McGraw Hill 1983.

\bibitem{DPR61}
M. Davis, H. Putnam and J. Robinson,
``The Decision Problem for Exponential Diophantine Equations'',
{\it Annals of Mathematic} 74 (1961) pp. 425-436.

\bibitem{Fe60}
S. Feferman, ``Arithmetization of Mathematics in a General Setting'',
{\it Fundamenta Mathematica}
19 (1960) pp. 35-92.


\bibitem{GD82}
H. Gaifman and C. Dimitracpoulos,
Fragments of Peano Arithmetic and the MDRP
Theorem,
{\it Logic and Algorithms},
Monagraphies de l'Ensignment Mathamtique, 30 (1982)
pp. 197-206.



\bibitem{Go31}
K. G\"{o}del,
`` \"{U}ber formal unentscheidbare S\"{a}tse der Principia
Mathematica und Verwandte Systeme I'',
{\it Monatshefte f\"{u}r Math. Phys.} 37 (1931) pp. 349-360.



\bibitem{Ha71}
P. H\'{a}jek,
On  Interpretability in Set Theory,
Part I in {\it Comm, Math, Univ. Carol} Volume 12 (1971) pp. 73--79
and 
Part II in  Volume 13 (1972) pp. 445-455.





\bibitem{HP91}
P. H\'{a}jek 
and P. Pudl\'{a}k,
{\it Metamathematics of First Order Arithmetic,}
Springer Verlag 1991. 



\bibitem{He30}
J. Herbrand, Recherches sur la theorie de la demonstration, 
{\it Travaux de la Soceite et des Lettres de Varsvoie}
III 33 pp. 33-160 (1930).


\bibitem{HB39}
D. Hilbert and P. Bernays,
{\it Grundlagen der Mathematic}, Springer 1939.



\bibitem{Ka91}
R. Kaye,
{\it Models of Peano Arithmetic}, Oxford University
Press, 1991. 

\bibitem{Kl38}
S.
Kleene,
``On the Notation of Ordinal Numbers'',
{\it Journal of  Symbolic Logic}
3 (1938), pp.
150-156.



\bibitem{Ko6b}
L.A. Ko{\l}odziejczyk,
``Some results on Weak Consistency Statements in Weak Theories'',
talk at the JAF25 conference in Clermon-Ferrand (2006)
whose slides are available
at hhtp://www.cs.us.es/glm/jaf25-CMBFD/slides-lk.pdf.

\bibitem{Ko7}
 L.A. Ko{\l}odziejczyk,
``On the Herbrand Notion of Consistency for Finitely
Axiomatizable Fragments 
of Bound Arithmetic'',
{\it Journal of Symbolic Logic}
71 (2006) pp. 624-638.



\bibitem{Kr87} J. Kraj\'{i}cek,
``A Note on Proofs of Falsehood'', 
{\it Archive for Math Logic}
 26 (1987)  169-176.


\bibitem{Kr95} J. Kraj\'{i}cek,
{\it Bounded Propositional Logic and Complexity Theory,} Cambridge 
Unniversity
Press, 1995.



\bibitem{KT74}
G. Kreisel and G. Takeuti,
``Formally Self-Referential Propositions  for Cut-Free Classical
Analysis'',
Dissertationes Mathematicae
118
(1974) pp. 1--55




\bibitem{Lo55} M. H. L\"{o}b, A Solution to a Problem by Leon Henkin,
{\it JSL}
20(1955) pp 115-118

\bibitem{Ma93}
Y. Matiyasevich, {\it Hilbert's Tenth Problem},
MIT Press (1993)


\bibitem{Ne86}
E. Nelson, {\it  Predicative Arithmetic,} Math Notes,
Princeton Press, 1986.


\bibitem{PD82}
J.  Paris and C. Dimitracopoulos,
``Truth definitions for $\Delta_0$ formulae'',
        {\it  Monagraphie de L'Enseignement Mathematique}  30 (1982)
pp. 317-329.


\bibitem{PD83}
J. Paris and C. Dimitracopoulos,
A Note on the Undefinability of Cuts, {\it  Journal of Symbolic Logic}
48 (1983) pp. 564--569.




\bibitem{PW81}
J.  Paris and A.  Wilkie,
``$\Delta_0$ Sets and Induction'', 
{\it 1981 Jadswin  Conference Proceedings,}
(Leeds University Press) 
pp. 237-248.


 \bibitem{Pu83}
 P. Pudl\'{a}k,
 Some Prime Elements in the Lattice of
 Interpretability
 Types,
 {\it Transactions of the American Mathematical
 Society} 280 (1983) pp. 255-275
 


\bibitem{Pu84}
P. Pudl\'{a}k,
``On Lengths of Proofs of Finisitic Consistency Statements in
First order Theories'',
{\it Logic Colloquium 84}, North Holland (1986)
pp. 165-196.



\bibitem{Pu85}
P. Pudl\'{a}k,
``Cuts, Consistency Statements and Interpretations'',
{\it Journal of Symbolic Logic}
50 (1985) 423-442


\bibitem{Pu96}
P. Pudl\'{a}k,
``On the Lengths of Proofs of Consistency'',
in {\it The Collegium Logicum: The Offical 1996 Annals of  Kurt G\"{o}del
Society} ( Volume 2),  Springer-Wien-NewYork,
pp. 65-86.


\bibitem{Sa1}
S. Salehi, Herbrand Consistency in Arithmetics with
Bounded Induction, Polish Academy of
Sciences  Ph D thesis, October 2001.



 \bibitem{Sm77}
 C.  Smory\'{n}ski, ``The Incompleteness Theorem'',
{\it The Handbook on Mathematical Logic},
1977,      {North Holland},
        pp.      {821--865}.


 \bibitem{Sm85}
 C.  Smory\'{n}ski, ``Non-standard Models and Related Developments
 in the Work of Harvey Friedman'', in {\it Harvey Friedman's
 Research in  Foundations of Math.},
 North Holland 1985, pp. 179-220.



\bibitem{So94}
R. Solovay, Private 
telephone conversations between  
Robert Solovay and Dan Willard 
(during April of 1994)
concerning Solovay's generalization of one of  Pudl\'{a}k's  theorems 
\cite{Pu85}, using also some of
Nelson's and Wilkie-Paris's methodologies \cite{Ne86,WP87}.
Solovay's unpublished theorem
 shows that essentially no axiom system that recognizes
Successor$(x)~=~x+1~$ as a total function
(and which treats multiplication and addition as 3-way relations)
 can prove a theorem
affirming its own consistency under Hilbert deduction.
The Appendix A of 
\cite{ww1} offers a 4-page interpretation of the  intuition behind
Solovay's unpublished idea.


\bibitem{Sv83}
V. \v{S}vejdar, Modal Analysis of Generalized
Rosser Sentences,
{\it Journal of Symbolic Logic} 48 (1983) pp. 986-999.



\bibitem{Sw3}
S. Swierczkowski,
Finite Sets and G\"{o}del's Incompleteness Theorem
Dissertationes Mathematicae
422 (2003)
pp. 1--58


\bibitem{Ta53}
G. Takeuti,
``On a Generalized Logical Calculus'',
{\it Japan Journal of Mathematics}
23 (1953) pp. 39--96.




\bibitem{Ta0} 
G. Takeuti,
``G\"{o}del Sentences of Bounded Arithmetic'',
{\it Journal of Symbolic Logic} 65 (2000) pp. 1338-1346

 \bibitem{TMR53}
 A. Tarski, A. Mostowski and R. M. Robinson,
 {\it Undecidable Theories}, North Holland, 1953.


\bibitem{Vi92}
A. Visser,
An Inside View of Exp (or The Closed
Fragment of The Provability Logic I$\Delta_0 +
\Omega_1$
With a Propositional Constant for Exp),
{\it Journal of Symbolic Logic} 57 (1992)
pp. 131--165


\bibitem{Vi93}
A. Visser,
``The Unprovability of Small Inconsistency'',
{\it Archive for Mathematical Logic} 32 (1993) pp. 275-298.

\bibitem{Vi5}
A. Visser,
 ``Faith and Falsity'',
 {\it Annals Pure \& Applied Logic}
131 (2005), pp. 103-131.


\bibitem{VH73}
P. Vop\v{e}nka and P. H\'{a}jek,
Existence of  a Generalized Semantic Model of
G\"{o}del-Bernays Set Theory,
{\it Bulletin de l'Academie Polonaise des
Sciences,
Mathmatiques, Astromiques et Physiques}
12 (1973) pp.1079-1086.




\bibitem{WP87}  
A. Wilkie and J. Paris,  
``On the Scheme of Induction for Bounded
Arithmetic'', 
 {\it Annals of Pure and Applied Logic}
(35) 1987, 261-302


\bibitem{ww93}
D. Willard,
``Self-Verifying Axiom Systems'', {\it
   Third Kurt G\"{o}del
Colloq}
(1993),
Springer-Verlag LNCS\#713,  325-336.




\bibitem{ww0}
D. Willard, 
``The Semantic Tableaux Version of the Second
Incompleteness Theorem Extends Almost to 
Robinson's 
Arithmetic 
Q'',
Springer Verlag LNCS\#1847, July 2000,
 pp. 415-430. 

\bibitem{ww1}
D. Willard, ``Self-Verifying  Systems, the Incompleteness
Theorem and the
Tangibility 
Principle'', in
{\it Journal of Symbolic Logic}
$~66~ (2001)\,$ pp. 536-596.


\bibitem{ww2}
D. Willard, 
``How to Extend The Semantic Tableaux And
Cut-Free Versions of the Second
Incompleteness Theorem 
Almost 
to 
Robinson's Arithmetic Q'', in $~$
{\it Journal of Symbolic Logic}
$~\,67~ (2002)~$ pp. 465--496.


\bibitem{ww4}
D. Willard, 
``A Version of the
Second Incompleteness Theorem For Axiom
Systems that Recognize Addition 
But Not Multiplication as a Total Function'',
{\it First Order Logic Revisited,}
Logos Verlag (Berlin) 2004, pp. 337--368.


\bibitem{ww5}
D. Willard, 
``An Exploration of the Partial Respects in which an Axiom
System Recognizing Solely Addition as a Total Function Can
Verify Its Own Consistency'', 
{\it Journal of Symbolic Logic} 70  (2005) pp. 1171-1209. 


\bibitem{ww5b}
D. Willard, 
``On the Available Partial Respects in which
 an Axiomatization
for Real Valued  Arithmetic Can  Recognize its 
Consistency'', 
{\it Journal of Symbolic Logic} 71 (2006)
pp. 1189-1199.



\bibitem{ww6}
D. Willard,  
``A Generalization of the Second Incompleteness 
Theorem and Some Exceptions to It'',
{\it Annals of Pure and Applied Logic}
141 (2006)
pp. 472-496.

\bibitem{ww7}
D. Willard,  
``Passive Induction and a Solution to a Paris-Wilkie
Open Question'',
{\it Annals of Pure and Applied Logic}
146 (2007) pp. 124-149.



\bibitem{Wr78}
C. Wrathall, ``Rudimentary Predicates and Relative Computation'',
Siam J. on Computing 7 (1978), pp. 194-209.


\end{thebibliography}



\end{document}

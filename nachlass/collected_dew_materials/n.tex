%%%%  septem10 correction  10.2am  MISSING WORD FIX UP 



\documentclass[bsl,meeting]{asl}

\pagestyle{plain}

\def\urladdr#1{\endgraf\noindent{\it URL Address}: {\tt #1}.}

\newcommand{\ep}[1]{Equation (\ref{#1})}
\newcommand{\NP}{}

\begin{document}
\thispagestyle{empty}


\NP  
\absauth{Dan E. Willard}
\meettitle{On 
the Utility of 
Partial Evasions
of the Second Incompleteness Theorem in the Modern
Digital Era}
\affil{Computer Science \& Mathematics Departments, 
University at Albany, NY 12222}
\meetemail{dwillard@albany.edu}

 \normalsize
 % \Large
% \baselineskip = 2.0 \normalbaselineskip 
 \baselineskip = 1.0 \normalbaselineskip 

We have published several articles
about 
generalizations of the Second Incompleteness  
 Theorem
and partial
evasions of it under
formalisms that own a partial
knowledge about their own self-consistency.  
This research has included six articles
in the JSL and APAL plus several additional papers,
all of which are 
cited by us in
paper \cite{ww18}.
In a context where $\alpha$ is a set of proper axioms and
$d$ is one of several possible deduction methods, the ordered pair
$(\alpha,d)$ will be called {\bf Self Justifying} iff:
\begin{description}
  \item[  i   ] one of  $~(  \alpha  , d  )$'s  theorems
(or possibly one of $\alpha$'s axioms)
will
state that the deduction method $ \, d, \, $ applied to the
axiom
system $ \, \alpha, \, $ 
is formally consistent.
\item[  ii   ]
   and also  the axiom system $ \, \alpha  \, $ is actually consistent.
\end{description}
We noted  the 
Fixed Point Theorem enables one to map 
any ordered pair $(\alpha,d)$ onto an axiom system $\alpha^d$
that includes all $\alpha$'s axioms plus an added axiom
statement  declaring:
\begin{quote} 
$\oplus~~~$ {\it
``There is no proof 
(using 
$d$'s deduction method)
of  $0=1$
from the  {\it union}
 of
the
 axiom system $\, \alpha \, $
with {\it this}
statement $~\oplus ~$ (looking at itself)''.}
\end{quote}
The difficulty with statement 
$\oplus$ is that it is typically false, {\bf  although it can be quite easily
encoded.} This is because if the ordered pair $(\alpha,d)$ is too
strong then the system   $\alpha^d$ will be rendered
inconsistent via a classic
G\"{o}del
 Diagonalization construction (e.g.
causing $(\alpha^d,d)$ to violate Part-ii of
Self-Justification's
 definition).

These issues  raise  {\bf legitimate concerns}
about whether  boundary-case
exceptions to the Second Incompleteness Theorem are 
an awkward  wrinkle
from a theoretical  standpoint.
 Aside from generalizing our earlier
results, our new paper suggests its proposals can be
 significant  from a {\it pragmatic perspective.}
For instance,
 the late physicist
Stephen Hawking
has predicted that global warming
and other dangers will likely be so severe
that civilization in our Solar System
will find it difficult to persist without
employing
 artificially
intelligent
computers, 
{\it in at least some respect.}

One would
ideally
 prefer computers to imitate a human's approximate
instinctive appreciation 
of  his own consistency.  
Our paper \cite{ww18} observes  self-justifying
computers can do this in  a pragmatic manner,
when $(\alpha,d)$ is sufficiently weak.
Moreover
{\it with quite comparable efficiency}, our formalisms can
nicely
 reconstruct  isomorphic 
analogs of
all  the  $\Pi_1$ theorems of Peano Arithmetic  under a 
slightly revised language.
\begin{thebibliography}{99}
\bibitem{ww18}
Dan E. Willard,
``About   the 
Chasm Separating the Goals of
Hilbert's Consistency Program From the
Second Incompleteness Theorem''
http://arXiv.org/abs/1807.04717

% (in the Cornell Archives).

\end{thebibliography}
\end{document}


% 2020 5  june 6 6.1 am ( aftr spell and subsequent corredions)
 



% 2017  august 30 8.5 am MINOR REVISONS PAGE 6
%%Removing rphraisng  skinny as Robert recomeneded
 


% vladimir mechanic 265-2212

% ``%ss% = sections seth recommendedI skip

%\documentclass[11pt]{article}
%\documentclass[10pt]{article}
% \documentclass[11pt]{article}
\documentclass[12pt]{article}


%%%%%%%%%% \documentstyle[11pt]{article}


\usepackage{amssymb}


\addtolength{\oddsidemargin}{-1.1in}
\setlength{\textheight}{9.4 in}
\setlength{\textwidth}{6.5 in}
%%%% above paper

%\setlength{\textwidth}{6,0 in}
%\setlength{\textwidth}{5,5 in}

\addtolength{\topmargin}{-0.75in}


%% 
%% \addtolength{\oddsidemargin}{-0.5 in}
%% \setlength{\textheight}{10.1 in}
%% \setlength{\textwidth}{7.5 in}
%% \addtolength{\topmargin}{-0.5in}






          \newcommand{\newthmwithin}[3]{\newtheorem{#1q}{#2}[#3]
              \newenvironment{#1}{\begin{#1q}\sf}{\end{#1q}}}

\newcommand{\newthm}[3]{\newtheorem{#1q}[#2q]{#3}
                        \newenvironment{#1}{\begin{#1q}\sf}{\end{#1q}}}
\newcommand{\newthmm}[3]{\newtheorem{#1q}[#2q]{#3}
                        \newenvironment{#1}{\begin{#1q}\rm}}

\newtheorem{theorem}{$~~~~$ Theorem}[section]
% \newtheorem{corollary}{Corollary}[section]
%\newtheorem{fact}{Fact}[section]
\newcommand{\makenewheading}[1]{\begin{tabbing} {\bf #1:}
\end{tabbing}}

\newtheorem{example}[theorem]{$~~~~$ Example}
\newtheorem{themx}[theorem]{$~~~~$ Theorem}
\newtheorem{corollary}[theorem]{$~~~~$ Corollary}
\newtheorem{lemma}[theorem]{$~~~~$ Lemma}
\newtheorem{remark}[theorem]{$~~~~$Remark}
\newtheorem{definition}[theorem]{$~~~~$Definition}
\newtheorem{fact}[theorem]{$~~~~$Fact}


\newtheorem{dff}[theorem]{$~~~~$ Definition}
\newtheorem{exx}[theorem]{$~~~~$ Example}
\newtheorem{lemm}[theorem]{$~~~~$ Lemma}
\newtheorem{propp}[theorem]{$~~~~$ Proposition}
\newtheorem{remm}[theorem]{$~~~~$Remark}
\newtheorem{ccr}[theorem]{$~~~~$Corrolary}
\newtheorem{coj}[theorem]{$~~~~$Conjecture}


\newtheorem{deff}[theorem]{$~~~~$Definition}





% \def\Box{ QED}
\def\nop{ }
\def\nyp{\newpage }
% \def\nxp{ }
\def\nxp{ Here $~$NXP }


%  \def\nyp{ }
% \def\nyp{ }

\def\bigc{$\,$of the unabridged version of this paper \cite{ww12}}

\def\nor1{Normed$\{~2^{ \zzz \theta  \, )} ~$,$~\sqrt{~2^{ \zzz \theta  \, )}}~\}$}

\def\pagxx{Page ?xx?}
\def\xor2{Normed$\{ ~\sqrt{~2^{ \zzz \theta  \, )}}~,~2~ \} $}
%\def\fffx{Fact \#}
\def\fffx{{\bf Fact *}}
\def\zhz{H }
%\def\fffx{Fact \#}
\def\appD{Appendix D }
\def\appxD{Appendix D}
\def\fffour{three }
\def\zazsta{ and EA-stability}

% \def\glamb{\xi}
\def\glamb{\lambda}
\def\glamb{P}
\def\glamb{\theta}
\def\pag2{Page 2}
%% \def\zzthe{\zeta}


\def\glamb{\zeta}
\def\zzthe{\theta}





\def\gggen{$( L^\xi  ,  \Delta_0^\xi   ,  B^\xi  ,  d  ,  G  )~$}
\def\gggcp{$( L^\xi  ,  \Delta_0^\xi   ,  B^\xi  ,  d  ,  G  )$}
\def\peta{\sigma}
\def\zzxz{~ \sharp ( \, }
\def\zzz{~ \sharp ( ~ }
\def\zip{\sharp }
\def\mheta{\theta^\bullet}
\def\mxi{\xi^\bullet}
%\def\mbi{\bullet}
\def\mbi{\bigdot}
\def\mbi{\bigoplus}
\def\xxi{$\, \xi^* \,$}




\def\tftt{~ \frac{1}{2}~ }
\def\sss{ }

\def\goodshit{\triangleright}
\def\bullshit{\triangleleft}
\def\foo{footnote \footnote}
\def\Uxp{\Upsilon}

\def\f55{ \normalsize  \baselineskip = 1.8 \normalbaselineskip } 


\def\f55{  \baselineskip = 1.1 \normalbaselineskip } 
\def\g55{  \baselineskip = 1.0 \normalbaselineskip } 
\def\s55{ \baselineskip = 1.0 \normalbaselineskip } 

\def\f55{  \baselineskip = 0.7 \normalbaselineskip } 



\def\bbskip{\bigskip}



\def\axst{\odot}
\def\rp{ p^\theta } 
\def\thsp{Theorem $ \, * \,$ }
\def\thss{Theorem $ \, *\,$'s }
\def\dexxt{\Delta}
\newcommand{\co}[1]{Corollary \ref{#1}}
\newcommand{\thx}[1]{Theorem \ref{#1}}
\newcommand{\cjx}[1]{Conjecture \ref{#1}}
\newcommand{\phx}[1]{Proposition \ref{#1}}
 \newcommand{\dfx}[1]{Definition \ref{#1}}
\newcommand{\lem}[1]{Lemma \ref{#1}}



%% \newcommand{\co}[1]{Corollary \ref{#1}}
%%  \newcommand{\thx}[1]{Theorem \ref{#1}}
\newcommand{\lxem}[1]{Lemma \ref{#1}}
\newcommand{\overx}[1]{\, \overline{ {#1} } \,} 


\newcommand{\el}[1]{Line (\ref{#1})}
\newcommand{\ex}[1]{Expression (\ref{#1})}
\newcommand{\ei}[1]{item (\ref{#1})}

\newcommand{\eq}[1]{(\ref{#1})}
\newcommand{\ep}[1]{Equation (\ref{#1})}
\newcommand{\thetlam}{ \theta }
\newcommand{\underx}[1]{\overline{~ {#1} ~}}
\newcommand{\appaa}{$App \forall$}
\newcommand{\appee}{$App \exists$}
\newcommand{\tll}[1]{Tab$- {#1} -$List}
\newcommand{\txl}[1]{Tab$- {#1}$}
\newcommand{\tlxl}[1]{Tab$- {#1}$ }
\newcommand{\sll}[1]{Short$- {#1} -$List}
\newcommand{\axx}[1]{NS$_D^{\,k,m}( ${#1}$)$}


\newenvironment{proof}{{\bf Proof:}}{$\Box$}
\newenvironment{sketchproof}{{\bf Sketch of Proof:}}{$\Box$}


\begin{document}

%\title{Rough Summary of March 2012 Research Before I Attended
%the AMS March 17-18 Conference}


%% \title{ On
%% How the Revival of a Diluted 
%% Version of
%% Hilbert's Consistency Program 
%% %is 
%% Should
%% Likely
%%  Be
%% % Plausible and
%% % Veru
%%  Germane
%% to Computer Science}


\title{$Very~~ Very~$
Informal Notes for Cameron and Nate from Dan}


%% \title{On  How
%%   A Novel Indeterminately Defined
%% %$~\theta~ \gimel$ 
%% Function Primitive 
%% Enables Some Axiom Systems
%% To
%% Appreciate Fragments of Their Own
%% Hilbert Consistency}
%% 
%% 

%% 
%% \title{On  How
%%   An Indeterminately Defined
%% $~\theta~ \gimel$ Function Prmitive 
%% Enables Some Axiom Systems
%% To
%% Appreciate Fragments of Their Own
%% Hilbert Consistency}
%% 
%% 







%%% 


%X%% \title{On  How the 
%X%% Ixntroducing of a 
%X%%  New $~\theta~$ Function Symbol
%X%% Into Arithmetic's Formalism Is 
%X%% %Likely 
%X%% Germane
%X%% to Devising Axiom Systems that Can 
%X%% Appreciate Fragments of Their Own
%X%% Hilbert Consistency}

%% \title{Why a Small Fragment of Hilbert's Consistency Program
%% Ought to Be Feasible
%% for Hilbert-like Deductive Methods
%% After A New $~\theta~$ Function Primitive
%% %AFTER A NEW ``$~\theta~$'' Function Primitive
%% Is Added to Arithmetic's Formalism}
%% 

%% \title{ A 2-Part Conjecture about
%% How  a Much-Diluted but Non-Trivial
%% %Variant
%% Fragment
%%  of
%% Hilbert's Consistency Program 
%% Is
%% Likely
%% %Plausible 
%% Feasible
%% for the
%% % Even the Challenging
%% Case of
%% Hilbert Deduction}



% 
% \title{ On
% How  a Much-Diluted but Non-Trivial
% %Variant
% Fragment
%  of
% Hilbert's Consistency Program 
% Is
% Likely
% %Plausible 
% Feasible
% for Even the 
% Challenging
% Case of
% Hilbert Deduction}
% 
% %Plausible and
% % Veru
% % Germane
% %to Computer Science}
% 
% 
% %%% \title{\large \bf On the Revival of a Much-Diluted 
% %%% but Non-Trivial
% %%% Version of
% %%% Hilbert's Consistency Program (And Its Applications
%%% for Computer Science, Mathematics and Philosophy)}




\def\beq{\begin{equation}}
\def\enq{\end{equation}}

\def\bel{\begin{lemma}}
\def\enl{\end{lemma}}


\def\bec{\begin{corollary}}
\def\enc{\end{corollary}}

\def\bed{\begin{description}}
\def\ennd{\end{description}}
\def\bee{\begin{enumerate}}
\def\ene{\end{enumerate}}


\def\bxbxd{\begin{definition}}
\def\bxbxdd{\begin{definition}}
\def\eedd{\end{definition}}
\def\bxbxdr{\begin{definition} \rm}
\def\bel{\begin{lemma}}
\def\enl{\end{lemma}}
\def\ent{\end{theorem}}



\author{  Dan E.Willard }
%Email = dew@cs.albany.edu.}}
%\newline
%Email = dan.willard.albany@gmail.com}}











%%%\date{Copyright 2012 by Dan E. Willard}

\date{State University of New York at Albany}

\maketitle

\setcounter{page}{0}
\thispagestyle{empty}

\normalsize




\baselineskip = 1.3\normalbaselineskip



\normalsize


\baselineskip = 1.0 \normalbaselineskip 
\def\bbint{\large \baselineskip = 1.6 \normalbaselineskip } 
\def\bbint{\large \baselineskip = 1.6 \normalbaselineskip }
\def\bbint{\normalsize \baselineskip = 1.3 \normalbaselineskip }



%%\baselineskip = 1.0 \normalbaselineskip 
%%\def\bbint{\large \baselineskip = 1.6 \normalbaselineskip } 
%%\def\bbint{\large \baselineskip = 1.6 \normalbaselineskip }
%%\def\bbint{\normalsize \baselineskip = 1.3 \normalbaselineskip }


\def\bbint{\normalsize \baselineskip = 1.27 \normalbaselineskip }



\def\bbint{\large \baselineskip = 2.0 \normalbaselineskip }


\def\bbint{\normalsize \baselineskip = 1.25 \normalbaselineskip }
\def\bbina{\normalsize \baselineskip = 1.24 \normalbaselineskip }



\def\bbint{\large \baselineskip = 2.0 \normalbaselineskip } 



\def\bbing{ }
\def\bbins{ }
\def\bbinm{ }

\def\bbint{\normalsize \baselineskip = 1.95 \normalbaselineskip } 



\def\bbing{ }
\def\bbins{ }
\def\bbinm{ }


\def\bbint{\large \baselineskip = 2.3 \normalbaselineskip } 
\def\bbing{ }
\def\bbins{ }
\def\bbinm{ }

\def\bbint{\normalsize \baselineskip = 1.7 \normalbaselineskip } 

\def\bbint{\large \baselineskip = 2.3 \normalbaselineskip } 
\def\bbinm{ \baselineskip = 1.18 \normalbaselineskip }


\def\bbint{\large \baselineskip = 2.0 \normalbaselineskip } 
\def\bbing{ }
\def\bbins{ }
\def\bbinm{ }
\def\bbinr{ }


\def\bbint{\normalsize \baselineskip = 1.25 \normalbaselineskip }
\def\bbina{\normalsize \baselineskip = 1.24 \normalbaselineskip }
\def\bbinr{ \baselineskip = 1.3 \normalbaselineskip }
\def\bbing{ \baselineskip = 1.28 \normalbaselineskip }
\def\bbins{ \baselineskip = 1.21 \normalbaselineskip }
\def\bbinm{  }

\def\ftl{ \baselineskip = 1.5 \normalbaselineskip }


\bbint

\parskip 5 pt




\noindent





\small


\baselineskip = 1.14 \normalbaselineskip 


 
\parskip 5pt

\baselineskip = 1.2 \normalbaselineskip 

%\setcounter{page}{0}



%%%%%%{\small








\large
\normalsize

 \baselineskip = 1.2 \normalbaselineskip 

%mmmmmmmmmmmmm

\begin{center}
\large
June 6, 2020
\end{center}

\begin{abstract}
\Large
\baselineskip = 1.65 \normalbaselineskip  
These notes are written quite informally
and they are meant to summarize the types of
probability distributions that will construct 
a decently 
``randomly''
formalized $\Theta$ function for my Cornell paper. This manuscript was
written in only a couple of hours time, and I therefore apologize
for many likely examples of carelessness in my QUITE INFORMAL extension
of \cite{ww16}'s results.
\end{abstract}









\def\ww22{\normalsize \baselineskip = 1.21\normalbaselineskip \parskip 4 pt}
\def\bb22{\normalsize \baselineskip = 1.19\normalbaselineskip \parskip 4 pt}
\def\zz22z{\normalsize \baselineskip = 1.19 \normalbaselineskip \parskip 3 pt}
\def\xx22{\normalsize \baselineskip = 1.17\normalbaselineskip \parskip 4 pt}
\def\vx22s{\normalsize \baselineskip = 1.16 \normalbaselineskip \parskip 3 pt} 
\def\vv22{\normalsize \baselineskip = 1.17 \normalbaselineskip \parskip 3 pt} 
\def\aa22{\normalsize \baselineskip = 1.15 \normalbaselineskip \parskip 3 pt} 
\def\g55{  \baselineskip = 1.0 \normalbaselineskip } 
\def\s55{ \baselineskip = 1.0 \normalbaselineskip } 
\def\sm55{ \baselineskip = 0.9 \normalbaselineskip } 
















\vspace*{- 1.0 em}


\def\waw11{\normalsize \baselineskip = 1.72\normalbaselineskip}
\def\waw11{\normalsize \baselineskip = 1.12\normalbaselineskip}
\def\waw11{\normalsize \baselineskip = 1.85\normalbaselineskip}



\def\waw11{\normalsize \baselineskip = 1.45\normalbaselineskip}


\def\waw11{\normalsize \baselineskip = 1.7\normalbaselineskip}

\def\waw11{\normalsize \baselineskip = 1.12\normalbaselineskip}


\def\g55{  \baselineskip = 1.50 \normalbaselineskip } 
\def\s55{ \baselineskip = 1.50 \normalbaselineskip } 
\def\sm55{ \baselineskip = 1.5 \normalbaselineskip } 


\def\g55{  \baselineskip = 1.50 \normalbaselineskip } 
\def\s55{ \baselineskip = 1.50 \normalbaselineskip } 
\def\sm55{ \baselineskip = 0.9 \normalbaselineskip } 






\def\aa22{\normalsize  \waw11 \parskip 6 pt} 
\def\bb22{\normalsize  \waw11 \parskip 5 pt}
\def\ww22{\normalsize \waw11 \parskip 4 pt}
\def\vv22{\normalsize  \waw11 \parskip 3 pt} 
\def\tt22{\normalsize  \waw11 \parskip 2 pt} 

\def\g55{  \baselineskip = 1.0 \normalbaselineskip } 
\def\b55{  \baselineskip = 1.0 \normalbaselineskip } 
\def\s55{ \baselineskip = 1.0 \normalbaselineskip } 
\def\sm55{ \baselineskip = 0.9 \normalbaselineskip } 






\def\mal{ \bf  }
\def\nal{\mathcal}

\def\cvrew{ \baselineskip = 1.6 \normalbaselineskip \parskip 3pt }

\def\ttt2c{ }
\def\tttc{ }

\def\tttc{\tiny \baselineskip = 0.8 \normalbaselineskip  \parskip 0pt }
\def\ttt2c{\tiny \baselineskip = 0.7 \normalbaselineskip  \parskip 0pt }
\def\tttc{ \baselineskip = 2.1 \normalbaselineskip  \parskip 5pt }
\def\ttt2c{ \baselineskip = 2.1 \normalbaselineskip  \parskip 5pt }

\def\tttc{ \baselineskip = 1.15 \normalbaselineskip  \parskip 5pt }
\def\ttt2c{ \baselineskip = 1.15 \normalbaselineskip  \parskip 5pt }


\def\tttc{ \baselineskip = 1.12 \normalbaselineskip  \parskip 4pt }
\def\ttt2c{ \baselineskip = 1.12 \normalbaselineskip  \parskip 4pt }


\def\tttc{ \baselineskip = 1.14 \normalbaselineskip  \parskip 3pt }
\def\ttt2c{ \baselineskip = 1.14 \normalbaselineskip  \parskip 4pt }

\def\cvt{ \baselineskip = 0.98 \normalbaselineskip }
\def\cv9{ \baselineskip = 0.99 \normalbaselineskip }
\def\cvs{ \baselineskip = 1.0 \normalbaselineskip }
\def\cvl{ \baselineskip = 1.0 \normalbaselineskip }
\def\cvh{ \baselineskip = 1.03 \normalbaselineskip }
\def\cvg{ \baselineskip = 1.00 \normalbaselineskip }


\def\cvt{ \baselineskip = 1.6 \normalbaselineskip }
\def\cv9{ \baselineskip = 1.6 \normalbaselineskip }
\def\cvs{ \baselineskip = 1.6 \normalbaselineskip }
\def\cvl{ \baselineskip = 1.6 \normalbaselineskip }
\def\cvh{ \baselineskip = 1.6 \normalbaselineskip }
\def\cvg{ \baselineskip = 1.6 \normalbaselineskip }
\def\cvb{ \baselineskip = 1.6 \normalbaselineskip }
\def\cvnew{ \baselineskip = 1.6 \normalbaselineskip }
\def\cvmew{ \baselineskip = 1.6 \normalbaselineskip }
\def\cvwew{ \baselineskip = 1.6 \normalbaselineskip \parskip 5pt }
\def\cvrew{ \baselineskip = 1.6 \normalbaselineskip \parskip 3pt }




\def\cvt{ \baselineskip = 1.22 \normalbaselineskip }
\def\cv9{ \baselineskip = 1.22 \normalbaselineskip }
\def\cvs{ \baselineskip = 1.22 \normalbaselineskip }
\def\cvl{ \baselineskip = 1.22 \normalbaselineskip }
\def\cvh{ \baselineskip = 1.22 \normalbaselineskip }
\def\cvg{ \baselineskip = 1.22 \normalbaselineskip }
\def\cvb{ \baselineskip = 1.22 \normalbaselineskip }
\def\cvnew{ \baselineskip = 1.4 \normalbaselineskip }
\def\cvmew{ \baselineskip = 1.35 \normalbaselineskip }
\def\cvwew{ \baselineskip = 1.4 \normalbaselineskip \parskip 5pt }
\def\cvrew{ \baselineskip = 1.22 \normalbaselineskip \parskip 3pt }


\def\cvt{ }
\def\cv9{ }
\def\cvs{ }
\def\cvl{ }
\def\cvh{ }
\def\cvg{ }
\def\cvb{ }
\def\cvnew{ } 
\def\cvmew{ }
\def\cvwew{ }
\def\cvrew{ }



\def\fend{ 

\medskip -------------------------------------------------------}


\def\g55{  \baselineskip = 1.0 \normalbaselineskip } 
\def\s55{ \baselineskip = 1.0 \normalbaselineskip } 
\def\sm55{ \baselineskip = 1.0 \normalbaselineskip } 
\def\h55{  \baselineskip = 1.08 \normalbaselineskip } 
\def\b55{  \baselineskip = 1.1 \normalbaselineskip } 

\normalsize

\baselineskip = 1.85 \normalbaselineskip 



%% Sleepy  

%\cvlpm %% Sleepy  
%\cvnew


%\small

%\parskip 0p

\parskip 2pt

\vspace*{- 1.0 em}

% \newpage

%\large

%\setcounter{page}{0}
\baselineskip = 1.04 \normalbaselineskip 
\parskip 2pt

\baselineskip = 0.96 \normalbaselineskip 
%\baselineskip = 0.90 \normalbaselineskip 

%\parskip 1pt
% 
\baselineskip = 2.16 \normalbaselineskip 
\baselineskip = 2.3 \normalbaselineskip 

\baselineskip = 0.95 \normalbaselineskip 
%\baselineskip = 0.95 \normalbaselineskip 
\baselineskip = 0.88 \normalbaselineskip 
\parskip 0pt
 


\noindent

% 
% 
% NNEW COMMENT
% 
% 
% The pdf version of this draft is verbatim identical to August's Version 3.
% The prior draft's abstract was incorrectly broadcast by Arxiv on the 
% Internet, after I pressed a wrong computer button. Thus, 
% Version 4 was issued.


\newpage

\def\gvs{ \normalsize \baselineskip = 1.4 \normalbaselineskip  \parskip    5pt}
\def\gvs{ \normalsize \baselineskip = 1.44 \normalbaselineskip  \parskip    5pt}
\def\gvs{ \large \baselineskip = 1.44 \normalbaselineskip  \parskip    5pt}
\def\gvs{ \normalsize \baselineskip = 1.44 \normalbaselineskip  \parskip    5pt}\def\gvs{ \normalsize \baselineskip = 1.74 \normalbaselineskip  \parskip    5pt}
\def\gvs{ \normalsize \baselineskip = 1.44 \normalbaselineskip  \parskip 5pt}

\def\gvs{   \baselineskip = 1.74 \normalbaselineskip  \parskip    5pt}

\def\gvs{ \normalsize \baselineskip = 1.44 \normalbaselineskip  \parskip 5pt}
\def\gvs{ \large \baselineskip = 2.0 \normalbaselineskip  \parskip 5pt}
\def\gvs{ \Large \baselineskip = 2.0 \normalbaselineskip  \parskip 5pt}
% \def\gvs{ \baselineskip = 2.0 \normalbaselineskip  \parskip 5pt}
\def\gvs{ \normalsize \baselineskip = 2.44 \normalbaselineskip  \parskip 5pt}
\def\gvs{ \normalsize \baselineskip = 2.04 \normalbaselineskip  \parskip 5pt}
\def\gvs{ \normalsize \baselineskip = 2.64 \normalbaselineskip  \parskip 5pt}
\def\gvs{ \Large \baselineskip = 1.6 \normalbaselineskip  \parskip 5pt}

%\def\gvs{ }


\gvs

\footnotesize


\def\gvs{ }
  

\normalsize \baselineskip = 0.98 \normalbaselineskip
\normalsize \baselineskip = 1.0 \normalbaselineskip
\normalsize \baselineskip = 1.01 \normalbaselineskip


\def\gvs{ \normalsize \baselineskip = 1.25 \normalbaselineskip  \parskip 4pt}

% \def\gvs{ \normalsize \baselineskip = 1.23 \normalbaselineskip  \parskip 5pt}

%\def\gvs{ \normalsize \baselineskip = 1.4  \normalbaselineskip  \parskip 6pt}

%\def\gvs{ \large \baselineskip = 1.5  \normalbaselineskip  \parskip 6pt}

\def\gvs{ \Large \baselineskip = 1.6  \normalbaselineskip  \parskip 6pt}
\def\gvs{ \normalsize \baselineskip = 1.6  \normalbaselineskip  \parskip 6pt}
\def\gvs{ \large \baselineskip = 1.6  \normalbaselineskip  \parskip 6pt}

\def\gvs{ \normalsize \baselineskip = 1.227 \normalbaselineskip  \parskip 3pt}
\def\gvs{ \large \baselineskip = 1.8  \normalbaselineskip  \parskip 6pt}

\def\gvs{ \normalsize \baselineskip = 1.5 \normalbaselineskip  \parskip 3pt}

% \def\gvs{ \large \baselineskip = 1.8  \normalbaselineskip  \parskip 6pt}

% \def\gvs{ \normalsize \baselineskip = 1.65 \normalbaselineskip  \parskip 3pt}

\def\gvs{ \large \baselineskip = 2.1  \normalbaselineskip  \parskip 6pt}

\def\gvs{ \normalsize \baselineskip = 2.1  \normalbaselineskip  \parskip 6pt}

%%%old 

 \def\gvs{ \normalsize \baselineskip = 1.227 \normalbaselineskip  \parskip 3pt}

 \def\gvs{ \large  \baselineskip = 1.6 \normalbaselineskip  \parskip 5pt}
%% march 31

\def\gvs{ \Large  \baselineskip = 1.8 \normalbaselineskip  \parskip 5pt}
\def\gvs{ \LARGE  \baselineskip = 1.8 \normalbaselineskip  \parskip 5pt}
\def\gvs{ \normalsize  \baselineskip = 2.0 \normalbaselineskip  \parskip 5pt}

\def\gvs{ \Large  \baselineskip = 2.0 \normalbaselineskip  \parskip 5pt}

\def\gvs{ \large  \baselineskip = 2.2 \normalbaselineskip  \parskip 5pt}

\def\gvs{ \normalsize \baselineskip = 2.4  \normalbaselineskip  \parskip 6pt}

\def\gvs{ \normalsize \baselineskip = 2.6  \normalbaselineskip  \parskip 6pt}
\def\gvs{ \normalsize \baselineskip = 2.2  \normalbaselineskip  \parskip 6pt}
\def\gvs{ \normalsize \baselineskip = 1.8  \normalbaselineskip  \parskip 5pt}
% \def\gvs{ \normalsize \baselineskip = 1.5  \normalbaselineskip  \parskip 5pt}

\def\sgvs{ \small \baselineskip = 1.33  \normalbaselineskip  \parskip 1pt}
\def\tttc{ }
%\baselineskip = 1.14 \normalbaselineskip  \parskip 4pt }
\def\ttt2c{ }
%\baselineskip = 1.14 \normalbaselineskip  \parskip 4pt }

\def\gv2{ \normalsize \baselineskip = 1.30  \normalbaselineskip  \parskip 3pt}



\def\gvs{ }


\def\gvs{ \normalsize \baselineskip = 2.1 \normalbaselineskip  \parskip 7pt}
\def\gvs{ \normalsize \baselineskip = 1.8 \normalbaselineskip  \parskip    7pt}

%\def\gvs{ \normalsize \baselineskip = 1.4 \normalbaselineskip  \parskip    5pt}

 \def\gvs{ \large \baselineskip = 1.7  \normalbaselineskip  \parskip 9pt}
\def\gvs{ \normalsize \baselineskip = 2.0  \normalbaselineskip  \parskip 9pt}

\def\gv2{ \normalsize \baselineskip = 1.30  \normalbaselineskip  \parskip 3pt}


\def\gvs{ \large \baselineskip = 1.7  \normalbaselineskip  \parskip 5pt}


\def\gvs{ \normalsize \baselineskip = 2.0  \normalbaselineskip  \parskip 8pt}
\def\gvs{ \Large \baselineskip = 2.5  \normalbaselineskip  \parskip 8pt}


%fffff


%fffff
\def\gvs{ \normalsize \baselineskip = 1.3 \normalbaselineskip  \parskip   5pt}

\def\gvx{ \normalsize \baselineskip = 1.23 \normalbaselineskip  \parskip    3pt}

\def\gvs{ \Large \baselineskip = 1.9 \normalbaselineskip  \parskip    5pt}

\section{Crudely Composed Notes}
% \section{Scientific Notes of Dan Willard Notarized on Nov 22,2016}
%%%%%%%%%% 1111111111111111}
\label{ss1}





\gvs

I will use the phrase ``probability distribution'' quite informally
in our discussion.  The goal is thus to ``invent'' {\it any type}
of Lebesgue measure that will assure that there is a probability
bounded below by some tiny constant $~c~$ where
\cite{ww16}'s $\Theta$ function will produce a consistent self-justifying
formalism.  {\it Even if that probability is tiny,} any discovered 
lower bound  $~c~> ~ 0 ~$, 
will assure that the  IQFS($\beta$)
formalism is consistent when $~\beta~$ 
holds true under the Standard Model. This is 
 because
IQFS($\beta$)'s
 Group 0, 1 and 2 axioms trivially hold true under the standard model
and its final Group-3 axiom sentence cannot be proven false 
when our probability framework can generate a model
where it holds with 
an explicit
probability lower bound lower bound of  $~c~> ~ 0 ~$.

(In other words since ZF Set Theory can formalize the validity of
G\"{o}del's Completeness  Theorem, the existence of some model
satisfying a  probability lower bound lower bound  $~c~> ~ 0 ~$
will be sufficient for establishing that
the
 Group-3 axiom's 
self-justification statement will not be contradicted.)


Please allow me to be informal here because I am trying to
quickly
 compose
a rough approximation of working notes, without delving into 
a bevy of 
tedious details.  

Let us recall that page 11 of \cite{ww16}
defines the $\zzthe(x)$ function-mapping  
%% haphazard
to be an
operation
 that maps powers of 2
onto powers of 2
subject to the following rules:

\vspace*{- 0.6 em}
{\parskip -6 pt
\beq
\label{walk1}
\forall ~~x~~~~~ \mbox{Power}(x) ~~~ \Rightarrow  ~~
\mbox{Power}(~ \zzthe(x)~)
\enq
\beq
\label{walk2}
\forall ~~x~~~~~ \zzthe(x)~ \neq ~ 1
\enq
\beq
\label{walk3}
\forall ~~x~~~ \forall ~~y ~~~~[~ x ~ \neq~ y ~
\wedge ~\mbox{Power}(x)~]
 \Rightarrow ~~ 
\zzthe(x)~ \neq ~\zzthe(y)
\enq
\beq
\label{walk4}
\forall ~~x~~~~~ \neg ~ ~\mbox{Power}(x)~~~~ \Rightarrow ~~~~ 
\zzthe(x)~=~0
\enq}
We want our probability
 distribution to have the property
that any recursively defined function has a zero
 probability of
occurring.  Thus the countable set of all recursively defined functions
will also have a probability zero of occurring.   BUT YET 
the $\zzthe(x)$ primitive will grow at a slow enough rate
that it is incapable of producing a fatal diagonalizing contradiction,
while satisfying the crucial constraints in Lines  \eq{walk1}-  \eq{walk3}.

I can immediately think of three likely ways of doing this.
{\it 
It is (?)  possible all three 
methods  will
work,
successfully.} Among the three plausible methods,
Method A is the simplest procedure, and Method C is the most
complicated. The virtue of Method C is that it is the one that
I am most confident about, although its procedure is a complex
hybrid of methods A and B.

All three of these 
randomized
methods will first generate the value of  
$ ~\zzthe(1)~$, and then calculate in chronological
oder the values of 
 $ ~\zzthe(2)~$,  $ ~\zzthe(4)~$,  $ ~\zzthe(8)~$ etc., 
This chronological order is important because
once  $ ~\zzthe(2^i)~$ 
is assigned a values of  $~2^K~$
then all  $~ j \, > \, i~$ are forbidden by rule 3
from mapping    $ ~\zzthe(2^j)~$ onto $~2^K~$.
This Rule 3 will be called the {\bf Exclusion Principle}
during our discussion of Methods A-C below:


\subsection{Method A: The ``Pairing Method''}

The Pairing method will begin by finding the two smallest
powers of 2, 
at least as large as 2,  where no $ j<i $ has  $ ~\zzthe(2^j)~$ 
correspond to  one of these two powers of 2,
which we write as $~2^{K_1}$ and 
 $~2^{K_2}$. It will then set    $ ~\zzthe(2^i)~$
 equal to one
of these two values,
 with each quantity receiving a 50 \%
probability of occurring. 

It would not surprise me if this Pairing rule (by itself)
would 
be sufficient to produce our
 needed final result.  It might, however,
be problematic because it would cause   $ ~\zzthe(2^i)~$
to always satisfy the following
possibly excessively
 tight constraint of:
\beq
\label{tight}
\forall ~~x~~~~~ \zzthe(x)
  ~ \leq ~ 4 ~ x^2
\enq

\subsection{Method B: The Almost-Stochastic Independence Method}

This method will assume that we have an infinite set of
real numbers greater than zero, $~p_1~,~p_2~,~p_3~,... $
such that $~p_j~$~ is the probability that 
$~ \zzthe(2^0)~=~ 2^j$.  (
The easiest example
\footnote{ \normalsize If
 $~p_j~=~ 2^{-j~} $  then
$\sum ~p_j~~=~~ \frac{1}{2}~+~  \frac{1}{4}~+~  \frac{1}{8}~+~...~~~ =~~~1.$}  
 of this
is when $~p_j~=~ 2^{-j~} $.) 
 Then for each 
subsequent 
power of 2, denoted as $2^L$, the same probability distribution
will be used, except that the  ``Exclusion Principle''
will be followed in precluding
repetitions from
occurring.
  Thus, let  the prior  powers of
2 be mapped onto the quantities of
 $2^{K_0}$   $2^{K_1}$, ...   $2^{K_{L-1}}$ 
and let $~S~$ be defined as below:
\beq
\label{s-sum}
S~~=~~ \sum_{i=0}^{L-1}{ p_{k_i}}
%
%%% p_{ \small K_0}~+~ p_{ \small K_1}~+~ p_{ \small K_2}~+~.... ~+~
%%%% p_{ \small K_{L-1}}
\enq
Then assuming $~2^j$~ is not one of the previously 
taken
       positions of
 $2^{K_0}$   $2^{K_1}$, ...   $2^{K_{L-1}}$, the method B will
have 
hold
$~ \zzthe(2^L)~=~ 2^j$ with an obviously adjusted  probability
of 
\beq
\label{adj-prob}
\frac{p_j}{1-S}
\enq
\subsection{Hybridizations of Methods A and B}
Again, we would not be surprised if either Methods A or B
(or both)
 would 
establish the conjectured self-justification property.
However, there are also 
other
plausible hybrid methods
that would establish this  property.
Here are two examples
\bed
\item{$~~  Method-C-1 $}. 
We follow Method A's approach with probability 0.5 and
method B's approach with probability 0.5.  
Thus, either of Method A's two choices occur with a 
precise
probability
        of
$~\frac{1}{4}~,~$ and Method B's setting
of 
$~ \zzthe(2^L)~=~ 2^j$ occurs with an exact  probability of:
\beq
\label{adj-prob1}
\frac{p_j}{2~(1-S)}
\enq
\item{$~~  Method-C-2 $}. 
We follow Method A's approach with probability of $~S~$ and
method B's approach with probability $~1-S~$.
Thus, either of Method A's two choices will hold with probability of
$~\frac{S}{2}~$ and Method B's setting of
$~ \zzthe(2^L)~=~ 2^J$ occurs with a  probability of
exactly $~p_j~$.
\end{description}

\section{Basic Strategy}

The basic strategy is to show that  if
axiom basis
$~\beta~$ holds valid in the standard model
then it is impossible for a minimal proof of $0=1$
to exist, Providing, just an overview,
this should be able to be proved,
{\it roughly,}
 by contradiction.

In particular, if $~\beta~$ holds valid in the standard model
then all the Group 0, 1 and 2 axioms will hold valid under the
Standard Model.   Hence
if the consistency preservation property fails, then
the Group-3
 {\it ``I am consistent"  axiom}
must be false under the standard model.
Thus, the proof of the falsification of the
 Group-3
 axiom
 must be able to self-reference itself
and
thereby
 establish its own incorrectness.

At this juncture, one must use a natural encoding for the
G\"{o}del 
numbers of a proof
(such as what was formally given in 
\cite{ww1,wwapal}'s
examples of G\"{o}del encodings).
  The proof of the existence of such a proof
will exceed the proof's length by a factor $\lambda > 4$
(or more)
 under all
natural encodings of proofs
(including the examples given in 
\cite{ww1,wwapal}).
 But the point  
is that our formalism
{\it  has only the} $\Theta$ operation as
a primitive,
for
 representing growth.
Thus, the natural probability
distributions from the prior section should establish something
to the effect that there is a
probability  lower 
bound
of
 $c > 0$,
such  that an
excessively fast
 growth-rate will be impossible.  

Thus leaving aside many
messy details, this lower bound will assure that there is stochastic
model where growth is precluded at a fast enough rate for some
demonstrated model to form the type of counterexample
that G\"{o}del's Completeness Theorem needs to show that the
 {\it ``I am consistent"  axiom} needs for corroborating its claim
for self-justification.

This result differs from my earlier work in that it needs ZF Set Theory
(rather than Peano Arithmetic) to corroborate what I call the 
Consistency Preservation Property.  That is fine and legal
because the system IQFS($\beta$) 
affirms its own consistency via
a 1-sentence axiom.
Thus, ZF Set Theory's knowledge about
Lebesgue measures 
should, likely,
indicate  IQFS($\beta$) 
is a competent
enough
 formalism to make no false claims.

\newpage
My apologies that the preceeding
short summary
 is not a formal proof.
It merely outlines,
{\it very roughly},
 what I have in mind.



\begin{thebibliography}{99}



 \normalsize
\parskip 5 pt
\baselineskip =  1.3 \normalbaselineskip



\bibitem{ww1}
Willard, D. E.:
 ``Self-verifying  systems, the incompleteness
theorem and the tangibiltiy reflection
principle'', in
{\it Journal of Symbolic Logic}
$~66~ (2001)\,$ pp. 536-596.
(Unlike our later articles, \cite{ww1,wwapal}
spends some time explaining how we generate
our analogs of
 G\"{o}del numbers
 for itemized sentences and proofs.)

\bibitem{wwapal}
Willard, D. E.:
``A generalization of the second incompleteness 
theorem and some exceptions to it''.
{\it Annals of Pure and Applied Logic}
141 (2006)
pp. 472-496.
(Unlike our later articles, \cite{ww1,wwapal}
spends some time explining how we generate
our analogs of
 G\"{o}del numbers
 for itemized sentences and proofs.)



\bibitem{ww16}
Willard, D. E.:
On  how the introducing of a 
 new $~\theta~$ function symbol
into arithmetic's formalism is 
germane
to devising axiom systems that can 
appreciate fragments of their own
Hilbert consistency.
{\it
Cornell Archives arXiv Report}      
1612.08071v5
         (2017).


\bibitem{ww20}
Willard, D. E.:
``On the Tender Line
Separating Generalizations and Boundary-Case Exceptions for the
Second Incompleteness Theorem under Semantic Tableaux
Deduction'',
a talk given 
on January 7 at the LFCS 2020 conference.
Early version in 
Volume 11972 of
 Springer's LNCS series, and longer version sent to
Cameron and Nate.


\end{thebibliography}
\end{document}


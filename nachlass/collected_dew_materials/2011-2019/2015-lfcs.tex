%%  2015 sept 9 after submission CHANGES ONE SENTENCE IN EXAMPLE 2.5

%% 2015 sept 7  2.20 pm (after finding addrees)
%% after reding conclusion to BOB
% chipped off end


%% www.cs.albany.edu/~dew/algor


%% 2015 home august 24 11.2  am 

%% 2015 home august 22 1.1  pm (single space)


 % 2015 july 4 3.4 am after spell 10.1 am after sinatra

 % 2015 july 2 3.15 pm 

%% 2015 july 2 2.50 pm upstairs 

%% 2015 july 1  10.30  am downstairs



%% notarized notes 2015 april 2 6.3  am april 4 notarize again 

%% home 2014 feb 8 1.15am (new email address)

%% home 2015 feb6 4.3 am   suny 2.40 pm home 6.15 pm


%% gmail dan.willard.albany and Prof.DanEdwardWillard
%% gmail password cpZ9ar48s


%%% SUNY JAN 11 Brad Copy 8.4  pm

%%  SUNY   jan11 5/30pm spell check

%% 2015 HOME jan 10  9.4  pm  pm New Abstratct

%% 512 6932

%% Towards a Restructuring of Hilbert's Consistency Program

% www.cs.albany.edu/~dew/algor/

%\documentclass[11pt]{article}
%\documentclass[10pt]{article}
% \documentclass[12pt]{article}
\documentclass[12pt]{article}


%%%%%%%%%% \documentstyle[11pt]{article}


\usepackage{amssymb}





% \addtolength{\oddsidemargin}{-0.95in}
 \addtolength{\oddsidemargin}{-0.9in}
%\addtolength{\oddsidemargin}{-1.0in}
% \addtolength{\oddsidemargin}{-0.95in}

\setlength{\textheight}{9.6 in}
\setlength{\textheight}{8.8 in}
\setlength{\textheight}{9.3 in}
\setlength{\textheight}{9.55 in}
%above too short


\setlength{\textwidth}{6.3 in}
%% PRINT

\setlength{\textwidth}{6.0 in}
\setlength{\textwidth}{5.4 in}


\setlength{\textwidth}{6.9 in}
%% \setlength{\textwidth}{6.7 in}

% \setlength{\textwidth}{6.3 in}


% \setlength{\textwidth}{7.0 in}


% \setlength{\textwidth}{7.0 in}
% Above IDeall


%% \setlength{\textwidth}{6.4 in}
%%%% above brad with 11 point

%\setlength{\textwidth}{6.0 in}
%\setlength{\textwidth}{5.7 in}

%\setlength{\textwidth}{6.4 in}

%\setlength{\textwidth}{5.5 in}

%\addtolength{\topmargin}{-1.0in}
\addtolength{\topmargin}{-0.95in}
%\addtolength{\topmargin}{-1.0in}
%\addtolength{\topmargin}{1.2in}

%\addtolength{\topmargin}{-.95in}
%\addtolength{\topmargin}{+.7in}
%%% delete above for pdf




          \newcommand{\newthmwithin}[3]{\newtheorem{#1q}{#2}[#3]
              \newenvironment{#1}{\begin{#1q}\sf}{\end{#1q}}}

\newcommand{\newthm}[3]{\newtheorem{#1q}[#2q]{#3}
                        \newenvironment{#1}{\begin{#1q}\sf}{\end{#1q}}}
\newcommand{\newthmm}[3]{\newtheorem{#1q}[#2q]{#3}
                        \newenvironment{#1}{\begin{#1q}\rm}}

\newtheorem{theorem}{$~~~~$ Theorem}[section]
% \newtheorem{corollary}{Corollary}[section]
%\newtheorem{fact}{Fact}[section]
\newcommand{\makenewheading}[1]{\begin{tabbing} {\bf #1:}
\end{tabbing}}

\newtheorem{example}[theorem]{$~~~~$ Example}
\newtheorem{themx}[theorem]{$~~~~$ Theorem}
\newtheorem{corollary}[theorem]{$~~~~$ Corollary}
\newtheorem{lemma}[theorem]{$~~~~$ Lemma}
\newtheorem{remark}[theorem]{$~~~~$Remark}
\newtheorem{definition}[theorem]{$~~~~$Definition}
\newtheorem{fact}[theorem]{$~~~~$Fact}


\newtheorem{dff}[theorem]{$~~~~$ Definition}
\newtheorem{exx}[theorem]{$~~~~$ Example}
\newtheorem{lemm}[theorem]{$~~~~$ Lemma}
\newtheorem{propp}[theorem]{$~~~~$ Proposition}
\newtheorem{remm}[theorem]{$~~~~$Remark}

\newtheorem{deff}[theorem]{$~~~~$Definition}





% \def\Box{ QED}
\def\nop{ }
\def\nxp{ }
\def\nxp{ Here $~$NXP }

\def\bigc{$\,$of the unabridged version of this paper \cite{ww12}}

\def\nor1{Normed$\{~2^{ \zzz \theta  \, )} ~$,$~\sqrt{~2^{ \zzz \theta  \, )}}~\}$}

\def\pag5{Page 5}
\def\pagxx{Page ?xx?}
\def\xor2{Normed$\{ ~\sqrt{~2^{ \zzz \theta  \, )}}~,~2~ \} $}
%\def\fffx{Fact \#}
\def\fffx{{\bf Fact *}}
\def\zhz{H }
%\def\fffx{Fact \#}
\def\appD{Appendix D }
\def\appxD{Appendix D}
\def\fffour{three }
\def\zazsta{ and EA-stability}

% \def\glamb{\xi}
\def\glamb{\lambda}
\def\glamb{P}
\def\glamb{\theta}
\def\pag2{Page 2}
%% \def\zzthe{\zeta}


\def\glamb{\zeta}
\def\zzthe{\theta}





\def\gggen{$( L^\xi  ,  \Delta_0^\xi   ,  B^\xi  ,  d  ,  G  )~$}
\def\gggcp{$( L^\xi  ,  \Delta_0^\xi   ,  B^\xi  ,  d  ,  G  )$}
\def\peta{\sigma}
\def\zzxz{~ \sharp ( \, }
\def\zzz{~ \sharp ( ~ }
\def\zip{\sharp }
\def\mheta{\theta^\bullet}
\def\mxi{\xi^\bullet}
%\def\mbi{\bullet}
\def\mbi{\bigdot}
\def\mbi{\bigoplus}
\def\xxi{$\, \xi^* \,$}




\def\tftt{~ \frac{1}{2}~ }
\def\sss{ }

\def\goodshit{\triangleright}
\def\bullshit{\triangleleft}
\def\foo{footnote \footnote}
\def\Uxp{\Upsilon}

\def\f55{ \normalsize  \baselineskip = 1.8 \normalbaselineskip } 


\def\f55{  \baselineskip = 1.1 \normalbaselineskip } 
\def\g55{  \baselineskip = 1.0 \normalbaselineskip } 
\def\s55{ \baselineskip = 1.0 \normalbaselineskip } 

\def\f55{  \baselineskip = 0.7 \normalbaselineskip } 



\def\bbskip{\bigskip}



\def\axst{\odot}
\def\rp{ p^\theta } 
\def\thsp{Theorem $ \, * \,$ }
\def\thss{Theorem $ \, *\,$'s }
\def\dexxt{\Delta}
\newcommand{\co}[1]{Corollary \ref{#1}}
\newcommand{\thx}[1]{Theorem \ref{#1}}
\newcommand{\phx}[1]{Proposition \ref{#1}}
 \newcommand{\dfx}[1]{Definition \ref{#1}}
\newcommand{\lem}[1]{Lemma \ref{#1}}



%% \newcommand{\co}[1]{Corollary \ref{#1}}
%%  \newcommand{\thx}[1]{Theorem \ref{#1}}
\newcommand{\lxem}[1]{Lemma \ref{#1}}
\newcommand{\overx}[1]{\, \overline{ {#1} } \,} 


\newcommand{\el}[1]{Line (\ref{#1})}
\newcommand{\ex}[1]{Expression (\ref{#1})}
\newcommand{\ei}[1]{item (\ref{#1})}

\newcommand{\eq}[1]{(\ref{#1})}
\newcommand{\ep}[1]{Equation (\ref{#1})}
\newcommand{\thetlam}{ \theta }
\newcommand{\underx}[1]{\overline{~ {#1} ~}}
\newcommand{\appaa}{$App \forall$}
\newcommand{\appee}{$App \exists$}
\newcommand{\tll}[1]{Tab$- {#1} -$List}
\newcommand{\txl}[1]{Tab$- {#1}$}
\newcommand{\tlxl}[1]{Tab$- {#1}$ }
\newcommand{\sll}[1]{Short$- {#1} -$List}
\newcommand{\axx}[1]{NS$_D^{\,k,m}( ${#1}$)$}


\newenvironment{proof}{{\bf Proof:}}{$\Box$}
\newenvironment{sketchproof}{{\bf Sketch of Proof:}}{$\Box$}


\begin{document}

%\title{Rough Summary of March 2012 Research Before I Attended
%the AMS March 17-18 Conference}


%% \title{ On
%% How the Revival of a Diluted 
%% Version of
%% Hilbert's Consistency Program 
%% %is 
%% Should
%% Likely
%%  Be
%% % Plausible and
%% % Veru
%%  Germane
%% to Computer Science}



% \title{ A 2-Part Conjecture about How  




\title{Why a Small Fragment of Hilbert's Consistency Program
Ought to Be Feasible
for Hilbert-like Deductive Methods
After
A New $~\theta~$ Function Primitive
%AFTER A NEW ``$~\theta~$'' Function Primitive
Is Added to
A New $~\theta~$ Function Primitive
Arithmetic's Formalism}


\title{Why a Small Fragment of Hilbert's Consistency Program
Ought to Be Feasible
for Hilbert-like Deductive Methods
After A New $~\theta~$ Function Primitive
%AFTER A NEW ``$~\theta~$'' Function Primitive
Is Added to Arithmetic's Formalism}


\title{On  How the Introducing of a 
 New $~\theta~$ Function Symbol
Into Arithmetic's Formalism Is Germane
to Devising Axiom Systems that Can 
Appreciate Fragments of Their Own
Hilbert Consistency}



%% 
%% \title{On the 
%% Likelihood
%% That a 
%% Curtailed but 
%% Well-Defined
%% Fragment
%%  of
%% Hilbert's Consistency Program 
%% Should be
%% Feasible
%% for the
%% Case of
%% Hilbert Deduction}


% \title{On the Almost-Certain Likelihood
% That a Sharply Curtailed but 
% Well-Defined
% %Significant
% %Variant
% Fragment
%  of
% Hilbert's Consistency Program 
% Ought to be
% %Plausible 
% Feasible
% for the
% % Even the Challenging
% Case of
% Hilbert Deduction}



% 
% \title{ On
% How  a Much-Diluted but Non-Trivial
% %Variant
% Fragment
%  of
% Hilbert's Consistency Program 
% Is
% Likely
% %Plausible 
% Feasible
% for Even the 
% Challenging
% Case of
% Hilbert Deduction}
% 
% %Plausible and
% % Veru
% % Germane
% %to Computer Science}
% 
% 
% %%% \title{\large \bf On the Revival of a Much-Diluted 
% %%% but Non-Trivial
% %%% Version of
% %%% Hilbert's Consistency Program (And Its Applications
%%% for Computer Science, Mathematics and Philosophy)}


% \title{\large \bf On the Revival of a Modified and Diluted Version of
% Hilbert's Consistency Program (Extended Abstract)} 



\def\beq{\begin{equation}}
\def\enq{\end{equation}}

\def\bel{\begin{lemma}}
\def\enl{\end{lemma}}


\def\bec{\begin{corollary}}
\def\enc{\end{corollary}}

\def\bed{\begin{description}}
\def\ennd{\end{description}}
\def\bee{\begin{enumerate}}
\def\ene{\end{enumerate}}


\def\bxbxd{\begin{definition}}
\def\bxbxdd{\begin{definition}}
\def\eedd{\end{definition}}
\def\bxbxdr{\begin{definition} \rm}
\def\bel{\begin{lemma}}
\def\enl{\end{lemma}}
\def\ent{\end{theorem}}



\author{  Dan E. Willard \thanks{This research 
was partially supported
by the NSF Grant CCR  0956495.
%Email = dew@cs.albany.edu.}}
%\newline
Email = dan.willard.albany@gmail.com}}











%%%\date{Copyright 2012 by Dan E. Willard}

\date{University at Albany}

\maketitle

\setcounter{page}{0}
\thispagestyle{empty}

\normalsize




\baselineskip = 1.3\normalbaselineskip



\normalsize


\baselineskip = 1.0 \normalbaselineskip 
\def\bbint{\large \baselineskip = 1.6 \normalbaselineskip } 
\def\bbint{\large \baselineskip = 1.6 \normalbaselineskip }
\def\bbint{\normalsize \baselineskip = 1.3 \normalbaselineskip }



%%\baselineskip = 1.0 \normalbaselineskip 
%%\def\bbint{\large \baselineskip = 1.6 \normalbaselineskip } 
%%\def\bbint{\large \baselineskip = 1.6 \normalbaselineskip }
%%\def\bbint{\normalsize \baselineskip = 1.3 \normalbaselineskip }


\def\bbint{\normalsize \baselineskip = 1.27 \normalbaselineskip }



\def\bbint{\large \baselineskip = 2.0 \normalbaselineskip }


\def\bbint{\normalsize \baselineskip = 1.25 \normalbaselineskip }
\def\bbina{\normalsize \baselineskip = 1.24 \normalbaselineskip }



\def\bbint{\large \baselineskip = 2.0 \normalbaselineskip } 



\def\bbing{ }
\def\bbins{ }
\def\bbinm{ }

\def\bbint{\normalsize \baselineskip = 1.95 \normalbaselineskip } 



\def\bbing{ }
\def\bbins{ }
\def\bbinm{ }


\def\bbint{\large \baselineskip = 2.3 \normalbaselineskip } 
\def\bbing{ }
\def\bbins{ }
\def\bbinm{ }

\def\bbint{\normalsize \baselineskip = 1.7 \normalbaselineskip } 

\def\bbint{\large \baselineskip = 2.3 \normalbaselineskip } 
\def\bbinm{ \baselineskip = 1.18 \normalbaselineskip }


\def\bbint{\large \baselineskip = 2.0 \normalbaselineskip } 
\def\bbing{ }
\def\bbins{ }
\def\bbinm{ }
\def\bbinr{ }


\def\bbint{\normalsize \baselineskip = 1.25 \normalbaselineskip }
\def\bbina{\normalsize \baselineskip = 1.24 \normalbaselineskip }
\def\bbinr{ \baselineskip = 1.3 \normalbaselineskip }
\def\bbing{ \baselineskip = 1.28 \normalbaselineskip }
\def\bbins{ \baselineskip = 1.21 \normalbaselineskip }
\def\bbinm{  }

\def\ftl{ \baselineskip = 1.5 \normalbaselineskip }


\bbint

\parskip 5 pt




\noindent





\small

%\begin{abstract
\baselineskip = 1.17 \normalbaselineskip 


 
\parskip 5pt

\baselineskip = 1.2 \normalbaselineskip 

%\setcounter{page}{0}



%%%%%%{\small








\large
\normalsize

 \baselineskip = 1.2 \normalbaselineskip 

%mmmmmmmmmmmmm


%  PERTECT TITLE ABOVE ALTHOUHG PERHAPS OLD MANUSCRIPT BETTEDR

\begin{abstract}

%\Large

% \baselineskip = 1.8 \normalbaselineskip 
%aaaaaaaaaaa

\large
\LARGE
 \normalsize


 It is known that the combined work of Pudl\'{a}k and Solovay
 \cite{Pu85,So94}, enhanced by some added techniques of Nelson and
 Wilkie-Paris \cite{Ne86,WP87}, implies no reasonable axiom system can verify
 its own Hilbert consistency, when it recognizes Successor as a total function
 and treats addition and multiplication as 3-way relations (as Example
 \ref{ex-2.3} will explain).  These considerations will lead us to examine
 unconventional axiomatizations for arithmetic that continue to view addition
 and multiplication as 3-way relations, but which replace the successor
 function symbol with an entirely new operator, called the ``$~\theta~$''
 primitive.

\medskip

%% This $~\theta~$ operator
%% will
%% allow us 
%% to encode any integer $~n~$ by a term $~T_n~$
%% whose length will exceed the $O(~$Log$~n~)$ length of a
%% binary encoding 
%% by
%% only  the
%% relatively
%%  small magnitudes formalized by
%% Proposition \ref{th-3.3}  and Remark \ref{rem-def-3.4}.
% Proposition 3.3  and Remark 3.6

It is likely that this paradigm can be combined 
with our prior results from \cite{wwapal} 
%%% 
%% with the prior results 
%% in our APAL 2006 paper
%% REMOVE NEXT lINE
to construct  axiom systems that are
seriously
diluted but
able to verify their Hilbert-style 
consistency
in some interesting fragmentary respects.

\end{abstract}

\bigskip
\bigskip
\bigskip
\LARGE

% ttttt THIS PAPER SHOULD BE MASTER DRAFDT for future articles.


\bigskip
\bigskip
\bigskip


\normalsize

{\bf Keywords:}
Bounded Arithmetic, 
G\"{o}del's Second Incompleteness Theorem, Hilbert's Second
Open Question,
Semantic Tableaux Deduction, 
and Hilbert
Deduction.



\bigskip


% {\bf Keywords:}
% G\"{o}del's Second Incompleteness Theorem, Consistency, Hilbert's Second
% Open Question,
% Hilbert-styled Deduction (and its Frege-like analogs).




% \bigskip
% 
% 
% 
% {\bf Mathematics Subject Classification:}
% 03B52; 03F25; 03F45; 03H13 
% 
% 
% 
% \bigskip
% \bigskip


 
% {\bf Please Cite this Paper as:}
% {\rm http://arxiv.org/abs/1108.6330}, 
%  appearing in Cornell Archives 








\def\ww22{\normalsize \baselineskip = 1.21\normalbaselineskip \parskip 4 pt}
\def\bb22{\normalsize \baselineskip = 1.19\normalbaselineskip \parskip 4 pt}
\def\zz22z{\normalsize \baselineskip = 1.19 \normalbaselineskip \parskip 3 pt}
\def\xx22{\normalsize \baselineskip = 1.17\normalbaselineskip \parskip 4 pt}
\def\vx22s{\normalsize \baselineskip = 1.16 \normalbaselineskip \parskip 3 pt} 
\def\vv22{\normalsize \baselineskip = 1.17 \normalbaselineskip \parskip 3 pt} 
\def\aa22{\normalsize \baselineskip = 1.15 \normalbaselineskip \parskip 3 pt} 
\def\g55{  \baselineskip = 1.0 \normalbaselineskip } 
\def\s55{ \baselineskip = 1.0 \normalbaselineskip } 
\def\sm55{ \baselineskip = 0.9 \normalbaselineskip } 
















\vspace*{- 1.0 em}


\def\waw11{\normalsize \baselineskip = 1.72\normalbaselineskip}
\def\waw11{\normalsize \baselineskip = 1.12\normalbaselineskip}
\def\waw11{\normalsize \baselineskip = 1.85\normalbaselineskip}



\def\waw11{\normalsize \baselineskip = 1.45\normalbaselineskip}


\def\waw11{\normalsize \baselineskip = 1.7\normalbaselineskip}

\def\waw11{\normalsize \baselineskip = 1.12\normalbaselineskip}


\def\g55{  \baselineskip = 1.50 \normalbaselineskip } 
\def\s55{ \baselineskip = 1.50 \normalbaselineskip } 
\def\sm55{ \baselineskip = 1.5 \normalbaselineskip } 


\def\g55{  \baselineskip = 1.50 \normalbaselineskip } 
\def\s55{ \baselineskip = 1.50 \normalbaselineskip } 
\def\sm55{ \baselineskip = 0.9 \normalbaselineskip } 






\def\aa22{\normalsize  \waw11 \parskip 6 pt} 
\def\bb22{\normalsize  \waw11 \parskip 5 pt}
\def\ww22{\normalsize \waw11 \parskip 4 pt}
\def\vv22{\normalsize  \waw11 \parskip 3 pt} 
\def\tt22{\normalsize  \waw11 \parskip 2 pt} 

\def\g55{  \baselineskip = 1.0 \normalbaselineskip } 
\def\b55{  \baselineskip = 1.0 \normalbaselineskip } 
\def\s55{ \baselineskip = 1.0 \normalbaselineskip } 
\def\sm55{ \baselineskip = 0.9 \normalbaselineskip } 






\def\mal{ \bf  }
\def\nal{\mathcal}

\def\cvrew{ \baselineskip = 1.6 \normalbaselineskip \parskip 3pt }


\def\cvt{ \baselineskip = 0.98 \normalbaselineskip }
\def\cv9{ \baselineskip = 0.99 \normalbaselineskip }
\def\cvs{ \baselineskip = 1.0 \normalbaselineskip }
\def\cvl{ \baselineskip = 1.0 \normalbaselineskip }
\def\cvh{ \baselineskip = 1.03 \normalbaselineskip }
\def\cvg{ \baselineskip = 1.00 \normalbaselineskip }


\def\cvt{ \baselineskip = 1.6 \normalbaselineskip }
\def\cv9{ \baselineskip = 1.6 \normalbaselineskip }
\def\cvs{ \baselineskip = 1.6 \normalbaselineskip }
\def\cvl{ \baselineskip = 1.6 \normalbaselineskip }
\def\cvh{ \baselineskip = 1.6 \normalbaselineskip }
\def\cvg{ \baselineskip = 1.6 \normalbaselineskip }
\def\cvb{ \baselineskip = 1.6 \normalbaselineskip }
\def\cvnew{ \baselineskip = 1.6 \normalbaselineskip }
\def\cvmew{ \baselineskip = 1.6 \normalbaselineskip }
\def\cvwew{ \baselineskip = 1.6 \normalbaselineskip \parskip 5pt }
\def\cvrew{ \baselineskip = 1.6 \normalbaselineskip \parskip 3pt }




\def\cvt{ \baselineskip = 1.22 \normalbaselineskip }
\def\cv9{ \baselineskip = 1.22 \normalbaselineskip }
\def\cvs{ \baselineskip = 1.22 \normalbaselineskip }
\def\cvl{ \baselineskip = 1.22 \normalbaselineskip }
\def\cvh{ \baselineskip = 1.22 \normalbaselineskip }
\def\cvg{ \baselineskip = 1.22 \normalbaselineskip }
\def\cvb{ \baselineskip = 1.22 \normalbaselineskip }
\def\cvnew{ \baselineskip = 1.4 \normalbaselineskip }
\def\cvmew{ \baselineskip = 1.35 \normalbaselineskip }
\def\cvwew{ \baselineskip = 1.4 \normalbaselineskip \parskip 5pt }
\def\cvrew{ \baselineskip = 1.22 \normalbaselineskip \parskip 3pt }


\def\cvt{ }
\def\cv9{ }
\def\cvs{ }
\def\cvl{ }
\def\cvh{ }
\def\cvg{ }
\def\cvb{ }
\def\cvnew{ } 
\def\cvmew{ }
\def\cvwew{ }
\def\cvrew{ }



\def\fend{ 

\medskip -------------------------------------------------------}


\def\g55{  \baselineskip = 1.0 \normalbaselineskip } 
\def\s55{ \baselineskip = 1.0 \normalbaselineskip } 
\def\sm55{ \baselineskip = 1.0 \normalbaselineskip } 
\def\h55{  \baselineskip = 1.08 \normalbaselineskip } 
\def\b55{  \baselineskip = 1.1 \normalbaselineskip } 

\normalsize

%\begin{abstract}
\baselineskip = 1.85 \normalbaselineskip 



%% Sleepy  

%\cvlpm %% Sleepy  
%\cvnew


%\small

%\parskip 0p

\parskip 2pt

\vspace*{- 1.0 em}

\newpage

%\large

%\setcounter{page}{0}
\baselineskip = 1.04 \normalbaselineskip 
\parskip 2pt

\baselineskip = 0.96 \normalbaselineskip 
%\baselineskip = 0.90 \normalbaselineskip 

%\parskip 1pt
% 
\baselineskip = 2.16 \normalbaselineskip 
\baselineskip = 2.3 \normalbaselineskip 

\baselineskip = 0.95 \normalbaselineskip 
%\baselineskip = 0.95 \normalbaselineskip 
\baselineskip = 0.88 \normalbaselineskip 
\parskip 0pt
 

\def\gvxs{ }


\def\gvxs{ \baselineskip = 1.0 \normalbaselineskip  \parskip 2pt}
\def\gvxs{ \baselineskip = 2.0 \normalbaselineskip  \parskip 5pt}
\def\gvxs{ \baselineskip = 1.0 \normalbaselineskip  \parskip 0pt}

\def\gvxs{ \baselineskip = 1.6 \normalbaselineskip  \parskip 5pt}
\def\gvxs{ \Large \baselineskip = 1.6 \normalbaselineskip  \parskip 7pt}
\def\gvxs{ \large \baselineskip = 1.6 \normalbaselineskip  \parskip 6pt}
\def\gvxs{ \normalsize \baselineskip = 1.6 \normalbaselineskip  \parskip 6pt}


\def\gvxs{ \normalsize \baselineskip = 2.1 \normalbaselineskip  \parskip 7pt}
\def\gvxs{ \normalsize \baselineskip = 1.8 \normalbaselineskip  \parskip    7pt}

%\def\gvxs{ \normalsize \baselineskip = 1.4 \normalbaselineskip  \parskip    5pt}


\noindent


\newpage

\def\gvxs{ \normalsize \baselineskip = 1.4 \normalbaselineskip  \parskip    5pt}
\def\gvxs{ \normalsize \baselineskip = 1.44 \normalbaselineskip  \parskip    5pt}
\def\gvxs{ \large \baselineskip = 1.44 \normalbaselineskip  \parskip    5pt}
\def\gvxs{ \normalsize \baselineskip = 1.44 \normalbaselineskip  \parskip    5pt}\def\gvxs{ \normalsize \baselineskip = 1.74 \normalbaselineskip  \parskip    5pt}
\def\gvxs{ \normalsize \baselineskip = 1.44 \normalbaselineskip  \parskip 5pt}

\def\gvxs{   \baselineskip = 1.74 \normalbaselineskip  \parskip    5pt}

\def\gvxs{ \normalsize \baselineskip = 1.44 \normalbaselineskip  \parskip 5pt}
\def\gvxs{ \large \baselineskip = 2.0 \normalbaselineskip  \parskip 5pt}
\def\gvxs{ \Large \baselineskip = 2.0 \normalbaselineskip  \parskip 5pt}
% \def\gvxs{ \baselineskip = 2.0 \normalbaselineskip  \parskip 5pt}
\def\gvxs{ \normalsize \baselineskip = 2.44 \normalbaselineskip  \parskip 5pt}
\def\gvxs{ \normalsize \baselineskip = 2.04 \normalbaselineskip  \parskip 5pt}
\def\gvxs{ \normalsize \baselineskip = 2.64 \normalbaselineskip  \parskip 5pt}
\def\gvxs{ \Large \baselineskip = 1.6 \normalbaselineskip  \parskip 5pt}

%\def\gvxs{ }


\gvxs

\footnotesize


\def\gvxs{ }
  

\normalsize \baselineskip = 0.98 \normalbaselineskip
\normalsize \baselineskip = 1.0 \normalbaselineskip
\normalsize \baselineskip = 1.01 \normalbaselineskip


\def\gvxs{ \normalsize \baselineskip = 1.25 \normalbaselineskip  \parskip 4pt}

% \def\gvxs{ \normalsize \baselineskip = 1.23 \normalbaselineskip  \parskip 5pt}

%\def\gvxs{ \normalsize \baselineskip = 1.4  \normalbaselineskip  \parskip 6pt}

%\def\gvxs{ \large \baselineskip = 1.5  \normalbaselineskip  \parskip 6pt}

\def\gvxs{ \Large \baselineskip = 1.6  \normalbaselineskip  \parskip 6pt}
\def\gvxs{ \normalsize \baselineskip = 1.6  \normalbaselineskip  \parskip 6pt}
\def\gvxs{ \large \baselineskip = 1.6  \normalbaselineskip  \parskip 6pt}

\def\gvxs{ \normalsize \baselineskip = 1.227 \normalbaselineskip  \parskip 3pt}
\def\gvxs{ \large \baselineskip = 1.8  \normalbaselineskip  \parskip 6pt}

\def\gvxs{ \normalsize \baselineskip = 1.5 \normalbaselineskip  \parskip 3pt}

% \def\gvxs{ \large \baselineskip = 1.8  \normalbaselineskip  \parskip 6pt}

% \def\gvxs{ \normalsize \baselineskip = 1.65 \normalbaselineskip  \parskip 3pt}

\def\gvxs{ \large \baselineskip = 2.1  \normalbaselineskip  \parskip 6pt}

\def\gvxs{ \normalsize \baselineskip = 2.1  \normalbaselineskip  \parskip 6pt}

 \def\gvxs{ \normalsize \baselineskip = 1.227 \normalbaselineskip  \parskip 3pt}

% \def\gvxs{ \normalsize \baselineskip = 2.4  \normalbaselineskip  \parskip 6pt}



 \def\gvxs{ \Large \baselineskip = 2.15 \normalbaselineskip  \parskip 2pt}
 \def\lvxs{ \Large \baselineskip = 2.18 \normalbaselineskip  \parskip 2pt}
 \def\svxs{ \Large \baselineskip = 2.11 \normalbaselineskip  \parskip - 2pt}
\def\hvxs{ \Large \baselineskip = 2.18 \normalbaselineskip  \parskip 3pt}










 \def\gvxs{ \large \baselineskip = 2.75 \normalbaselineskip  \parskip 2pt}
 \def\lvxs{ \large \baselineskip = 2.78 \normalbaselineskip  \parskip 2pt}
 \def\svxs{ \large \baselineskip = 2.71 \normalbaselineskip  \parskip - 2pt}
\def\hvxs{ \large \baselineskip = 2.78 \normalbaselineskip  \parskip 3pt}


 \def\fvxs{ \normalsize \baselineskip = 1,46 \normalbaselineskip  \parskip 2pt}
 \def\fvxs{ \normalsize \baselineskip = 1,47 \normalbaselineskip  \parskip 2pt}
 \def\gvxs{ \normalsize \baselineskip = 1,47 \normalbaselineskip  \parskip 2pt}
 \def\lvxs{ \normalsize \baselineskip = 1,48 \normalbaselineskip  \parskip 2pt}
 \def\svxs{ \normalsize \baselineskip = 1,41 \normalbaselineskip  \parskip - 2pt}
 % \def\svxs{ }
\def\hvxs{ \normalsize \baselineskip = 1,48 \normalbaselineskip  \parskip 3pt}
\def\bvxs{ \normalsize \baselineskip = 1.5 \normalbaselineskip  \parskip 1pt}



 \def\fvxs{ \normalsize \baselineskip = 1,66 \normalbaselineskip  \parskip 2pt}
 \def\fvxs{ \normalsize \baselineskip = 1,67 \normalbaselineskip  \parskip 2pt}
 \def\gvxs{ \normalsize \baselineskip = 1,67 \normalbaselineskip  \parskip 2pt}
 \def\lvxs{ \normalsize \baselineskip = 1,68 \normalbaselineskip  \parskip 2pt}
 \def\svxs{ \normalsize \baselineskip = 1,61 \normalbaselineskip  \parskip - 2pt}
 % \def\svxs{ }
\def\hvxs{ \normalsize \baselineskip = 1,68 \normalbaselineskip  \parskip 3pt}
\def\bvxs{ \normalsize \baselineskip = 1.6 \normalbaselineskip  \parskip 1pt}



 \def\fvxs{ \normalsize \baselineskip = 0.96 \normalbaselineskip  \parskip 2pt}
 \def\fvxs{ \normalsize \baselineskip = 0.97 \normalbaselineskip  \parskip 2pt}
 \def\gvxs{ \normalsize \baselineskip = 0.97 \normalbaselineskip  \parskip 2pt}
  \def\svxs{ \normalsize \baselineskip = 0.91 \normalbaselineskip  \parskip - 2pt}
 % \def\svxs{ }
\def\rvxs{ \normalsize \baselineskip = 0.98 \normalbaselineskip  \parskip 3pt}
\def\hvxs{ \normalsize \baselineskip = 0.98 \normalbaselineskip  \parskip 3pt}
\def\bvxs{ \normalsize \baselineskip = 1.0 \normalbaselineskip  \parskip 3pt}
\def\nvxs{ \normalsize \baselineskip = 1.0 \normalbaselineskip  \parskip 2pt}

\def\hvxs{ \normalsize \baselineskip = 1.35 \normalbaselineskip  \parskip 3pt}
\def\bvxs{ \normalsize \baselineskip = 1.35 \normalbaselineskip  \parskip 3pt}
\def\nvxs{ \normalsize \baselineskip = 1.35 \normalbaselineskip  \parskip 2pt}


 % \def\svxs{ }

%moving
\def\rvxs{ \normalsize \baselineskip = 0.98 \normalbaselineskip  \parskip 3pt}
\def\lvxs{ \normalsize \baselineskip = 0.99 \normalbaselineskip  \parskip 2pt}
\def\hvxs{ \normalsize \baselineskip = 0.98 \normalbaselineskip  \parskip 3pt}
\def\bvxs{ \normalsize \baselineskip = 1.0 \normalbaselineskip  \parskip 3pt}
\def\nvxs{ \normalsize \baselineskip = 1.01 \normalbaselineskip  \parskip 2pt}
\def\tvxs{ \normalsize \baselineskip = 1.01 \normalbaselineskip  \parskip 2pt}

\def\rvxs{ \normalsize \baselineskip = 2.1 \normalbaselineskip  \parskip 3pt}
\def\lvxs{ \normalsize \baselineskip = 2.1 \normalbaselineskip  \parskip 3pt}
\def\hvxs{ \normalsize \baselineskip = 2.1 \normalbaselineskip  \parskip 3pt}
\def\bvxs{ \normalsize \baselineskip = 2.0 \normalbaselineskip  \parskip 3pt}
\def\nvxs{ \normalsize \baselineskip = 2.01 \normalbaselineskip  \parskip 2pt}
\def\tvxs{ \normalsize \baselineskip = 2.01 \normalbaselineskip  \parskip 2pt}

%moving
\def\rvxs{ \normalsize \baselineskip = 0.98 \normalbaselineskip  \parskip
  3pt}
\def\lvxs{ \normalsize \baselineskip = 0.99 \normalbaselineskip  \parskip 2pt}
\def\hvxs{ \normalsize \baselineskip = 0.98 \normalbaselineskip  \parskip 3pt}
\def\bvxs{ \normalsize \baselineskip = 1.0 \normalbaselineskip  \parskip 3pt}
\def\nvxs{ \normalsize \baselineskip = 1.01 \normalbaselineskip  \parskip 2pt}
\def\tvxs{ \normalsize \baselineskip = 1.01 \normalbaselineskip  \parskip 2pt}

\def\tempvxs{ \normalsize \baselineskip = 2.1 \normalbaselineskip  \parskip 3pt}

%\def\tempvxs{ }

\lvxs
\nvxs

%%% IGNORE BELOW:

%% NOTE TO ME: Page 5 is cited in paper. needs to  BE UPDATED
%%% AND prehaps GO BACK OLD FORM.


%%% iiiii
\section{Introduction}

\label{ss1}
\label{ss2}

\vspace*{- 0.5 em}

We have published a series of articles about generalizations
and boundary-case exceptions for the Second Incompleteness Theorem
in \cite{ww93}-\cite{ww14}. One theme of this literature was that
such boundary-case exceptions will arise when multiplication
is treated as a 3-way relation by a system which verifies its own consistency
%in a 
under
semantic tableaux deduction.
% instead of Hilbert deduction 
%%%%%%% context.
A 15-page summary of
this research  appeared 
in
\cite{ww14}, but the latter is 
not
% unnecessary to examine  as 
a prerequisite for reading this paper.


%%  our prior research 
%% about this topic
%% was provided in our
%% Wollic-2014 paper \cite{ww14}, but it is unnecessary for a reader
%% to examine \cite{ww14} as a prerequisite for this paper.

\smallskip

The main shortcoming in our prior research was that our formalisms
were mostly unable to recognize their own consistency under 
Hilbert-style deductive methods. This was because a version
of the Second Incompleteness Theorem, 
due to the
combined work of Pudl\'{a}k, Solovay, Nelson and Wilkie-Paris
\cite{Ne86,Pu85,So94,WP87}, indicated the mere assumption that the successor
operation was a total function was sufficient to trigger off the power
of the Second Incompleteness Theorem
for logics that verified their own consistency relative to Hilbert
deduction. In particular, only our paper \cite{wwapal}
% (which was 
%by coincidence invited by editor Sergey Artemov to appear in APAL)
was 
% exclusively 
devoted to addressing this problem.
Its ISCE formalism could verify its own Hilbert consistency, but it
unfortunately required deploying an infinite number of separate constant
symbols because the presence of even a successor function symbol
would trigger off the Second Incompleteness Effect for systems corroborating
their own Hilbert consistency.

The current article will introduce a modification of \cite{wwapal}'s
ISCE formalism, called IQFS, that 
addresses this
% should help partially resolve this
challenge. It will introduce a 
% new
function symbol, called the ``$~\theta~$''
primitive, that will enable an arithmetic to construct the infinite 
collection of 
% positive 
integers, 
{\it without using} counterparts of any of the
%  conventional growth 
% function 
% primitives of 
successor, addition or multiplication
function-mappings.

\medskip

It is almost certain that \cite{wwapal}'s proof of ISCE's
consistency preservation property will generalize for the
current article's IQFS formalism. 
We will provide no proof supporting our  conjecture in this
extended abstract, but the intuition 
behind it
% supporting our conjecture 
will become quite evident. 
Its gist will be that our proposed
new $~\theta~$
primitive  
%will be 
should likely be
germane to 
%{\it some special} 
{\it some unusual} 
axiom systems owning a
{\it small 
% quite
fragmentized}  knowledge about their own consistency.


%% 
%% The proof of 
%% IQFS's self-justifying properties will, 
%% however, be much longer
%% than
%% %
%%  \cite{wwapal}'s proof of its analogous Theorem 3
%% (if the latter result is proven with the same level of detail
%% and
%% %  100 \% 
%% rigor that Theorem 3's proof did receive).
%% We will sketch in this short conference abstract the intuitive
%% reason why IQFS should satisfy 
%% an analog of ISCE's consistency preservation property, but
%% % provide 
%% no formal proof
%% will be provided.

% a formal consistency preservation property, exactly analogous to ISCE.

%%  I do
%% not want to write up such a proof in a context where I am currently
%% suffering from Diabetes and Hypertension. It is essentially
%% 99 \%
%% certain that this article's proposed new 
%% $~\theta~$ operator should support our conjecture.
%% The purpose of this article will be to invite the
%% research community to investigate how one can 
%% expand the mini-formalism from 
%% \cite{wwapal}'s Theorem 3 to rigorously prove our
%% stated conjecture.

% 
% \section{More Detailed Description of Goals}

%\label{ss2}


% This article will define a new
%  ``$~\theta~$'' 
% function symbol
% that should enable unusual logics
% %% 
% %% % 5-10 \% 
% %% boundary-case
% %% effect where a 
% %% system can own
% %% 
% to own a
% {\it diluted but tangible} knowledge about 
% their own
% % its 
% %own 
% consistency.

\bigskip


%% {\bf More Detailed Description of Goals}
%% %
%% % It is  known the
%% %  
%% G\"{o}del's

%More precisely, 

During this article, we will often note
the 
Second Incompleteness Theorem 
was
%known to be
 published in two 
% quite 
different 
 forms during
1931-1939.
Its initial 1931 variant, formalized by Theorem XI
in  G\"{o}del's
% millineal 
paper \cite{Go31},
% 
% 
% Its Theorem XI,
% later known as the ``Second  Incompleteness
% Theorem'',
% %,
% %appearing in G\"{o}del's millennial paper \cite{Go31}. 
% 
demonstrated
% that
no extension
of
% axiom systems,
% roughly corresponding to
the
%  Russell-Whitehead 
Principia Mathematicae formalism
% $\, P \,$
could
% could
 verify
its own consistency.
The widely quoted more general
result, that 
every consistent r.e. 
extension
 of Peano Arithmetic must
be unable to prove a theorem affirming its
own consistency, 
was 
first
published 
%% 
%% (see \footnote{ Boolos states in \cite{Bool}
%% that it has been open to scholarly debate
%% whether or not the 1939
%% Hilbert-Bernays generalization of the Second Incompleteness Theorem
%% is or (is not) a straightforward generalization of
%% G\"{o}del's initial result} ) 
%% 
in the 1939 edition of
the Hilbert-Bernays
textbook \cite{HB39}.

% It has been considered
% to be the definitive demonstration of the broad reach of
% the Second Incompleteness Effect.

%%  
%% It also established, beyond any reasonable doubt, that any type
%% of formalism possessing a conventional knowledge of its own consistency,
%% must rely upon a 
%% foundational structure
%%  fundamentally different from Peano Arithmetic. 
%% (This is because the 
%% Hilbert-Bernays 
%% textbook formalized the forerunner of
%% what has now been known as the 
%% Hilbert-Bernays Derivability Conditions \cite{HB39,HP91,Lo55,Mend},
%% as a mechanism for
%% % foreseeing 
%% envisioning
%% the 
%% astonishing
%% broad generality of the
%% Second Incompleteness Effect.) 
 

It is, thus, fascinating that Hilbert,
as the co-author of 
% an important 
a
%very 
% historic
generalization of the Second Incompleteness Theorem,
never withdrew the
% chose to  never fully withdraw his 
1926 justification
 \cite{Hil26}
for his consistency program:
\begin{quote}
\small
\baselineskip = 0.9 \normalbaselineskip 
$*~$~ 
{\it ``Where 
else 
would 
reliability and truth be found 
if  even mathematical thinking fails?   The definitive nature
of the infinite has become necessary,  not merely for the special
interests of individual sciences, but rather { for the
honor} of human understanding''}
% itself.''}
\end{quote}


%% Indeed, 
%% %Instead,
%% Hilbert 
%% always insisted  some
%% new formalism would revive his consistency program
%% and had its
%% motto
%% ({\it ``Wir m\"{u}ssen wissen,  $~$Wir werden wissen''} )  
%% engraved on his tombstone.


% Moreover, it 
% 
It 
is also 
known
\cite{Da97,Go5,Yo5}
that G\"{o}del
was
% also
doubtful about the generality of the Second Incompleteness
Theorem for at least two years after its publication.
He thus inserted the following
cautious caveat into
his famous
1931 
% millennial
paper  \cite{Go31}:
% whose closing section 
%%% 
%%% One of the closing paragraphs of
%%%  \cite{Go31} 
%%% thus
%%% included
%
%
%%% contained the following cautious disclaimer:
%caveat:
% \newpage
\begin{quote}
\small
\baselineskip = 0.9 \normalbaselineskip 
\it
$~**~~$ 
``It must be
% expressly 
noted that
Theorem XI
%'s incompleteness result
(e.g. the Second Incompleteness Theorem) 
represents no contradiction of the formalistic
standpoint of Hilbert. For this standpoint
presupposes only the existence of a consistency
proof by finite means, and  there might
conceivably be ... ''
%  finite proofs} which cannot
% be stated in P or in ...''
\end{quote}
The 
% above 1931
statement $**$ has 
had
%been subject to 
numerous
%many
different
interpretations
 \footnote{
 Some 
 scholars
 have interpreted
 $\,**\,$
 as
 %as, possibly,'
 anticipating
 % 
 %  attempts
 % to confirm Peano Arithmetic's consistency,
 % via 
 % either
 % 
 Gentzen's formalism or 
  G\"{o}del's Dialetica interpretation.}.
All
 G\"{o}del's 
biographers
\cite{Da97,Go5,Yo5}
%%%have
have noted
% 
% 
% 
% (with
% some 
% scholars
% viewing it as germane to
% Gentzen's formalism or 
%  G\"{o}del's Dialetica interpretation).
% 
% \gvxs
% \nvxs
% \bvxs
% %\parskip 1pt
% 
% \noindent
% The 
% % above 1931
% statement $**$ has 
% had
% %been subject to 
% numerous
% %many
% different
% interpretations
% (with
% some 
% scholars
% viewing it as germane to
% Gentzen's formalism or 
%  G\"{o}del's Dialetica interpretation).
% % 
% % \footnote{
% % Some 
% % scholars
% % have interpreted
% % $\,**\,$
% % as
% % %as, possibly,'
% % anticipating
% % % 
% % %  attempts
% % % to confirm Peano Arithmetic's consistency,
% % % via 
% % % either
% % % 
% % Gentzen's formalism or 
% %  G\"{o}del's Dialetica interpretation.}.
% % 
% All
%  G\"{o}del's 
% biographers
% \cite{Da97,Go5,Yo5}
% %%%have
% noted
% % 
% % 
% 
% 
%
his
% % that G\"{o}del's
initial intention
was
to
establish
%achieve
Hilbert's proposed objectives, before 
%he proved
proving
%proving
% G\"{o}del proved
a result 
% 
% however,
% %%%%%his
% G\"{o}del
% did  originally 
% seek 
% % goal was 
% to
% establish
% %achieve
% Hilbert's proposed objectives before 
% proving
% % G\"{o}del proved
% a result 
% 
leading
%that led 
in the opposite direction.
Yourgrau \cite{Yo5}
records
%furthermore,
 how
von Neumann 
% surprisingly
%did
{\it ``argued 
against G\"{o}del 
himself''}
in the early 1930's,
 about the definitive 
%%%  achievement of a'
 termination of Hilbert's
consistency program,  
which
{\it ``for several years''} after \cite{Go31}'s publication,
G\"{o}del 
{\it ``was cautious not to prejudge''}.
It is known 
 G\"{o}del 
began to 
more fully
endorse 
the Second Incompleteness
Theorem
during a 1933
%% Vienna
lecture  \cite{Go33},
and he 
% told biographers he
strongly
%completely 
% fully
embraced it
after learning about Turing's work
\cite{Tur36}.

\smallskip

Our research
in \cite{ww93}-\cite{ww14}
%has been 
has been
% is
related to issues 
%analogous
 similar 
to those
%that were 
raised by Hilbert and 
G\"{o}del
in
 statements  $*$ and $**$.
This is because it is counter-intuitive and awkward to
%presume that 
explain how
human beings can maintain the
psychological drive and 
needed energy-desire to cogitate, without
being stimulated by 
{\it some type (?)} of instinctive faith in their own
consistency. The new ``$~\theta~$'' primitive, introduced in the 
current article, will
further
 reinforce this perspective.

%  (under a definition of 
% % formal  consistency
% such
% % this concept 
% that is suitably 
% gentle and
% % delicate
% soft
%  to 
% be consistent with the 
% Incompleteness Theorem's requirements).

% preclude a violation of the 
% restrictions imposed by the Incompleteness Theorem).


\smallskip

We emphasize 
 that 
%our current 
the present
paper will differ from
all our prior research (except for \cite{wwapal}'s
trial-balloon result) 
{\it by changing the focus from
semantic tableaux deduction to a Hilbert-style
deductive methodology.}

\smallskip

%\parskip 0pt


%% 
%% Accordingly, our research in
%% \cite{ww93}-\cite{ww14} 
%% has explored both generalizations and
%% boundary-case exceptions of the Incompleteness Effect, so as
%% to determine what type of boundary-case evasions are permitted.
%% Our prior research in \cite{ww93}-\cite{ww14} 
%% had used mostly cut-free forms of deduction to
%% evade the 
%% restrictions imposed by the
%% Second Incompleteness Effect. The current article will instead
%% focus on more pristine Hilbert-Frege methods of deduction.
%% They are likely to support an evasion of the Second Incompleteness
%% Effect when our axiom systems replace the traditional  
%% growth properties of the addition, multiplication and successor 
%% function symbols with our new $~\theta~$ primitive.
%% 
%% \smallskip
%% 
%% The motivation for this replacement will be
%% explained during the next section of this article.
%% It is needed
%% essentially
%% because
%% a
%%  version of the Second Incompleteness Theorem,
%% due to the
%% combined work of Pudl\'{a}k, Solovay, Nelson and Wilkie-Paris
%% \cite{Ne86,Pu85,So94,WP87},
%% %will show 
%% demonstrates
%% that 
%% if an axiom system $~\alpha~$
%% proves any 
%% of \eq{totxtefs} - \eq{totxtefm}'s totality statements
%% then it is incapable of confirming its own consistency
%% under a Hilbert-style deductive method.
%% 
%% 
%% 
%% 
%% 
%% \smallskip

Our results will suggest  it is possible to obtain a
{\it part-way 5-10 \%
positive} interpretation
for what
Hilbert and G\"{o}del 
% advanced
 were seeking 
% a Consistency Program
to establish
%%  seeking to accomplish
% contemplating 
in their 
statements
$*$ and $**$, within a context where 
it is  known that the 
Second Incompleteness Effect precludes a full achievement of
%these 
Hilbert's
objectives
from ever transpiring.
The last five minutes of
a 
recent
 60-minute YouTube 
presentation
 by Harvey Friedman
\cite{Fr14},
entitled 
% the 
{\it ``The Blessing and Curse of Kurt  G\"{o}del''},
suggested that it is
%the themes of Hilbert and G\"{o}del's
%remarks $*$ and $**$ by indicating that 
%it is 
interesting to explore futuristic partial 
boundary-like evasions of the Second Incompleteness Theorem,
despite the stunning strength of 
G\"{o}del's result.
It is within this context where our proposed use of a new
$~\theta~$ primitive 
symbol to replace the growth properties
of the traditional addition, multiplication and successor function
symbols may be of potential interest.

\smallskip

The development of our $~\theta~$ primitive 
was 
% partially 
influenced by a private email
communication 
we
% had 
received
from  
Pavel Pudl\'{a}k \cite{Pupriv},
as \textsection \ref{ss4}
% \ref{ss3}  \& \ref{ss4} 
%shall 
will
explain.
We also emphasize that the 
%conventional 
usual
interpretation of
the
% Second 
Incompleteness Theorem, as precluding
Hilbert's Consistency Program from
% ever 
achieving its initially
specified objectives, is certainly correct.
{\it Our only caveat} is that 
some  
{\it very  tiny} 5-10 \%
% perhaps
{\it fragmentized part} 
of Hilbert's and G\"{o}del's aspirations in
$*$ and $**$ ought to be viable.

%fragment of its objectives ought to be viable.


%% the latter should not lead  one
%% to
%% ignoring the role that a
%% human's instinctive faith in his/her's internal
%% consistency 
%% %crucially stimulates and motivates
%% plays in stimulating and motivating
%% human cognition.

%% 
%% It is
%% from this 
%% special
%% perspective where our prior research and
%% new 
%% results 
%% %2-part conjecture 
%% will
%% % does 
%% suggest that 
%% an approximate
%% %at least a
%% 5-10 \%
%% fragment 
%% % fraction
%% of what Hilbert and G\"{o}del 
%% %suggested in 
%% had
%% sought
%% in $*$ and $**$ 
%% %could
%% should be
%%  plausibly 
%% %is
%% %% be formally 
%%  feasible.
%% 


%\gvxs

\vspace*{- 0.6 em}

\section{Starting Perspective}
%  222222}
\label{ss3}

\vspace*{- 0.6 em}

%%! 
%%! This article will be written in a style so that its
%%! overall theme (if not full details)
%%! should become 
%%! {\it  quickly} comprehensible to a reader who has
%%! examined 
%%! only
%%!  one of the
%%! % introductory 
%%! logic textbooks by say Enderton,
%%! Fitting, H\'{a}jek-Pudl\'{a}k,
%%!  Mendelson
%%! or Papadimitriou \cite{End,Fi96,HP91,Mend,Papa}.
%%! %% 
%%! %% We will rely mostly upon the
%%! %% precise
%%! %% deductive calculi notation employed
%%! %% in Section 2.4 of Enderton's textbook,
%%! %% but any of 
%%! %% the 
%%! %% similar
%%! %% Hilbert-style deductive calculi of 
%%! %%  H\'{a}jek-Pudl\'{a}k,
%%! %%  Mendelson
%%! %% or Papadimitriou \cite{HP91,Mend,Papa}.
%%! %% will also be suitable for achieving our results.
%%! 
%%! %% 
%%! %%  
%%! %% ( Papadimitriyou's textbook
%%! %% generously states it employs a deductive notation
%%! %% that 
%%! %% has
%%! %% stemmed from its predecessor in 
%%! %% Enderton's textbook.)
%%! 
%%! 



%% In order to make our 
%% results
%% %research 
%% apply to the 
%% formalism

It is helpful to employ a flexible vocabulary so
% that our
our 
results
will apply 
%research 
%to the 
%formalisms
to any of the
%% 
%% accessible to
%% some 
%% readers who are acquainted with only one of the 
%% 
textbook 
formalisms of
% settings outlined by 
%  
say 
Enderton,
Fitting, H\'{a}jek-Pudl\'{a}k,
or Mendelson
\cite{End,Fit,HP91,Mend}.
% 
% ,
% %% or
% %%  Papadimitriou 
% %% \cite{End,Fit,HP91,Mend,Papa},
% %% 
% %% 
% %the widest
% % possible  
% %audience, 
% it is 
% helpful
% % useful to 
% use
% %employ 
% a
% % very 
% flexible
%  vocabulary.
% %%
% %%that 
% %%allows a reader to 
% %%%quickly 
% %%%translate results 
% %%traverse
% %%from
% %%one textbook to another.
% % Therefore, let us define a
Let us 
% thereby
call an
%\newpage
%
%\noindent
ordered pair $(\alpha,d)$ a
    {\bf ``Generalized Arithmetic''}
% therefore
iff its 
% first and second 
two components 
%% 
%%  Each of the 
%% textbooks \cite{End,Fi96,Mend,Papa,HP96} have
%% employed
%% substantially
%%  different variants of Basis and 
%%  Deductive-Apparatus structures. 
%% 
are 
% described
formalized
% defined 
as below:
%% 
%% Their
%% definitions in
%%  Items (1) and (2)
%% %simple 
%% %%%definitions of these two notions
%% %given below,
%%     allow one to easily translate 
%% %theorems 
%% formalisms
%% % methodologies
%% from
%% one textbook 
%% % source
%% to another:
%% 
%% %\njp
%% % \newpage
%% \parskip 2pt
\bee
\item
The {\bf ``Axiom Basis''} $~\alpha~$ 
for an arbitrary arithmetic
shall be defined as
the set of 
{\it proper axioms} employed by the 
formalism  $(  \alpha  , d  )$.
\item
An arithmetic's {\bf ``Deductive Apparatus''} $~d~$ 
is defined as
the 
{\it combination} of its formal rules for inference
and 
%its
the
 built-in
 logical axioms ``$~L_d~$''
%  (that are 
% implicitly 
employed by these rules.
\ene

%%%\item
%%%The term {\bf ``Deductive Apparatus''} $~d~$ will
%%%refer to the 
%%%{\it combination} of the rules of inference
%%%used by an arithmetic and its
%%%the logical axioms ``$~L_d~$'' that 
%%%render meaning to
%%%%are an automatic part of
%%%$~d\,$'s machinery.


\begin{exx}
\label{ex-2.1}
%\label{ex-basis}
\rm
This notation 
allows one to
 conveniently separate  the logical axioms
$~L_d~,~$ associated
with  $(  \alpha  , d  )~$, from 
 $\, \alpha \,$'s
 ``basis axioms''.
%basis axioms
It also allows one to  isolate
and compare
%  , conveniently, 
various
apparatus techniques,
%technique,
% employed in the exact formalisms 
including the
 $~d_E~$,  
 $~d_M~$, 
 $~d_H~$,  
and  $~d_F~$
methods
%that we will now define:
defined below:
%% 
%%  Three
%% examples of this are illustrated below,
%% in a context where
%% are the deductive apparatus machineries defined
%% in  Enderton's, Mendelson's and Fitting's textbooks
%% \cite{End,Fi96,Mend}.
%% 
\bed
\item[   i. ]
The  $~d_E~$ apparatus,
formalized in 
\textsection
 2.4 of  Enderton's textbook, 
% will
uses only  modus ponens
as a rule of inference.
The latter will be accompanied
by 
a
4-part 
system of
 logical axioms,
called $~L_{d_E}~$, $\,$ to endow
 $~d_E~$ 
with an
ability to  support
% apparatus
% agility so that it  supports
%can satisfy
%the analog of 
G\"{o}del's Completeness Theorem.
%% '
%% (similar to  other'
%% % full-scale '
%% deductive methodologies).'
%% 
%%%% 
%%%% (Papadimitriyou's
%%%% % in-depth  exploration
%%%% textbook \cite{Papa}  about
%%%% % examination  of 
%%%% the Logic-Computer interface
%%%% relies
%%%% explicitly 
%%%%  upon
%%%% % uses  
%%%% Enderton's  
%%%% % underlying 
%%%% apparatus mechanism.)
%%%% 
%% %uses
%% relies upon
%%  Enderton's 
%% approach   $d_E$.)  

%% %relies 
%% does rely
%% upon 
%% Enderton's apparatus
\item[  ii. ]
The  $~d_M~$ 
apparatus in
\textsection 2.3 
of  Mendelson's textbook
and  the $d_H$ 
 apparatus
in  \textsection 0.10
of the H\'{a}jek-Pudl\'{a}k's
 textbook
employ a more compressed set of logical axioms
than $\, d_E \,$,
but
they 
instead
use
two rules of inference
% (formalizing separately
( modus ponens and generalization).
%% plus a smaller set of logical axioms, which Mendelson
%% has called A1-A5.
%% Also, the $d_H$ 
%%  apparatus 
%% on pages ???? 
%% of the 
%%  H\'{a}jek-Pudl\'{a}k textbook
%% uses a slightly different variation of a generalization. 
%% (In the end, 
In the end, 
% both
 $~d_M~$ 
and  $~d_H~$ 
prove the same
% set of 
theorems
as   $~d_E~$ with 
only
%  minor and 
unimportant changes in
proof length.
\item[ iii. ]
The 
``semantic tableaux''
 $\,d_F \,$ 
apparatus in
Fitting's 
and Smullyan's 
textbooks 
\cite{Fit,Smul}
was
 the 
% main 
focus of our 
investigations in \cite{ww93,ww1,ww5,ww6,ww14}.
It will be rarely used
in the current article,
however.
Unlike
 $~d_E~$,  $~d_M~$  and  $~d_H~$, it
employs no logical axioms.
It instead
 uses a more complicated rule of inference.
This tableaux apparatus 
% and also Resolution, have  been 
 has 
% been found to  have many 
many
%a wide array of
applications
% underfor 
in
automated deduction,
although it is 
less efficient than
 $  d_E  $,  $  d_M  $  and  $  d_H  $   
in
% under
% extremal
worst-case 
environments.
% settings.
%circumstances.
\ennd
\end{exx}

\tvxs

\begin{dff}
\label{def-2.2}
\rm
Each of the
% deductive 
methods of
 $  d_E  $,  $  d_M  $  and  $  d_H  $   
have the property that if a theorem $\, \Psi \,$
has a proof
with  length $~L~$ 
 from an arbitrary 
axiom basis $~\alpha~$ 
under one of these deductive systems,
then it will have a proof from these other formalisms
with lengths bounded by Polynomial$(L)$.
The term 
{\bf ``Hilbert-style''} deductive method will,
thus, refer to any  deductive 
% apparatus will refer to any other
apparatus $~d~$ that
has a modus ponens rule and
employs
% similarity has its 
proof lengths
% being 
equivalent to within a polynomial magnitude
to
%of
the comparable proof lengths from $d_E$,  $d_M$  and  $d_H$.
\end{dff}


%% and which 
%% also
%% assures 
%% that the proofs of any
%% two theorems $~\Phi~$ 
%% and $~\Psi~$ 
%% (under $d$ from any
%% axiom basis $~\alpha~$) 
%% will
%% always 
%% have
%% % by more than a constant factor 
%% the sum of the lengths of the proofs
%% of $~\Phi \rightarrow \Psi ~$ and $~\Phi~$ 
%% % under $~d~$ from $\alpha$ always 
%% formally
%% bound the length of
%% $~\Psi\,$'s proof. 
%% \end{dff}


\begin{exx}
\label{ex-2.3}
%\label{ex-basis}
\rm
Some added notation is
 needed to
explain why
% help outline
% an important distinction between 
a Hilbert style
deductive apparatus, such as $\,d_E\,$,  $\,d_H\,$
 or $\,d_M\,$, should be distinguished from
 $d_F$'s
``tableaux'' apparatus.
Let
% the symbols
 $Add(x,y,z)$ and    $Mult(x,y,z)$
once again 
% will
denote 
two 
3-way predicate symbols
specifying
that
$x+y=z$ and
$x*y=z$.
Also, let us recall 
that 
an
axiom basis
 ``$\, \alpha \,$''
is said to
{\bf recognize}
successor, 
 addition  and multiplication
as {\bf Total Functions} if 
%%%%%%%%% $\, \alpha \,$ 
it
includes
\eq{totdefxs} - \eq{totdefxm}
as theorems.

% {\small
{\vspace*{- 0.6 em}
{
\small
\beq 
\label{totdefxs}
\forall x ~ \exists z ~~~Add(x,1,z)~~
\enq
\vspace*{- 1.7 em}
\beq 
\label{totdefxa}
\forall x ~\forall y~ \exists z ~~~Add(x,y,z)~~
\enq
\vspace*{- 1.7 em}
\beq 
\label{totdefxm}
\forall x ~\forall y ~\exists z ~~~Mult(x,y,z)~
\enq }
}

\vspace*{- 1.4 em}

\noindent
%Then an 
In this context, an
``axiom basis''
$\alpha$
will be called 
{\bf Type-M} if it contains
\eq{totdefxs}-\eq{totdefxm}
% \ref{totdefxs}-\ref{totdefxm}
as theorems,  
{\bf Type-A} if it contains 
%only
\eq{totdefxs} and \eq{totdefxa} as theorems,
and {\bf Type-S} if it contains
only \eq{totdefxs} as a
 theorem. 
%Moreover, 
Also,
$\alpha$ 
%will be 
is
called 
{\bf Type-NS} if it  can prove
none of these theorems.
%In this context,
%The
%Items (a) and (b) illustrate 
%%%Below are illustrated several
%%%implications of this notation:
The implications of this notation
are formalized by Items a and b:

%% 
%% 
%% , below, 
%% %will 
%% illustrate how
%% a
%% %% 
%% %%  the
%% %% prior
%% %%  literature has
%% %% 
%% %% 
%% ``Hilbert-style''
%% deductive apparatus, such as $\,d_E\,$
%%  or $\,d_M\,$, supports very different generalizations
%% of the Second Incompleteness Theorem
%% than  $\,d_F\,$'s
%% ``tableaux-style'' apparatus:  
%% 
%%  the prior literature most germane
%% to our current article is summarized as follows:
%% 
%% 
%% The relationship of these constructs to 
%% self-justification 
%% is explained by
%% items (a) and (b):
\bed
\item[   a.  ]
The
%%  
%% above 
%% evasions of the Second Incompleteness
%% Theorem are known to be near-maximal in a mathematical sense.
%% This is because
%% the
%% 
combined work of Pudl\'{a}k, Solovay, Nelson and Wilkie-Paris
\cite{Ne86,Pu85,So94,WP87},
as formalized by statement $\, ++ \,$,
%has  implied  
implies
no
natural 
Type-S system can recognize  its
own  consistency
under
any of $\, d_E \,$'s, $\, d_H \,$'s  
or
 $\, d_M \,$'s
  Hilbert-style 
versions
of deduction:
\begin{quote}
{\bf ++ }
\small
% \footnotesize
 \baselineskip = 0.9 \normalbaselineskip 
{\it 
(Solovay's  
modification
%Generalization 
\cite{So94}
%1994 Generalization \cite{So94}
%of a 1985 theorem 
of Pudl\'{a}k \cite{Pu85}'s formalism 
with
%using 
%some of 
Nelson and Wilkie-Paris \cite{Ne86,WP87}'s
methods)} :
Let $ \, \alpha \, $ denote 
%%! any consistent
% a logic
an axiom  
basis  
% axiom system 
%basis  system 
%supporting
% which contains 
able to prove
Line
\eq{totdefxs}'s
Type-S statement
and 
assuring
%which assures 
%that 
the successor operation
%always 
does satisfy
both 
% the axioms of 
 $  \,   x'     \neq 0     $ and
the identity
$     x'     =     y' \Leftrightarrow x=y $.
$~$Then $~\alpha~$
cannot verify its
%will be unable to recognize its
%own  
consistency
under any Hilbert-style 
%deductive 
apparatus $d$,
%whenever
if  it
 treats addition and multiplication
as 3-way relations, 
satisfying 
the usual % identity,
associative, commutative 
 distributive 
and identity axioms.
%   -axiom
% properties.
\end{quote}
Essentially, Solovay \cite{So94} 
privately communicated 
to us 
in 1994
%to us
an analog of $++$'s result. 
%but 
Many authors
have noted Solovay
 has 
been
%often
reluctant to publish
% several of 
his 
nice 
privately communicated
results 
on many occasions
%in several contexts
\cite{BI95,HP91,Ne86,PD83,Pu85,WP87}. 
Thus,
%polished
approximate  analogs of 
%statement
 $++$
 were  explored 
subsequently
by
Buss-Ignjatovic,
H\'{a}jek 
and
\v{S}vejdar in \cite{BI95,Ha7,Sv7},
as well as in Appendix A of 
our paper
\cite{ww1}.
Also, 
Pudl\'{a}k's initial 1985 article  \cite{Pu85} 
% implicitly
did capture 
essentially
the 
{\it great majority} 
%most
%%%  much 
of $++$'s 
% main
% underlying
formalism,
and
Friedman did 
related work
in
\cite{Fr79a}.

\item[   b.  ]
Part of what makes
% the Pudl\'{a}k-Solovay discovery in
 $++$ interesting is that 
\cite{ww93,ww1,ww5,wwapal}
%Willard
developed various 
% separate
methods for 
basis systems
%%% $\alpha$ 
to confirm their own consistency, whose
main
further  improvements are 
prohibited by either the invariant $++$ 
or by
\cite{ww2,ww7}'s hybridization of 
$++$'s formalism with some further methods of 
 Adamowicz-Zbierski
\cite{AZ1}. (As a consequence of these facts, it is known
that 
some 
Type-A 
and Type-NS
arithmetics can verify their
respective
 semantic tableaux and 
 Hilbert-styled consistencies,
but 
Type-S arithmetics cannot verify their Hilbert consistency and
most Type-M systems cannot verify their semantic tableaux
consistency.)
 \ennd
\end{exx}

% natural hybridizations is precluded by $++$. These results involve
% either a Type-NS
% % basis 
% system 
% 
%  verifying its own consistency
% under 
% any  of the 
%  $d_E$ or  $d_H$
%  or $d_M$'s
% Hilbert-style methods,
% or   a Type-A 
% %basis 
% system \cite{ww93,ww1,ww5,ww6,ww14} 
% verifying
%  its
% % own 
% self-consistency
% under $d_F$'s tableaux 
% %deductive 
% apparatus.
% Also, Willard \cite{ww2,ww7} observed how one could
% refine $++$ with Adamowicz-Zbierski's
% methodology \cite{AZ1} to show 
%  Type-M  systems
% cannot recognize their semantic tableaux consistency.
% \ennd
% \end{exx}

\lvxs
\nvxs


% \tempvxs


%% A more detailed 15-page summary,
%% % of our prior research, 
%% germane to
%% % the
%%  Item (b),
%% % , above, 
%% %can be found in our article
%% appears in
%%   \cite{ww14}.

% NNN NEED TO REWRITE NEXT TWO PARAGRAPHS


A full 15-page summary
% of our prior research, 
of
 Item (b)'s results
% , above, 
%can be found in our article
can be found in
  \cite{ww14}.
It does not need
 to be read,
however,
 as a prerequisite for understanding
the current paper. This is because our goal
%%%  in the current paper
will be to explore axiom basis systems that can recognize their
own 
Hilbert-styled
consistency, and the invariant $++$ indicates that each of the
classes of Type-S, Type-A and Type-M 
arithmetics
are irrelevant to 
this objective.
%our goals.


%% %% formalisms 
%% own
%% % contain 
%% excessive
%% growth properties that 
%% lie outside 
%% our goals.

%this goal.

%are incompatible with our goals.
 
Instead, the \textsection \ref{seee3} will introduce a new
growth function symbol, called the ``$~\theta~$'' primitive,
that allows us to reside within the domain of a Type-NS arithmetic
because {\it none of the  identities in
Lines 
 \eq{totdefxs}-\eq{totdefxm}} 
will be provable consequences
of $\theta$'s speedy but unconventional growth properties.

This $\theta$ primitive will be attractive because
Proposition \ref{th-3.3}
and Remark \ref{rem-def-3.4} will imply  it supports
respective $O(~$Log$^3 \, n \,)$ and $O(~$Log$ \, n \,)$
growth speeds for constructing arbitrary integers $~n$
(depending on what linguistic notation one uses for encoding integers).
As a result,
%%% of this fact,  
\textsection \ref{ss32}
 will conjecture that
a seemingly minor ``IQFS'' modification of \cite{wwapal}'s
ISCE formalism is an arithmetic that possesses some interesting abilities
to confirm its own 
% Hilbert-style 
Hilbert 
consistency.

We might add that the discussion in this article will be
{\it entirely self-contained}
because 
 \textsection \ref{ss32}
 will summarize  \cite{wwapal}'s
definition of the ISCE formalism.

\begin{deff}
\label{def-2.4}
\rm
Let
$~\alpha~$ again 
denote an axiom basis 
and $~d~$ 
designate
 a
deduction apparatus.
%  
%  During our discussion about the open questions
%  raised by Hilbert's and 
%   G\"{o}del's
%  statements 
%  $*$ and $**\,$,
%  an 
%  %
%  % Then the  
%  
In this context, an
ordered pair
 $~(  \alpha  , d  )$
will
be called  {\bf Self Justifying} when:
\begin{description}
% \xxitch
% \small
  \item[  i   ] one of $ \, \alpha \,$'s  theorems
(or at least one of its axioms)
will
state that the deduction method $ \, d, \, $ applied to the
system $ \, \alpha, \, $ will 
produce a consistent set of theorems, and
\item[  ii   ]
     the axiom system $ \, \alpha  \, $ is in fact consistent.
\end{description}
\end{deff}


\begin{exx}
\label{ex-2.5}
 \baselineskip = 1.04 \normalbaselineskip 
\rm
% Using 
% Definition \ref{def-2.4}'s
%  notation, our research  
% 
Our research  
in
\cite{ww93,ww1,ww5,wwapal,ww9,ww14}
developed
%\cite{ww93}-\cite{ww14}
%has  consisted of
% developing
ordered pairs  $~(  \alpha  , d  )$
that 
were
%are
``Self Justifying''.
It 
%  has 
also explored
how the Second Incompleteness Theorem formalizes
limits beyond which such formalisms cannot transgress.
For any  $\,(\alpha,d) \,$, 
it is 
easy 
to construct a 
% second
%%% axiom 
system $ \, \alpha^d \, \supseteq  \,  \alpha  \, $
 that  satisfies
the
Part-i 
condition.
%of 
% this definition.
For instance,  $ \, \alpha^d \, $  could
consist of all of $~\alpha \,$'s axioms plus 
an added {\bf $\,$``SelfRef$(\alpha,d)$''$\,$} sentence,
defined as stating:
%%%
%%% the following 
%%% %%% added
%%% further
%%% sentence,  
%%% called
%%% %%% that we call
%%% {\bf SelfRef$(\alpha,d)~$}:
\begin{quote}
\small
% \baselineskip = 0.95 \normalbaselineskip  
%\xxitch
$\oplus~~~$ 
There is no proof 
(using 
$d$'s deduction method)
of  $0=1$
from the  {\it union}
 of
the
 axiom system $\, \alpha \, $
with {\it this}
sentence  ``SelfRef$(\alpha,d) \,$'' (looking at itself).
\end{quote}
Kleene 
\cite{Kl38} 
discussed
how
to
encode
approximate
 analogs of this
{\bf $\,$``SelfRef$(\alpha,d)$''$\,$}
statement.
%%% SelfRef$(\alpha,d)$'s
%%% self-referential statement.
Both Kleene and 
Rogers  \cite{Kl38,Ro67}
% 
% Each of
% Kleene, 
% Rogers and Jeroslow 
%  \cite{Kl38,Ro67,Je71}
% 
 noted
$\alpha ^d$ 
may
% , however,  
be inconsistent
(despite SelfRef$(\alpha,d)$'s assertion),
thus causing 
it
to violate Part-ii of   self-justification's
definition.
This is because if the 
%  ordered
 pair $(\alpha,d)$ is too strong
then a classic G\"{o}del-style diagonalization argument can
be applied to the axiom system 
$\alpha^d~~=~~ \alpha \, + \, $ SelfRef$(\alpha,d)$,
where the added presence of 
% statement 
SelfRef$(\alpha,d)$'s axiomatic   statement 
will cause 
%% this extended version of 
$\, \alpha^d\,$, ironically,
 to
 become automatically inconsistent.
Thus, the machinery of the sentence
``SelfRef$(\alpha,d)$'' is relatively easy to 
encode,
%make well-defined 
via an application of the Fixed Point Theorem,
but it
is
ironically 
%%%%%{\it most often  
{\it 
typically
%usually
useless! }
\end{exx}

%\newpage


Unlike our earlier work, which focused 
 mostly around a 
semantic
tableaux apparatus for deduction,
the current paper
will
explore
%paper will explore
\dfx{def-2.2}'s
more pristine Hilbert-style methodologies.
%% 
%% analogous to 
%% Example
%% \ref{ex-2.1}'s 
%% textbook
%% methods.
%% 
% of 
% $d_E$,  $d_M$
% and  $d_H$.
%%! 
%%! in
%%! the textbooks by 
%%!  Enderton,  H\'{a}jek-Pudl\'{a}k,
%%! Mendelson and Papadamiriyou \cite{End,HP91,Mend,Papa}.
There are, of course, many types of generalizations
of the Second Incompleteness Theorem known to
arise in  Hilbert-like  settings
\cite{BS76,Bu86,BI95,Fe60,Fr79a,Go31,Ha7,Ha11,HP91,HB39,Lo55,Kr87,Kr95,Pa71,Pa72,Pu85,Pu96,So94,Sv7,Vi5,WP87,ww1,wwapal}.
Each such
generalization
formalizes
a paradigm where 
self-justification is infeasible
under a Hilbert-style apparatus.

\smallskip

Our 
main
prior research about
%%! 
%%! main work about arithmetics displaying knowledge
%%! about their 
%%! 
Hilbert consistency appeared in \cite{wwapal}.
Its ISCE$(\beta)$ 
system could recognize its
own 
Hilbert consistency and 
%%could 
prove analogs of 
the  $\Pi_1$ theorems of
any 
% r.e. 
extension $~\beta$ of
Peano Arithmetic.
%  's $\Pi_1$ theorems.
It unfortunately required
% the 
use of  infinitely many
% number of 
constant symbols, with
 ISCE$(\beta)$ 
using one built-in constant
% symbol 
$C_i$
for each power of 2.
(Such a series of constants seriously
 deviated from 
Hilbert's
intended
 goals
in statement $*\,$, as
we will explain during
\textsection \ref{ss32}'s detailed review of
\cite{wwapal}'s results.)


%  An alternative in  \cite{wwapal},
% called ISINF$(\beta)$, required use of only three constant symbols,
% but its proof lengths were impractically long. 


%%! required
%%! % in excess of
%%! an impractical
%%!  $O(N)$ length proof to construct an integer $N$.

\smallskip

Prior to \cite{wwapal}'s publication, 
Pavel Pudl\'{a}k
\cite{Pupriv}
examined this article and asked 
%us
% the crucial question about 
whether
we could improve 
upon ISCE
% 's properties
 by using Ajtai's observations
\cite{Aj94}
 about
the Pigeon Hole principle.
%%  
%% Sam Buss \cite{Bupriv} also asked us
%% a 
%% %similar 
%% related
%% question
%% (during a more 
%% informal 
%% %abbreviated
%% conversation).
%% %(in a more informal manner).
%% 
Our prior
partial
answer to  Pudl\'{a}k's
question
was offered  
%issue
%appeared
%in Sections 6 and 7 of \cite{wwapal}.
in Sections 6 of \cite{wwapal}.
A 
% very 
different type of reply will be offered
% 
% We will offer 
% % an alternate much  
% a 
% % much
% more sophisticated
% and different 
% type of reply
% % analysis
% %  formalism 
% 
in
the current 
paper.
%article.


%%! an abbreviated version of a similar
%%! question after we verbally summarized to him \cite{wwapal}'s
%%! planned results.

% in 1997.


\begin{deff}
\label{def-2.6}
\rm
Throughout our discussion, a
% A
primitive $~F~$ will be called a 
{\bf Q-Function} 
iff is is 
sufficiently ambiguous
for there to exist an uncountably infinite set of
% different 
distinct
{\it 
plausible  sequences} of
ordered pairs in expression \eq{wow} where
$~F(i)=a_i~$  is allowed as a
% logically 
permissible
%plausible 
%formalization 
representation of $F$
under some fixed axiom system $~\gamma~$. 
\begin{equation}
\label{wow}
  (0,a_0) 
 ~,~  (1,a_1)   ~,~  (2,a_2)   ~,~  (3,a_3)   ~,~  (4,a_4)~ ...
\end{equation} 
\end{deff}

% \gvxs2

\vspace*{- 0.6 em} 
%It turns out most

%  
% Most
% Q-Function symbols are
% unsuitable for 
% analyzing
% %producing a positive resolution to
% Hilbert's  Second Open Question or most 
% issues in
% % other prominent
% % % mathematical 
% % questions within 
% mathematics.

Most
Q-Function symbols are
awkward to employ.
 This is because the
presence of an 
 uncountably
 infinite
  number of
different 
plausible  sequences,
formalized by Line
\eq{wow} for solving
$~F(i)=a_i~,~$  is 
typically more of a burden than a benefit.
A
possible
%potential
 exception to this general rule
of thumb
 will be
provided by
 the next
section's $~\theta~$ operator: It
% because it
 will,
conveniently,
 lie outside the scope of
 $\, ++ \,$'s generalization of the Second Incompleteness
Theorem.
This fact will ultimately lead to our
main  conjecture
about stronger variants of Type-NS logics
recognizing their own Hilbert consistency.



% an enticing manner.


%  It 
% will
% % should 
% provide an
% % enticing
%  avenue for
% Type-NS axiom systems to recognize their own
% Hilbert consistency (if
% \textsection \ref{ss5}'s 
% ``IQFS''
% %anticipated
% conjecture is
% correct). 


%% 
%% and suggest a mechanism whereby an efficient form of
%% ``Type-NS''self-justifying 
%% arithmetic
%% can recognize its own Hilbert consistency,
%% without
%% viewing
%% % recognizing 
%% %%%%%% any of  addition, multiplication and
%% even
%% successor  as
%% a total function.

%In other words,
% \smallskip


%% 
%% 
%% \medskip
%% Thus in a context where the partial drawbacks of
%% our 
%% new $~\theta~$ primitive
%% will be beyond doubt, it will
%% % simultaneously
%% renew the
%% serious
%%  question about whether a  
%% {\it 
%% part-way 
%% 5-10 \% fragment} of
%% %positive} interpretation
%% %can be assigned to
%% Hilbert's and G\"{o}del's
%% goals in $*$ and $**$
%% can be acheived.


%% 
%% Our
%% suggestion
%% % conjecture
%%  will be that
%% Q-functions 
%% might
%% allow one to assign a
%% {\it 
%% % part-way 
%% 5-10 \%
%% positive} interpretation
%% for what
%% Hilbert and G\"{o}del 
%% were
%% seeking
%% %referring to
%% % a Consistency Program
%% % to establish
%% %%  seeking to accomplish
%% % contemplating 
%% in their 
%% famous
%% %often quoted
%% statements
%% $*$ and $**$.
%% 

%% 
%% It will enable us to develop ground terms for formulating
%% any integer $~N~$ using
%%  $O\{~$Log$(N)~\}~$ 
%% logical symbols,
%% in a context where
%% {\it  none of the} addition, multiplication or
%% successor  function symbols are employed
%% by $~\theta \,$'s analog of an 
%%  $O\{~$Log$(N)~\}~$ 
%% lengthed 
%% binary-like 
%% encoding 
%% for integers.
%% % of an integer as a binary number. 
%% This  alternate
%%  $O\{~$Log$(N)~\}~$ 
%% format
%% for encoding an integer $~N~$ is 
%% potneitlally useful
%% %fascinating
%% because  
%% Item $\, ++ \,$'s generalization of the Second Incompleteness
%% Theorem, due to the
%% combined work of Pudl\'{a}k, Solovay, Nelson and Wilkie-Paris
%% \cite{Ne86,Pu85,So94,WP87},
%% does not preclude evasions of its invariant when a
%% ``Type-NS''
%% axiom system ceases to recognize 
%%  addition, multiplication and
%% successor  as total functions.


 
% Most
% Q-Function symbols are
% unsuitable for 
% analyzing
% %producing a positive resolution to
% Hilbert's  Second Open Question.
% A special class of Q-Functions
% %  , however,
% will
% walk through
%   Cantor's 
% % the
% world of the Uncountably Infinite
% %% in a more 
% in an
% enticing manner,
% however.
% %% however.
% %It 
% They
% will suggest a
% %%% much
%  {\it  diluted but non-trivial}
% variant
% of the aspirations
% which 
% %that 
% Hilbert and G\"{o}del 
% expressed
%  in
% $*$ and $**$ 
% are 
% %applicable to
% feasible under
% % plausible in the context of 
% Hilbert Deduction.
% (This
% % Q-function a
% analysis will be 
% %%%%%%%%%%%%%% quite
% % entirely 
% different
% from 
% \cite{ww14}'s 
% examination of
% %formalisms
% %the formalisms appearing in our Wollic-2014 paper
% %%% because 
% %it will replace 
% semantic tableaux deduction
% because it will
% apply  
% uniquely 
% to 
% Definition \ref{def-2.2}'s 
% ``Hilbert-styled'' deduction methods.)

 
%%% is replaced by the more efficient
%%% %with the more pristine 
%%% Hilbert-style deductive methodology.)

%can be achieved.


%%! This will suggest
%%! a {\it limited}  
%%! and very-much {\it down-sized} version of the formalism that
%%!  Hilbert
%%! advocated
%%! is
%%! likely
%%! %probably
%%! feasible
%%! and 
%%! germane to 
%%! the
%%! future
%%! %  computational 
%%! needs of automated
%%! theorem provers.
%%! Our 
%%! exploration 
%%!  will also provide a
%%! % quite
%%!  new interpretation of the
%%! meaning of the statements $*$, $**$ and
%%!  $***$.

% by Hilbert and G\"{o}del.



% \section{Revisiting a World which Hilbert called
% {\it ``Cantor's Paradise''}}

%\section{Main Formalism}

% \section{Deploying a New ``$~\theta~$'' Primitive}

% \vspace*{- 0.9 em} 



%\section{Need for a New ``$~\theta~$'' Primitive}


%\section{Mysterious New ``$~\theta~$'' Primitive}


% \section{The Surprisingly Useful ``$~\theta~$'' Primitive}

%\section{The  ``$~\theta~$'' Primitive and Its Potential Uses}

\vspace*{- 0.5 em} 
\section{Arithmetic Under The  ``$~\theta~$'' Primitive}
\label{ss4}
\label{seee3}

%333333333333333333333333333

\vspace*{- 0.5 em} 

%\vspace*{- 0.9 em} 


% OLD Title was {\it Notation and Basic Concepts}

% Throughout this 
% paper,
% %article,
% % a 

A function 
 $\, H \, $ 
will be     called
a
 {\bf Non-Growth} operation 
iff  
$ H(a_1,a_2,...a_j) 
\leq   Maximum(a_1,a_2,...a_j)$
for all  $a_1,a_2,...a_j$.  Six  examples of  
 non-growth functions are:
\bee
%\small
\footnotesize
\parskip - 3pt
 \baselineskip = 0.6 \normalbaselineskip 
\item
{\it Integer Subtraction} 
where ``$~x-y~$'' is defined to equal zero 
in {the special case} where
 $~x \leq y,$
\item
{\it Integer 
Division}
where ``$~x \div y~$'' equals
$~x~$ when $~y=0~$ and
it equals $~\lfloor ~x/y ~\rfloor~$ otherwise,
\item
$Root(x,y)~$ which equals $~ \lfloor ~x^{1/y}~ \rfloor$ when $~y\geq 1~$
%% 
%% and
%% it equals $~x~$ when $~y=0.$
%% 
(and zero otherwise),
\item
$Maximum(x,y),~~$
\item
$ Logarithm(x)~=~\lfloor ~$Log$_2(x)~ \rfloor~$
and
\item
$Count(x,j)~~=~~$the number of ``1'' bits
among $~x$'s rightmost $~j~$ bits.
\ene
%% 
%% 
%% \bee
%% \baselineskip = 0.8 \normalbaselineskip 
%% 
%% \item
%% {\it Integer Subtraction} 
%% where ``$~x-y~$'' is defined to equal zero 
%% in {\it the special case} where
%%  $~x \leq y,$
%% \item
%% {\it Integer 
%% Division}
%% where ``$~x \div y~$'' equals
%% $~x~$ when $~y=0,~$ and
%% it equals $~\lfloor ~x/y ~\rfloor~$ otherwise,
%% \item
%% $Root(x,y)~$ which equals $~ \ulcorner ~x^{1/y}~ \urcorner$ when $~y\geq 1,~$
%% and
%% it equals $~x~$ when $~y=0.$
%% \item
%% $Maximum(x,y),~~$
%% \item
%% $ Logarithm(x)~=~\lfloor ~$Log$_2(x)~ \rfloor~$ when $~x \geq 2,~$
%% and  zero otherwise. 
%% \item
%% $Count(x,j)~~=~~$the number of ``1'' bits
%% among $~x$'s rightmost $~j~$ bits.
%% \end{enumerate}
%% 
These operations were called
either the 
{\bf ``Grounding''} 
or {\bf ``Ground-Level''} 
 functions
in our articles \cite{ww1,ww5,ww14}.
% We will use the
We will rely upon the
% The 
latter nomenclature in the current article
 because the notion of a  ``Ground-Level''
function should not be confused with the
% very 
quite
different notion
of a ``Grounded Term'' (employed by Definition
\ref{def-3.4}). 

%  TWO DEFS or ONE ???????? 



Our
starting  language $L^G$ 
will
% shall 
also contain
the
two atomic 
symbols
%% relations 
of ``$~=~$'' and ``$~\leq~$'' and three
built in constants symbols, $~C_0~$, $~C_1~$ and $~C_2~$, 
for representing
the values of 0, 1 and 2. 
Within this context, 
 Expressions 
% Lines
\eq{newadd} and \eq{newmult} formalize how addition and multiplication
can be encoded as two 3-way predicates,
%%  in  $L^G$,
 denoted as
Add$(x,y,z)$ and Mult$(x,y,z)$.
% 
% (Their
% %  unusual
% particular
% definitions
% are 
% % quite
% %highly
% useful because they
% allow our ``Type-NS'' arithmetic to evade
% satisfying
% %  Lines 
%  \eq{totdefxs}-\eq{totdefxm}'s
% forbidden
% function-existence
%  conditions.)
% 
%%%%undesirable constraints.)
% 
% they do not imply addition
% and multiplication are total functions (e.g.
% they permit our arithmetic to be a 
% ``Type NS system''.) 
% 
%  further conditions.)
% 
% (These 
% definitions
% % for Add$(x,y,z)$ and Mult$(x,y,z)$
% {\bf notably  allow} a
% %two 3-way predicates are consistent with a 
% ``Type NS system'' to
% {\it evade satisfying} Lines  \eq{totdefxs}-\eq{totdefxm}
% {\it forbidden}
% constraints.) 
% %  further conditions.)

\newpage
%bbbbbb
{ \small
\vspace*{- 1.2 em} 
\beq 
\label{newadd}
z ~ -~x~~=~~ y~~~~ \wedge ~~~~ z~\geq~x
\end{equation}

\vspace*{- 1.2 em} 
\begin{equation}
\label{newmult}
[~(x=0    \vee    y=0 ) \Rightarrow z=0~ ]~ ~\wedge ~~ 
[~(x \neq 0 \wedge y \neq 0~) ~ \Rightarrow ~
(~ \frac{z}{x}=y  ~\wedge \, ~  \frac{z-1}{x}<y~~)~]
\end{equation} }
$~~~~$Also, 
our constant symbols,
 $~C_1~$ and $~C_2~,~$ for representing
the quantities 1 and 2, 
will  allow us to
formalize the following further
%useful 
%%%%%%%% function 
operations:
\bed
\small
\topsep -4pt
\itemsep -1pt
\it
\item[   $~$a.  ] 
Pred$(x)~~=~~x-1$
(in a context where 
%Item 1's
%the prior paragraph's
our prior
definition for Subtraction implies
Pred(0)$=0$.$~)$   
\item[   $~$b.  ] 
Half$(x)~~=~~x \div 2$
(in a context 
where 
``$ ~x \div 2~$'' equals technically 
$~\lfloor \, \frac{x}{2} \, \rfloor~$ 
under 
%our
% the prior paragraph's 
our
notation).
% of Item 2's definition for Division).  
\item[   $~$c.  ] 
Pred$^n(x)$ defined to be $~n~$ iterations of the
Predecessor operation
\item[   $~$d.  ] 
Half$^{ \,n}(x)$ defined to be $~n~$ iterations of the
halving operation.
%%
%%\item[   $~$e.  ] 
%%Bit$(x,j)~=~$ $Count(x,j)~-~Count(x,j-1)~$
%%(e.g. the value of $~x \,$'s rightmost
%% $j-$th bit)
%%\item[   $~$f.  ] 
%%Min$(x,y)~~=~~x~-~(y-x)~~$
%%(e.g. a quantity that represents the minimum of $x$ and $y$
%%because our definition of ``integer subtraction'' 
%%implies  $~y-x \,= \, 0 ~$ whenever $\, x \geq y$.$~~)$  
%%
\ennd

\nvxs

Let us say
that
 a function symbol $H(x_1,x_2...x_j)$
is
 {\bf Growth Permitting}
iff for each integer $~k \geq 2~$ there exists a 
``growth-tuple''
$(a_1,a_2...a_j)$ 
satisfying
  $~H(a_1,a_2...a_j) \, >\, k~$
and 
also
% simultaneously 
having
each $ \, a_i \leq k$ \, 
It will  be necessary 
%% for  us 
to employ
 either an infinite number
of constant symbols or some Growth-Permitting function 
so that an extension of the 
language $L^G$ can construct the 
full
%infinite
collection of
 integers of $~3,4,5,6~....~$.




%% One  awkward aspect of this notation is that 
%% it provides
%%  no guarantee
%% that    integers larger than 2 will exist         without the 
%% presence of some 
%% further
%% methodology for producing larger integers.


% 
\smallskip


One method for resolving this problem was presented in \cite{wwapal}.
% 
% It employed  an infinite number of further constant symbols. The
% latter's
%  ISCE$(\beta)$ 
%  system was
% %   shown to be 
% compatible with                     self-justification,
% but such an infinite number of constant symbols clearly trespassed on
% Hilbert's goal of using a 
% %strictly 
% finite-sized formalism.
% 
Its  ISCE$(\beta)$ axiom basis
deployed  an infinite number of 
% further 
distinct
constant symbols. It
was
compatible with   self-justification,
but deviated from 
%{\it very sharply from} 
Hilbert's
intended
 goals 
because it employed
% by employing 
%an  
a {\it highly awkward}
infinite number of 
distinct
 constant symbols.
(The reader will 
better
appreciate this point when
\textsection \ref{ss32}
reviews 
% the
% properties of 
ISCE's defining formalism.
This  difficulty is fundamental 
% to avoid 
because
the Invariant $++$'s generalization of the
Second Incompleteness Theorem indicates that
Type-S arithmetics are unable to confirm their
own Hilbert consistency.)



\medskip



%% The 
%% % self-justifying
%% ``ISINF'' formalism 
%% % in
%% of 
%% \cite{wwapal} 
%% offered an alternate method for resolving this difficulty.
%% %% in the context of a self-justifying logic. 
%% It
%% % required the use of
%% used
%%  {\it only
%% three} constant symbols.  It   could prove analogs of all
%% of Peano Arithmetic's
%% $\Pi_1$ theorems, but almost all
%% of
%%  its proofs
%% unfortunately
%%  had lengths longer 
%% than the number of atoms in the universe.
%% Most other approaches, for resolving this dilemma, 
%% % are 
%% were
%% also problematic
%% because
%% Example \ref{ex-2.3}'s
%% invariant
%%  $~++~$
%% % 
%% % which Example \ref{ex-2.3}
%% % attributed to the joint work of 
%% %   Pudl\'{a}k, Solovay, Nelson and Wilkie-Paris,
%% % 
%% showed that essentially every Type-S arithmetic is unable to
%% recognize its
%% own consistency under a Hilbert-style deductive apparatus.
%% 
%\smallskip


% The challenge posed by $++$ 
% is, thus,
% substantial.
% % certainly formidable.

Our goal in 
%the current 
this
article
will be to suggest 
how
%that 
a Q-function primitive $F(x)$,
that has
% an extraordinarily 
a deliberately
ambiguous
 function definition,
can help overcome the constraints that $++$ imposes. 
Such an
% ultra-ambiguous defined 
unusual
primitive $~F~$ will have
an  uncountable number
of vectors, analogous to Line \eq{wow}, that are permitted 
solutions to
$F$'s
definition.
Our basic goal
% in this article 
will be to outline 
how this unusual concept is
likely  germane to the
self-justifying
axiom system satisfying
{\it  a very diluted} but not
immaterial subset of
%% 
%% why it is
%% likely such Q-functions will enable one to build
%% surprisingly efficient self-justifying logics that
%% partially (but not fully) achieve the 
%% %{\it  diluted portion} of the
%% 
%aspirations  
%that
Hilbert's and G\"{o}del's goals in 
%expressed in 
statements 
$*$ and
$**~$.

\smallskip

%\normalsize \baselineskip = 0.98 \normalbaselineskip
%\gvxs


%% \el{wow}'s 
%% dizzying
%% $\aleph_1$ distinct solutions.
%% We will traverse in the opposite direction in this article
%% because 
%% Definition \ref{def-2.6}
%% %\eq{wow} 
%% formalizes 
%% % fascinating 
%% %  a possible
%% an 
%% avenue
%% available
%% for evading $++$'s generalization of the 
%% %Second 
%% Incompleteness
%% Theorem.

% {\it in at least a partial sense.}


%%  
%% %They 
%% This level of multiplicity will be
%% %These will turn out to be 
%% useful in the present setting
%% because
%% %% 
%% %% Our hypothesis is that such
%% %% solutions, while awkward and  disadvantageous
%% %% in many 
%% %% evident
%% %% %%%%%obvious 
%% %% respects,
%% %% should not be discarded.
%% %% This is 
%% %%  because 
%% %% 
%% it
%% can
%% formalize a type of 
%% allowed
%% growth-permitting function,
%% that is
%% not prohibited by $++$'s generalization of the Second Incompleteness
%% Theorem. 
%% 
%% 

%\smallskip

\begin{deff}
\label{def-3.1}
\rm
Let us say 
% an integer-valued
a function symbol
$F(x_1,x_2,...x_j)$ is {\bf `` 1-Definitive ''} iff it has only one
solution under its definition by an axiom system $\gamma$. 
Let us call $~F~$  {\bf ``Indeterminate''} otherwise.
%(The remainder of this article 
(Mathematicians
obviously typically avoid using 
function definitions
%%%%%%%%%, $~F~$ 
with 
%that have 
%%%%  having
even two solutions, not to
speak of
what will be
\el{wow}'s 
surprising
%dizzying
quantity
of
%potential
%possible
%% possibly $\aleph_1$ distinct 
 potentially infinitely many distinct 
% number of 
solutions.
This is
because such objects are typically more of a burden than a benefit.
Our conjecture
% , in this paper, 
will be that a special new
Indeterminate function, called ``$~\theta~$'', will 
be different and
offer
% an
%%%a surprisingly
%%% efficient means to eschew $++$'s prohibitions.)
a 
%surprisingly efficient 
means to eschew $++$'s 
%prohibitions.)
prohibitions with pleasing levels of
% quantitative
efficency.)
% efficency.)


\end{deff}


% considered in the current article.
% Our 
% % main
% conjecture will be that this unconventional
% approach is 
% germane to the challenge posed
% by  $++$'s 
% %
% % broad-scale
% generalization of the 
% %Second 
% Incompleteness
% Theorem
% %
% %This is because our conjecture will be that
% %
% because the  new
% proposed $~\theta~$ 
% primitive
%  will represent
% an efficient Indeterminate 
% function
% that eschews  $++$'s prohibitions.) 
% \end{deff}

% helpful when addressing 
% the challenge 
% We will traverse in 
% an unconventional
% %  the opposite 
% direction in this article
% because
% our conjecture will be that
% Indeterminate functions 
% with $\aleph_1$ different 
% allowed solutions for
% \el{wow} 
% formalize
% %%  
% %% Definition \ref{def-3.1}
% %% % {def-2.6}
% %% %\eq{wow} 
% %% will formalize 
% %% a 
% %% possible
% %% %feasible
% %% %plausible
% %% 
% an
% avenue
% % available
% for evading $++$'s 
% broad-scale
% generalization of the 
% %Second 
% Incompleteness
% Theorem.)
% 
% (Our conjecture in this article will be that
% %%
% %%The next two sections
% %%will explore
% %%how 
% %%
% \el{wow}'s indeterminate ``Q-Function'' symbol $F$
% %%%can clarify
% is germane to
% the
% aspirations 
% that 
% Hilbert and G\"{o}del
% % expressed
% stated
%  in 
% % statements 
% $*$ and
% $** \, $, $\,$when
% one employs an 
% operator
% $ \, F \, $ 
% that
% owns
% % has
% $\aleph_1$ distinct 
% available
% solutions). 
% \end{deff}


%%are viable,
%%especially in an automated theorem proving setting,
%%when Ambiguous function operatives are judiciously
%%employed.
%%

%%!  
%%! can
%%! %%and germane about automated
%%! %%deduction, will 
%%! % theorem proving,
%%! %computational deduction,
%%! be satisfied by a self-justifying logic that employs
%%! one 
%%! %single $\aleph_1$ 
%%!   Growth-Permitting function $F(x)$,
%%! that is
%%! % inherently 
%%! ``ambiguous'',
%%!  {\it accompanied 
%%! by} a finite number of non-growth primitives
%%! % {\it  non-growth}
%%!  $G_1,~G_1,~... G_k$
%%! that are ``unambiguous''.
%%! 
%%! It will also be explained how such results should have useful 
%%! applications for automated theorem proving, even when they
%%! employ 
%%! only
%%! diluted forms of  self-justifying logics.

%% 
%% \vspace*{- 1.0 em}
%% 
%% \subsection{Main Notation Conventions}
%% % about Cantor's Paradise}
%% % \large
%% % \baselineskip = 1.8 \normalbaselineskip 
%% 
%% %\vspace*{- 0.7 em}

%\gvxs

{\bf More Notation:}
$~$Let us say
an axiom system $~\alpha~$ 
has 
{\bf Infinite Far Reach} iff
it relies upon
{\it only a  finite number} of
axioms to
% distinct constant symbols
% (and/or axiom sentences) to 
prove 
for each $n$
the
\el{farreach}'s invariant.

%for each particular integer $n$.

\vspace*{- 0.8 em} 

\beq
%% \small
\label{farreach}
\exists ~~x~~~ \mbox{Pred}^n(x)~\geq ~1
\enq

%\newpage


\nvxs

\noindent
The ``ISINF'' axiomatic       framework 
from
   \cite{wwapal} 
was
    a self-justifying
system with Infinite Far Reach.
%% 
%% The opening paragraph of
%% \cite{wwapal}'s Section 6
%% %%% quite 
%% %was frank 
%% warned the reader
%% about ISINF's limitations. 
%% These arose because
%% 
%% 
%% It
%% used the word ``unnatural'' to describe 
%% the ISINF system.
%% Such caution
%% % deliberately
%% % self-deprecating term 
%% was appropriate because
%% 
Unfortunately, this result was mostly useless because
nearly all
   theorem-proofs
%of trivial theorems0000000 
from ISINF 
were
longer than the number of atoms in the universe.

\newpage
\parskip 0pt

The reason
\cite{wwapal} defined ISINF,  
%ISINF was worthy of mention,
despite 
its evident impractical  characteristics,
% 
% such
% %%% these
% % plainly 
% %%%  obvious 
% limitations,
% 
was 
because
ISINF 
demonstrated some
unusual
 self-justifying logics,
knowledgeable about their own Hilbert consistency,
were
{\it technically} 
able to
prove all of Peano Arithmetic's $\Pi_1$ theorems
together with the
existence of
the infinite set of integers $ \, 1,2,3,... \,  \, $.
This result 
% is interesting because it casts 
did cast
% casts 
a
new 
perspective
%light 
on  $\,++\,$'s  
invariant
% $++$  (appearing on \pag2)
by showing how
{\it  some  unusual}
Type-NS
forms of self-justifying arithmetics
did
escape $++$'s almost-ubiquitous 
 reach
by managing to possess infinite far reach
 without taking
% {\it without recognizing} 
Successor as a total function.


%%  of the current article.
%% The latter result indicated that Type-S arithmetics, recognizing merely
%% Successor as a total function, are unable to confirm their own
%% Hilbert consistency. 
%% Yet,
%% %%  despite this fact, 
%% ISINF was able to produce an 
%% {\it  eye-squinting} caveat because it 
%% supported the above ``Infinite Far Reach'' property
%% without 
%% needing
%% %being able to prove
%%  Line
%% \eq{totdefxs}'s declaration that successor is a total function.

%\smallskip

We sent an advanced copy of \cite{wwapal}
to
Pudl\'{a}k.
He
appreciated the nature of the challenge we faced.
% 
% ,
% concerning the delicate nature of self-justifying 
% arithmetics that are
% able to prove
% % satisfy 
% \eq{farreach}'s invariant
% {\it for each fixed $~n~$} while 
% being prohibited 
% by $++$
% from
% recognizing  successor as a total
% function.
% % (due to $++$'s restrictions).
% 
% 
Pudl\'{a}k's
subsequent  
%private 
%His
emailed communications
\cite{Pupriv}
suggested 
that we look at 
Ajtai's
work 
\cite{Aj94}
about a
%the 
Pigeon-Hole function
 $~ \glamb(x)~$ defined by the identities
\eq{zm1} and \eq{zm2}.

% \newpage
\vspace*{- 1.2 em} 
\beq
%% \small
\label{zm1}
\forall ~~x~~~~~ \glamb(x)~ \neq ~ 0
\enq

\vspace*{- 1.2 em} 

\beq
\label{zm2}
%% \small
\forall ~~x~~~ \forall ~~y~~~~ x ~ \neq~ y ~~ \Rightarrow ~~
\glamb(x)~ \neq ~\glamb(y)
\enq
The relevance of 
$~\glamb~$ 
% Pigeon-Hole functions
can be
best 
%readily 
appreciated
% if 
when
%we let
$~\glamb^n(x)~$ 
 denotes
% the 
a
term
 $~\glamb(~\glamb(~ ... \glamb(x)))~$
consisting of $~n~$ iterations of the $~\glamb~$ operator.
Then the below
% the
%%% \el{DUMB1}'s composite
term $~S_n~$
% , defined below, shall
will
% then 
satisfy
Pred$^n(~S_n~)~\geq ~1.~$ 
%% 
%% An axiom system, employing the primitive 
%% operation
%%  $~ \glamb~,~$ 
%% can thus 
%% can easily
%% prove
%% Line \eq{farreach}'s 
%% assertion.
%% %claim.
%% %under almost all conventional logics.
%% 
%% 
\beq
% \vspace*{- 0.5 em}
\label{DUMB1}
S_n~~~=~~~\mbox{Max}[~\glamb(0)~,~\glamb^2(0)~,~\glamb^3(0)~,~...~~\glamb^n(0)~]
\enq
Pudl\'{a}k
observed
that
%the
% Pigeon-Hole function 
 $~ \glamb(x)~$  
will
grow too slowly
(under well-defined non-standard models)
% (in the worst case)
for
one to be able to 
deduce
successor is a total function
from its properties.
%%   
%% % further observed that it is known 
%%  \footnote{
%% \tiny
%%  \baselineskip = 0.94 \normalbaselineskip
%%  The operation $\glamb(x)$ will grow
%% at a slower rate than Successor, 
%% if it equals $x+1$ for all standard
%% numbers $~x~$ and if $\glamb(x)=x-1$ 
%%  when $~x~$ is
%% a non-standard integer. This seemingly minute detail
%% implies  one cannot infer 
%% Successor is a total function from
%%  $\glamb$'s behavior.}.
%% 
%% 
%% 
%%  since the latter is contradicted by a
%%  model  where
%%  all  non-standard
%% numbers have 
%% %their 
%% sizes bounded by some fixed  
%% % non-standard 
%% number B.
%% (This 
%% subtle 
%% %detail, 
%% raised by
%% Pudl\'{a}k's email \cite{Pupriv}, was fascinating because
%% it 
%% %shows that
%% raised the question about whether
%%  a partial exception to
%% Example \ref{ex-2.3}'s
%% invariant $++$
%% %% on \pag2,  
%% might plausibly exist.)  }.
%% 
%% 
%% thus, 
%% suggests the 
%% Pudl\'{a}k-Solovay 
%% version of the Second Incompleteness
%% Theorem (stated on \pag2) 
%% might
%% %%%%%should
%% allow for
%% potential
%%  exceptions 
%% to it
%% arising from the 
%% %delicate 
%% formal
%% behaviour of
%% some
%% %% 
%% %% presence of
%% %% %some
%% %% these permissible
%% %% 
%% %% 
%%  non-standard 
%% variants of
%% % interpretations for 
%% the  Pigeon-Hole function $\glamb$.  }.
%% 
%% 
%that 
%prove 
%%% 
%%% 
%%% 
%%%  (in the worst case)
%%% for
%%% one to be able to 
%%% deduce
%%% successor is a total function
%%% from its properties  
%%% % further observed that it is known 
%%%  \footnote{
%%% \tiny
%%%  \baselineskip = 0.94 \normalbaselineskip
%%%  The operation $\glamb(x)$ will grow
%%% at a slower rate than Successor, 
%%% if it equals $x+1$ for all standard
%%% numbers $~x~$ and if $\glamb(x)=x-1$ 
%%%  when $~x~$ is
%%% a non-standard integer. This seemingly minute detail
%%% implies  one cannot infer 
%%% Successor is a total function from
%%%  $\glamb$'s behavior.}.
%% 
%% 
%%  since the latter is contradicted by a
%%  model  where
%%  all  non-standard
%% numbers have 
%% %their 
%% sizes bounded by some fixed  
%% % non-standard 
%% number B.
%% (This 
%% subtle 
%% %detail, 
%% raised by
%% Pudl\'{a}k's email \cite{Pupriv}, was fascinating because
%% it 
%% %shows that
%% raised the question about whether
%%  a partial exception to
%% Example \ref{ex-2.3}'s
%% invariant $++$
%% %% on \pag2,  
%% might plausibly exist.)  }.
%% 
%% 
%% thus, 
%% suggests the 
%% Pudl\'{a}k-Solovay 
%% version of the Second Incompleteness
%% Theorem (stated on \pag2) 
%% might
%% %%%%%should
%% allow for
%% potential
%%  exceptions 
%% to it
%% arising from the 
%% %delicate 
%% formal
%% behaviour of
%% some
%% %% 
%% %% presence of
%% %% %some
%% %% these permissible
%% %% 
%% %% 
%%  non-standard 
%% variants of
%% % interpretations for 
%% the  Pigeon-Hole function $\glamb$.  }.
%% 
%% 
%that 
%prove 
His insightful email \cite{Pupriv} asked
whether 
the inequality 
Pred$^n(~S_n~)~\geq ~1~$
might
%would,
thus, 
% still
enable a formalism,
% based around
utilizing the
 $\, \glamb \,$ operative, 
to 
somehow
improve upon \cite{wwapal}'s results ?


% our
% formalisms could be 
% revised
% %modified 
% so that 
% % the  Pigeon-Hole function 
%  $~ \glamb(x)~$ 
%  could improve upon \cite{wwapal}'s results.

%%
%%(possibly using Ajtai's methodologies \cite{Aj-focs}).
%%Sam Buss raised, interestingly,  a 
%%partially
%%similar 
%%issue during an informal conversation
%% \cite{Bu-priv} in 1977.
%%
%%\smallskip
%%
%%These questions
%%% by
%%%Pudl\'{a}k and Buss 
%%were insightful because they isolated 
%%an
%%important juncture where $++$'s underlying methodology does not apply.
%%A partial answer to these questions appeared in 
%%\cite{wwapal}'s closing section, but a more comprehensive full
%%answer  has always eluded us.

%This is  because there always seemed to appear
%one wrinkle of details that precluded a full proof.


\smallskip


It  was
initially
 unclear 
%%%%% to us
whether a positive answer to 
Pudl\'{a}k's
 probing
 question would resolve ISINF's main difficulties.
This is
because
% Expression 
\eq{DUMB1}'s
term
$~S_n~$  requires $O(~n^2~)$ logic symbols to encode
% essentially 
an integer quantity
greater than
 $~n~$
(since its term
$~\glamb^j(0)~$ uses $O(j)$ logic symbols).
%an integer quantity that exceeds the quantity $~n~$ in size.
Thus once again, the quantity $~2^{100}~,~$ whose binary encoding
requires 100 bits,  would require in excess of 
 $~2^{100}~$ bits to encode. 
Such large  quantities are obviously undesirable.


% Such impractical quantities are obviously distant
% from what is desired.


% Such quantities, exceeding the number of atoms in the universe,
% were troubling because our 
% general
% goal has been to 
% construct self-justifying arithmetics
% with  pragmatic features.


%%  that 
%%  possessed, at least,
%% some
%% partial  facets of
%%  pragmatic value.


% 
% find a partial
% answer to Hilbert's
% Year-1900 Second
% Problem  
%  that would 
%  possess, at least,
% some
% partial  facets of
%  pragmatic value.
% 

\medskip
\nvxs

The remainder of this section will outline how a different type of
Q-Function operator will 
be
%  much 
better than
 $~ \glamb~$ for meeting our needs.  
During our discussion,
Power$(x)$ will denote 
a primitive specifying
% that
 $~x~$ is
a power of
$~2~$.  
Its formal encoding
in $L^G\,$'s language
 is illustrated by \eq{wep2}. 
%% 
%% It is 
%% %formally 
%% encoded
%% by 
%% \eq{wep2} 
%% because
%% %under
%% our Grounding language
%% has
%% ``Logarithm$(x) \,$''$ ~ = ~ \lfloor \,$Log$_2(x) \, \rfloor \,$.
\beq
\vspace*{- 0.6 em}
\label{wep2} 
%\small
x=1 ~~~\vee ~~~ \mbox{Logarithm}(~x~)~\neq~\mbox{Logarithm}(~x-1~)
\enq
In this context,
 $\zzthe(x)$ 
will denote the analog of
the $\glamb(x)$ function  
%% haphazard
that walks among the powers of 2 in a manner 
similar to 
$\glamb(x)$'s
%  haphazard
 walk through conventional
integers. 
It is 
% formally
defined by \eq{walk1}-\eq{walk4}.
% 
% It will thus satisfy
% the axiomatic constraints below (which are 
% $\zzthe(x)$'s analog of the more modest constraints given in
% % sentences 
% \eq{zm1} and \eq{zm2}).
% The most important difference between these two constructs
% is that axiom \eq{walk1} requires that 
%  $\zzthe(x)$ maps power of 2 onto powers of 2.

{
%\small

\vspace*{- 0.6 em}
{\parskip -6 pt
\beq
\label{walk1}
\forall ~~x~~~~~ \mbox{Power}(x) ~~~ \Rightarrow  ~~
\mbox{Power}(~ \zzthe(x)~)
\enq
\beq
\label{walk2}
\forall ~~x~~~~~ \zzthe(x)~ \neq ~ 1
\enq
\beq
\label{walk3}
\forall ~~x~~~ \forall ~~y ~~~~[~ x ~ \neq~ y ~
\wedge ~\mbox{Power}(x)~]
 \Rightarrow ~~ 
\zzthe(x)~ \neq ~\zzthe(y)
\enq
\beq
\label{walk4}
\forall ~~x~~~~~ \neg ~ ~\mbox{Power}(x)~~~~ \Rightarrow ~~~~ 
\zzthe(x)~=~0
\enq} }

\vspace*{- 1.2 em}
\noindent
{\it It needs to be emphasized} that 
 \eq{walk1} -- \eq{walk4} will be the
{\it only vehicle} our
proposed formalisms will
%self-justifying axioms 
%will 
have available to construct
integers $\geq \, 3$. $~$These
axioms will be called
%They will be henceforth called
the {\bf $~$Up-Walking$~$} axioms. 
(The axiom \eq{walk4} 
is
% does not,
%% , technically, 
unnecessary
 to construct any
integer  $\geq \, 1\, $, but it is helpful 
for 
% because
% it 
% % allows us to 
% formalizes 
formalizing
how our methodology will treat  integers
which are not powers of 2.)

% \gvxs
% \svxs

\smallskip


Both
$\glamb$ 
 and $\zzthe$ are
% unusual
% complicated
% entities 
Q-functions
%because they 
that
own
% potentially
 infinitely many distinct 
% $\aleph_1$
%  distinct 
vectors, 
%analogous 
similar
to 
 Line \eq{wow},
for representing them.
We will soon see
their 
underlying
% computational
 complexities 
% properties 
are 
%sharply 
%quite
surprisingly
different.


%% 
%% they have sharply
%% contrasting
%% %%very
%% %sharply 
%% %% different
%% computational
%% % complexity 
%% properties.
%% 

%% Both
%% the 
%% Q-functions 
%% % the operators 
%% $\glamb$ 
%%  and $\zzthe$ are 
%% %awkward 
%% challenging
%% to analyze
%% %challenging
%% %daunting
%% because there are
%% % a  
%% % dizzying
%% $\aleph_1$ distinct 
%% vectors, analogous to 
%%  Line \eq{wow}, 
%% that are
%% %where their definitions permits 
%% representations of these functions.
%% %% 
%% %% Also, we may combine either operation with our 
%% %% language $L^G$'s grounding function-primitives to formulate a term
%% %% $~T_n~$ that defines any arbitrary integer $~n~$.
%% %% 
%% We will soon see that
%% there is, however, a
%% distinction
%% %  major difference 
%% between these
%%  two concepts
%% from a
%% % computational 
%% complexity perspective.

\begin{definition}
\label{defx-3.2}
\rm
Let $~L^Q~$ 
and $~L^{Q^*}~$ 
denote the 
extensions
of $~L^G\,$'s Grounding language that contain the
respective
additional 
function symbols of  
  $\zzthe$
 and
$\glamb$. Then 
$~~L^Q~$ shall be called the {\bf  Q-Grounding} language, and 
 $~~L^{Q^*}~$ 
will be called the  {\bf  Q* Grounding} language. 
\end{definition}

\begin{propp}
\label{th-3.3}
In contrast to the
Q* Grounding language 
that requires $O(~n^2~)$ function symbols
for defining a term $~T^*_n~$ for representing the integer
$~n,~$ the    Q-Grounding language
%% will need no more than 
needs
% uses 
only
$O(~$Log$^{ \, 3\,} \,n~)$ symbols to 
encode
%formalize 
a term 
$~T_n~$ representing
$~n$.
\end{propp}

\vspace*{- 1.0 em}

\begin{center}
% \small
% Our proof of \phx{th-3.3} 
\phx{th-3.3}'s 
proof
will rely upon the following notation  convention:
\end{center}

\vspace*{- 0.8 em}

\begin{definition}
\label{def-3.3}
\rm
Let
 $~\zzthe^j(x)~$
denote the term
 $~\zzthe(~\zzthe(~ ... \zzthe(x)))~$
where there are 
$~j~$ iterations of the 
 $~\zzthe~$ operation.
% Throughout this article,
Then
%for any  $~j \, \geq 1~,~$ 
%the symbol 
$~E_j~$ 
will
% shall
% will 
denote
the quantity produced by
\eq{ej-def}'s division operation: 

\vspace*{- 0.6 em}

\beq
\small
\label{ej-def}
 \frac{~\mbox{Max}
~[~\zzthe^{\, j \,}(1)~,~~\zzthe^{\, j-1 \, }(1)~,~... ~\zzthe(1)~~] }
{~~\mbox{Half}^{\,j\,} ~ \{ ~\mbox{Max}~[
 ~\zzthe^{\, j \,}(1)~,~~\zzthe^{\, j-1 \, }(1)~,~... ~\zzthe(1)~]~ 
 ~ \}~~ } 
%
% \mbox{Max}(~\zzthe^j(1),~\zzthe^{j-1}(1),~... ~\zzthe^1(1)~ 
\enq
It is each to see
$  E_j  =  2^j  $ for every
$j   \geq 1$.
This is
 because \el{ej-def}'s
twice-repeating
 term
% object  of
``$\,  \mbox{Max}
~[~\zzthe^j(1),  \zzthe^{j-1}(1),...\zzthe(1) \,]\,$''
% is at least as large as $\, 2^j\,$.
is a power of 2 exceeding  $\, 2^j\,$.
%% 
%% The definitions of the
%% % Q-Grounding 
%% functions of ``Half'', ``Max'' and
%% ``$~\zzthe~$''   imply 
%% $~E_j~=~2^j~$ for each
%% $j \, \geq 1$. 
%% 
For the additional case where $~j=0~,~$
we will 
% formally
define  $~E_0~=~1~$ (by 
using the 
%%
%%setting it equal to 
%%our
%%%the
%%
built-in constant symbol
of $~C_1~$).
\end{definition}

%% , which 
%% is intended to
%% %formally 
%% represent the integer of ``1'').



{\bf Proof of  \phx{th-3.3}:}
%The justification of   \phx{th-3.3} is an 
Easy consequence of
\dfx{def-3.3}'s machinery. Thus if $~n~$ is a power of
2 of the form $~2^j~$ then
% the preceding
% definition's
expression $~E_j~$ is a term representing $~n \,$'s value
that employs
  $O(~$Log$^{ \, 2\,} \,n~)$ 
logical
symbols. On the other hand, if  
 $~n~$ is not a power of
2 then it can be defined 
with   $O(~$Log$^{ \, 3\,} \,n~)$ symbols by 
setting
$~E_j~$ equal to the least power of 2 greater than $~n~$ and
subtracting from $~E_j~$ those powers of 2 that are needed to
produce $\,n\,$'s  value. 
For example since $76~=~128~-~32~-~16~-~4~,\,$ it can
be formalized as a term $T_{76}$ defined by
$~E_7 \, - \,E_5 \, - \,E_4 \, - \,E_2 ~$.

% $~~~~\Box$

% \baselineskip = 1.8 \normalbaselineskip 

\begin{definition}
\label{def-3.4}
\rm
A term in mathematical logic
is defined to be a syntactic object,  built
out of
solely
 symbols for representing 
functions,
constants and variables.
% 
% The nomenclature in
% % classical
% logic has 
% %formally 
% defined
% a {\it ``term''} to be a syntactic object,  built
% out of symbols for representing 
% functions,
% constants and variables.
% 
% 
Such an object is called
% either 
a {\bf ``Ground Term''} 
%% (or for precision a
%%% {\bf ``Tree-Oriented Ground Term''} )
when it  is built {\it out of solely}
function and
 constant symbols.
For example in our Q-Grounding language (which 
uses
%owns only
$  C_0  $, $  C_1  $ and $  C_2  $ 
as its
built-in
 constants),
% symbols), 
the expression
%of 
``$\, C_2-  C_1\,$'' 
is 
% such 
a Ground term.
Two more complex 
examples of 
Ground terms are
``Max$(   C_2 ,  C_1 - C_0)$'' 
and ``Max$( ~\zzthe(C_1)~,~C_2 ~)$''.
Also,
expression $~E_j~$ 
in Line \eq{ej-def}
should be viewed as 
a Ground term (when one 
views
its
use of the
symbol
 ``1'' as an %informal 
abbreviation
for the
constant
 ``$~C_1~$''). 
\end{definition}


\begin{remm}
\label{rem-def-3.4}
{\bf (sharply improving upon 
 \phx{th-3.3}'s result) : }
\rm
A longer version of this article
will 
%technically 
distinguish between two
kinds of Ground terms, which it calls the
%
%{\bf Comment of Definition \ref{def-3.4}'s Notation:} 
%We will  distinguish  between two
%kinds of Ground terms in Section \textsection \ref{ss6},
%called its 
{\bf ``Tree-Oriented''} and 
{\bf ``Dag-Oriented''} formats.
The latter will differ from a more 
conventional tree structure
by having a 
Directed Acyclic Graph structure replace 
a logic's
usual 
 tree format for defining its quantitative values.
It will turn out that
 Dag-Oriented Ground Terms
will allow one to compress multiple repeating
terms into single objects.
This will
%and thus 
reduce the number of logical symbols in 
 \phx{th-3.3}'s 
$O(~$Log$^{ \, 3\,} \,n~)$ sized 
%ground 
terms to a 
 more compact
$O(~$Log$\, \,n~)$ quantity.
(This is almost analogous to the 
% (in a context where the pointers to our
$O(~$Log$\, \,n~)$ 
size of an integer's binary encoding,
except that we will need  $O(~$LogLog$\, \,n~)$
further bits to encode the pointers to 
each 
specified
object.)
 \end{remm}


%% quantity
%% $O(~$Log$\, \,n~)$ 
%% objects will require  $O(~$LogLog$\, \,n~)$
%% bits per pointer). 
%% %(analogous to the classical binary encoding of an integer).

% 
%  {\it This is the same length
% as would occur in a conventional
% $O(~$Log$\, \,n~)$ sized binary encoding of an integer.}
% We will refer to this improvement later in the current
% article because it will suggest that 
%  \phx{th-3.3}'s 
% $O(~$Log$^{ \, 3\,} \,n~)$ sized ground terms attain a length
% not too different from the binary encoding of an integer,
% after further refinements are undertaken.
% 



\begin{definition}
\label{def-3.5}
\rm
A ground term
% $~T~$ will be 
is
called an 
{\bf ``Observable''} 
object iff it has a
%{\it only one} 
% an
unique
interpretation of its 
quantified value in the
%meaning  in our
Q-Grounding language.
It 
%will be
 is
called an
{\bf ``Unobservable''} iff it has multiple 
%plausible 
such
interpretations
due to $\zzthe$'s ``indeterminate'' definition
(e.g. see Definition \ref{def-3.1}). 
\end{definition}

%%% (due to the 
%%% %uncountably 
%%% ambiguous nature of 
%%% % our built-in function 
%%% $~\zzthe~~$).
%%% \end{definition}

\begin{exx}
\label{ex-3.6}
\rm
The previously mentioned ground term
Max$( ~\zzthe(C_1)~,~C_2 ~)$ is an ``unobservable''
because it can assume any of the plausible integer values
of $~2 \, , \, 4 \, , \,8  \,  , \,16  \, 
 \, ... ~$.
On the other hand,

\newpage
 
\gvxs
\nvxs
\parskip 0pt

\noindent
\el{xoo} 
%is 
provides
an 
example of an
``observable''
that
 represents
 the integer value of ``3''.
(This is because 
its 
twice-repeating
term 
``$~\mbox{Max}[ ~C_2~,  ~\zzthe(C_2)~, ~\zzthe^{ \, 2 \,}(C_2)~]~$'' is bounded
below by 4, causing the left and right sides of its subtraction
operation to differ by 
% an amount of 
exactly 3.) 
\beq
%% \small
\label{xoo}
\mbox{Max}[ ~C_2~, ~\zzthe(C_2)~, ~\zzthe^{ \, 2 \,}(C_2)~] ~~-~~
\mbox{Pred}^{\, 3 \,} \{~\mbox{Max}[ ~C_2~, ~\zzthe(C_2)~, ~\zzthe^{ \, 2 \,}(C_2)~]~\} 
\enq
Our notation 
%thus 
also
implies that Line \eq{ej-def}'s
 expression $~E_j~$ 
is an
 ``observable''. This implies, in turn, that 
 \phx{th-3.3}'s term $~T_n~$ is an  ``observable''
 (employing 
conveniently
no more than 
 $O(~$Log$^{ \, 3\,} \,n~)$ 
logical
symbols).
\end{exx}
% 
% 
% For example since $76~=~128~-~32~-~16~-~4~,\,$ 
% it follows that $~ T_{76}~$
% corresponds to the term
% $~E_7 \, - \,E_5 \, - \,E_4 \, - \,E_2 ~$,
% $~$where each $~E_j~$ employs only 
%  $O(~$Log$^{ \, 2\,} \,j~)$ symbols.
% %%%%%\end{exx}


\nvxs
\hvxs


Thus, \dfx{def-3.5} and Example \ref{ex-3.6} have illustrated
%that 
how
the realm of ``observable'' objects is a 
% very 
broad and accessible world,
of 
potential usefulness.
% 
% non-trivial
% %% pragmatic
%  significance.
% 
It allows every integer $~n~$ to be represented by a
% reasonably small 
term $~T_n~$ with
% an
a tight
 $O(~$Log$^{ \, 3\,} \,n~)$ length
(in a context where
Remark \ref{rem-def-3.4}'s refinement
% Section \textsection \ref{ss6}'s more elaborate formalism
%will allow us to 
will 
reduce this length
almost to a more compact
% 
% nearly
%  to a 
% % far 
% more
% attractive  
% 
$O(~$Log$ \,n~)$ magnitude).


The distinction between
``Observables'' and ``Unobservables''
% ground terms 
will 
%also 
%% 
%% cast a
%% delightful
%% 
offer a
 new perspective on
the aspirations
% that 
which
Hilbert and G\"{o}del 
expressed
in 
their
statements $*$ and $**$.
%
% under our proposed formalism.
It
 will suggest
how the Second Incompleteness Theorem
can
%  remain to 
be seen as a majestic result
from a purist perspective, while
a {\it well-defined fragment} of
%their 
what
Hilbert and G\"{o}del
sought in 
 $*$ and $**$
%aspirations
%% in  statements $*$ and $**$
can 
likely
%almost certainly
be 
%part-way
satisfied (in at least a 
% well-defined 
limited sense).
% 
% 
% 
% 
% \begin{remm}
% \label{rem-3.7}
% {\bf (explaining the goals of this paper):$~$}
% \rm
% Let us say 
% that
% a basis axiom system $~\alpha~$ owns
% a {\it ``Finitized Perspective''}  of the Natural Numbers
% if it requires only a 
% {\it finite number} of proper axioms
% to construct the full set of integers
% $~0,1,2,3~... ~$. All conventional arithmetics have this property.
% %It is useful to divide such 
% Such
% logics
% %arithmetics
% fall into two categories,
% called {\it Single} and {\it Double-Formatted} systems.
% %as defined below: 
% They are defined below: 
%  %These constructs are defined below:
% \bed
% \item[   a.  ]
% %An axiom basis $~\alpha~$
% %will be called
% {\bf Single-Formatted Arithmetics} consist of
% axiomatic basis systems
% % $~\alpha~$
% %all of 
% whose 
% %iff all its 
% ground terms are 
% all 
% Observables.
% (Most conventional arithmetics
% %%%% will 
% %fall into this 
% lie in this
% category
% %when the 
% because they
% employ the
%  growth
% %  function
% properties 
%  of
% the  Successor
% operation
% %function
%  in
% % a straightforward 
% %the
% %% a  conventional
% the traditional
% manner.)
% %% since the 
% %% the simple  growth function of Successor
% %% easily
% %% generates
% %% all the natural numbers). 
% %are {\it ``Single-Formatted Formatted''} logics. 
% \item[   b.  ]
% {\bf Double-Formatted Arithmetics} 
% % representing 
% represent
% systems
% %%%consisting of
% %%%%axiomatic 
% %%%logics
% %%%%%basis systems
% whose ground terms 
% may be either
%  Observables
% or Unobservables.
% (Axiomizations
% for Q-Grounded logics 
% %%  of
% %% the
% %% % our
% %% Q-Grounding language
% are
% %%% will 
% %obviously 
% %%% be
% ``Double-Formatted''
% because they
% allow $\theta$'s analog of 
% \el{wow}'s function symbol $F$ 
% to have
% an uncountable number of 
% different allowed
% representations).
% % 
% % (Our
% % Q-Grounding language 
% % gives support to such a system.
% % This is because it
% % can  have its function primitives
% % defined by a finite number of 
% % proper axioms,)
% % %axiom-sentences.) 
% % \ennd
% The distinction between 
% categories
% %Items 
% (a) and (b) is
% significant
% % important
% because 
% Example \ref{ex-2.3}
% %%%  \pag2 
% %%% had 
% already explained how
%  statement $++$'s generalization
% of the Second Incompleteness Theorem applies to 
% any formalism recognizing Successor as a total function.
% Thus, Item (b)'s Double-Formatted logics 
% are useful, if one wishes to consider alternatives
% to
% %formalism that do not recognize
% successor as a total function.
% %More precisely, 
% In this context,
% 
This will be because
Hilbert's 
%  famous 
%Year-1900 
Second 
Open
Problem
can be viewed
 as a  {\it  2-part question}, 
composed of sub-queries Q-1 and Q-2:
%%%%%
%%%%% {\it  2-part question}.
%%%%%The separation of Hilbert's question into two parts,
%%%%%called Q-1 and Q-2, will allow 
%%%%%%% 
%%%%%%% This 
%%%%%%% bipartite
%%%%%%% distinction 
%%%%%%% is useful because it
%%%%%%% can enable 
%%%%%%% 
%%%%%the academic community to better
%%%%%  with
%%%%% what Hilbert and G\"{o}del were
%%%%%seeking to accomplish
%%%%%in 
%%%%%their
%%%%%statements
%%%%%of $*$, $**$ and $***$.
\bed
\small
\item[ {\bf Question Q-1$~~$}] {\it Are any axiom systems
able to
 prove 
theorems 
verifying
 their own consistency in a robust sense?$~~$}
The answer to Q-1 is clearly ``No'' because the combination
 G\"{o}del's initial 1931 result \cite{Go31} with 
%the 
%further
Hilbert-Bernays's result 
\cite{HB39} 
and the Pudl\'{a}k-Solovay invariant $++$
(from Example \ref{ex-2.3})
%% \pag2) 
imply 
arithmetics of ordinary strength cannot prove
their own consistency in a robust sense.
\item[ {\bf Question Q-2$~~$}]
 {\it Can 
logic systems
%arithmetic logics
%axiomatizations of Arithmetic
%  , at least, 
%somehow 
``appreciate'' 
% (not formally ``prove'')
 their
own consistency in some 
{\bf REDUCED} sense, that is diluted
but not fully immaterial?}
$~~\,$The answer to 
%question 
Q-2 is 
complex
%%% more complex than Q-1
%less clear-cut
because 
%several types of 
some
arithmetics,
such as \cite{ww93,ww1,ww5,wwapal,ww9,ww14}'s paradigms,
 can
formalize
% ``recognize''
their 
own consistency 
using Example \ref{ex-2.5}'s
% a 
Fixed-Point {\it ``I am consistent''}
axiom.
Moreover,
 Definition \ref{def-3.5}'s 
% further 
separation of
the concepts  of ``Observables'' from ``Unobservables''
%
%the notions of Observable from Unobservable objects 
%
raises
some
% further
% very
subtle issues beyond these distinctions.
\ennd


%% %  sentence $\,\oplus\,$.
%% %%Using
%% %%%%%%%%%Under
%% % Using the notation from 
%% Under
%% Lines
%% \eq{totdefxs}--\eq{totdefxm}'s notation,
%% these paradigms include:
%% % both:
%% \bee
%% \small
%% \baselineskip = .86 \normalbaselineskip 
%% \item
%%  Type-A arithmetics
%% \cite{ww93,ww1,ww5,wwapal,ww9,ww14}
%% %capable of 
%% recognizing their self-consistencies under
%% either the deductive mechanics of semantic tableaux or one
%% of its cousins. (See especially \cite{ww14}'s
%% recent Wollic-2014 paper.)
%% \item
%%  Type-NS arithmetics recognizing their Hilbert consistency,
%% such as the formalisms of \cite{ww1,wwapal}
%% %further 
%% improved, possibly,
%% with the added techniques introduced in
%% this  article.
%% \ene
%% 
%% \ennd

One theme
in 
% the remainder  of 
this article will be that
the
Second Incompleteness Theorem represents a
$100 \, \% $ 
full
% comprehensive
reply to question Q-1 
but only a
% and a
 90 \% 
% adequate
 reply to question Q-2.   
Our
tiny
% only 
%tiny
 caveat to Q-2 will be related to Hilbert's 
insistence that {\it some type of ``new formalism''} 
% was needed
will be needed
to explain how
% it is 
humans
motivate themselves to engage
in cognition. 

%The next section of this article will
%  note 

Our discussion will
observe
% that
mathematicians 
% had
made no distinction between 
Unobservable and  Observable ground terms during the
% early 
1930's.
% We 
It
will suggest 
a 
% tiny new 
revised
interpretation can be assigned to the 
% historic 
%often-quoted
statements $*$ and $** \,$
of
%by 
Hilbert and G\"{o}del,
when one 
views
them
 from the perspective of 
arithmetics that 
rely upon indeterminate growth functions,
similar to $\theta$.

%employ $\theta-$like
% growth functions.

% logics that allow deploying unobservable  ground terms.

 
% owns two types of ground terms.

% looks more closely at these two types of
% ground terms.


%% where the remaining  5-10 \% fraction of
%% unresolved issues is connected to the 
%% fundamental
%% distinction
%%  separating
%% % separation of
%% Unobservable from  Observable ground terms.


% that the final
% tiny remaining
%  5-10 \%
% gap, pointed to
% in
% % by
% Hilbert's and G\"{o}del's 
% statements $*$ and $** \,$,  can be viewed as
% being related to this 
% fundamental
% distinction.


%% Some 
%% %other insightful 
%% different
%% approaches to these dilemmas

%Some other 

Other insightful
approaches
to
% the 
Incompleteness paradigms
are related to 
 Gentzen's perspectives about
transfinite induction 
under his $\epsilon_0$ ordinal
\cite{Ge36,Ta87}, the 
%% 
%% 
%% explore
%% how \cite{wwapal}'s results for a Single-Formatted logic
%% can be revised
%% % with our new $~\zzthe~$ function
%% under a
%% 
%% Before 
%% broaching
%%  this topic it should be mentioned that
%% %0fascinating
%% other approaches to
%% %efforts to partially 
%%  the Second Incompleteness Theorem 
%% % do
%% have centered around
%% 
 Kreisel-Takeuti's    ``CFA''
system \cite{KT74}
and
the {\it interpretational frameworks} of
Friedman,
Nelson, Pudl\'{a}k and Visser
\cite{Fr79b,Ne86,Pu85,Vi5}.
These systems are unrelated to 
our 
%% main
%\cite{ww93}--\cite{ww14}'s 
methods.
%approach.
They
do not use
% Kleene-like 
{\it ``I am consistent''} axiom-sentences,
similar to Example \ref{ex-2.5}'s
``SelfRef'' statement.
Also,
they
%apply to 
employ
``cut-free'' logics
(rather 
than
a 
% preferable 
Hilbert-style 
deductive 
apparatus).
%that 
%%%%%%%%%%% explored 
%%%%%%%%%%% in 
%%%%%%%%%%% \textsection \ref{ss32} ).
%%%we are considering).
%%
%%Instead, CFA uses the 
%%special
%%properties of ``second order'' generalizations of Gentzen's
%%{\it cut-free}
%%Sequent Calculus, 
%%and 
%%the
%%interpretational approach
%%formalizes how some systems 
%%recognize their
%% Herbrand consistency 
%%on localized sets of integers,
%%which 
%%unbeknownst to 
%%themselves,
%%includes all
%%integers.
%%
%%%These
%   alternate 
%%%approaches 
Their 
%alternate 
% very
 fascinating
perspectives  
should
% certainly,
 be examined by researchers
interested in the
% Second 
Incompleteness Theorem,
but they are unrelated to 
our 
objectives
%IQFS formalism.

\smallskip


%% During our 
%% % description 
%% investigation
%% of IQFS, 

Also during the next chapter,
the reader should keep
in mind that 
Proposition \ref{th-3.3}'s
% characterization of the 
$O(~$Log$^3 \, n \,)$
lengths for encoding $T_n$'s ground terms can be reduced to
% an essentially 
% a more compact
%nearly an
 essentially an
  $O(~$Log$ \, n \,)$ complexity,
when these terms are encoded 
using
%under
 Remark \ref{rem-def-3.4}'s 
more compressed 
Directed Acyclic Graph
% formalism.
methodology.
This fact will make IQFS's formalism look
% much 
% significantly more tempting.
%%%% very
quite
  tempting.

% cccc ddddddd 

%% the next chapter's discussion.
%% 
%% 
%% Their insights are important but 
%% unrelated to 
%% our particular
%% % the next section's 
%% %specific analysis of
%% %%% type of
%% Hilbert-styled self-justifying effects,
%% explored in the next chapter.
%% 



\gvxs

\vspace*{- 0.6 em}

\section{Proposed New IQFS Formalism}

 \label{ss32}
\label{ss5}

\vspace*{- 0.6 em}


\nvxs

% \rvxs


\parskip 1 pt

The only aspect of our prior research that will be 
related to our proposed new IQFS formalism
is the ISCE framework,
defined in \cite{wwapal}'s Sections 3 \& 4.
The next several paragraphs will review 
\cite{wwapal}'s results, 
so that a reader 
can omit examining \cite{wwapal}. 


% will not need to examine 
% \cite{wwapal}'s 
% formal treatment.
%results.
%%%%%%%%%%%%%%%for the reader's convenience.
%% 
%% This section will
%% review \cite{wwapal}'s results in sufficient detail
%% so that a reader need not examine \cite{wwapal}'s formal 
%% text,
%% 
%% %%%%%definition of the ISCE axiom system. 
%% 
%% During our discussion, 
%% 

% \lvxs
% \parskip 1 pt

\smallskip

During our 
discussion,
%review of \cite{wwapal}'s results,
$~L^G~$ will  again denote
our
 ``Grounding-level'' 
language 
that 
formalizes
%employs
\textsection \ref{seee3}'s
six non-growth
%  functions of
operations of
 Subtraction, Division,
Maximum, Logarithm, Root and Count determination.
%%% 
%%%  the six ``Grounding-level'' 
%%% % non-growth 
%%% functions defined on Page 5.
%  
%  consisting
%   of 
%  the
%  Subtraction, Division,
%  Maximum, Logarithm, Root and Count operations.
%  
Also, $\,C_0\,$, $\,C_1\,$ and $\,C_2\,$ will denote
three constant symbols designating the
integers 
%values 
of 
``0'', ``1'' and  ``2''.
In a context where Pred$(x)$ is an abbreviation for
``$\,x \,- \, 1\,$'',
%% (or more precisely ``$\,x \,- \, C_1\,$'' ), 
the ISCE axiom system
% from \cite{wwapal}
used
\eq{start}'s axiom
 statement 
to define
 $\,C_0\,$, $\,C_1\,$ and $\,C_2~$:
% these three constants:  
\begin{equation}
\label{start}
\small
\mbox{Pred}(    C_0    )  =  C_0~ \, \wedge ~ \,
C_1 \neq C_0~ \, \wedge ~ \,
\mbox{Pred}(    C_1    )  =  C_0 ~ \, \wedge ~ \,
\mbox{Pred}(    C_2    )  =  C_1 
\end{equation}
%Also,
The  challenge 
\cite{wwapal} 
faced was its formalism could
not use any of the
%  conventional 
operations 
%function-operations 
of
successor, addition or multiplication to infer the existence
of larger integers from the initial constants of 
$\,C_0\,$, $\,C_1\,$ and $\,C_2\,$
(without violating $++$'s generalization of the Second Incompleteness
Theorem).
% 
%  This was because
% the Pudl\'{a}k-Solovay result $++$ 
% indicated
%   the 
% presumption  successor is a total function 
% precludes   
% most
% %axiom 
% systems
% from recognizing their
% own Hilbert
% consistency.

\smallskip


Our article
\cite{wwapal} 
considered two 
methods for achieving these tasks,
%alternatives
%to a conventional Successor 
%function symbol 
% for overcoming these difficulties,
 called
the  {\bf Additive} and  {\bf Multiplicative Naming}
conventions.
They defined 
some
further constant symbols $~C_3,~C_4,~ C_5,~ ...~$
and  $~C^*_3,~ C^*_4,~ C^*_5,~ ...~$
where 
%respectively
$~C_j~=~2^{j-1}~$ and $~C^*_j~=~2^{\,  2^{ \,j-2}}~$.

\smallskip


The definition of these
% new 
constants
%  symbols
is 
easy
%straightforward
under $L^G\,$'s
% Grounding-level 
language.
% called $L^G~,~$
%all of whose function objects are non-growth primitives. 
This is because
Lines
\eq{newadd}  and \eq{newmult}
%had 
specify how
% that
two 3-way predicates, called
Add$(,x,y,z)$  and  Mult$(,x,y,z)\,$, 
%can 
%do
encode the identities of
% can be encoded to specify,  respectively,
$x=y+z$ and $x*y=z$.
Our additive and multiplicative 
% naming 
conventions
can,
%will,
 then, define 
 $~C_3,~C_4,~ C_5,~ ...~$
and  $~C^*_3,~ C^*_4,~ C^*_5,~ ...~$
%by  using
via
an infinite number of instances  of
%utilizing respectively
 \eq{addcov} and
\eq{multcov}'s
%{\it infinitely long}
axioms:
%%%%%%%%%% schemas:
% two infinite schemas of  axiom-sentences:
%% 
%% that belong to our 
%%  ``Additive'' and  ``Multiplicative
%%  Naming Conventions'',
%% then the values for 
%% $~C_j~$ and 
%% $~C^*_j~$ can be easily derived from $j-2$ instances of
%% %respectively \eq{addcov} and \eq{multcov}'s 
%% these
%% schema:
%% 


{
\small
\beq
\label{addcov}
\mbox{Add}(~C_{j-1}~,~C_{j-1}~,~C_{j}~)
\enq
\beq
\label{multcov}
\mbox{Mult}(~C^*_{j-1}~,~C^*_{j-1}~,~C^*_{j}~)
\enq}
The methodology in
 \cite{wwapal} 
%% employed \eq{addcov}  and  \eq{multcov}'s schema in a context where it 
presumed
% assured
%the Y of 
the ``names'' for its constants $ C_j $ 
and $ C^*_j $ 
had nice compact encodings using $O(~Log(j)~)$ bits.  
Its formalism calculated
%, thereby, 
the values of ``unnamed'' integers from 
named entities via the {\it non-growth} Subtraction and
Division primitives. For instance since $~20~=~32-8-4~,~$
the quantity 20 
can be encoded as $~C_6-C_4-C_3$.
%%%%%%%%%%%%%%%% under \eq{addcov}'s naming convention.


%% required
%% $O(~Log(j)~)$ bits.  
%% Thus, the length of these encodings was 
%% much 
%% smaller
%% than the respective
%% numbers
%% % magnitudes of 
%%  $2^{j-1}~$ and $2^{2^{j-2}}$ 
%% %that 
%% these constants represent.

\smallskip


The challenge \cite{wwapal} 
faced was to determine whether 
%it was possible to formulate 
self-justification
was possible
under 
%% either
\eq{addcov}'s
% ``Additive'' 
or  \eq{multcov}'s 
% ``Multiplicative'' 
%% naming
schema.
It found 
%that 
 \eq{multcov}'s 
multiplicative
% naming 
convention was  incompatible 
with self-justification (due to its 
%%very 
speedy growth rate),
but
%In contrast,
\eq{addcov}'s additive 
% naming 
schema did
% conveniently,
 permit self-justification. 

\medskip

Our new proposed IQFS
axiom system is easiest to describe, if we first
review \cite{wwapal}'s definition of ISCE
and then
explain how our IQFS framework 
%% 
%% will improve upon
%% it (by not requiring
%% the definition of an infinite number of separate
%% constant symbols).
%% 
can  incrementally refine it.
The extension of our base-language $~L^G~$
that includes the Additive Naming Convention (ANC)'s
additional constants 
 $~C_3,~C_4,~ C_5,~ ...~$
will be called
an {\bf ANC-Based Language}.
It will be denoted
as  $~L^{ANC}~$. 
Also if
 $\, t \,$ denotes   any term in $\, L^{ANC} \,$'s   
language, then 
the quantifiers in 
the two wffs of
$~ \forall ~ v \leq t~~ \Psi (v)~$ and
$\exists ~ v \leq t~~ \Psi (v)$
will be  called $\, L^{ANC} \,$'s   
{\bf ``Bounded 
Quantifiers''}.


\begin{deff}
\label{def-3.8}
\rm
The analogs of
% a
 conventional
% arithmetic's
$\Delta_0$, $\Pi_n$ and $\Sigma_n$ 
formulae
in the
language $L^{ANC}$ will be denoted as
$\Delta^{ANC}_0$,
 $\Pi^{ANC}_n$
 and $\Sigma^{ANC}_n$.
Thus,
a formula will be defined to be
$\Delta^{ANC}_0$  iff all its quantifiers are bounded.  
The
%%%%%%%%%  below 
definitions 
of  $\Pi^{ANC}_n$ and $\Sigma^{ANC}_n$
formulae are
%  also  
quite conventional:
\bee
%\small
\parskip -2 pt
\baselineskip = 0.8 \normalbaselineskip 
\item
\small
Every  
$\Delta_0^{ANC}$ formula is considered to
be 
also 
a
$\Pi_0^{ANC}$  and 
an
$\Sigma_0^{ANC}  $ expression.
%% 
%% ``$~\Pi_0^{ANC}~ \,$''  and 
%% %  also 
%%  ``$~\Sigma_0^{ANC}~ \, $''. 
%% 
\item
A
formula
is  called
 $ \,\Pi_n^{ANC} \,$
when it
% is 
can be
encoded as 
$\forall v_1 ~ ...~ \forall v_k ~ \Phi$  
where
%with
$\Phi$ is  $\Sigma_{n-1}^{ANC}$
\item
A formula
is  called
 $\Sigma_n^{ANC}$
when it can be encoded as 
$\exists v_1~ ...~ \exists v_k ~ \Phi,$  where
$\Phi$ is  $\Pi_{n-1}^{ANC}$.
\ene
\end{deff}



%%\begin{deff}
%%\label{def3.9}
%%\rm

 \parskip 0pt


Given an initial axiom system $\beta,$
the Theorem 3 of \cite{wwapal} defined a 
self-justifying logic, called
ISCE$(\beta)$ 
that could prove all 
$~\beta\,$'s $\Pi_1^{ANC}$ theorems and 
verify its own consistency under a Hilbert-style deductive
apparatus. It consisted of the following four
groups of axioms:
%
%  \newpage
\begin{description}
\small
 \parskip 2pt
\item
{\bf GROUP-ZERO:} 
This 
schema
%  axiom group 
will
use \el{start}'s axiom to define the constants of
$\,C_0\,$, $\,C_1\,$ and $\,C_2\,$ 
and 
%employed
%an infinite number of instances of
\el{addcov}'s Additive Naming 
schema
%convention
to define 
 the further constants
 $    C_3,  C_4,  C_5,  ...   $  
\item
{\bf GROUP-1:}
It is convenient  to
define
 ISCE's Group-1 and Group-2
axioms using a notation that 
will support \cite{wwapal}'s Theorem 3
in a 
% slightly
 more 
general sense than
appeared in \cite{wwapal},
%%% under a slightly different notation convention,
% is transparently equivalent
% (but slightly different) from \cite{wwapal}'s counterpart,
so our
% a 
new
% proposed
%%%% second
 ``IQFS''
formalism
(appearing later in this section)
% shall
will
%proposal shall
% framework will 
be easier to define.
Let us
therefore
 say  a $\Pi_1^{ANC} $  sentence is {\bf Simple}
iff the only built-in constants it employs are
$\,C_0\,$, $\,C_1\,$ and $\,C_2$.
Then ISCE's Group-1 scheme will
allowed to
 be any finite set of 
simple $\Pi_1^{ANC} $  axioms, called $~S~,~$
that is consistent with Group-zero schema and
which
 has
the following 
two
properties:
\bee
\item
  The union of $~S~$ with ISCE's Group-Zero
axioms
%will be capable of proving 
%do 
will
prove
all true $\Delta^{ANC}_0 $
sentences.  
%
% statements which
% are true.
\item
  The union of $~S~$ with ISCE's Group-Zero
scheme
%will be capable of proving 
will
% do
 prove
that
the ``=" and ``$\leq$" predicates
% own 
support
their conventional
transitivity, reflexivity, symmetry and total ordering
properties.
\ene
Any finite set 
$\Pi_1^{ANC} $  axioms
with the above properties can be used to define $~S~$
and 
support
%prove 
an analog of
\cite{wwapal}'s Theorem 3,
by a trivial generalization
of
\cite{wwapal}'s results.
% 
% \footnote{A formal proof of this generalization of
% \cite{wwapal}'s results is 
% %absolutely 
% entirely
% routine.
% %  and omitted here for the sake of brevity.}
% For the sake of brevity, it  is
% omitted.} 
% 
%  of
% the methodologies from Sections 3 and 4 of
% \cite{wwapal}. (Thus,
% any such finite set $~S~$ supporting Conditions (1) and (2)
% can  be employed
% by
% ISCE's
% Group-1 part.)
%%% 
%%% and it is unimportant which
%%% particular defining
%%%  set is used.
% 
% BBB111
%  
% This
% % schema
% axiom group 
% consisted of a finite
% set of 
% $\Pi_1^{ANC} $ axioms
% %, CALLED $F$,
%  defining ISCE's
% Grounding  function primitives. 
% %This means that
% For each  such function $G$ and set of numbers 
% $    {k},   {k_1},    {k_2}, ...    {k_m}$,
% %the combination of 
% the Group-Zero and Group-1 axioms 
% %must
% will
%  imply 
% $ G(    {k_1},    {k_2}, ...    {k_m}) \,=\,    {k}   $ when
% this sentence is true
% \footnote{ \f55 
% Our
%  $\Pi^{ANC}_1$
% encoding for the
% Group-1 scheme  needs,
% technically, 
% % employ
%  only
% employ
%  the three constant symbol $C_0$, $C_1$ and $C_2$ for the
% union of all 
% the
%  Group-Zero and Group-1 axioms
% to satisfy 
% their
% %its
% %the 
% above requirements.} .
% The Group-1 schema
% of \cite{wwapal} 
% will also
% assign the ``=" and ``$<$" predicates
% their conventional
% %  logical
% properties.
% %footnoted property.}
% %%
% %%(Any finite 
% %%set of  $\Pi_1^{ANC} $
% %%sentences  meeting these conditions is
% %%suitable.)
% %%
\item
{\bf GROUP-2:}
Let
$\ulcorner \, \Phi \, \urcorner$ denote $\Phi$'s G\"{o}del number, and
$\mbox{HilbPrf}_{ \beta  }(x,y)$
denote a 
%%%%%%%%%%%  $\Delta _0^{ANC+}$ 
$\Delta _0^{ANC}$ 
formula indicating  $y$ is a
Hilbert-styled
proof
from axiom system $\beta  $ of the theorem 
$x$.
%
% Suppose that
%$~\beta~$ uses the same Grounding function symbols as
%ISCE$^{ANC}(\beta)$,
%and it therefore generates
%a set of 
%% $\Pi_1^{ANC+} $ theorems.
% $\Pi_1^{ANC} $ 
%theorems.
%
For each 
%$\Pi_1^{ANC+} $
$\Pi_1^{ANC} $
 sentence  $\Phi$,
the Group-2 schema
for ISCE$(\beta)$ 
%
%was defined in  \cite{wwapal} 
%did 
will 
contain
% an 
one
axiom of the form:
%% 
%% \begin{equation}
%% \small
%% \label{group2nold}
%% \forall ~x~\forall ~y~
%% ~~\{~~[~~ \sigma_{~ \ulcorner \, \Phi \, \urcorner
%%  ~}(x)~\wedge~
%% \{~ \mbox{HilbPrf}~_\beta
%% ~(~ x ~,~y~)~~]~~
%% \Rightarrow ~~ \Phi~~ \}
%% \end{equation}
%% % {\bf IMPORTANT CLARIFICATION:} 
%% %{\small 
%% %%{{\bf DECIPHERING LINE \eq{group2nold}:$~$}
%% {{\bf Clarification:$~$ }
%%  \el{group2nold} is {\it helpful}
%% because   ISCE(\beta)$  can             infer
%% \eq{group2old}'s {\it simpler statement}
%% directly
%% from the combination of
%% \eq{group2nold},
%% % it,
%% the Group-1 schema and \el{deltf}'s definition of
%% ``$~\sigma~$''.}
%% 
\begin{equation}
% \small
\label{group2old}
\forall ~y~~~\{~ \mbox{HilbPrf}~_\beta
~(~ \ulcorner \Phi \urcorner ~,~y~)~~
\Rightarrow ~~ \Phi~~\}
\end{equation}
\item
{\bf GROUP-3:}
This last part of
%%%%%%%%%%%%%%%  \cite{wwapal}'s
ISCE$(\beta)$
% formalism 
was
 a single 
self-referencing
$\Pi_1^{ANC}$
sentence  
stating:
 %% essentially declaring:
\begin{quote}
% \small
%%%%%%%%%%%%% $ \oplus ~ \oplus ~~~$
$ \oplus  \oplus ~~~$
 ``There 
%is 
exists
no
Hilbert-style proof of 0=1 from the union of the Group-0, 1 and 2
axioms  with {\it THIS  SENTENCE} (referring to itself)''.
\end{quote}
\end{description}
%{\bf CLARIFICATION:}
{\bf Clarifying $ \oplus  \oplus$'s Meaning:}
 $~$Several of our articles
\cite{ww1,ww5,wwapal,ww9}
employed
self-referential
  $\Pi_1^{ANC}$ constructions,
similar 
to 
%%%%%%%%%the sentence
 $ \oplus  \oplus \,$,
as Example \ref{ex-2.5} had mentioned.
%% 
%% whose 
%% % precise implications were outlined in
%% significance was explained by
%% %formalized by 
%% Example \ref{ex-2.5}.
%% 
A reader can find
several
%detailed
slightly different
 illustrations about how 
$~ \oplus  \oplus ~ $
%  $\, \oplus  \oplus $'s 
%   self-referential statement 
is encoded in  these articles.


% 
% Each of these articles provide examples of
% how analogs for
% $\, \oplus  \oplus $'s 
% self-referential
% statement
% are encoded.


 
% If the reader wishes to see
% a formal encoding  for
% $\, \oplus  \oplus $'s 
% %self-referential
% % Fixed-Point 
% statement,
% %it 
% one such example
% is provided by  
% \cite{wwapal}'s 
%  Lemma 1.
% 


\begin{deff}
\label{def-3.9x10}
\rm
Let $~I(~\bullet~)~$ denote
an operation that maps
an initial axiom basis $\, \beta \,$ onto an alternate
system  $\,I(\beta)\, $.
(One example of
such an operation is the
  ISCE$( \, \bullet \, )$ 
framework,
that maps 
an initial axiom basis of 
  $~\beta~$ onto 
the alternate formalism of
 ISCE$(\beta).~)~$ 
Such an operation  $~I(~\bullet~)~$
is  called {\bf Consistency Preserving}
iff  $\,I(\beta)\, $ is consistent whenever 
the union of
 $\beta$ with the Groups 0 and 1 axiom schemas is
consistent.
\end{deff}


%Most of our research in 
% \cite{ww93}-\cite{ww14}
% has 

Several of our research projects 
%centered around
%had 
employed
 \dfx{def-3.9x10}'s
framework.
For instance, 
%% 
%% the 
%% 
%% 
%% Its
%% %%%  main
%% % central 
%% focus in
\cite{wwapal} 
demonstrated
%consisted of showing
 the   ISCE$( \, \bullet \, )$ 
mapping was consistency preserving.
Thus if PA+ denotes the extension of
Peano Arithmetic that 
includes
PA's traditional  Addition and Multiplication
functions
%% 
%% 1n addition to the conventional
%% functions of addition and multiplication
%% contains 
%% 
%% 
plus $L^G\,$'s six
added
%previously mentions
 Grounding-level function 
primitives,
%functions,
then
 ISCE$( \, $PA+$ \, )$
will 
be automatically
%be
 consistent
(because  PA+ was consistent).
% consistent whenever PA+ is consistent.
Hence while Peano Arithmetic is unable to
verify its own consistency,
%  (on account of G\"{o}del's
% seminal 1931 discovery), 
it is sufficiently agile to
prove the following relative-consistency statement:
\begin{center}
%% \small
$\#~~~$ If PA is consistent then 
 ISCE$( \, $PA+$ \, )$ is  
 self-justifying.
 \end{center}
This
%The above
% statement
 relative-consistency statement
%does offer 
provides
a partial 
positive
answer to
the
Q-2 version of Hilbert's Second  Question.
It 
captures
% Brad change encapsulizes 
one
% positive 
respect
in which
%such as  
ISCE$( \, $PA+$ \, )$
can {\it appreciate} its own consistency.
% 
% \newpage
% 
% \svxs
% 
% \noindent
%This is because it formalizes one respect 
This respect is, obviously,
only
of a limited nature
because $++$'s generalization of the Second
Incompleteness Theorem indicates 
that
no Type-S arithmetic
can
% simultaneously 
recognize 
% {\it both} 
its Hilbert consistency and
take
successor 
to be
 a total function.
%The consistency-preservation property of 
% ISCE$( \, \bullet \, )$
%dies, however,
It does,  however,
 raise the following
enticing
 question:
\newpage

\lvxs
\parskip 0pt

\begin{quote}
$\# \, \#~ $
\small
Can the infinite number of
distinct
 constant symbols, employed by
ISCE's Group-Zero schema, be reduced to a finite size
by a Type-NS Self-Justifying Logic,
without resorting to \cite{wwapal}'s inefficient
``ISINF'' 
methodology (which requires 
a proof 
having an expensive
 $\Omega(N)$ length for constructing integers $N$
whose binary encoding uses $O(~$Log$(N)~)$ bits) ?
\end{quote}
The remainder of this section will outline how an encouraging
answer to 
$\, \# \, \# \, $'s query
is likely to
%%%should,
% conveniently
arrive,
%be plausible
when one 
% carefully
%delicately 
modifies ISCE's formalism
with  the Q-function operative of $~\zzthe~$.

\begin{deff}
\label{def-3.10}
\rm
Let $L^Q$ 
% once 
again denote the extension of
$~L^G\,$'s Grounding language that includes
the
% further
 Q-function symbol of $\, \theta $.
Then
$\Delta^Q_0$,
 $\Pi^Q_n$ and $\Sigma^Q_n$
will,
intuitively,
%similarly
 denote the
%  1-to-1 
analogs of
\dfx{def-3.8}'s
$\Delta^{ANC}_0$,
 $\Pi^{ANC}_n$ and $\Sigma^{ANC}_n$'s
formulae
in $~L^G\,$'s  language.
In particular, if $~\Phi~$ 
is one of an
$\Delta^{ANC}_0$,
 $\Pi^{ANC}_n$ or $\Sigma^{ANC}_n$
formula,
then
% the formula 
$~\Phi^Q~$
will be called
% respectively
$\Delta^Q_0$,
 $\Pi^Q_n$ or $\Sigma^Q_n$
when
%if 
it
differs from $~\Phi~$ 
only
by
 replacing each constant $~C_J~$
from the set $~C_3,C_4,C_5...~$ 
with Line \eq{ej-def}'s 
% mathematically equivalent term of
term  $~E_{J-1}~$. 
\end{deff}

\parskip 2pt

%% 444444444444444

\begin{example}
\label{ex-3.11}
\rm
Suppose $~\Phi$ 
is one of a 
$\Delta^{ANC}_0$,
 $\Pi^{ANC}_n$ or $\Sigma^{ANC}_n$
sentence that employs the three constant symbols
of $C_4$,   $C_6$ and  $C_{10}\,$
for
 representing the
three numbers
of 8, 32 and 512. 
Let us recall 
that
 $E_3$,   $E_5$ and  $E_9\,$
%  do
formulate these three quantities
under  Line \eq{ej-def}'s notation.
Then $~\Phi^Q$  will have an 
identical definition as
 $~\Phi$ 
except each $C_j$ is replaced by
$E_{j-1}$.


A formula is,
moreover,
 defined to lie in one
of the 
$\Delta^Q_0$,
 $\Pi^Q_n$ or $\Sigma^Q_n$
classes 
{\it only if} it is constructed in such a manner.
This fact
% brad assures 
ensures
that all the terms employed in these
three classes of sentences are 
{\it ``Observable''} terms.
Hence ``Unobservable'' ground terms are allowed in
$~L^Q\,$'s language, but {\it they are excluded}
from occurring in the 
{\it ``end-product''} 
$\Delta^Q_0$,
 $\Pi^Q_n$ or $\Sigma^Q_n$
theorems 
that 
%%will now be discussed.
%it proves !
do
encapsulate
%formalize
 the {\it intended use of 
%its
this 
formalism.}
\end{example}


\begin{deff}
  \label{def-3.12}
\rm
The term {\bf IQFS($ ~\bullet~$)$~$}
will  refer
to the self-justifying analog of
  ISCE($ ~\bullet~$)$~$
%that will be employed 
under  $L^Q\,$'s
language.
(The acronym ``IQFS'' stands for 
``Introspective Q-Function Semantics''.)
In a context where $~\beta~$ is 
%some i
an initial axiom
system that proves theorems 
%under
in 
the 
language  $L^Q$, the 
system
%formalism 
 IQFS($ \, \beta\,$)
%$~$
will 
be defined as a
 4-part 
formalism,
analogous to  ISCE($\beta$),
except for the following 
%relatively modest 
three 
relatively
modest
changes: 
\bed
\small
\parskip 0pt
\item[  a.  ]
The Group-Zero schema of
 IQFS will
differ from ISCE's analog
by replacing 
\el{addcov}'s ``Additive Naming'' schema with  
the
Up-Walking axioms,
given in Lines  \eq{walk1}--\eq{walk4}.
(This is because
the language  $L^Q$ differs from
 $L^{ANC}$ by 
having the 
 Q-function operator of $~\zzthe~$
define the formal quantities that are represented by
the constant symbols
of $~C_3,C_4,C_5~~....~$ 
under  $L^{ANC}.~~)$ 
%% 
%% Otherwise both
%% these
%% Group-Zero 
%% schemes will be 
%% identical.
%% Thus,
%%  they 
%% will 
%% both 
%% use \el{start}'s axiom to define the 
%% three initial constants of
%% $\,C_0\,$, $\,C_1\,$ and $\,C_2\,~$.
%% 
\item[  b.  ]
All the $\Pi_1^Q$ axioms lying in IQFS's
Group-1 and Group-2 schemes will be
% identical
analogous to their counterparts
under ISCE, except  they
will
 employ 
\dfx{def-3.10}'s machinery for translating 
 $ \,\Pi_1^{ANC} \,$ 
sentences into
essentially their
% equivalent
 $ \,\Pi_1^Q \,$ counterparts. 
\item[  c.  ]
The Group-3 axiom of  IQFS
will be similar to ISCE's Group-3 
{\it ``I am consistent''} 
axiom-statement, except 
the latter's notion of ``I'' will reflect the above
changes in the Groups 0, 1 and 2 schemes. 
It
%Thus, the new
%Group-3 axiom 
will,
thus,
be a $\Pi_1^Q$ sentence declaring that
{\it ``There is no 
Hilbert-style
proof of 0=1 from the union of the preceding axioms
with THIS SENTENCE (looking at itself)''.}
\ennd
\end{deff}

%\noindent

\bvxs
\parskip 2pt

{\bf REVISITING THE Q-2 VERSION OF
  HILBERT'S SECOND OPEN
QUESTION FROM THE PERSPECTIVE OF  ``IQFS'' .}
$~$
Let us recall
% that 
\textsection \ref{ss4} indicated 
that   Hilbert's Second Open Problem could be
divided into two sub-queries, 
that were
called Q-1 and Q-2.
The former query asked whether axiom systems could
verify their own consistency in a robust sense, and the latter
inquired whether some 
{\it weaker but non-trivial} forms of
self-justification might exist.
The 
Q-1
paradigm
% 
% former query
% addressed the larger part of Hilbert's
% open question. It
% % We already noted that the Q-1 version of this query
% 
was definitively resolved in a negative direction
by the combination of G\"{o}del's initial Second Incompleteness
Effect,
its
generalization
appearing
% documented 
in the 
Hilbert-Bernays textbook \cite{HB39} and  the
Result $++$ due to the combined work of
 combined work of Pudl\'{a}k, Solovay, Nelson and Wilkie-Paris
\cite{Ne86,Pu85,So94,WP87}.
In contrast,
we noted in
\textsection \ref{ss4} 
 that there
were other
issues raised by the Q-2 version of Hilbert's
question that were 
not yet
fully
resolved.

% \gvxs
 
\smallskip

It is within 
this
context where 
Definition \ref{def-3.12}'s IQFS framework is helpful.
The strong similarity between the definitions of ISCE and IQFS,
{\it by itself,} 
suggests that IQFS is likely to satisfy a
consistency-preservation property analogous to ISCE.
Moreover, all the techniques that were used to prove
either $++$'s generalization of the Second Incompleteness
Theorem
% 
%  (due to the combined 
% work of Pudl\'{a}k, Solovay, Nelson and Wilkie-Paris
% \cite{Ne86,Pu85,So94,WP87}) 
% 
or the related subsequent 
results of
% 
% generalizations
% %% of the Second Incompleteness Effect 
% that
% Example \ref{ex-2.3} attributed
% to 
% 
Buss-Ignjatovic,
H\'{a}jek, 
\v{S}vejdar 
and Willard 
 \cite{BI95,Ha7,Sv7,ww1}
% \cite{BI95,Ha7,Sv7,ww1,wwlogos}
lose their relevance in IQFS's context.

\medskip

This is because a longer version of the current article
demonstrates
the Groups 0, 1 and 2 axioms of IQFS
{\it  are unable}
to prove 
% an invariant implying
successor is a total function, 
while the 
incompleteness results
of   \cite{BI95,Ha7,Ne86,Pu85,So94,Sv7,WP87,ww1}
require taking
successor as a total function.

% is
% precisely what is needed for these formalisms to 
% become applicable
%  to IQFS.

% (because they require a counterpart of a successor
% function operation).



\smallskip

The properties of IQFS
 are
interesting
% especially intriguing 
from a 
% Computer Science 
Complexity perspective
because
\phx{th-3.3} showed 
% that 
every integer $\,n\,$ 
can
%could 
be encoded
%under it by
by
%via
a
 term $T_n$ that has an $O\{~[~$Log$(n)~]^3~\}$ length.
This is
unlike the 
%much 
%%%%%%%%%% far worse 
asymptote  $\Omega(n^2)$ that results
when the 
$~\zzthe$ primitive 
(from Lines  \eq{walk1}-\eq{walk4})
is replaced by 
Lines \ref{zm1} and  \ref{zm2}'s
less efficient
primitive of 
$~\glamb~$.  
Moreover, Remark \ref{rem-def-3.4}
indicated
% our
that these
 $O\{~[~$Log$(n)~]^3~\}$ lengths
could be reduced
to almost an  $O\{~$Log$(n)~\}$ size if 
$L^Q\,$'s Ground terms were encoded as 
Directed Acyclic Graphs (rather than as tree-like objects).

We consider it 99 \% likely that
{\it BOTH} 
Definition \ref{def-3.12}'s
%precise
specified
formulation of 
% the
 IQFS($ ~\bullet~)$
%construct 
and its 
% more compressed Dag 
implied Dag
refinement
(using  Remark \ref{rem-def-3.4}'s additional machinery)
%modification
will satisfy 
a consistency preservation property analogous to ISCE.
If this 
2-part
conjecture is correct, it will support our
hypothesis that the Q-2 version of Hilbert's Second Open
Question 
% would 
does
support some 
{\it fragmentary}
positive results,
in the context of
% with regards to
Hilbert-styled deductive methods using the
$~\zzthe$ primitive for formulating growth among integers.

Moreover, a reader should not be
especially
 concerned that the Group-2
axiom schemas for ISCE and IQFS involve employing
an infinite number of separate incarnations
of \el{group2old}'s axiom schema.
This is because these
Group-2 schemas
can be
% should be able to be
nicely
 reduced to a 
purely
finite size, with almost no loss 
in
%of useful
information. This was done in \cite{ww14} 
for the Group-2
scheme
of its IS$_D(\beta)$ formalism, 
with the latter
%%%%%%%%%%%%%%%% still 
%where the
% 
% germane Group-2 scheme was
% reduced to one
% single  axiom sentence 
% while the resulting 
% 
% latter
%formalism still
% produced
producing
isomorphic counterparts
of all of $~\beta \,$'s
full set of
 $\Pi_1$ theorems
(e.g. see 
%Sections 5 and 6 
\textsection $\,$5
of    \cite{ww14}).
The same methods will
% trivially 
%routinely
easily
generalize for
%%%% the
% 
% Analogs of the techniques from Sections 5 and 6 of 
% \cite{ww14} 
% % will easily 
% apply to each of the 
% 
 IQFS, 
% axiom framework, 
if it does satisfy
Definition \ref{def-3.9x10}'s
Consistency Preservation property (as we
conjecture it does).

% it does).

\vspace*{- 1.0 em}

%\section{Broader Perspectives Produced by These Results}


\section{Concluding Remarks}

\vspace*{- 0.8 em}

\lvxs

\label{nnnew}

%\Large
% \baselineskip = 1.8 \normalbaselineskip 

There is no question that the 
% Second 
Incompleteness Theorem
%does imply
% demonstrates
illustrates 
that 
90-95 \% of the initial objectives of
Hilbert's Consistency Program were overly ambitious.
It would, nevertheless, be of interest if
some 5-10 \%
fragment of 
Hilbert's
% initially 
intended 
goals
% objectives 
were
partially
 achieved.

%% bbbbbb

\smallskip


This is because it is difficult to fathom how humans
can
maintain
their psychological motive to engage in cognition without owning some
type of
% qualified 
instinctive faith in their own consistency.
%% 
%% Moreover, the close similarity between the defining structures of the
%% ISCE and IQFS frameworks strongly suggests
%% \cite{wwapal}'s proof of ISCE's consistency preservation property
%% should generalize for both 
%% IQFS  and 
%%  IQFS$^*$ under a more elaborate
%% % and sophisticated
%% inductive machinery.
%% 
Moreover, it is fascinating that 
the distinction between
Unobservable and Observable ground terms, using 
Proposition \ref{th-3.3}'s  and Remark \ref{rem-def-3.4}'s
 $\, \theta \, $ operator,
% 
% whose $O(~$Log$^3\,n )$ and $O(~$Log$~n )$ complexities
% are characterized
% by Proposition \ref{th-3.3}  and Remark \ref{rem-def-3.4},
% 
%%%%%%
$\,$does 
seem to 
lend credibility to a
% fraction 
{\it  partial  subset}
% {\it fragment}
of the goals
that
Hilbert and G\"{o}del 
advanced
%% were seeking 
in
% aspiring to in
their
statements $*$ and $**~$.

\smallskip

\nvxs

Also, the last five minutes of a YouTube lecture
by Harvey Friedman,
entitled
% the 
{\it ``The Blessing and Curse of Kurt  G\"{o}del''},
raised the question 
\cite{Fr14}
of whether 
some type of 
{\it sharply  circumscribed} boundary-case exception
to 
the Second Incompleteness
Theorem 
might be possible.


% 
% could be evaded with some
% % new
% non-recursive  function symbol
% (which under
% \cite{Fr14}'s
% % Friedman's
% hypothetical
%  example
% involved deploying the laws of Physics
% instead of
% %  rather than 
% Lines  \eq{walk1}-\eq{walk4}'s
% indeterminate definition).



% (in a context where the broader ambitions of these
% two statements 
% are clearly untenable).

%% infeasible


%Thus,

% \medskip


Our proposed IQFS 
% axiom system 
framework
is intended to
be no full remedy,
when the
traditional growth properties of the  addition,
multiplication and successor function operations are replaced
by an
alternative  $~\theta~$ function symbol.
It is only a partial solution, similar to our
alternative class of 
partially
positive
 results in
\cite{ww93,ww1,ww5,ww14},
involving 
%axiom systems 
arithmetics
that
 sacrifice
%      sacrificed 
their
understanding that multiplication is a total function
for the sake of gaining an appreciation of their 
semantic
tableaux consistency.

\smallskip

Neither of these
% results 
formalisms
are perfect, and 
imperfections 
will 
% be ever-present
always be 
 present
%result
%be inevitable
when one
%  explores
considers
 the
%  tight
 dilemma posed by the
% 
% must
%  always
% % have to 
% be tolerated
% 
% when examining the  dilemma posed by the
Second Incompleteness 
Effect. 
It is within such a  context that
{\it a well-defined fragment} of 
%what 
the goals
% which
that
Hilbert and 
G\"{o}del sought in
$*$ and $**$ 
should be
%% 
%% looks 
%% %part-way 
%% like it is probably
%% 
%         realistically feasible 
possible to 
reach
%realize
under
%  some
certain
% % might be plausibly 
% %possible 
% should be 
% possible 
% to obtain
{\it meticulously defined weak-logic settings$\,$,}
if IQFS satisfies an
analog of ISCE's
 consistency-preservation
property (as we conjecture it 
will almost certainly
 do). 

%\textsection \ref{ss5}'s conjectures about IQFS do
%old to be ture.


%  special
% % broad
% % potential 
% interest.
% 


   \medskip

{\bf Acknowledgments:}
%%%%As  several Sections 1-4, 
%\textsection \ref{ss2}, 
I am
% much
%very 
grateful to
%was influenced by an emailed letter from 
Pavel Pudl\'{a}k
for suggesting 
\cite {Pupriv}
I investigate how to apply
% an analog of 
Ajtai's study
\cite{Aj94} of Pigeon-Hole effects 
for
refining my prior results about self-justifying logics.
(The combination of
 Pudl\'{a}k's
insightful             suggestion
% \cite {Pupriv}
and our
subsequent
% further 
        distinguishing
between the
$~\glamb   ~$ and $~\theta~$ operators
has led to the 
conjectured
 improvement of 
\cite{wwapal}'s ISCE formalism.)
% I am very grateful to Pudl\'{a}k for making this
% suggestion. 
I also thank Bradley Armour-Garb
and Seth Chaiken
 for
%  many
%%%%%%%%%%%%%%%%%%%%%%%%%%%%%%%%%%%%%%%%%%%%%%%%%%%%%%%several
comments
%  about how to improve 
that improved 
 the 
%this paper's
presentation.


\footnotesize
% \tiny
\parskip -3 pt

%\baselineskip =  0.92 \normalbaselineskip 
\baselineskip =  0.5 \normalbaselineskip 
%\baselineskip =  0.65 \normalbaselineskip 
\bibliographystyle{abbrv}
\bibliography{aa}

 \end{document}
\newpage

rrrrrrrrrrrrrr


\Large


On How the Introducing of a New $~\theta~$ Function Symbol Into Arithmetic's
Formalism Is Germane to Devising Axiom Systems that Can Appreciate Fragments
of Their Own Hilbert Consistency


Why a Small Fragment of Hilbert's Consistency Program
Ought to Be Feasible
for Hilbert-like Deductive Methods
After A New $~\theta~$ Function Primitive
Is Added to Arithmetic's Formalism

AFTER A NEW ``$~\theta~$'' Function Primitive


 \baselineskip = 1.5 \normalbaselineskip 

It is known that the combined work of Pudlak and Solovay, enhanced by some
added techniques of Nelson and Wilkie-Paris, implies no reasonable axiom
system can verify its own Hilbert consistency, when it recognizes Successor as
a total function and treats addition and multiplication as 3-way relations.
These considerations will lead us to examine unconventional axiomatizations
for arithmetic that continue to view addition and multiplication as 3-way
relations, but which replace the successor function symbol with an entirely
new operator, called the $\theta$ primitive.



It is likely that this paradigm can be combined with the prior results in our
APAL 2006 paper to construct axiom systems that are seriously diluted but able
to verify their Hilbert-style consistency in some interesting fragmentary
respects.



%% This operator will allow us to encode any integer  n  by a term $T_n$ whose
%% length will exceed the O(Log n) length of a binary encoding by only the
%% relatively small magnitudes formalized by our Proposition 3.3 and Remark 3.6.
%% It is likely that this paradigm can be combined with the prior results in our
%% APAL-2006 paper to construct axiom systems that are seriously diluted but able
%% to verify their Hilbert-style consistency in certain interesting respects.



{\bf Keywords:}

G\"{o}del's Second Incompleteness Theorem, 
Hilbert's Second Open Question,
Bounded Arithmetic, 
Distinction Between Semantic Tableaux and Hilbert Deduction, 
Weak Arithmetics.

\end{document}

% \parskip 2pt

The 
significance
% roles 
of
% Observations
(a) and (b)
%in our research
%from the current example, 
will become
% more 
evident
as this article progresses. 
Essentially, our
prior research
% ,
% best summarized in \cite{ww14},
%  has 
%
% had 
focused
% mostly 
on Type-A arithmetics
that could verify their consistency under
% either 
semantic tableaux deduction 
and/or its near cousins.
%% 
%% or some near-cousin of this concept
%% (e.g. see \cite{ww14}'s  summary
%% of \cite{ww93}-\cite{ww9}'s results).
%% The 
%% 
(A 15-page summary of this research appears in
\cite{ww14},
 but 
it
%the latter
does need to be examined.)
%%% 
%%%  is 
%%% not
%%% % unnecessary to examine  as 
%%% a prerequisite for reading this paper.
%%% 
%%% 
%%%  but this technical
%%% material does not need to be examined.) 
%%% does hot need to examine
%%% this material for the 
%%% 
%%% 
Our
% new
$~\theta~$ operator, defined in the next section, will
raise the question about whether a
% surprising
% powerful 
new
class of 
%new 
Type-NS systems 
will
%may 
satisfy an analogous
property in the context of
Definition \ref{def-2.2}'s more pristine
Hilbert-style methodology for deduction. 

\medskip

% have
% a similar property.(The article \cite{ww14} offers a nice 16-page summary
% of our prior results 
% \cite{ww93}-\cite{ww9}
% about Type-A arithmetics, but 
% none of these results will
% be
%  needed
% %  to be examined 
% during our current article's exploration of 
% the properties of the new % 
% $~\theta~$ operator.)

% It will be unnecessary for a reader to examine any of our

 
%% % year-2014 
%% Wollic-2014 paper \cite{ww14}
%% summarized and extended our
%% results about
%% semantic tableaux consistency.
%% % and this 
%% The current 
%% new 
%% year-2015
%% paper 
%% will, now,
%% explore whether
%%  systems  can
%% also corroborate
%% their Hilbert-styled consistency
%% under certain well-defined circumstances.

%% (and seek to explore the restrictions $++$ imposes upon
%% Hilbert-styled deduction).

% 
% (The latter topic
% % is 
% %very 
% %entirely
% %different 
% differs
% from the former
% because
% constraint  $++$ 
% applies only 
% to its
% particular
%  domain.)

%in the second context.)


%% The constraints imposed by $++$ 
%% are challenging 
%% because Type-NS arithmetics 

This
topic
% subject 
is 
interesting
%challenging 
because 
%% essentially all
Type-S arithmetics
are forbidden by $++$ from
verifying the consistency
of their own Hilbert-styled deductions
%%%%%%%%%%%%%%%%%%% (and conventional 
%forms of
(while
Type-NS formalisms are
%typically 
 usually 
quite weak).
%% 
%% (Thus, our efforts
%% to design
%% self-verifying systems
%%  must focus on
%% Type-NS arithmetics).
%% 
Our new 
$~\theta~$ operator,
together with
Proposition \ref{th-3.3}
and Remark \ref{rem-def-3.4}, 
% and \ref{th-6.1}
will suggest a 
%possible
% plausible 
partial
solution to this 
problem by
% daunting challenge by
illustrating how
an {\it unusual class} of 
Type-NS arithmetics can efficiently construct the
full set of integers $~0,1,2,3,...~$
% by finite means 
{\it without using}
any of the successor, addition or multiplication
% functional 
operations.

% function symbols. 

As a result, we will suggest 
a
%  a {\it part-way} 
% that an interesting
%%% non-trivial (although diluted)
{\it small fragment}
of what Hilbert
and G\"{o}del
% sought 
% referred to
did seek
%sought
in 
statements 
$*$ and $**$ 
% will
%  be formally  achieved 
% become tempting 
is likely
% be 
viable
under Definition \ref{def-2.4}'s formalism.



% 
% This
% topic
% % subject 
% is challenging because 
% $++$'s 
% Type-NS arithmetics 
% %  obviously 
% have sharply circumscribed powers
% (demonstrating the 
% broad reach
% % ubiquitous  nature 
% of
% %the Second Incompleteness Theorem's reach).
% G\"{o}del's 
% second theorem).
% %%  
% %% The current article will
% %% show, however,  that 
% %% some Type-NS arithmetics are 
% %% substantially
% %%  stronger than previously 
% %% anticipated
% %% (and they will have useful applications 
% %% in
% %% computer science settings).
% %% 
% %% Thus in a context where 
% %% the power of both G\"{o}del's initial
% %% Second Incompleteness Theorem and $++$'s strengthening of it
% %% are stunning
% %% and
% %% have pervasive implications,
% %% we will show that a 
% %% {\it partial-and-much-less-than-full}
% %% fragment 
% %% of what Hilbert
% %% desired
% %%  in statements $*$ and $**$ an be
% %% positively  achieved.
% %% 
% The current article will
% %show, however, 
% suggest,
% however, 
% % that 
% some Type-NS arithmetics are
% % , however,  
% % significantly 
% %% substantially
% more far-reaching
% than 
% previously 
% anticipated. Thus,  a 
% % well-defined 
%   {\it
% partial but non-trivial} fragment of what Hilbert
% and G\"{o}del
% % sought 
% % referred to
%  anticipated
% in 
% statements 
% $*$ and $**$ will
% %  be formally  achieved 
% % become tempting 
% look
% % be 
% viable
% under Definition \ref{def-2.4}'s formalism.



\end{document}

% \textsextion

%\setlength{\textwidth}{5.0 in}

\gvxs

Line 1


Line 1


Line 1


Line 1


%% eeeeee

\newpage

A theme of this article will be that
% distinction 
the distinguishing
between questions Q-1 and Q-2 and 
the separation of Observables from
Unobservables
is 
related
% likely central 
to the mystery
% that has enshrouded 
enshrouding
the Second Incompleteness Theorem.
This is
%is germane to the aspirations of automated theorem proving
%will be germane to this article 
because there 
%is no doubt
can be no doubt that
% can be no question 
%%%%%%%% that 
the Second
Incompleteness Theorem is  fully
robust
% result 
from a purist 
%pristine 
mathematical perspective.
Yet,
it is still problematic to fully
% 
%  simultaneously
% % at the same time,
% it is 
% hard to 
% entirely
% 
dismiss
 Hilbert's 1926
suggestion that 
 some 
specialized forms of logics should
%declaration 
%% 
%% concerns
%% in $\,*\,$ 
%% that 
%% {\it ``the honor of human understanding''}
%% requires
%% examining
%% % explaining 
%% % considering
%% how  logic systems can
%% 
possess
a type of well-defined 
 knowledge about their
own 
internal
consistency.
(This is because it is
highly
 awkward to explain how and why
human beings 
are able to
%can 
%manage to 
motivate 
their 
%cogitations,
cognitive process,
% themselves to think,
 if they do not own
some type of 
% instinctive
internal
knowledge about their own
 consistency.)

% sufficient
% % enough 
%  knowledge about their
% % own 
% internal
% consistency
% to motivate 
% cognition.

% Bad change above
%cogitation.
% themselves to cogitate. 

%%It is also
%%especially 
%%% very 
%%tempting 
%%to divide Hilbert's Year-1900
%%Open Question into its Q-1 and Q-2 separate parts
%% during the 21st century,
%%as computers share with humans cogitative abilities.
%%
%%Maybe DELETE above sentence ??? 
%\end{remm}

% \baselineskip = 1.8 \normalbaselineskip 

\smallskip

The next 
section will 
formalize our
% new
 proposed IQFS formalism.
% 
% describe our 
% 2-part
% conjecture about how 
% %a
% Double-Formatted Logics are 
% likely to 
% %produce some 
% cast
% new perspectives
% on 
% this topic.
% %the nature of the Second Incompleteness Theorem.
% 
Before starting this subject, it should be mentioned
that other unusual interpretations of the Second Incompleteness
Theorem have followed
from Gentzen's perspectives about
transfinite induction 
under his $\epsilon_0$ ordinal
\cite{Ge36,Ta87}, the 
%% 
%% 
%% explore
%% how \cite{wwapal}'s results for a Single-Formatted logic
%% can be revised
%% % with our new $~\zzthe~$ function
%% under a
%% 
%% Before 
%% broaching
%%  this topic it should be mentioned that
%% %0fascinating
%% other approaches to
%% %efforts to partially 
%%  the Second Incompleteness Theorem 
%% % do
%% have centered around
%% 
 Kreisel-Takeuti's    ``CFA''
system \cite{KT74}
and also
the {\it interpretational frameworks} of
Friedman,
Nelson, Pudl\'{a}k and Visser
\cite{Fr79b,Ne86,Pu85,Vi5}.
These systems are unrelated to 
our 
%% main
%\cite{ww93}--\cite{ww14}'s 
methods.
%approach.
They
do not use
Kleene-like {\it ``I am consistent''} axiom-sentences.
Also,
they
%apply to 
employ
``cut-free'' logics
(rather 
than
a 
% preferable 
Hilbert-style 
deductive 
apparatus).
%that 
%%%%%%%%%%% explored 
%%%%%%%%%%% in 
%%%%%%%%%%% \textsection \ref{ss32} ).
%%%we are considering).
%%
%%Instead, CFA uses the 
%%special
%%properties of ``second order'' generalizations of Gentzen's
%%{\it cut-free}
%%Sequent Calculus, 
%%and 
%%the
%%interpretational approach
%%formalizes how some systems 
%%recognize their
%% Herbrand consistency 
%%on localized sets of integers,
%%which 
%%unbeknownst to 
%%themselves,
%%includes all
%%integers.
%%
%%%These
%   alternate 
%%%approaches 
Their 
%alternate 
% very
 fascinating
perspective  
should
% certainly,
 be examined by researchers
interested in the
Second 
Incompleteness Theorem,
although 
%but
it is
%% 
% they are 
unrelated to 
our particular
% the next section's 
%specific analysis of
%%% type of
Hilbert-styled self-justifying effects,
studied in the current article.


%% systems 
%% formalizing
%% %verifying 
%% their
%% own consistency
%% %%%%%Definition \ref{def-2.2}'s
%% %%% approximate 
%% under
%% Hilbert-styled 
%% deduction.


%deduction.
%  Hilbert deduction.

%methods.
%formalism.


%% It is,
%% % They
%% %are, 
%% however,  not germane to the next section's
%% perspective.

%methodology.
%main formalisms.
%methods.
%results.

 % \baselineskip = 1.8 \normalbaselineskip 

%\section{
%\small 
%Improving \cite{wwapal}'s Results with a 
%``Double-Formatted'' Logic  }

\newpage
xxxxxxxxxxx


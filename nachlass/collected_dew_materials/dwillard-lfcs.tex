%% 9sept 2.1 pm       Final Copy Ready Submit AFTER SPELL DONE
%%% AND dwillard-lfcs will NOW be READY
%%   Artemov "recently" added to page 1

\documentclass[12pt]{article}
%\documentclass[10pt]{article}
% \documentclass[11pt]{article}
%\documentclass[12pt]{article}


%%%%%%%%%% \documentstyle[11pt]{article}


\usepackage{amssymb}


\addtolength{\oddsidemargin}{-0.8in}
\setlength{\textheight}{9.2 in}
%\setlength{\textwidth}{6.0 in}
 \setlength{\textwidth}{6.7 in}
%%%% above paper

%\setlength{\textwidth}{6,0 in}
%\setlength{\textwidth}{5,5 in}

\addtolength{\topmargin}{-0.75in}


%% 
%% \addtolength{\oddsidemargin}{-0.5 in}
%% \setlength{\textheight}{10.1 in}
%% \setlength{\textwidth}{7.5 in}
%% \addtolength{\topmargin}{-0.5in}






          \newcommand{\newthmwithin}[3]{\newtheorem{#1q}{#2}[#3]
              \newenvironment{#1}{\begin{#1q}\sf}{\end{#1q}}}

\newcommand{\newthm}[3]{\newtheorem{#1q}[#2q]{#3}
                        \newenvironment{#1}{\begin{#1q}\sf}{\end{#1q}}}
\newcommand{\newthmm}[3]{\newtheorem{#1q}[#2q]{#3}
                        \newenvironment{#1}{\begin{#1q}\rm}}

\newtheorem{theorem}{$~~~~$ Theorem}
%%% \newtheorem{theorem}{$~~~~$ Theorem}[section]
% \newtheorem{corollary}{Corollary}[section]
%\newtheorem{fact}{Fact}[section]
\newcommand{\makenewheading}[1]{\begin{tabbing} {\bf #1:}
\end{tabbing}}

\newtheorem{example}[theorem]{$~~~~$ Example}
\newtheorem{themx}[theorem]{$~~~~$ Theorem}
\newtheorem{corollary}[theorem]{$~~~~$ Corollary}
\newtheorem{lemma}[theorem]{$~~~~$ Lemma}
\newtheorem{remark}[theorem]{$~~~~$Remark}
\newtheorem{definition}[theorem]{$~~~~$Definition}
\newtheorem{fact}[theorem]{$~~~~$Fact}


\newtheorem{exx}[theorem]{$~~~~$ Example}
\newtheorem{lemm}[theorem]{$~~~~$ Lemma}
\newtheorem{propp}[theorem]{$~~~~$ Proposition}
\newtheorem{remm}[theorem]{$~~~~$Remark}
\newtheorem{ccr}[theorem]{$~~~~$Corrolary}
\newtheorem{coj}[theorem]{$~~~~$Conjecture}
\newtheorem{comm}[theorem]{$~~~~$Comment}


\newtheorem{deff}[theorem]{$~~~~$Definition}





% \def\Box{ QED}
\def\aaa{\beta}
\def\ccc{Class}


\def\ulxyz{\lceil}
\def\urxyz{\rceil}

\def\ulxyz{\ulcorner}
\def\urxyz{\urcorner}


\def\nop{ }
\def\nyp{\newpage }
% \def\nxp{ }
\def\nxp{ Here $~$NXP }


%  \def\nyp{ }
% \def\nyp{ }

\def\bigc{$\,$of the unabridged version of this paper \cite{ww12}}

\def\nor1{Normed$\{~2^{ \zzz \theta  \, )} ~$,$~\sqrt{~2^{ \zzz \theta  \, )}}~\}$}

\def\pagxx{Page ?xx?}
\def\xor2{Normed$\{ ~\sqrt{~2^{ \zzz \theta  \, )}}~,~2~ \} $}
%\def\fffx{Fact \#}
\def\fffx{{\bf Fact *}}
\def\zhz{H }
%\def\fffx{SFact \#}
\def\appD{Appendix D }
\def\appxD{Appendix D}
\def\fffour{three }
\def\zazsta{ and EA-stability}

% \def\glamb{\xi}
\def\glamb{\lambda}
\def\glamb{P}
\def\glamb{\theta}
\def\pag2{Page 2}
%% \def\zzthe{\zeta}


\def\glamb{\zeta}
\def\zzthe{\theta}





\def\gggen{$( L^\xi  ,  \Delta_0^\xi   ,  B^\xi  ,  d  ,  G  )~$}
\def\gggcp{$( L^\xi  ,  \Delta_0^\xi   ,  B^\xi  ,  d  ,  G  )$}
\def\peta{\sigma}
\def\zzxz{~ \sharp ( \, }
\def\zzz{~ \sharp ( ~ }
\def\zip{\sharp }
\def\mheta{\theta^\bullet}
\def\mxi{\xi^\bullet}
%\def\mbi{\bullet}
\def\mbi{\bigdot}
\def\mbi{\bigoplus}
\def\xxi{$\, \xi^* \,$}




\def\tftt{~ \frac{1}{2}~ }
\def\sss{ }

\def\goodshit{\triangleright}
\def\bullshit{\triangleleft}
\def\foo{footnote \footnote}
\def\Uxp{\Upsilon}




\def\f55{  \baselineskip = 1.1 \normalbaselineskip } 
\def\g55{  \baselineskip = 1.0 \normalbaselineskip } 
\def\s55{ \baselineskip = 1.0 \normalbaselineskip } 

\def\f55{  \baselineskip = 0.7 \normalbaselineskip } 



\def\bbskip{\bigskip}



\def\axst{\odot}
\def\rp{ p^\theta } 
\def\thsp{Theorem $ \, * \,$ }
\def\thss{Theorem $ \, *\,$'s }
\def\dexxt{\Delta}
\newcommand{\co}[1]{Corollary \ref{#1}}
\newcommand{\thx}[1]{Theorem \ref{#1}}
\newcommand{\cjx}[1]{Conjecture \ref{#1}}
\newcommand{\phx}[1]{Proposition \ref{#1}}
 \newcommand{\dfx}[1]{Definition \ref{#1}}
\newcommand{\lem}[1]{Lemma \ref{#1}}



%% \newcommand{\co}[1]{Corollary \ref{#1}}
%%  \newcommand{\thx}[1]{Theorem \ref{#1}}
\newcommand{\lxem}[1]{Lemma \ref{#1}}
\newcommand{\overx}[1]{\, \overline{ {#1} } \,} 


\newcommand{\el}[1]{Line (\ref{#1})}
\newcommand{\ex}[1]{Expression (\ref{#1})}
\newcommand{\ei}[1]{item (\ref{#1})}

\newcommand{\eq}[1]{(\ref{#1})}
\newcommand{\ep}[1]{Equation (\ref{#1})}
\newcommand{\thetlam}{ \theta }
\newcommand{\underx}[1]{\overline{~ {#1} ~}}
\newcommand{\appaa}{$App \forall$}
\newcommand{\appee}{$App \exists$}
\newcommand{\tll}[1]{Tab$- {#1} -$List}
\newcommand{\txl}[1]{Tab$- {#1}$}
\newcommand{\tlxl}[1]{Tab$- {#1}$ }
\newcommand{\sll}[1]{Short$- {#1} -$List}
\newcommand{\axx}[1]{NS$_D^{\,k,m}( ${#1}$)$}


\newenvironment{proof}{{\bf Proof:}}{$\Box$}
\newenvironment{sketchproof}{{\bf Sketch of Proof:}}{$\Box$}


\begin{document}


\title{On a 3-Part ``Tripod'' Styled Reply
  to Hilbert's Mysterious Second Problem}




\def\beq{\begin{equation}}
\def\enq{\end{equation}}

\def\bel{\begin{lemma}}
\def\enl{\end{lemma}}


\def\bec{\begin{corollary}}
\def\enc{\end{corollary}}

\def\bed{\begin{description}}
\def\ennd{\end{description}}
\def\bee{\begin{enumerate}}
\def\ene{\end{enumerate}}


\def\bxbxd{\begin{definition}}
\def\bxbxdd{\begin{definition}}
\def\eedd{\end{definition}}
\def\bxbxdr{\begin{definition} \rm}
\def\bel{\begin{lemma}}
\def\enl{\end{lemma}}
\def\ent{\end{theorem}}



\author{  Dan E.Willard }
%\thanks{This research 
%was partially supported
%by the NSF Grant CCR  0956495.}}

%Email = dew@cs.albany.edu.}}
%\newline
%Email = dan.willard.albany@gmail.com}}











%%%\date{Copyright 2012 by Dan E. Willard}

\date{State University of New York at Albany}

\maketitle

\setcounter{page}{0}
\thispagestyle{empty}

\normalsize




\baselineskip = 1.3\normalbaselineskip



\normalsize


\baselineskip = 1.0 \normalbaselineskip 
\def\bbint{\large \baselineskip = 1.6 \normalbaselineskip } 
\def\bbint{\large \baselineskip = 1.6 \normalbaselineskip }
\def\bbint{\normalsize \baselineskip = 1.3 \normalbaselineskip }



%%\baselineskip = 1.0 \normalbaselineskip 
%%\def\bbint{\large \baselineskip = 1.6 \normalbaselineskip } 
%%\def\bbint{\large \baselineskip = 1.6 \normalbaselineskip }
%%\def\bbint{\normalsize \baselineskip = 1.3 \normalbaselineskip }


\def\bbint{\normalsize \baselineskip = 1.27 \normalbaselineskip }



\def\bbint{\large \baselineskip = 2.0 \normalbaselineskip }


\def\bbint{\normalsize \baselineskip = 1.25 \normalbaselineskip }
\def\bbina{\normalsize \baselineskip = 1.24 \normalbaselineskip }



\def\bbint{\large \baselineskip = 2.0 \normalbaselineskip } 



\def\bbing{ }
\def\bbins{ }
\def\bbinm{ }

\def\bbint{\normalsize \baselineskip = 1.95 \normalbaselineskip } 



\def\bbing{ }
\def\bbins{ }
\def\bbinm{ }


\def\bbint{\large \baselineskip = 2.3 \normalbaselineskip } 
\def\bbing{ }
\def\bbins{ }
\def\bbinm{ }

\def\bbint{\normalsize \baselineskip = 1.7 \normalbaselineskip } 

\def\bbint{\large \baselineskip = 2.3 \normalbaselineskip } 
\def\bbinm{ \baselineskip = 1.18 \normalbaselineskip }


\def\bbint{\large \baselineskip = 2.0 \normalbaselineskip } 
\def\bbing{ }
\def\bbins{ }
\def\bbinm{ }
\def\bbinr{ }


\def\bbint{\normalsize \baselineskip = 1.25 \normalbaselineskip }
\def\bbina{\normalsize \baselineskip = 1.24 \normalbaselineskip }
\def\bbinr{ \baselineskip = 1.3 \normalbaselineskip }
\def\bbing{ \baselineskip = 1.28 \normalbaselineskip }
\def\bbins{ \baselineskip = 1.21 \normalbaselineskip }
\def\bbinm{  }

\def\ftl{ \baselineskip = 1.5 \normalbaselineskip }


\bbint

\parskip 5 pt




\noindent





\small


\baselineskip = 1.14 \normalbaselineskip 


 
\parskip 5pt

\baselineskip = 1.2 \normalbaselineskip 

%\setcounter{page}{0}



%%%%%%{\small








\large
\normalsize

 \baselineskip = 1.2 \normalbaselineskip 

%mmmmmmmmmmmmm


 




\begin{abstract}
\baselineskip = 4.2 \normalbaselineskip  
\Large
Hilbert's mysterious year-1900 Second Problem asked
mathematicians to devise a methodology whereby
Peano Arithmetic  can confirm its own consistency.
G\"{o}del's famous 1931 paper showed that a fully
positive reply can never be made to Hilbert's question.
This article will explain how Hilbert's question is
such a complicated issue that it can be better receive
a 3-way styled  ``Tripod'' reply.

\bigskip

We  also provide substantial evidence  that G\"{o}del would likely
agree with  the main opinions   expressed in this article.

\end{abstract}

\bigskip

\bigskip

\bigskip

\bigskip

 
 \normalsize
 {\bf Keywords and Phrases:}
 G\"{o}del's Second Incompleteness Theorem,  Hilbert's Second
 Problem, Consistency, Smullyan-Fitting Semantic Tableau Deduction,
 Hilbert-Frege Deduction.


\bigskip

\bigskip

%% 
%% {\bf Mathematics Subject Classification:}
%% 03B52; 03F25; 03F45; 03H13 
%% 
%% \bigskip


\def\ww22{\normalsize \baselineskip = 1.21\normalbaselineskip \parskip 4 pt}
\def\bb22{\normalsize \baselineskip = 1.19\normalbaselineskip \parskip 4 pt}
\def\zz22z{\normalsize \baselineskip = 1.19 \normalbaselineskip \parskip 3 pt}
\def\xx22{\normalsize \baselineskip = 1.17\normalbaselineskip \parskip 4 pt}
\def\vx22s{\normalsize \baselineskip = 1.16 \normalbaselineskip \parskip 3 pt} 
\def\vv22{\normalsize \baselineskip = 1.17 \normalbaselineskip \parskip 3 pt} 
\def\aa22{\normalsize \baselineskip = 1.15 \normalbaselineskip \parskip 3 pt} 
\def\g55{  \baselineskip = 1.0 \normalbaselineskip } 
\def\s55{ \baselineskip = 1.0 \normalbaselineskip } 
\def\sm55{ \baselineskip = 0.9 \normalbaselineskip } 
















\vspace*{- 1.0 em}


\def\waw11{\normalsize \baselineskip = 1.72\normalbaselineskip}
\def\waw11{\normalsize \baselineskip = 1.12\normalbaselineskip}
\def\waw11{\normalsize \baselineskip = 1.85\normalbaselineskip}



\def\waw11{\normalsize \baselineskip = 1.45\normalbaselineskip}


\def\waw11{\normalsize \baselineskip = 1.7\normalbaselineskip}

\def\waw11{\normalsize \baselineskip = 1.12\normalbaselineskip}


\def\g55{  \baselineskip = 1.50 \normalbaselineskip } 
\def\s55{ \baselineskip = 1.50 \normalbaselineskip } 
\def\sm55{ \baselineskip = 1.5 \normalbaselineskip } 


\def\g55{  \baselineskip = 1.50 \normalbaselineskip } 
\def\s55{ \baselineskip = 1.50 \normalbaselineskip } 
\def\sm55{ \baselineskip = 0.9 \normalbaselineskip } 






\def\aa22{\normalsize  \waw11 \parskip 6 pt} 
\def\bb22{\normalsize  \waw11 \parskip 5 pt}
\def\ww22{\normalsize \waw11 \parskip 4 pt}
\def\vv22{\normalsize  \waw11 \parskip 3 pt} 
\def\tt22{\normalsize  \waw11 \parskip 2 pt} 

\def\g55{  \baselineskip = 1.0 \normalbaselineskip } 
\def\b55{  \baselineskip = 1.0 \normalbaselineskip } 
\def\s55{ \baselineskip = 1.0 \normalbaselineskip } 
\def\sm55{ \baselineskip = 0.9 \normalbaselineskip } 






\def\mal{ \bf  }
\def\nal{\mathcal}

\def\cvrew{ \baselineskip = 1.6 \normalbaselineskip \parskip 3pt }

\def\ttt2c{ }
\def\tttc{ }

\def\tttc{\tiny \baselineskip = 0.8 \normalbaselineskip  \parskip 0pt }
\def\ttt2c{\tiny \baselineskip = 0.7 \normalbaselineskip  \parskip 0pt }
\def\tttc{ \baselineskip = 2.1 \normalbaselineskip  \parskip 5pt }
\def\ttt2c{ \baselineskip = 2.1 \normalbaselineskip  \parskip 5pt }

\def\tttc{ \baselineskip = 1.15 \normalbaselineskip  \parskip 5pt }
\def\ttt2c{ \baselineskip = 1.15 \normalbaselineskip  \parskip 5pt }


\def\tttc{ \baselineskip = 1.12 \normalbaselineskip  \parskip 4pt }
\def\ttt2c{ \baselineskip = 1.12 \normalbaselineskip  \parskip 4pt }


\def\tttc{ \baselineskip = 1.14 \normalbaselineskip  \parskip 3pt }
\def\ttt2c{ \baselineskip = 1.10 \normalbaselineskip  \parskip 2pt }
\def\ttt2c{ \baselineskip = 0.98 \normalbaselineskip  \parskip 0pt }

\def\cvt{ \baselineskip = 0.98 \normalbaselineskip }
\def\cv9{ \baselineskip = 0.99 \normalbaselineskip }
\def\cvs{ \baselineskip = 1.0 \normalbaselineskip }
\def\cvl{ \baselineskip = 1.0 \normalbaselineskip }
\def\cvh{ \baselineskip = 1.03 \normalbaselineskip }
\def\cvg{ \baselineskip = 1.00 \normalbaselineskip }


\def\cvt{ \baselineskip = 1.6 \normalbaselineskip }
\def\cv9{ \baselineskip = 1.6 \normalbaselineskip }
\def\cvs{ \baselineskip = 1.6 \normalbaselineskip }
\def\cvl{ \baselineskip = 1.6 \normalbaselineskip }
\def\cvh{ \baselineskip = 1.6 \normalbaselineskip }
\def\cvg{ \baselineskip = 1.6 \normalbaselineskip }
\def\cvb{ \baselineskip = 1.6 \normalbaselineskip }
\def\cvnew{ \baselineskip = 1.6 \normalbaselineskip }
\def\cvmew{ \baselineskip = 1.6 \normalbaselineskip }
\def\cvwew{ \baselineskip = 1.6 \normalbaselineskip \parskip 5pt }
\def\cvrew{ \baselineskip = 1.6 \normalbaselineskip \parskip 3pt }




\def\cvt{ \baselineskip = 1.22 \normalbaselineskip }
\def\cv9{ \baselineskip = 1.22 \normalbaselineskip }
\def\cvs{ \baselineskip = 1.22 \normalbaselineskip }
\def\cvl{ \baselineskip = 1.22 \normalbaselineskip }
\def\cvh{ \baselineskip = 1.22 \normalbaselineskip }
\def\cvg{ \baselineskip = 1.22 \normalbaselineskip }
\def\cvb{ \baselineskip = 1.22 \normalbaselineskip }
\def\cvnew{ \baselineskip = 1.4 \normalbaselineskip }
\def\cvmew{ \baselineskip = 1.35 \normalbaselineskip }
\def\cvwew{ \baselineskip = 1.4 \normalbaselineskip \parskip 5pt }
\def\cvrew{ \baselineskip = 1.22 \normalbaselineskip \parskip 3pt }


\def\cvt{ }
\def\cv9{ }
\def\cvs{ }
\def\cvl{ }
\def\cvh{ }
\def\cvg{ }
\def\cvb{ }
\def\cvnew{ } 
\def\cvmew{ }
\def\cvwew{ }
\def\cvrew{ }



\def\fend{ 

\medskip -------------------------------------------------------}


\def\g55{  \baselineskip = 1.0 \normalbaselineskip } 
\def\s55{ \baselineskip = 1.0 \normalbaselineskip } 
\def\sm55{ \baselineskip = 1.0 \normalbaselineskip } 
\def\h55{  \baselineskip = 1.08 \normalbaselineskip } 
\def\b55{  \baselineskip = 1.1 \normalbaselineskip } 

\normalsize

\baselineskip = 1.85 \normalbaselineskip 



%% Sleepy  

%\cvlpm %% Sleepy  
%\cvnew


%\small

%\parskip 0p

\parskip 2pt

\vspace*{- 1.0 em}

% \newpage

%\large

%\setcounter{page}{0}
\baselineskip = 1.04 \normalbaselineskip 
\parskip 2pt

\baselineskip = 0.96 \normalbaselineskip 
%\baselineskip = 0.90 \normalbaselineskip 

%\parskip 1pt
% 
\baselineskip = 2.16 \normalbaselineskip 
\baselineskip = 2.3 \normalbaselineskip 

\baselineskip = 0.95 \normalbaselineskip 
%\baselineskip = 0.95 \normalbaselineskip 
\baselineskip = 0.88 \normalbaselineskip 
\parskip 0pt
 


\noindent

% 
% 
% NNEW COMMENT
% 
% 
% The pdf version of this draft is verbatim identical to August's Version 3.
% The prior draft's abstract was incorrectly broadcast by Arxiv on the 
% Internet, after I pressed a wrong computer button. Thus, 
% Version 4 was issued.


\newpage

\def\gvs{ \normalsize \baselineskip = 1.4 \normalbaselineskip  \parskip    5pt}
\def\gvs{ \normalsize \baselineskip = 1.44 \normalbaselineskip  \parskip    5pt}
\def\gvs{ \large \baselineskip = 1.44 \normalbaselineskip  \parskip    5pt}
\def\gvs{ \normalsize \baselineskip = 1.44 \normalbaselineskip  \parskip    5pt}\def\gvs{ \normalsize \baselineskip = 1.74 \normalbaselineskip  \parskip    5pt}
\def\gvs{ \normalsize \baselineskip = 1.44 \normalbaselineskip  \parskip 5pt}

\def\gvs{   \baselineskip = 1.74 \normalbaselineskip  \parskip    5pt}

\def\gvs{ \normalsize \baselineskip = 1.44 \normalbaselineskip  \parskip 5pt}
\def\gvs{ \large \baselineskip = 2.0 \normalbaselineskip  \parskip 5pt}
\def\gvs{ \Large \baselineskip = 2.0 \normalbaselineskip  \parskip 5pt}
% \def\gvs{ \baselineskip = 2.0 \normalbaselineskip  \parskip 5pt}
\def\gvs{ \normalsize \baselineskip = 2.44 \normalbaselineskip  \parskip 5pt}
\def\gvs{ \normalsize \baselineskip = 2.04 \normalbaselineskip  \parskip 5pt}
\def\gvs{ \normalsize \baselineskip = 2.64 \normalbaselineskip  \parskip 5pt}
\def\gvs{ \Large \baselineskip = 1.6 \normalbaselineskip  \parskip 5pt}

%\def\gvs{ }


\gvs

\footnotesize


\def\gvs{ }
  

\normalsize \baselineskip = 0.98 \normalbaselineskip
\normalsize \baselineskip = 1.0 \normalbaselineskip
\normalsize \baselineskip = 1.01 \normalbaselineskip


\def\gvs{ \normalsize \baselineskip = 1.25 \normalbaselineskip  \parskip 4pt}

% \def\gvs{ \normalsize \baselineskip = 1.23 \normalbaselineskip  \parskip 5pt}

%\def\gvs{ \normalsize \baselineskip = 1.4  \normalbaselineskip  \parskip 6pt}

%\def\gvs{ \large \baselineskip = 1.5  \normalbaselineskip  \parskip 6pt}

\def\gvs{ \Large \baselineskip = 1.6  \normalbaselineskip  \parskip 6pt}
\def\gvs{ \normalsize \baselineskip = 1.6  \normalbaselineskip  \parskip 6pt}
\def\gvs{ \large \baselineskip = 1.6  \normalbaselineskip  \parskip 6pt}

\def\gvs{ \normalsize \baselineskip = 1.227 \normalbaselineskip  \parskip 3pt}
\def\gvs{ \large \baselineskip = 1.8  \normalbaselineskip  \parskip 6pt}

\def\gvs{ \normalsize \baselineskip = 1.5 \normalbaselineskip  \parskip 3pt}

% \def\gvs{ \large \baselineskip = 1.8  \normalbaselineskip  \parskip 6pt}

% \def\gvs{ \normalsize \baselineskip = 1.65 \normalbaselineskip  \parskip 3pt}

\def\gvs{ \large \baselineskip = 2.1  \normalbaselineskip  \parskip 6pt}

\def\gvs{ \normalsize \baselineskip = 2.1  \normalbaselineskip  \parskip 6pt}

%%%old 

 \def\gvs{ \normalsize \baselineskip = 1.227 \normalbaselineskip  \parskip 3pt}

 \def\gvs{ \large  \baselineskip = 1.6 \normalbaselineskip  \parskip 5pt}
%% march 31

\def\gvs{ \Large  \baselineskip = 1.8 \normalbaselineskip  \parskip 5pt}
\def\gvs{ \LARGE  \baselineskip = 1.8 \normalbaselineskip  \parskip 5pt}
\def\gvs{ \normalsize  \baselineskip = 2.0 \normalbaselineskip  \parskip 5pt}

\def\gvs{ \Large  \baselineskip = 2.0 \normalbaselineskip  \parskip 5pt}

\def\gvs{ \large  \baselineskip = 2.2 \normalbaselineskip  \parskip 5pt}

\def\gvs{ \normalsize \baselineskip = 2.4  \normalbaselineskip  \parskip 6pt}

\def\gvs{ \normalsize \baselineskip = 2.6  \normalbaselineskip  \parskip 6pt}
\def\gvs{ \normalsize \baselineskip = 2.2  \normalbaselineskip  \parskip 6pt}
\def\gvs{ \normalsize \baselineskip = 1.8  \normalbaselineskip  \parskip 5pt}
% \def\gvs{ \normalsize \baselineskip = 1.5  \normalbaselineskip  \parskip 5pt}

% \def\sgvs{ \small \baselineskip = 1.33  \normalbaselineskip  \parskip 1pt}
\def\tttc{ }
%\baselineskip = 1.14 \normalbaselineskip  \parskip 4pt }
%\baselineskip = 1.14 \normalbaselineskip  \parskip 4pt }

\def\gv2{ \normalsize \baselineskip = 1.30  \normalbaselineskip  \parskip 3pt}



\def\gvs{ }


\def\gvs{ \normalsize \baselineskip = 2.1 \normalbaselineskip  \parskip 7pt}
\def\gvs{ \normalsize \baselineskip = 1.8 \normalbaselineskip  \parskip    7pt}

%\def\gvs{ \normalsize \baselineskip = 1.4 \normalbaselineskip  \parskip    5pt}

 \def\gvs{ \large \baselineskip = 1.7  \normalbaselineskip  \parskip 9pt}
\def\gvs{ \normalsize \baselineskip = 2.0  \normalbaselineskip  \parskip 9pt}

\def\gv2{ \normalsize \baselineskip = 1.30  \normalbaselineskip  \parskip 3pt}


\def\gvs{ \large \baselineskip = 1.7  \normalbaselineskip  \parskip 5pt}


\def\gvs{ \normalsize \baselineskip = 2.0  \normalbaselineskip  \parskip 8pt}
\def\gvs{ \large \baselineskip = 2.0  \normalbaselineskip  \parskip 8pt}


%fffff


\def\gvx{ \large \baselineskip = 1.6 \normalbaselineskip  \parskip    3pt}

%fffff
\def\gvs{ \Large \baselineskip = 1.8 \normalbaselineskip  \parskip   5pt}

\def\gvs{ \LARGE \baselineskip = 2.0 \normalbaselineskip  \parskip   5pt}

\def\gvs{ \normalsize \baselineskip = 2.0 \normalbaselineskip  \parskip   5pt}
\def\gvs{ \normalsize \baselineskip = 1.4 \normalbaselineskip  \parskip   5pt}

 \def\gvs{ \large \baselineskip = 2.0 \normalbaselineskip  \parskip   5pt}
 \def\gvs{ \Large \baselineskip = 2.0 \normalbaselineskip  \parskip   5pt}
\def\gvs{ \normalsize \baselineskip = 2.0 \normalbaselineskip  \parskip   5pt}
\def\gvs{ \normalsize \baselineskip = 2.5 \normalbaselineskip  \parskip   5pt}

\def\gvs{ \normalsize \baselineskip = 1.4 \normalbaselineskip  \parskip   5pt}
\def\gvs{ \large      \baselineskip = 1.7 \normalbaselineskip  \parskip   5pt}
\def\gvs{ \normalsize \baselineskip = 1.7 \normalbaselineskip  \parskip   5pt}


% \def\gvs{ \normalsize \baselineskip = 1.07 \normalbaselineskip  \parskip =  5pt}

%\def\gvs{ \large \baselineskip = 2.4 \normalbaselineskip}
%\def\gvs{ \normalsize \baselineskip = 3.1 \normalbaselineskip}

%\def\gvs{ \normalsize \baselineskip = 3.4 \normalbaselineskip}

%\def\gvs{ \normalsize \baselineskip = 1.07 \normalbaselineskip  \parskip  5pt}


%\def\gvs{ \normalsize \baselineskip = 2.07 \normalbaselineskip}

\def\gvs{ \large \baselineskip = 1.6 \normalbaselineskip}

\def\gvs{ \normalsize \baselineskip = 1.8 \normalbaselineskip}

\def\gvs{ \Large \baselineskip = 1.8 \normalbaselineskip}

\def\gvs{ \normalsize \baselineskip = 2.2 \normalbaselineskip}
\def\gvs{ \normalsize \baselineskip = 2.7 \normalbaselineskip}
%\def\gvs{ \normalsize \baselineskip = 2.0 \normalbaselineskip}

\def\gvs{ \normalsize \baselineskip = 2.9 \normalbaselineskip}
\def\gvs{ \normalsize \baselineskip = 3.0 \normalbaselineskip}

\def\gvs{ \normalsize \baselineskip = 2.0 \normalbaselineskip}
\def\gvs{ \normalsize \baselineskip = 1.7 \normalbaselineskip}

\def\gvs{ \normalsize \baselineskip = 1.07 \normalbaselineskip}
\def\gvs{ \normalsize \baselineskip = 1.02 \normalbaselineskip}
\def\gvs{ \normalsize \baselineskip = 1.0 \normalbaselineskip}
\def\gvs{ \normalsize \baselineskip = 1.01 \normalbaselineskip}

\def\nvs{ \Large \baselineskip = 2.5 \normalbaselineskip}

\def\nvs{ \normalsize \baselineskip = 1.5   \normalbaselineskip}

%

% \def\nvs{ \large \baselineskip = 1.8   \normalbaselineskip}

%%%\def\nvs{ \normalsize \baselineskip = 1.10   \normalbaselineskip}

\def\nvs{ \normalsize \baselineskip = 1.08   \normalbaselineskip}

% \def\nvs{ \normalsize \baselineskip = 2.2   \normalbaselineskip}
\def\nvs{ \normalsize \baselineskip = 0.96   \normalbaselineskip}

 % fffffffff



\label{ss11}
\gvs
\parskip 3pt
\tttc

\def\dvs{ \normalsize \baselineskip = 2.5   \normalbaselineskip}
\def\dvs{ \normalsize \baselineskip = 3.0   \normalbaselineskip}
 \def\dvs{ }
\newpage

\section{Introduction}
%  11111111111111111 }

\label{aaa1}
\nvs
\baselineskip = 1.02   \normalbaselineskip
\parskip 4pt
\tttc
\dvs

This article is a continuation of a series of papers
that began
with the 1993 article \cite{ww93} 
and continued until and through the year-2021 article \cite{ww21}.
This series,
which included six papers appearing in the JSL and APAL,
had focused on discussing
generalizations and boundary case exceptions for the
Second Incompleteness Theorem.
The two goals of this paper will be to explore the underlying
philosophy that motivated this series and to explain how
it is related to several generalizations and
boundary-case exceptions to the
Second Incompleteness Theorem , including some important new
results formalized by Artemov  \cite{Ar19,Ar21}.


Our general theme will be that Hilbert's year-1900 Second Problem
 is too complex an issue to receive a 1-directional or
 even 2-directional
 answer. Instead, it will require a
 3-directional answer, called the ``Tripod Reply'' to
 Hilbert's question. 


 
 The  first leg of this 3-part response to
 Hilbert's Second Problem
will rest on the combination
 of G\"{o}del's initial version of his Second Incompleteness
 Theorem and its numerous generalizations.
 % cite????.
 They
collectively
establish that 
many logical formalisms, besides Peano Arithmetic (PA),
are unable to corroborate their own consistency
in a fully extensive respect.
While these generalizations of G\"{o}del's Second 
Incompleteness paradigm are indisputably important,
the second section of this article will explain
%% how
Hilbert and G\"{o}del
strongly doubted they constituted a full answer to  
  Hilbert's year-1900 problem.

%%% citations in above paragraph MAYBE NOT? PROBABLY NOT NOT NOT NOT
  
 A second leg of a 3-part reply to Hilbert's Second
Problem
 was formalized by  
 Artemov
            recently
 in \cite{Ar19,Ar21}.
 He noted Peano Arithmetic (PA) can prove
 an infinite
%  -sized
 schema of
 theorems, whose collective
 union
 confirms PA's own consistency.
Let us call Artemov's method a
Step-By-Step Infinite-Schema  approach  
 {\bf ($~$SBSIS$~$)}.
This technique is an extension of the 
 Justification Logics explored by Artemov and Beklemishev 
 \cite{Ar1,AB5,Be5}. It was  motivated by
Artemov's
 observation
 that
the year-1900 logic community (including Hilbert)
were not aware  that PA would require an
 infinite number of proper axioms.
$~$Thus, an  SBSIS schema-driven logic approach
is a valid reply to Hilbert's year-1900 second problem,
although its potential significance
was not
well appreciated during the
era when Hilbert  posed his Second  Problem.

Artemov's analysis
\cite{Ar19,Ar21} 
uses Tarski's partial definitions
of truth for sentences with a bounded number of quantifiers
as an intermediate step. It essentially
constructs a sequence of finite subsets of
Peano Arithmetic $~S_1~\subset ~S_2~\subset ~S_3~\subset~ ...~$
where
\bee
\item
$~~~$PA$~=~~S_1~\cup ~S_2~\cup ~S_3~\cup~ ...~$
\item
\label{step2}
  Each $~S_{j+1}~$ can prove
a $\Pi_1$ theorem asserting
  there exists no proof from
  $~S_j~$
  of 0=1, in a context where all  logical axioms
  in the concerned proof from $~S_j~$
are no
  more complex than $\Pi_j$
  or $\Sigma_j$ statements. (We will henceforth
  call this theorem $T_{j+1}$. )
  \ene
While  
Artemov's SBSIS-style response to Hilbert's
year-1900 second question
is intriguing,
one would ideally still like access to 
a system that does not rely upon his infinite series
$S_1 \, S_2 \, S_3  ...~$.

\nvs

\dvs


\parskip 9 pt

A third possible leg  of a proposed Tripod
reply to Hilbert's second problem  
involves formal systems using Fixed Point axiomatic
sentences that confirm their own consistency.
For example,
% the articles
\cite{ww93,ww1,ww5,wwapal,ww21}
examined systems
strictly weaker than PA,
that 
 verified their own self-consistency,
under mostly semantic tableau deduction
\cite{Fi96,Sm95}.



This third leg,
called the {\bf Declarative Approach},
rests on using a self-referencing 
{\it ``I am consistent''} axiomatic declaration,
so that a formalism can confirm its own consistency.
This approach, studied in
\cite{ww93,ww1,ww5,wwapal,ww21},
will  essentially be defined by
the current paper's statement $~\oplus~$.
Its advantage is
that its declaration of self-consistency is compressed into
one single sentence,
while its {\bf non-trivial drawback} is that its particular  statement $~\oplus~$
(defined in \textsection\ref{aaa3})
can be successfully used  only when the 
surrounding formalism is sufficiently weak.

In other words, each of the three legs of a unified ``Tripod''
response to Hilbert's Second Problem
% do
own drawbacks, as well as significant
virtues. A 
multi-facet
Tripod reply is attractive
because it 
straddles nicely between
%these
all  three
legs, simultaneously.


\section{The Main Weakness of the First Leg}

%%%% 2222222222222 

\label{aaa2}

It is  known  Hilbert always suspected  the
Second Incompleteness Theorem, while correct, would
ultimately  display established exceptions.
For instance, Hilbert \cite{Hil26} wrote:
\begin{quote}
\small
\baselineskip = 1.0 \normalbaselineskip
\ttt2c 
$*~$
{\it ``
Let us admit that the situation in which we presently
find ourselves with respect to paradoxes is in the long
run intolerable. Just think: in mathematics, this paragon of
reliability and truth, the very notions and inferences,
as everyone learns, teaches, and uses them, lead to absurdities.
And 
where 
else 
would 
reliability and truth be found 
if  even mathematical thinking fails?''}
\end{quote}
Also, it is known 
G\"{o}del's seminal 1931 paper appeared to
agree with
the goals of Hilbert's
Consistency Program
in one of its closing paragraphs:
\begin{quote}
\small
\ttt2c 
$~**~~$ 
{\it ``It must be expressly noted that
Theorem XI''}
(i.e. the Second Incompleteness Theorem) 
{\it ``represents no contradiction of the formalistic
standpoint of Hilbert. For this standpoint
presupposes only the existence of a consistency
proof by finite means, and {\it there might
conceivably be finite proofs} which cannot
be stated in P or in ... ''}
\end{quote}
Several 
biographies
of 
 G\"{o}del
\cite{Da97,Go5,Yo5}
have observed 
 G\"{o}del's 
 intention (prior to 1930)
was to
establish
Hilbert's proposed objectives, before
he developed
his famous
nearly
opposing theorem. 
Indeed,
 Yourgrau's
biography
 \cite{Yo5}
 of G\"{o}del 
had traversed beyond this point.
It
recorded
 how
von Neumann 
found it necessary during the early 1930's to
{\it ``argue 
against G\"{o}del 
himself''}
 about the definitive 
 termination of Hilbert's
consistency program,  
which
{\it ``for several years''} after \cite{Go31}'s publication,
G\"{o}del 
{\it ``was cautious not to prejudge''}.

In a context where
G\"{o}del
published fewer than 85 pages during his
career,
historians 
will
probably
never
% be able to
fully characterize
the nature of
G\"{o}del's
finished  philosophical position about his Second Incompleteness
Theorem. From informal notes taken during a 1933 
 Vienna
 lecture  \cite{Go33}, it is known
 % that 
 G\"{o}del was more positive about its
% long-range
 implications
in 1933 than he was in 1931.

On the other hand, Gerald Sacks   recorded
a
% quite
stunning
80-minute YouTube recollection 
about his interactions with
G\"{o}del in the year 2007.
These recordings
explicitly  
state that
G\"{o}del
held  {\it ``contrarious''} opinions about some of
his 
most
famous results.
Also Sacks explicitly recalled (see footnote
\footnote{These two quotations
can be found in
  the 7-th and and 9-th minutes of the YouTube recording
  \cite{YouSa14}
  that Gerald Sacks had made about Kurt
  G\"{o}del.
  } )
G\"{o}del  
  communicating
private
opinions
% to Sacks
that were
{\it ``almost the opposite of what every one
else would have expected''.}

These remarks by Sacks deserve to be taken
seriously,
given that Sacks 
interacted  twice
with  G\"{o}del
at the Institute of Advanced Studies
(once in 1959-1960 and a second time
during the 1970s). Moreover, Sacks went beyond the preceding quoted
remarks.
Other
memorable comments  from
 \cite{YouSa14}'s
  YouTube lecture
are that:
\bed
\item{   a)  } 
 G\"{o}del
 {\it ``did not  think''}
the objectives of Hilbert's Consistency Program 
{\it ``were erased''} 
by
the Incompleteness Theorem,
\item{   b)  } 
G\"{o}del believed (according to Sacks) 
 it left
  Hilbert's program 
{\it ``very much alive and
even more interesting than it initially was''}. 
\ennd
Also,  Nerode \cite{Ne20}    
indicated in private communications that
Stanley Tennenbaum
shared similar conversations 
with G\"{o}del, as those remembered by
Sacks.
These
conversations
%%%  with  Tennenbaum also, apparently, 
%% expressed
likewise noted
the
need for some type of revival of
Hilbert's
Consistency Program.

Thus,
 the last two paragraphs,
have
summarized the quandary
that Symbolic
Logic has faced since 1931.
They suggest some type of
 Tripod-like response
to Hilbert's  Second  Problem
may become necessary to formulate
a more comprehensive
response to Hilbert's question.

\section{Starting Notation with the WCB \& USEGR Paradigms}

%%  3333333
  \label{aaa3}

  We will summarize the contents of the prior articles
  \cite{wwapal,ww21} so it will be
 unnecessary to read these papers.
The particular annotated paragraphs in
\cite{ww21} had used
decimal reference  numbers
  such as ``Example 3.1'' or ``Definition 3.2''. A non-decimal paragraph
  notation will be used in the current paper,
thus having its
  first  annotated paragraph instead
called ``Definition 1''. This type of
adjusted notation
should
 cross-reference the prior paper \cite{ww21}
in a manner that avoids
any ambiguity and/or
  confusion.

  
\begin{deff}
  \label{defx1}
\rm  
An
ordered pair $(\alpha,D)$ is called a
{\it Generalized Arithmetic Configuration}
(abbreviated as a {\bf ``GenAC'' })
when  its 
first and second 
components 
are 
defined 
as 
follows:
\bee
\item
The {\bf Axiom Basis} ``$~\alpha~$'' 
for a 
 GenAC
is defined as
its set of
 proper axioms. 
\item
The second component 
 ``$\, D \,$''
  of a 
 GenAC,  called its 
 {\bf Deductive Apparatus},
is
defined as
the union of its
 logical axioms ``$\,L_D$'' with its
        rules for obtaining inferences.
\ene
\end{deff}    

The Example 3.1 from \cite{ww21} provided several examples of
GenAC formalisms. Its Definition 3.2 defined a GenAC $(\alpha,D)$
to be {\bf ``Self-Justifying''} when:
\begin{description}
  \item[  i.   ] one of  $~(  \alpha  , D  )$'s  theorems
(or possibly one of $\alpha$'s axioms)
states that the deduction method $ \, D, \, $ applied to the
basis
system $ \, \alpha, \, $ 
produces a consistent set of theorems, and
\item[  ii.   ]
     the GenAC formalism $ \,( \alpha,D)  \, $ is
actually, in fact,
 consistent.
\end{description}

\begin{exx}
  \label{ex3}
\rm  
For any  $\,(\alpha,D) \,$, 
it is
easy
to construct a 
system $ \, \alpha^D \, \supseteq  \,  \alpha  \, $
 that  satisfies
the
Part-i 
condition
in an isolated context  where the Part-ii condition is
 not also
satisfied.
Thus,
  $ \, \alpha^D \, $  can
consist of all of $~\alpha \,$'s axioms plus 
the added {\bf $\,$``SelfRef$(\alpha,D)$''$\,$} sentence,
defined below: 
\begin{quote} 
$\oplus~~~$ 
There is no proof 
(using 
$D$'s deduction method)
of  $0=1$
from the  {\it union}
 of
the
 axiom system $\, \alpha \, $
with {\it this}
sentence  ``SelfRef$(\alpha,D) \,$'' (looking at itself).
\end{quote}
Each of
Kleene and Rogers
\cite{Kl38,Ro67}
noticed
how
to
encode
 analogs of 
SelfRef$(\alpha,D) \,$'s  above statement,
which we will often 
 call an 
 {\bf $\,$``I AM CONSISTENT'' 
 axiom.}
 The catch is
%%%  that
$\alpha ^D$ 
may
be inconsistent
(e.g. 
violate
 Part-ii of   self-justification's
definition
 despite
the assertion in
 SelfRef$(\alpha,D)$'s 
particular
statement).
This is because if the 
 pair $(\alpha,D)$ is too strong
then a
quite conventional
G\"{o}del-style diagonalization argument can
be applied to the axiom basis of
$~\alpha^D~=~ \alpha \, + \, $ SelfRef$(\alpha,D), ~$
where the added presence of the statement 
SelfRef$(\alpha,D)$ 
will cause this extended version of 
$\, \alpha\,$, ironically,
 to
 become automatically inconsistent.
 Thus, while an
 encoding for statement $~\oplus \,$'s {\it ``I am consistent''} declaration 
 is a  routine consequence of the Fixed Point Theorem,
its computational implementation is
 potentially
devastating.

% consequences.
 
\end{exx}

\begin{definition}
\label{defy3}
\rm
Let
 $Add(x,y,z)$ and    $Mult(x,y,z)$ 
denote two 3-way predicates  specifying 
 $x+y=z$ and $x*y=z$.
Let us say
 a formalized
system of
$~\alpha~$
recognizes successor,  addition  and multiplication
as {\bf Total Functions} iff 
it can  prove all of
\eq{totdefxs} - \eq{totdefxm}
as theorems:
\end{definition}
{ \small
\baselineskip =  .9 \normalbaselineskip 
\beq 
\label{totdefxs}
\forall x ~ \exists z ~~~Add(x,1,z)~~
\enq
\beq 
\label{totdefxa}
\forall x ~\forall y~ \exists z ~~~Add(x,y,z)~~
\enq
\beq 
\label{totdefxm}
\forall x ~\forall y ~\exists z ~~~Mult(x,y,z)~
\enq }
\noindent
Also a GenAC system $(\alpha,D)$ will be called
{\bf Type-M} 
formalism
iff it proves
\eq{totdefxs} - \eq{totdefxm}
as theorems, {\bf Type-A} if it proves
only \eq{totdefxs} and \eq{totdefxa},
and
it will be called
 {\bf Type-S} if it proves
only \eq{totdefxs} as a
 theorem. 
Furthermore,
 $(\alpha,D)$ 
 will be 
called 
{\bf Type-NS} iff it proves
none of \eq{totdefxs} - \eq{totdefxm}. 




%% bbbbbbbbbbbbbbb  DONE AND DELETE

\bigskip
{\bf BROADER OUTLOOK:$~$}
The remainder of this chapter will step backwards 
roughly 1,500 years in time, and explore the relationship
between Definitions 1 and \ref{defy3}  with
India's famous 
``Wheat-and-Chess-Board''  paradigm (often denoted 
as the
{\bf ``WCB''} phenomenon.)

This paradigm will
 help clarify the persistent confusion that has
 surrounded G\"{o}del's
   Second  Incompleteness
   Theorem. It will also help us 
formulate a new interpretation for the
   Remarks $*$ and $**$ that
Hilbert and  G\"{o}del had made.


 According to Wikipedia,
 scholars are
 %actually
 uncertain about exactly when the WCB fable had
 emerged in ancient India. (It may possibly
 have originated as early as 400-600 AD.)

 % but  scholars have debated the accuracy of such a date.)



 Under one variant of the WCB fable, a Brahmin named
 Sissa ibn Dahir invented
a type of
 Indian predecessor
to the game of Chess. At some
subsequent
 juncture, the king
 of Sissa's province
 offered to award him
a prize 
for the  game
that
 Sissa had invented. Sissa
 replied that it would be sufficient to place one
 grain of wheat on the first square of a Chess Board,
 two grains
 on its second square, four grains on its third square, etc.

 There are several versions of the Sissa fable, and no one knows
 which (if any ?) is accurate. One variant  involves
 the king executing Sissa after  discovering the last
chess-board
 square would
 require  $~2^{63}~$ grains of wheat. Another
 variant concludes with
the
 king smiling upon realizing the nature
of Sissa's 
 puzzle and declaring that this
puzzle
is  more fascinating
and stimulating
than the game of chess (itself).

 While many  precise details have been lost over time,
 the underlying 
message of the WCB  
 fable has survived for  roughly
 1,500 years.
This is because it is thought
to
 convey an instructive lesson about
the unsustainable
nature of an
 exponential growth.

 For instance,
 the 64 squares
of a Chess Board
has been found to 
require  an amount of wheat that is
 a factor of 1,000 greater than the world's full
 %  annual
 year-2019 production.
 Moreover, a direct analog of the WCB paradigm, involving
a few hundred doubling operations,  will require more  wheat
than there are  atoms in the  Universe.
 
Our point is that this fable would not have survived the
millennial test of time if there were not many examples in human
history where individuals
have
accidentally got involved in
exponential growth processes that started at a gradual rate
but then ran wild.  One of these examples seems to concern 
Hilbert's statement $~*~$ (i.e. see Section \ref{aaa2} ).
It appears to  ask   mathematicians to  perform
an
analog of Sissa's
unsustainable  WCB task.

This is
because if one seeks to perform
several hundred squarings of the number 2 than its
binary encoding will own more zero-digits than there are
atoms in the universe. This can be appreciated by comparing
the sequences 
$   x_0,~   x_1,~   x_2,~     ...    $ 
and  $   y_0, ~  y_1, ~  y_2,     ...  $
     defined below:
\vspace*{- 0.7 em}
\beq
\label{zs}
x_0~~~=~~~2~~~=~~~y_0
\enq
\beq
x_{i}~~~=~~~x_{i-1}~+~x_{i-1}
\label{as}
\enq
\beq
y_{i}~~~=~~~y_{i-1}~*~y_{i-1}
\label{bs}
\enq
For $\, i \, > \,0 \,$, 
$\,$let $ \, \phi_{i} \, $ 
and $ \, \psi_{i} \, $ 
denote the  
sentences in 
\eq{as} and \eq{bs}
respectively.
Then if
  $ \, \phi_{0} \, $ and
$ \, \psi_{0} \, $ 
denote \eq{zs}'s
sentence, it is apparent that
 $ \, \phi_0, \, \phi_1, \, ... \, \phi_n \, $
imply
 $ \, x_n \, = \, 2^{ n+1} \, , \, $ while 
 $ \, \psi_0, \, \psi_1, \, ... \, \psi_n \, $
 imply $ \, y_n \, = \, 2^{2^n} \, $.
Thus, the  latter sequence
will grow
at an
exponentially 
faster
rate than 
the former.
(E.g. the respective  quantities of
Log$_{ \, 2 \,}(\, y_n \,) \, = \, 2^{n} \, $ 
and
Log$_{ \, 2  \,}(\, x_n \,) \, = \, {n+1} \, $ 
 represent
the  lengths for the binary codings 
for
$ \, y_n \, $
 and 
$ \, x_n \, $.)

In other words,
 $ \, y_n\,$'s 
     binary     
encoding 
will have a length
$\,  2^{n} \,  $,
roughly analogous to the n-th square belonging to
the WCB's chess board paradigm.
Leaving aside technical details that were largely
explored in \cite{ww21} and shall be briefly
reviewed later, we are suggesting that
Hilbert's statement $*$ is partially analogous to
 Sissa's WCB paradigm.

The latter is not meant to deny that there are some intriguing
interpretations of the Hilbert and G\"{o}del statements $*$ and $**$
deserving, certainly, partial sympathy.
This is because there are other types of 
amended axiomatizations of arithmetic that avoid 
\el{bs}'s malignant assumption that multiplication is
a total function.

The theme  of this article is, thus,  largely that one needs to be
more flexible when approaching the 
 Hilbert and G\"{o}del statements $*$ and $**$.
Our proposed ``Tripod Reply'' to
 Hilbert's Second Problem 
 will thus be
 a gentle      hybrid-styled reply, that
accepts   $*$ and $**$ half-way and
 views Hilbert's Second Problem
from all three Tripod  perspectives
summarized in Section \ref{aaa1}.


  
\begin{deff}
  \label{defx4}
 \rm 
During the remainder of this article, the acronym
{\bf ``USEGR''} will refer to an {\it Unsustainable Exponential Growth Rate},
similar
to the dizzying number of zero digits
that was produced by \el{bs}'s sequence of
  $   y_0, ~  y_1, ~  y_2,   ~  ... ~ $.
We will argue that such sequences resemble 
Sissa's WCB paradigm,
and a mathematical realist should avoid them as much as
feasible.
\end{deff}




Our theme will be that a
1-way response to Hilbert's Second Open Problem
was observed  by  Sacks to be 
questionable
when his remarks 
(a) and (b)  
(from Section \ref{aaa2})
noted G\"{o}del viewed  the
 Second Incompleteness Theorem
as essentially 
a too
pessimistic reply
 to Hilbert's second problem.
Also,
 a broader
 2-way response is inadequate  because the various
self-justifying formalisms that Willard developed
between the times of \cite{ww93}'s  initial 1993
%%%  announcement
presentation
and that of  \cite{ww21}'s
recent  LFCS-2020 paper
are weaker 
than would be  preferred.

Artemov's third type of response
uses
his elegant 
infinite ranged SBSIS-styled
exceptions \cite{Ar19,Ar21}
to the Second Incompleteness Theorem. It is 
almost
%% very close to
ideal.
Yet, it also  has drawbacks
(see footnote\footnote{Section 
   \ref{aaa1}    explained how
   \cite{Ar19,Ar21} also had
some
   hidden difficulties. They are
  that Peano Arithmetic (PA)
  can  verify only the consistency of finite subsets
  of itself. Thus,
\cite{Ar19,Ar21}'s 
  perspective 
  is
  useful, but it does not
fully
  explain how PA
  can formalize the consistency of its
  infinitely drawn
  expanse.
  In this context, we shall argue 
  a hybridized
  Tripod-style response is a
more
 comprehensive
reply to
  Hilbert's Second  Problem.} )
because its  SBSIS approach
is a mixture of tempting positive results and compromises.
This
is the  reason the current article advocates using
several perspectives simultaneously,
under
our  hybrid-styled 3-part
``Tripod''
reply to Hilbert's
120 year-old problem. 

\section{Characterizations of Several Base Languages}

    %%% 4444444444444444

  \label{aaa4}
% \label{sect4}

  Sissa's 1,500-year old ``WCB'' paradigm,
as well as
  Definition \ref{defx4}'s   
  ``USEGR'' effect
have  similar themes.
This is because both involve
unsustainable exponential growth processes,
having   often
unwelcome and
unanticipated effects.
In particular, their circumstances can be made applicable
to essentially all the
known variants of the Second Incompleteness Theorem
\cite{AZ1,Ar1,Be14,BS76,Bu86,BI95,Fe60,Fr79a,Ha11,HP91,KT74,Pa71,PD83,Pa72,Pu85,Pu96,So88,So94,Sv7,Vi5,WP87,ww1,ww2,ww7,wwapal,ww16}.  
% cite?????????.
One  
of its
most
interesting variants
arose during a  1994
private telephone conversation
with Robert Solovay
\cite{So94}.
% that he chose never to publish.
Its formalism  (due to Solovay)
can be thought of as a generalization
  of a methodology that was initially developed by Pudl\'{a}k in
  \cite{Pu85} and which was further explored
by  Nelson and Wilkie-Paris in \cite{Ne86,WP87}.
%%% Essentially,  it is
It essentially amounts to
the following observation:
\begin{description}
\item{\bf Theorem ++ $~$ : }
{\it 
$~~~$
(Solovay's  
modification
\cite{So94}
of Pudl\'{a}k \cite{Pu85}'s formalism 
using some of 
Nelson and Wilkie-Paris \cite{Ne86,WP87}'s
methods)} :
Let 
$ \, (\alpha,D) \, $ 
denote 
a 
Type-S
GenAC system
which assures
the successor operation
will
% provably
satisfy
both 
 $  \,   x'     \neq 0     $ and
$     x'     =     y' \Leftrightarrow x=y $.
$~$Then
$ \, (\alpha,D  ) \, $  
cannot verify its own
consistency
whenever
simultaneously
 $D$ is some type of
a Hilbert-Frege
deductive
apparatus and
$~\alpha~$
 treats addition and multiplication
as 3-way relations, 
satisfying 
their usual % identity,
associative, commutative, 
 distributive 
and identity 
axioms.
\end{description}
%\end{quote}
\medskip
Essentially, Solovay \cite{So94}
had
privately communicated to Willard  
an approximate analog of Theorem $++$.
(This communication was similar to
several
other often-privately-communicated comments 
that the
% \newpage
% \noindent
literature  \cite{BI95,HP91,Ne86,PD83,Pu85,WP87} 
has often attributed to 
Robert Solovay's unpublished observations.)
It also should be mentioned that
partial
analogs of 
 $++$'s statement
 were  explored 
subsequently
 by  Buss-Ignjatovi\'{c},
H\'{a}jek 
and
\v{S}vejdar in \cite{BI95,Ha11,Sv7},
as well as in Appendix A of 
the paper
\cite{ww1} 
and in \cite{wwapal}.

Furthermore, we stress that
Pudl\'{a}k's initial 1985 article  \cite{Pu85} 
had captured
the majority 
of $++$'s 
implications, chronologically before Solovay's observations.
Also,
Friedman did nicely
 related
work as early as 1979
 in
\cite{Fr79a}.

\bigskip

In order to explain how
Sissa's 1,500-year old ``WCB'' paradigm
and Definition \ref{defx4}'s similar  ``USEGR'' effect
are related to  this material, it will be useful to
employ
Definition \ref{defx5}'s notation. (Its predecessor
can be found 
in \cite{wwapal}, but the latter did not discuss the
 ``USEGR'' effect.)

\begin{deff}
\label{defx5}
\rm
A  function $ F(a_1, \ldots , \, a_j)$
is called
a {\bf Non-Growth} operation
when 
$ F(a_1, \ldots , \, a_j) 
\leq  Maximum(a_1 , \ldots ,  \, a_j)$
holds.
Seven  examples of  
non-growth functions are:
\bee
 \parskip 0pt
\small 
 \baselineskip = 0.70 \normalbaselineskip
\item
{\it Integer Subtraction} 
(where $~x-y~$ is defined to equal zero when
 $~x \leq y~),~~$
\item
{\it Integer 
Division}
(where $x \div y$ 
equals
$~x~$ when $y=0$, and
it equals $~\lfloor ~x/y ~\rfloor~$ otherwise),
\item
$~Maximum(x,y),$
\item
$~ Log_{ \, \spadesuit \, }(x)~$ 
 which
is an abbreviation for
$~\lceil~$Log$_2(~x+1~)~\rceil~~$ under 
the conventional
notation.
%%% 
%%% 
%%% (The  footnote 
%%% \footnote{
%%% The H\'{a}jek-Pudl\'{a}k textbook \cite{HP91} uses the 
%%% notation ``$~\mid  x  \mid~$'' 
%%%  to 
%%% designate
%%%  what we shall call ``$~ Log_{ \, \spadesuit \, }(x)~$''
%%% Thus for $~x \geq 1 \,$, 
%%%  $~ Log_{ \, \spadesuit \, }(x)~$
%%% denotes
%%% the number of symbols that will 
%%% encode the number $~x\,$, when it is written 
%%% in  a binary format. }
%%% explains the 
%%% significance
%%% of this
%%% concept.)
%%% 
%%%
\item
$\,~Root(x,y) \, =  \, \lceil  \, x^{1/y} \,  \rceil$ 
\item $~Count(x,j)$   designating the number of
physical
 ``1'' bits
stored among $    \,  x$'s rightmost $    \, j    \,  $ bits.
\item 
  $~Bit(x,i)~$
  designating  the $i-$th rightmost bit
  of the string $x$ (as explained by footnote \footnote{ The 
$~Bit(x,i)~$ operator is technically unnecessary
because it can be encoded as:
$\mbox{Bit}(x,i)~~=~~   \mbox{Count}(x,i) ~-~  \mbox{Count}(x,i-1)$.}
 $~$.)
\ene  
These function were called 
{\bf Grounding Functions} in \cite{wwapal}.
Also, $L^0  \,$ will denote a quite weak
language 
built from the Grounding functions, together with
three constant symbols $c_0$,  $c_1$ and   $c_2$
(representing  0, 1 and 2) and 
the   ``$ \, = \, $''
and ``$ \, \leq \, $'' primitives.
\end{deff}

Since  $L^0$ contains absolutely no growth
functions, it is so weak that it cannot even
formalize the integer 3 as an isolated term.
Thus,
\cite{wwapal} found it necessary
to strengthen  $L^0$'s language
with the two additional Type-NS 3-way predicates
$Add(x,y,z)$ and $Mult(x,y,z)$ (that formalize the concepts
of   ``$~x+y=z~$'' and ``$~x*y=z~$''). Their two formal
definitions are given below:
\beq
\small
  \label{newadd}
z ~ - ~ x ~~ =~~y ~~~~\wedge~~~~ z~~ \geq~~ x
\enq 
\begin{equation}
\small
  \label{newmult}
 \{ ~(x=0    \vee    y=0 ) \Rightarrow z=0~ \} ~ ~\wedge ~~ 
  \{ ~(x \neq 0 \wedge y \neq 0~) ~ \Rightarrow ~
(~ \frac{z}{x}=y  ~\wedge \, ~  \frac{z-1}{x}<y~~)~ \}
\end{equation}
More precisely to help Type-NS self-justifying axiom systems encode
integers distinctly larger than 2,
%%% the year-2006 paper
\cite{wwapal}
introduced some additional constant
symbols,
of $~a_2,~a_3,~a_4~....~$
and $~b_2,~b_3,~b_4~....~$  
into  $L^0~$'s language,  satisfying  the following
three constraints:
\newpage
\beq
\label{new1}
a_2~~=~~b_2~~=~~c_2~~=~~2
\enq
\beq
\label{new2}
a_{j+1}~~=~~a_{j}~+~a_{j} ~~~~\mbox{ for } j \geq 2
\enq
\beq
\label{new3}
b_{j+1}~~=~~b_{j}~*~b_{j} ~~~~\mbox{ for } j \geq 2
\enq
In particular, 
\el{new2} was called
 an
 {\bf ``Additive Naming Convention''  (ANC)}
 in \cite{wwapal}, 
 and
\el{new3} was called an {\bf ``Multiplicative Naming Convention'' (MNC)}.

The precise encoding for Lines
\eq{new2} and \eq{new3}  in \cite{wwapal}
is delicate 
because the earlier Theorem  ++
(of Pudl\'{a}k and Solovay)  implied
a Self-Justifying axiom system cannot treat either addition
or multiplications as total functions. 
This difficulty was resolved by applying
the  $Add(x,y,z)$ and  $Mult(x,y,z)$ 3-way predicates
(from Lines \eq{newadd} and \eq{newmult}$~).$
Thus,   reworded forms of Lines \eq{new2}
and
\eq{new3}, evading  $++$'s
generalization of the Second Incompleteness Theorem, 
appear below:
\beq
\label{new4}
\label{addcov}
ADD(a_j,a_j,a_{j+1}) ~~~~\mbox{ for any } j \geq 2
\enq
  \beq
  \label{new5}
  \label{multcov}
Mult(b_j,b_j,b_{j+1}) ~~~~\mbox{ for any } j \geq 2
\enq

\begin{deff}
\label{defx6}
\rm
The term {\bf Additive Naming Language} refers to the
extension of the  
$L^0~$ language
that defines the constant symbols
$~a_2,~a_3,~a_4~....~$
with the formalisms from
Lines \eq{new1}
and \eq{new4}. This revised version
of $L^0~$'s language will be denoted as $L^{\rm ANC}.$
Likewise
the phrase {\bf Multiply-Additive Naming Language}  refers to the
revision of the  
$L^0~$'s language
that defines all the constant symbols of
$~b_2,~b_3,~b_4~....~$
and $~a_2,~a_3,~a_4~....~,~$  
using the formalisms of
%via
Lines \eq{new1}, \eq{new4} and \eq{new5}. 
Its language
% will be
is
called  $L^{\rm MNC}.$
\end{deff}

An examination of these two naming conventions was
%% essentially
the core topic in \cite{wwapal}.
Roughly summarized, the Theorem 3 from \cite{wwapal} 
showed that  the language $L^{\rm ANC}$
can house
robust forms of self-justifying systems, and its Theorem 4
showed that the similar evasions of the Second Incompleteness
Theorem are impossible under  $L^{\rm MNC}$. 


What was missing from \cite{wwapal}
was an intuitive explanation about why  $L^{\rm MNC}~$'s failure
did not out-shine the partial (but limited)
success that $L^{\rm ANC}$
achieved. In other  words, one needs to ask:
{\it Which of these two languages 
provide a better standpoint
for
  approximating conventional
 theorem-proving?}

Our reply to the preceding question is that the language 
$L^{\rm ANC}$ is preferable because
Definition \ref{defx4}'s
 awkwardly
exponential
``USEGR''
growth suggests one must 
look merely at 
the  600-th item in the sequence 
$~b_2,~b_3,~b_4~....~$  to find an
element
whose
binary encoding-length exceeds the number of atoms lying inside
the universe.

\begin{remm}
  \label{remx}
  \rm
  \dvs
More specifically, we are suggesting the reader keep in mind
this dichotomy
when the next section
compares the Parts (a) and (b) of its Propositions
\ref{prop11} and  \ref{prop12}.
This is because the negativistic  Propositions
\ref{prop11}.b and  \ref{prop12}.b will
% shall
appear to be
% even more
divorced from reality, when their 
excessive
USEGR  growth rates
are noticed to
resemble
Sissa's  quite
outlandish exponential
 rates of growth. 
%% $\, 2^{64}  \,- \, 1 \,$ grains of
%% wheat.
\end{remm}

\section{More Details}

    %%% 555555555555555555555555

  \label{aaa5}

  
The primary goal 
in \cite{wwapal} 
was to determine when
%it was possible to formulate 
self-justification
was possible
under 
either
\eq{addcov}'s
 ``Additive'' 
or  \eq{multcov}'s 
 ``Multiplicative'' 
 naming
schema.
Some added notation is needed to briefly review
\cite{wwapal}'s
main
results.
In a context where
 $\, t \,$ denotes   any term in $\, L^{ANC} \,$'s   
language that does not contain the variable symbol of ``$~v~$'',
the quantifiers in
%%% the  wffs of
$~ \forall ~ v \leq t~~ \Psi (v)~$ and
$\exists ~ v \leq t~~ \Psi (v)$
will be  called $\, L^{ANC} \,$'s   
{\bf ``$~$v-Restricting 
  Quantifiers$~$''}. Then Definition \ref{def-3.8}
formalizes the 
analogs of
%% a
conventional arithmetic's
$\Delta_0$, $\Pi_n$ and $\Sigma_n$ 
formulae
in  $L^{ANC}\,$'s 
language:

\begin{deff}
\label{def-3.8}
\rm
Any formula in
$L^{ANC}\,$'s 
language will be called  $\Delta^{ANC}_0$
iff all its quantifiers variables $v$ meet the preceding
v-Restricting  constraint.  The  $\Pi^{ANC}_n$
 and $\Sigma^{ANC}_n$.
sentences of  $L^{ANC}$
are
then
defined by
%%% Items
1-3.
(They are analogous to
conventional arithmetic's
%%% $\Delta_0$, $\Pi_n$ and $\Sigma_n$ 
counterparts):
\bee
%\small
%\parskip -2 pt
%\baselineskip = 0.8 \normalbaselineskip 
\item
Every  
$\Delta_0^{ANC}$ formula is
%considered to be 
also 
a
$\Pi_0^{ANC}$  and 
an
$\Sigma_0^{ANC}  $ expression.
%% 
%% ``$~\Pi_0^{ANC}~ \,$''  and 
%% %  also 
%%  ``$~\Sigma_0^{ANC}~ \, $''. 
%% 
\item
A
formula
is
called
% defined to be
 $ \,\Pi_n^{ANC} \,$
when it
% is 
can be
encoded as 
$\forall v_1 ~ ...~ \forall v_k ~ \Phi$  
where
%with
$\Phi$ is  $\Sigma_{n-1}^{ANC}$.
\item
A formula
is
called
% defined to be
 $\Sigma_n^{ANC}$
when it can be encoded as 
$\exists v_1~ ...~ \exists v_k ~ \Phi,$  where
$\Phi$ is  $\Pi_{n-1}^{ANC}$.
\ene
\end{deff}



%%\begin{deff}
%%\label{def3.9}
%%\rm

\bigskip
 


Given an initial axiom system $\beta,$
the Theorem 3 of \cite{wwapal}
% defined 
%formalized
did formalize
a 
self-justifying logic, called
{\bf ISCE$(\beta)$}, 
that could prove all 
$~\beta\,$'s $\Pi_1^{ANC}$ theorems and 
which could also
verify its own consistency under any Hilbert-Frege
style deductive
apparatus.
The axiom basis for ISCE$(\beta)$ 
was comprised, formally,  of the following four
distinct
groups of axioms:
%
%  \newpage
\begin{description}
 \parskip 0pt
\item
{\bf GROUP-ZERO:} 
This 
schema
%  axiom group 
will formally represent the integers of 0,1 and 2
with the defined constant-symbols of
$\,c_0\,$, $\,c_1\,$ and $\,c_2\,$. 
It will also apply
\el{addcov}'s Additive Naming 
schema
to formally define 
 the further constants of
 $    a_2,~  a_3,~ a_4,~  ...   $  
\item
{\bf GROUP-1:} 
This axiom group  will consist of a
finite set of  
 $\Pi_1^{ANC} $ sentences, whose union with Group-zero
axioms is consistent and
can prove every $\Delta_0^{ANC}$ sentence that
holds true in the standard model.
(It was explained in \cite{wwapal} that any
set $~H~$, meeting these conditions,
may formalize the  
Group-1 axioms.)
\item
{\bf GROUP-2:}
Let
$\ulcorner \, \Phi \, \urcorner$ denote $\Phi$'s G\"{o}del number, and
$\mbox{HilbPrf}_{ \beta  }(x,y)$
denote a 
$\Delta _0^{ANC}$ 
formula indicating  $y$ is a
Hilbert-styled
proof
from axiom system $\beta  $ of the theorem 
$x$.
For each 
$\Pi_1^{ANC} $
 sentence  $\Phi$,
the Group-2 schema
will 
contain
an 
% approximate
encoding for 
\eq{group2old}'s
$\Pi_1^{ANC} $
axiom:
\begin{equation}
\label{group2old}
\forall ~y~~~\{~ \mbox{HilbPrf}~_\beta
~(~ \ulcorner \Phi \urcorner ~,~y~)~~
\Rightarrow ~~ \Phi~~\}
\end{equation}
\item
{\bf GROUP-3:}
This last part of
%%%%%%%%%%%%%%%  
\cite{wwapal}'s
ISCE$(\beta)$
formalism
is
% was
 a single 
self-referencing
$\Pi_1^{ANC}$
sentence  
stating:
 %% essentially declaring:
\begin{quote}
% \small
%%%%%%%%%%%%% $ \oplus ~ \oplus ~~~$
$ \oplus  \oplus ~~~$
 ``There 
%is 
exists
no
Hilbert-style proof of 0=1 from the union of the Group-0, 1 and 2
axioms  with {\it THIS  SENTENCE} (referring to itself)''.
\end{quote}
\end{description}
More details about  $ \oplus  \oplus$'s exact formal 
encoding ``fixed point'' encoding will not be mentioned here
because our earlier articles
\cite{ww1,ww5,wwapal} discussed this topic quite thoroughly.

  
\begin{deff}
  \label{defx8}
  \rm
  The symbols
  $\Pi^{MNC}_n$,
 $\Sigma^{MNC}_n$
  and $\Delta_0^{MNC}$
will denote the obvious analogs of  
Definition \ref{def-3.8}'s
  $\Pi^{ANC}_n$,
 $\Sigma^{ANC}_n$
and $\Delta_0^{ANC}$ sentences,
when the broader language $L^{MNC}$
of the Multiplicative Naming Convention
replaces the initial language of   $L^{ANC}$.
%
% utilized by the Additive Naming Convention.
%
Also, the symbol
ISCE$^{\rm MNC}(\beta)$
will denote the  intended
generalization of the preceding  
ISCE$(\beta)$
system where 
all references to the language   $L^{ANC}$
and its Additive Naming Convention
are
%% changed into
replaced by
statements about
the language   $L^{MNC}$,
along with
its Multiplicative Naming Convention,
 its
associated
 $\Pi^{MNC}_n$ 
and $\Delta_0^{MNC}$ sentences
and a natural $\Pi^{MNC}_1$ counterpart of
ISCE$(\beta)$'s particular
Group-3 {\it ``I am consistent''}
axiom.
(It  turns out that Proposition
\ref{prop11} will formalize that 
ISCE$^{\rm MNC}(\beta)$
typically
differs from 
ISCE$(\beta)$ by being
inconsistent.)
\end{deff}

\begin{deff}
  \label{defx9}
  \rm
Let $~I(~\bullet~)~$ denote
an operation that maps
an initial axiom basis $\, \beta \,$ onto an alternate
system  $\,I(\beta)\, $.
(One example of
such an operation is the
  ISCE$( \, \bullet \, )$ 
framework,
that maps 
an initial axiom basis of 
  $~\beta~$ onto 
the alternate formalism of
 ISCE$(\beta).~)~$ 
Such an operation  $~I(~\bullet~)~$
is  called {\bf Consistency Preserving}
iff  $\,I(\beta)\, $ is consistent whenever 
the union of
 $\beta$ with the Groups 0 and 1 axiom schemas is
consistent.
\end{deff}

\begin{propp}
  \label{prop11}
\rm
The   ISCE$(~\bullet~)~$
and  ISCE$^{\rm MNC} (~\bullet~)~$ transformations
have different properties, insofar as only the former is
Consistency Preserving. In particular, if PA$^{Ground}$
denotes the
extension of Peano Arithmetic that includes
both the conventional arithmetic operators and the
seven Grounding functions, then the following two invariants
do hold:
\bed
\item[  a.  ]
  ISCE(PA$^{Ground~}$)
  is a consistent system.
\item[  b.  ]
ISCE$^{\rm MNC}$(PA$^{Ground~}$) fails to be  consistent.
\ennd  
\end{propp}

It is unnecessary to prove  Propositions \ref{prop11}.a
and \ref{prop11}.b
 because  they both are easy consequences
of the Theorems 3 and 4 from \cite{wwapal}.
Instead, we will discuss in this paper how 
Proposition \ref{prop11} and some
 of its
generalizations
are related
to a needed Tripod-style response to Hilbert's
% famous
second problem.

It firstly should be noted that
Definition \ref{defx4}'s  very fast ``USEGR'' growth is
essential for appreciating   the sharp contrast  between 
Proposition \ref{prop11}.a
and 
Proposition \ref{prop11}.b.
This is because
the generalization of the Second Incompleteness Theorem, in 
Proposition \ref{prop11}.b,
arises because the MNC supports 
an unfortunately fatal
USEGR rate of
exponential  growth among numbers.

It next should be remarked that some half-analogs of 
Proposition \ref{prop11}'s twin 2-part statement for semantic
tableau deduction were established by \cite{ww1,ww2,ww5,ww9}.
Moreover, the recent year-2021 article \cite{ww21}
summarized and extended
these  articles.
Thus, let 
 IS$_{\rm Tab}(~\beta~)$
 be
\cite{ww21}'s
 4-part axiom system,
whose relationship to the ISCE$(\beta)$
system
is as follows:
\bed
\item[    i. ]
  The Group-1 and Group-2
  schemes
  for   IS$_{\rm Tab}(~\beta~)$
and  ISCE$(\beta)$
are
  essentially identical.
\item[   ii. ]
  The Group-Zero
  scheme
  for   IS$_{\rm Tab}(~\beta~)$
is stronger than that of
  ISCE$(\beta)$ because it replaces the 
  Additive Naming Convention with
a stronger 
more compact ``Type-A''
statement  declaring that the operations of
  Addition and Double$(x)~=~x+x~$ are total functions.
\item[   iii. ]
  The Group-3
  scheme
  for   IS$_{\rm Tab}(~\beta~)$
is, however, weaker that its analog under
ISCE$(\beta)$ because  it only recognizes its
self-consistency under a semantic tableau form of deduction.
\ennd

%%%% ggggggggggggggggggggggg

 \baselineskip = 1.12 \normalbaselineskip 

 \dvs



 \parskip 8pt


 
 \noindent
 In essence,
  IS$_{\rm Tab}(~\beta~)$
  is a self-justifying formalism that avoids being inconsistent
  by using a different type of trade-off
  than 
  ISCE$(\beta)$, where its Group-Zero schema  is stronger while
its Group-3 statement uses a weaker deductive methodology.

%%%%% \newpage
%nnnnnnn

\bigskip

% \parskip 11pt

Also, we will employ the Example 5.1 from \cite{ww21}, where
IS$^M_{\rm Tab}(~\beta~)$ denotes the natural modification
of  IS$_{\rm Tab}(~\beta~)$ wherein:
\bee
\item
  The  Group-Zero axiom further recognizes integer-multiplication
  as a total function.
\item
  The Group-3 axiom is the same as that described in (iii)
  except  that its Group-3
  {\it ``I am consistent''} statement has been  adjusted 
  to recognize
  that the Group-Zero formalism now treats
  integer-multiplication as a total function.
  \ene
  Then  Proposition \ref{prop12}
  is
%  represents
  \cite{ww21}'s  counterpart
to
%   for
Proposition \ref{prop11}.
(We again remind the reader about  Remark$~$\ref{remx}'s
warnings about excessively
demanding
USEGR  growth rates.
It is the
 intuitive
reason behind the two inconsistency
results 
appearing in both
the Parts (b) of the
Propositions \ref{prop11}
and \ref{prop12}.)


\begin{propp}
  \label{prop12}
  \rm
  The
 IS$_{\rm Tab}(~$PA$^{\rm Ground}~)$
 and  IS$^M_{\rm Tab}(~$PA$^{\rm Ground}~)$
 systems
 differ
  insofar as:
\bed
\item[  a.  ]
   IS$_{\rm Tab}(~$PA$^{\rm Ground}~)$
  is  consistent.
\item[  b.  ]
   IS$^M_{\rm Tab}(~$PA$^{\rm Ground}~)$
  fails to be  consistent.
  \ennd
\end{propp}


 Formal   analogs of Propositions \ref{prop12}.a and   \ref{prop12}.b
 were established in \cite{ww21}, and we shall not repeat that
 discussion here. Instead, our discussion will focus on the two consistency
 results established by
 Propositions \ref{prop11}.a and   \ref{prop12}.a.
 The Remark \ref{rm13} will explain how there exists a quite natural
 interface between these
 two results
with
 the 
SBSIS-styled approach that  Artemov had
advocated in   \cite{Ar19,Ar21}.
 



%% \def\fggg{ \normalsize  \baselineskip = 2.4 \normalbaselineskip } 

%%% \fggg

\begin{remm}
  \label{rm13}
 \baselineskip = 1.17 \normalbaselineskip 
 \rm
 \dvs
     {\bf (about how the Second and Third Legs of a ``Tripod''
      Reply to Hilbert's Second  Problem will interact
Quite Nicely
       with Each Other) :}
%%% in a Beautiful Manner) :}        
Let us recall that Step \ref{step2}
of  Artemov's
SBSIS-styled   \cite{Ar19,Ar21}
constructions
(summarized earlier
in \textsection\ref{aaa1})
%%% proved
constructed
a class of $\Pi_1-$like theorems, 
$~T_1,~T_2,~T_3,~...~$,
which confirmed 
the consistency of  the respective
 $S_0,~S_1,~S_2,~...~~$ subsets of PA.
It turns out that the  validity of these
 $~T_j~$ theorems
is  known
 to both
 the preceding
ISCE$(~$PA$^{\rm Ground}~)$
and
IS$_{\rm Tab}(~$PA$^{\rm Ground}~)$ formalisms from
Propositions \ref{prop11}.a
and \ref{prop12}.a
(see Footnote
  \footnote{This is because the Group-2 schemes of
ISCE$(~$PA$^{\rm Ground}~)$
and
IS$_{\rm Tab}(~$PA$^{\rm Ground}~)$
% do
both
formally
recognize the validity of all of PA's $\Pi_1$ theorems.
Thus using a direct analog of Artemov's
proposed
methodology in \cite{Ar19,Ar21},
they can
prove each of
\cite{Ar19,Ar21}'s
theorems of
$~T_1,~T_2,~T_3,~...~$.  }
$~$).
  The remainder of this section will explain
  how this subtle interconnection causes the Second and Third legs of
a proposed Tripod Reply to Hilbert's Second Open Question   
to interact
quite eloquently
%%% nicely
with each other.
\bigskip
\end{remm}


The central point is that both Artemov's formalisms in 
\cite{Ar19,Ar21} and Willard's approach
in \cite{ww93}-\cite{ww21}
own distinct partial drawbacks. It turns out that one can
essentially 
overcome
these deficiencies by using Remark \ref{rm13}'s
theoretical hybrid methodology.
In particular, the partial drawbacks which 
 need to be addressed
are listed below:
\bed
\item[   a.  ]
  The partial drawback to Artemov's 
SBSIS-styled method \cite{Ar19,Ar21}
is that it does not produce one single theorem, asserting
a formalism's overall consistency. (Instead, it produces a sequence of
theorems  $~T_1,~T_2,~T_3,~...~$,
whose approximate union
comprises the desired {\it ``I am consistent''} statement.)
\item[   b.  ]
  The comparable drawback in Willard's
self-justifying
  formalisms in
\cite{ww93}-\cite{ww21}'s
  28-year long
  series of 
  articles 
  is that none of these
papers involve a system as powerful as Peano Arithmetic
  declaring its own self-consistency.
  \ennd
  The nice aspect of
Remark \ref{rm13}'s
  hybridization of these two methods is   
  that its
  methodology
  views the two formalisms from
%%  in  
Propositions \ref{prop11}.a
and \ref{prop12}.a as declaring their self-consistencies under certain
%%% particular
unifying perspectives,
while owning a formalized awareness about
Artemov's
broader expanding  series of
theorems   $~T_1,~T_2,~T_3,~...~$.
This paradigm overcomes the main challenges
that were
posed by Items (a) and (b).

{\it For the sake of clarity, 
Remark \ref{rm13}'s observations
certainly should not be viewed
as being a full panacea}. Thus,
the 
Second Incompleteness Theorem certainly imposes
%%% some very
severe
restrictions
upon any attempts to
%%% fully
evade it.
Thus
while not fully embracing the  philosophical positions
of Hilbert's and G\"{o}del's
statements of
$*$ and $**$, our perspective in Remark \ref{rm13}
certainly conveys definite strong partial support for 
Hilbert's and G\"{o}del's general approach.

In other words, a 3-way ``Tripod'' reply for Hilbert's Second
Open Problem appears to be attractive
because a
more
compressed
% abbreviated
one-or-two leg reply to Hilbert's
foundational
Second Problem
appears to be essentially
%%%much
over-simplistic.

\section{Additional Curious Aspects about Self-Justification}

    %%% 66666666666666666666666 

  \label{aaa6}


Contrary to the impression that may have been conveyed by
the 28-year-long literature
about self-justification
\cite{ww93}-\cite{ww21},
these papers {\it were
 not actually}
 intended to completely 
 reject the assumption
that integer-multiplication is a total function.
This is because several {\it near-cousins} of  integer-multiplication
% were
are
allowed
within the languages of  \cite{ww93}-\cite{ww21}. 

This point was first made in \cite{ww93}'s  initial 1993 paper. 
Its Item (viii) on page 326
suggested using a function called ``AndMultiply$(x,y,z)$''
The latter function first
computes an integer $~v~$, derived by
taking the
multiplicative product of $~x~$ and  $~y~$, and it then
computes the Bit-Wise-AND of
$~v~$ and  $~z~$ so that its final
output
is produced.

% of  ``$~w~$'' can be produced.

This ``AndMultiply$(x,y,z)$'' operation can be viewed as
a non-growth operator
because AndMultiply$(x,y,z)~\leq~$Max$(x,y,z)$. Thus, it can
be easily incorporated into the Type-A
%%% ``U-Grounding''
Self-Justifying
% logical
formalisms of \cite{ww1,ww5,ww21}. This raises the following
tantalizing issue:

%question:

\begin{description}
\item{ +++ }  
\it  Could  the main mysteries about a formalism's inability to
  simultaneously recognize its semantic tableau consistency and
  multiplication as a total function be
 philosophically
  resolved by
letting
``AndMultiply'' replace
% the
% traditional
% role of
the classic Integer-Multiply operation?
\end{description}

\noindent
Furthermore, if ``AndAddition$(x,y,z)$''
% is
designates
the straightforward analog of the
primitive
``AndMultiply$(x,y,z)$''
for Integer-Addition,
then the AndAddition$(x,y,z)$ operator can be
easily
incorporated into the Type-NS Grounding-language
``ISCE'' system (which did
appear in both this article and in its year-2006
predecessor \cite{wwapal} ).

These two  AndAddition$(x,y,z)$
and AndMultiply$(x,y,z)$ operations are not as powerful as
the more conventional 
classic Integer-Addition and Integer-Multiplication
operations.
However, their potential in
several
pragmatic engineering environments
should not be under-estimated.
% overlooked.

Also, it should be mentioned that the results in
\cite{ww93}-\cite{ww21} can be further
%extended
amplified
by using
\cite{ww6}'s Floating-Point-With-Rounding Multiplication
operation (FPWRM). This FPWRM primitive suggests
\cite{ww6}
that
floating point arithmetics, unlike classic  integer arithmetics,
are  compatible with  self-justifying logics
using semantic tableau deduction.

\section{Overall Perspective}

%%% 7777777777777777

\label{aaa7}

The  preceding discussion has focused mainly
on only one of the three legs of a more elaborate
Tripod Response to Hilbert's Second Problem.
The other two legs, involving  generalizations
of G\'{o}del's
 Second Incompleteness Theorem
%%% cite?????? 
and Artemov's
step-by-step
SBSIS-like approach \cite{Ar19,Ar21},
have been surveyed only very much more briefly.
Together, these three legs
formulate a triple
 reply to Hilbert's Second Problem,
which is
significantly
more far-reaching than a more isolated
type of
one-or-two leg
response.

Collectively,
these three legs
clarify the nature of the statements  
$*$ and $**$, which 
Hilbert and G\"{o}del had  articulated.
Moreover, the Question +++ (in
\textsection\ref{aaa6}) is quite tantalizing.

In essence, Hilbert's Second Problem 
is so complex that
an isolated one-or-two legged
reply  is insufficient to
answer Hilbert's fascinating question.
Moreover,
Gerald Sacks
has
recalled 
G\"{o}del 
expressing
opinions that were
{\it ``almost the opposite of what every one
  else would have expected''} him to make
(see again \textsection\ref{aaa2} ).
% for details).
Thus, he would presumably quite
directly
approve the philosophical
positions that Remark \ref{rm13} and Section \ref{aaa6}
%% (in the current article)
have formulated,
since G\"{o}del
repeated
several
analogs of his
often-quoted controversial
1931
remark $\, **\,$
during numerous private
conversations
 G\"{o}del shared
with Gerald
Sacks  \cite{YouSa14}.


%\baselineskip = 1.0  \normalbaselineskip
%\noindent
%  \medskip

{\bf Acknowledgment:}
I thank Seth Chaiken for
helping   improve the
presentation.  


%I thank Robert  Willard for
%help   improving th
% presentation.  


% \newpage

\footnotesize
\baselineskip = 0.80 \normalbaselineskip 




 \begin{thebibliography}{99}

  %rrrrrrrrrrrrrrrrrrrrrrrrrrrrrrrrr
   
\baselineskip =  1.01 \normalbaselineskip

\parskip 1 pt



\bibitem{AZ1}
Adamowicz, Z., 
Zbierski,  P.:
  On Herbrand consistency in weak theories. 
{\it Archive for Mathematical Logic}
40(6):  399-413 (2001).    


\bibitem{Ar1}
Artemov,  S.:
Explicit provablity and constructive semantics.
{\it 
Bulletin of Symbolic Logic}
7(1): 1-36 (2001).

\bibitem{Ar19}
Artemov,  S.:
    The provability of consistency. {\it
Cornell Archives arXiv Report}      
1902.07404v5
         (2020).

\bibitem{Ar21}
Artemov,  S.:
Missing Proofs and the Provability of Consistency. 
Video presentation at the {\it 2021 T\"{u}bingen conference
``Celebrating 90 Years of G\"{o}del's Incomppleteness
Theorem''.} For the video recording see:
\newline
https://www.youtube.com/watch?v=Nw4nERAqi2U



\bibitem{AB5}
Artemov,  S.,
Beklemishev,  L. D.:
Provability logic. In
{\it
  Handbook of Philosophical Logic, Second Edition},    
       pp.     189-360
  (2005).

\bibitem{Be5}
Beklemishev,  L. D.:
Reflection principles and provability algebras in formal arithmetic.
{\it Russian Mathematical Surveys} 
60(2):  197-268  (2005).   



\bibitem{Be14}
Beklemishev,  L. D.:
Positive provability for uniform reflection principles.
{\it   Annals Pure \& Applied Logic} 
   165(1): 82-105  (2014).


   
\bibitem{BS76}
Bezboruah, A., 
 Shepherdson, J. C.:
G\"{o}del's second incompleteness theorem for Q.
{\it Journal of Symbolic Logic} 
 41(2):  503-512 (1976). 


\bibitem{Bu86}
Buss, S. R.:
 {\it Bounded Arithmetic.} 
Studies in Proof Theory, Lecture Notes 3, disseminated
by  Bibliopolis
as revised version of Ph. D. Thesis
    (1986).



\bibitem{BI95}
Buss, S. R., 
Ignjatovi\'{c},  A.:
Unprovability of consistency statements in fragments of
bounded arithmetic.
{\it  Annals Pure \& Applied Logic } 
 74(3): 221-244 (1995).

\bibitem{Da97}
Dawson,   J.  W.:
         {\it Logical Dilemmas: The Life and Work of
Kurt G\"{o}del},
   AKPeters Press
          (1997).


\bibitem{Fe60}
Feferman, S.:
 Arithmetization of mathematics in a general setting.
   {\it Fundmenta Mathematicae}
49: 35-92 (1960).


\bibitem{Fi96}
Fitting, M.: 
{\it First Order Logic and Automated Theorem Proving.}
Springer-Verlag (1996).


\bibitem{Fr79a}
Friedman, H. M.: 
 On the consistency, completeness and correctness
problems.
 Ohio State  Tech Report (1979).
 See also  Pudl\'{a}k \cite{Pu96}'s summary of this result.



\bibitem{Go31}
G\"{o}del, K.:
 \"{U}ber 
formal unentscheidbare S\"{a}tze der Principia
Mathematica und verwandte Systeme I.
{\it Monatshefte f\"{u}r Mathematik und Physik} 38:  349-360 (1931).




\bibitem{Go33}
G\"{o}del, K.:
     The present situation in the foundations of
mathematics.
 In  {\it Collected Works Volume III: Unpublished Essays and Lectures}
(eds. Feferman, S.,
Dawson, J. W.,
Goldfarb, W.,
Parsons, C.,
Solovay, R. M.),
        pp.       {45--53},
     Oxford    University Press (2004).


\bibitem{Go5}
Goldstein,        R.:
   {\it Incompleteness: The Proof and Paradox of Kurt G\"{o}del}.
        {Norton} Press
         (2005).



\bibitem{Ha11}
H\'{a}jek, P.:
              Towards metamathematics of weak arithmetics over
fuzzy logics.
{\it Logic Journal of the IPL}
      19: 467-475
          (2011).

\bibitem{HP91}
H\'{a}jek, P.,
Pudl\'{a}k, P.:
{\it Metamathematics of First Order Arithmetic.}
Springer (1991). 

\bibitem{Hil26}
Hilbert, D.:
                 \"{U}ber das unendliche.
{\it
   Mathematische Annalen}    
      95: 161-191
 (1926).


 

\bibitem{Kl38}
Kleene,  S. C.:
On  notation for ordinal numbers.
{\it Journal of Symbolic Logic} 
3(1): 150-15 (1938).



\bibitem{KT74}
Kreisel, G., 
Takeuti,  G.:
Formally self-referential propositions  for cut-free classical
analysis.
{\it Dissertationes Mathematicae}
118: 1-50
(1974).


\bibitem{Ne86}
Nelson, E.:
 {\it  Predicative Arithmetic.}
Mathematics Notes,
 Princeton University Press
  (1986)

\bibitem{Ne20}
Nerode, A:
 Some useful comments by A. Nerode  on Janaury 7, 2020
when he heard Willard's conference  talk
at the LFCS 2020  conference. Nerode said
  he could reinforce
Gerald Sacks's YouTube
presentation \cite{YouSa14} (because
Nerode heard Stanley  Tennenbaum
make similar comments
about
G\"{o}del).


\bibitem{Pa71}
Parikh, R.:
 Existence and feasibility in arithmetic.
  {\it Journal of Symbolic Logic} 
   36(3): 494-508
(1971).

\bibitem{PD83}
Paris, J. B.,  
Dimitracopoulos,  C.:
A note on the 
undefinability of cuts. 
{\it Journal of Symbolic Logic} 
48(3):
  564-569
(1983).


\bibitem{Pa72}
Parsons, C.:
On $n-$quantifier elimination.
{\it Journal of Symbolic Logic} 
37(3):  466-482
 (1972).


\bibitem{Pu85}
Pudl\'{a}k, P.:
Cuts, consistency statements and interpretations.
{\it Journal of Symbolic Logic} 
50(2):   423-441
(1985).

\bibitem{Pu96}
Pudl\'{a}k, P.:
On the lengths of proofs of consistency.
 {\it Collegium Logicum: 1996 Annals of the Kurt G\"{o}del
Society} (Volume 2, pp 65-86).  Springer-Wien-NewYork
(1996).


\bibitem{Ro67}
Rogers, H. A.:
 {\it 
Theory of Recursive Functions and Effective 
Computibility.} McGrawHill (1967).


\bibitem{YouSa14}
  Sacks, G.:
  Reflections on G\"{o}del. A YouTube recorded talk
  delivered 
  on 11$~$April$~$2007 as Lecture 3 in
  {\it The Thomas and Yvonne Williams Symposia for
    the Advancement of Logic}. 
The easiest way to find a video recording of the
Upenn-Sacks lecture is to do a google-search on the
keywords of
 {\it ``Gerald Sacks,
   Reflections on Goedel, YouTube''}. This video is also
 among several recordings stored at the deposit  site of:
 
{\small
  
https://itunes.apple.com/us/itunes-u/lectures-events-williams-lecture/id431294044

}

\bibitem{Sm95}
Smullyan, R.: 
{\it First Order Logic}. Dover Books (1995).




\bibitem{So88}
Solovay, R. M.:
Injecting Inconsistencies into models of PA.
{\it   Annals Pure \& Applied Logic } 
  44(1-2): 102-132 (1989).


\bibitem{So94}
Solovay, R. M.:
Private
Telephone  
conversations
during April of  1994
between Dan Willard and
 Robert M. Solovay.
During those conversations
Solovay described 
an unpublished
 generalization of one of  Pudl\'{a}k's  theorems 
\cite{Pu85},
using 
some 
methods
of Nelson and Wilkie-Paris \cite{Ne86,WP87}.
(The Appendix A of 
\cite{ww1} offers a 
4-page summary of
our interpretation of Solovay's remarks.
Several other articles 
\cite{BI95,HP91,Ne86,PD83,Pu85,WP87}
have also noted
that Solovay  often
has  chosen to privately communicate noteworthy
 insights that he has elected not
to formally publish.)


\bibitem{Sv7}
\v{S}vejdar, V.:
 An interpretation of Robinson arithmetic in its
Grzegorczjk's weaker variant.
{\it Fundamenta Informaticae} 81:  347-354
(2007).

\bibitem{Vi5}
Visser, A.:
Faith and falsity.
{\it        Annals Pure \& Applied Logic } 
131(1-3): 103--131
     (2005).


\bibitem{WP87}  
Wilkie, A. J.,
Paris, J. B.:  
On the scheme of induction for bounded
 arithmetic. {\it  Annals Pure \& Applied Logic } 
 35: 261-302 (1987).

\baselineskip =  1.0 \normalbaselineskip

\parskip 0 pt


 
\bibitem{ww93}
Willard, D. E.:
Self-verifying axiom systems.
 In Proceedings of
the   Third Kurt G\"{o}del
Colloquium
(with eds.
Gottlob, G.,
Leitsch, A.,
Munduci, D.),
{\it   LNCS } vol.
713.
pp.  325-336,
Springer, Heidelberg
(1993).
https://doi.org/10.1007/BFb0022580.



  

\bibitem{ww1}
Willard, D. E.:
Self-verifying  systems, the incompleteness
theorem and the tangibility reflection
principle.
{\it Journal of Symbolic Logic} 
66(2):  536-596
(2001).



\bibitem{ww2}
Willard, D. E.:
How to extend the semantic tableaux and
cut-free versions of the second
incompleteness theorem 
almost 
to 
Robinson's arithmetic Q.
{\it Journal of Symbolic Logic} 
67(1): 465--496
(2002). 


%% 
%% \bibitem{wwlogos}
%% Willard, D. E.:
%% A version of the
%% second incompleteness theorem for axiom
%% systems that recognize addition 
%% but not multiplication as a total function.
%% In {\it First Order Logic Revisited}
%% (eds.
%%  Hendricks, V.,
%% Neuhaus, F.,
%% Pedersen, S. A.,
%% Sheffler, U.,
%% Wansing, H.),
%% pp. 337--368,
%% Logos Verlag,
%%    Berlin
%%  (2004). 


\bibitem{ww5}
Willard, D. E.:
An exploration of the partial respects in which an axiom
system recognizing solely addition as a total function can
verify its own consistency.
{\it Journal of Symbolic Logic} 
70(4):  1171-1209  (2005).

\bibitem{wwapal}
Willard, D. E.:
A generalization of the second incompleteness 
theorem and some exceptions to it.
{\it  Annals Pure \& Applied Logic } 
141(3):  472-496
(2006).



\bibitem{ww6}
Willard, D. E.:
On the available partial respects in which
 an axiomatization
for real valued  arithmetic can  recognize its 
consistency.
{\it Journal of Symbolic Logic} 
 71(4):  1189-1199 
(2006).


\bibitem{ww7}
Willard, D. E.:
Passive induction and a solution to a Paris-Wilkie 
open question.
{\it  Annals Pure \& Applied Logic } 
146(2):  124-149
(2007). 


\bibitem{ww9}
Willard, D. E.:
Some specially formulated axiomizations for
I$\Sigma_0$ 
manage to
evade
the Herbrandized version of the second incompleteness theorem,
{\it Information and Computation}
207(10): 1078-1093
(2009).



\bibitem{ww16}
Willard, D. E.:
On  how the introducing of a 
 new $~\theta~$ function symbol
into arithmetic's formalism is 
germane
to devising axiom systems that can 
appreciate fragments of their own
Hilbert consistency.
{\it
Cornell Archives arXiv Report}      
1612.08071
         (2016).


%%% 
%%% \bibitem{ww18}
%%% Willard, D. E.:
%%% About   the 
%%% chasm separating the goals of
%%% Hilbert's consistency program from the
%%% second incompleteness theorem.
%%% {\it
%%% Cornell Archives arXiv Report} 
%%% 1807.04717
%%%          (2018). (This quite preliminary annnouncement of
%%% \cite{ww20}'s results
%%% appears  in an
%%%  {\it  essentially roughly written} summary-abstract
%%% form.)


\bibitem{ww21}
Willard, D. E.:
About the Characterization of a
     Fine Line That Separates
      Generalizations and Boundary-Case Exceptions for the
 Second Incompleteness Theorem under Semantic Tableau
 Deduction'',
 {\it Journal of Logic and Computation}
 31(1):375-392 (2021).

%%  
%% OLD DELETED
%% ``On the tender line
%% separating generalizations and boundary-case exceptions for the
%% second incompleteness theorem under semantic tableaux
%% deduction'',
%% a talk given 
%% on January 7 at the LFCS 2020 conference.
%% It preceded the current article, and
%% its shorter manuscript can be found on
%% pp.  268-286 of 
%% Volume 11972 of
%%  Springer's LNCS series.

\bibitem{Yo5}
Yourgrau, P.:
              {\it A World Without Time: The Forgotten Legacy of
G\"{o}del and Einstein}.
(See page 58 for the passages we have quoted.)
        {Basic Books}
          ( 2005).



\end{thebibliography}

\end{document}




\bibliographystyle{abbrv}

\bibliography{bb}

\end{document}


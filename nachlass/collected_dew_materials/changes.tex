
Suggested Changes for Dan Willard's Wiki Page

Thanks for your help with Dan Willard's wiki page, and below
is a list of four minor new updates (one of which I suspect you will
find to be quite interesting and surprisingly intriguing):

1) Please change the first clause in the ``Contributions'' Section
to read as given below. Thus, the initial words of this passage will be
the exact same as your words.  The only change is the new suggested words on the
second line below. This added new short passage starts with phrase of ``as well as''

``Although trained as a mathematician and employed as a computer scientist
     as well as an expert in the proof-theoretic aspects of Symbolic Logic,...''

2) Just before the ``SELECTED PUBLICATIONS'' section, please add the following tiny
new section entitled ``MORE DETAILS about WILLARD'S LOGIC RESEARCH''. (You may
either copy the below passage verbatim or edit it, if you so wish. I suspect you
will find the below passage to be surprisingly interesting and informative):

Willard's most recent published paper during his logic investigations appeared
during 2021 in Oxford University's Journal of Logic and Computation (publication 8
whose exact citation you can find in the Item A-1 of my vitae). This article
both extends Willard's prior research, and it also summarizes the implications
of his prior and on-going research. It explains more clearly than before the
motivation for Willard's unorthodox exploration
of boundary-case exceptions to G\"{o}del's Second Incompleteness Theorem.
It thereby contains a pointer to an 80-minute year-2007 You-Tube lecture,
delivered by Prof. Gerald Sacks (who had visited Kurt G\"{o}del for two extended 
periods at the Institute for Advanced Studies and also did hold two simultaneous
joint appointments with Harvard and MIT). Sacks explains that
the reticent Kurt G\"{o}del had expressed startling
private opinions ``that were almost the  opposite of what everyone
else would have expected''. Sacks thus indicated that  G\"{o}del believed some type
of expanded Gentzen-like exception to  G\"{o}del's G-2 result would eventually
accompany  G\"{o}del's G-2 formalism.  In essence, Willard's logic research, which
started in 1993, was based on intuitions that were approximately similar
to the comments that Gerald Sacks had much later attributed to Kurt G\"{o}del during
Sacks's year-2007 You-Tube lecture. More details about this 80-minute You-Tube lecture
and its exact relationship to Willard' research can be found in Willard's recent
year-2021 JLC article.

3. Please replace my old vitae with the  attached  updated year-2021 version
of this vitae.

4. Please insert an eighth article into my list of selected publications on the
wiki page.  It correspond to Item A-1 in my vitae's publication list.

Again, I thank you very warmly for setting up my wiki page, and I have
carefully described these changes so that you can easily undertake them.
It is obviously significant that a Harvard-MIT professor, who was a close
friend of Kurt G\"{o}del, recorded a year-2007 You-Tube lecture, that was consistent
with a theory that I started advocating in 1993. To save you work, my Item 2
has outlined a new paragraph that you can add to my wiki-page. Thus, you can either
employ Item 2 verbatim or revise it, if you so wish. You can telephone me at
518-475-1622 or email me at dan.willard.albany@gmail.com  ---- Very Best Regards, Dan

